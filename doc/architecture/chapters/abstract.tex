%% The following is a directive for TeXShop to indicate the main file
%%!TEX root = diss.tex

\chapter{Abstract}

\tm{This needs to be re-written to conform to the new story.}

Modern computer storage systems have evolved from\reto{I expect a from ...
    to}\tm{It does have a from/to structure but apparently it isn't clear enough.} a simple model in which names denoted
the information necessary to know both \textit{where} a file was stored as well
as \textit{what} the file contains.\reto{maybe want to use "path" here?
    I'm not so sure, if I follow the "what the file contains" aspect here.
    One possible approach would be to use the path as an example that "related" files are having a common path prefix i.e. they are in the same directory.
    Convoluting the location and relationship.
    with path you may be able to do the path => uri transition when moving to other storage silos
}
\tm{I would argue that using the path focused model is too narrow.  We
    definitely have name spaces that do not use paths (e.g., S3, which is an object
    store).  Assuming that the only thing we have is a path is a bit limiting.
}
As computer storage systems have evolved we
have seen an increase in the diverse functional behavior offered for storing
files.  This has led to a situation in which users use distinct storage silos.
Each of these storage silos uses its own naming scheme, provides support for
specific usage, and has its own unique meta-data support.

The size, type, location, and capabilities of storage silos continues to
proliferate.  Storage silos can be local to a computer, shared with other local computers,
form geo-replicated distributed storage, consist of object storage or
traditional hierarchical file organization.  Non-traditional storage systems
have emerged through the use of collaborative tools such as Teams, Slack, and
Discord, each of which creates another location for storing in\-for\-ma\-tion shared
between human users.\reto{template error?}\tm{This seems to be an issue with
    hyphenation.  I normally don't try to address this early on unless it is
    interfering with the understanding.  I have added soft hyphens in the word so
    hopefully \LaTeX will hyphenate it properly.}

These storage silos do not provide mechanisms for tracking how information is
stored across these various silos.  This means is a user trying to locate a
document she saved while collaborating with a colleague will need to search
through each of these storage silos that she routinely uses: was the document
attached as an e-mail, sent via a collaboration service, stored in a shared
cloud storage location will have a difficult time locating that document six
months later. This problem is only exacerbated as documents migrate because the
version she does find might not be the correct one.

This work proposes to develop a model to separate naming from location, which
enables the construction of dynamic cross-silo human usable name\-spaces and show
how that model extends the utility of computer storage to better meet the needs of
human users.

% Consider placing version information if you circulate multiple drafts
\vfill
\begin{center}
    \begin{sf}
        \fbox{Revision: \today}
    \end{sf}
\end{center}
