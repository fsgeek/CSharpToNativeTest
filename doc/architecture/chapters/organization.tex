\chapter{Organization Level Naming}
\label{ch:organization}

The second contribution that I propose for my thesis is to identify,
implement, and evaluate naming at the organizational level.  This system will
expand upon the work done for \autoref{ch:local} with an emphasis on considering
the issues in naming for organizations.  The key issues that I propose
addressing are: organizational roles and adaptability, working collaboration,
and security.

\begin{description}
    \item[Organizational roles and adaptability] --- individuals play some role within an
        organization, which means that the responsibilities of a given role may be
        given to a new member of the organization.  Part of the assignment of
        existing responsibilities to someone new includes ``ownership'' of data and
        information that are part of the role.  In this scenario the ability to find
        things becomes important in a way different than considered in
        \autoref{ch:local} because there is no cognitive model for the
        organizational structure of the existing information.

    \item[Working collaboration] --- teams of people may be called upon to
        collaborate on various projects over time.  Consideration of this model is
        core to the design principles I set forth previously (\autoref{ch:model})
        but were not a core consideration for the earlier proposed work
        (\autoref{ch:local}).

    \item[Security] --- the initial model of security for local naming
        (\autoref{ch:local}) was deliberately simple, since the owner of the data
        had access to it, and others did not.  With multiple members in an
        organization, security becomes a more relevant consideration. The focus here
        is not on \emph{data} security, as that has been well-addressed previously,
        but rather \emph{meta-data} security.  As we collect more meta-data within
        the system, it becomes possible to infer information about the data. Because
        an organization might have information that is considered sensitive,
        security is a consideration for \system.
\end{description}

\tm{This is still way too vague.}

