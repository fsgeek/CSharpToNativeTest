\chapter{Plan}
\label{ch:plan}

\reto{
    I recall, we had this to be structured in work packages (of course with some more information):\\%
    WP1: Literature Survey\\%
    - Duration: 3m\\%
    - State: Completed\\%
    WP2: Model definition\\%
    - Duration: 6m\\%
    - State: On going\\%
    WP3: storage silo implementation\\%
    - Duration: 6m\\%
    - State: not started. \\%
    WPN: write thesis\\%
    - Duration: 3m\\%
    - State: not started. \\%
}

\section{Contributions}
\label{ch:introduction:sec:contributions}

\begin{epigraph}
    \textit{I don't let the cleaners in\ldots because \textbf{I} know where
        everything in this room is.  All the books, the papers --- and the moment
        they start cleaning, those things get hopelessly organized and tucked away
        and I can never find them again.} --- Crown of Midnight, Sarah J. Maas
\end{epigraph}

This thesis proposal describes what I intend to contribute to our understanding
of pragmatic file systems:

\begin{enumerate}
    \item \label{contribution:model} An explicit model for what computer storage
          naming is: how and why we name data objects, what a name \emph{should be} in a
          computer storage context to correspond how names are used and formed \reto{that's probably going to be super related to Saltzers work... };
          and
    \item \label{contribution:arch} An architecture and design of a potential
          naming system that supports the naming model (\autoref{contribution:model}
          that is sufficiently flexible to work across the full range of current and
          prospective computer storage needs: in-memory compute systems, small
          devices, computer workstations, enterprise scale storage systems, and
          distributed, geo-replicated cloud storage systems, which permits the
          construction of new purpose-built storage systems with the strong naming
          support as well
          as integration of existing storage systems; and
    \item \label{contribution:purpose-built} Implement and evaluate a novel
          storage system implementation based on this design (\autoref{contribution:arch}) that
          demonstrate strong naming support within a single storage silo model; and
    \item \label{contribution:existing} Implement and evaluate an implementation
          of the richer naming system (\autoref{contribution:arch}) on an existing
          computer storage system; and
    \item \label{contribution:combined-system}  Implement and evaluate the
          combination of both the novel storage system implementation
          (\autoref{contribution:purpose-built}) and the enhanced legacy system
          (\autoref{contribution:existing}) into a unified system consistent with the
          model (\autoref{contribution:arch}).

\end{enumerate}

To provide these contributions I must implement a number of novel new
approaches to naming in computer storage systems.  In this chapter I provide a
high level overview of my plan to accomplish this.

\section{Research Questions}
\label{ch:plan:sec:research-questions}

\reto{Somehow I feel like the research questions should go into the problem statement section.
    (unless that's required to behere)
}

This plan is focused on answering the following research questions:

\begin{itemize}
    \item \textbf{Can a storage silo implementation that separates naming, meta-data,
              and storage management into distinct components be constructed without
              sacrificing performance?}\label{ch:plan:rq:naming-separation-performance}  It is easy to talk about \emph{functionality} but
          there is a real risk with a radically new architecture that it will
          not be usable due to the performance costs. Part of the challenge in
          answering this research question is to pick appropriate benchmarks.  My goal
          is to demonstrate there is \emph{no} significant cost associated with I/O
          performance (less than 5\%), and that there is at most modest cost
          associated with meta-data performance (creating, opening, deleting,
          renaming) versus existing systems (less than 10\%).  While I think it would
          be useful to work with someone interested in data visualization to
          demonstrate improvements in data analysis and access, I do not propose doing
          this as part of my own work.  I discuss this in more detail in
          \autoref{ch:plan:sec:storage-silo-implementation}.

    \item \textbf{Can a federated meta-data service be constructed to provide meaningful
              security guarantees against information leakage?}
          \label{ch:plan:rq:security} By increasing the exposure
          of information via enhanced/augmented meta-data there are clear benefits for
          compliance measures, but these are achieved at the expense of making
          additional meta-data available.  Part of the original design was to address
          these concerns and this research question evaluates the efficacy of that
          design.  To do this will require examining the security model my system
          uses, the potential information exposure, and at least a first-order
          analysis of the impact of information leaks.  Depending upon my analysis of
          the answer to this question it may be necessary to augment the \system
          security model to better protect the information.

    \item \textbf{How can \system, implemented as a cross-silo naming and meta-data service,
              provide measurable benefits to those using this system?}
          \label{ch:plan:rq:cross-silo-naming-benefits} The research
          question itself is vague at this point because while I can suggest
          anticipated benefits in terms of cross-silo data location it is difficult to
          identify the ``benefits'' let alone quantify them.  Despite this, both will
          be necessary, which makes this a ``high risk'' research question. \tm{I
              don't think this is a good research question as written, so I need to
              re-work this\ldots again.  The difficulty here is that measuring this is not
              going to be particularly easy, unless we turn this into an \ac{HCI}
              activity, such as by looking at ``abandonment rates'' when people start
              searching for things.  That \emph{may} be what is required, though, to make
              this argument because a performance eval of the software doesn't make much
              sense --- to what would I compare this?}

\end{itemize}

My model for \system is quite broad.  Building the pieces
to satisfy this architecture from scratch is a monumental undertaking that
outstrips the time available while I complete my PhD.

The next step then is to determine what is possible given current technological
components and what extensions are necessary to expand those components to
answer the research questions. I propose achieving this by combining those
existing technological components, identifying key elements that are necessary,
and analyzing how to minimize the risk involved by focusing on answering the
research questions rather than re-implementing functionality that \emph{already
    exists} and is sufficient to the task.

One key area in which I expect to develop new techniques is meta-data extraction
and construction from existing data and storing that meta-data into pre-existing
technology components.  I describe this further in the following sections.

\section{Storage Silo Implementation}
\label{ch:plan:sec:storage-silo-implementation}

While my objective is to construct \system, which will support multiple silos
combined with federated naming, activity, and meta-data services, my proposed
initial step is to do this on a single system consisting of two storage silos.

This section constitutes an initial ``straw person'' proposal for what I
anticipate constructing.  \MIS{I would call this the MVP --- how do we most
    simply demonstrate the key ideas behind this work in a constrained setting.}
In doing so I expect to evaluate at least two of the following research questions:

\begin{itemize}
    \item \textbf{Can the separation of naming from the file system(s) be achieved
              without compromising performance?} This is related to the broader question articulated earlier in this
          chapter (\autoref{ch:plan:rq:naming-separation-performance}.)

    \item \textbf{Can this two-silo naming system be used to demonstrate more effective
              data location?}  The original motivation is that rich, dynamic naming will
          enable ``finding my stuff''.  One approach to this would be to build on top
          of prior work done in our department related to \emph{data curation} to see
          if a richer naming system can be used to simplify that
          task~\cite{Vitale_2020}.

    \item \textbf{Can a non-traditional storage mechanism be effective when used as a storage silo
              within this separated naming and meta-data service model?}  This
          specifically speaks to the non-hierarchical name system idea that was
          previously proposed in ``Hierarchical File Systems Are
          Dead''~\cite{seltzer2009hierarchical}.  While not a primary motivator for
          the main thesis, I am intrigued by the idea of taking a persistent memory
          key-value store along the lines of ``Modernizing File System through
          In-Storage Indexing''~\cite{koo2021modernizing} where the key-value store is
          in persistent memory, which allows me to leverage my own prior
          work. \MIS{you should take a look at the Inversion file system of
              Mike Olson (circa 1990-3). It's basically building a file system
              on top of an RDBMS -- think about exactly what you are trying to
              do on the KV store that is different.
          }\tm{I have looked at inversion previously.  I do not think that
              maps onto the ideas I've expressed elsewhere about this approach,
              but perhaps this isn't a good research question for \emph{this}
              proposal.
          }

    \item \textbf{Does the explicit separation of meta-data and naming from the file
              system enable the creation of dynamic per-user name\-spaces?} Traditional
          name\-spaces are static in that the names of objects within them
          are not changed by the system, yet relationship based name\-spaces could be
          dynamic.  Further, the namespace of interest to me now may not be the same
          as the namespace of interest to someone using the same data but in the
          context of a different role --- by not tying the namespace to the data does
          it provide a more functional namespace?  \tm{This question is not fleshed
              out enough at this point.}

    \item \textbf{Does capturing activity context related to the way that data is
              constructed and used allow construction of enhanced data management
              mechanisms?} One obvious way of demonstrating this might be to show where it
          does permit that type of enhanced data management mechanism.  For example,
          this might be related to the data curation aspect of prior work. It is
          likely there are other similar data management activities that might yield a
          useful basis for finding information of interest.

\end{itemize}

I anticipate that this work will require developing specific components, which
largely correspond to the services described in
\autoref{ch:model:sec:architecture:subsec:services}.

One key aspect that is not well-defined in the model section but is
important to do as part of this work is to define what an \emph{activity
    context} is, how it is created, and how it is used within the system.  This will
be an important aspect of this work, even if the actual implementation is quite
simple, possibly capturing only a tiny amount of information.\MIS{I continue to
    claim that an AC is the result of a query applied to a rich MD store.}\tm{This
    is an area in which I think the anthropological linguistic perspective is
    useful. Activity context in a human sense would involve who is present, what
    you've discussed before that, where you are located, etc.  So my model is quite
    a bit different than a query capture, but this is a good thing to explore
    further.
}
Initial
techniques for doing this can likely be based upon existing information
gathering systems, such as \emph{eBPF}\tm{add to glossary, define out}\MIS{I
    find glossaries unhelpful; you want to define terms in context. No reader is
    going to remember them from the glossary nor are they going to want to turn back
    to look them up.
} in Linux
or \emph{ETW}\tm{add to glossary, define out} on Windows, both of which are
pre-existing systems with well-defined mechanisms for extracting information
from production systems.

To review, the services that will be implemented in this phase of my work are:

\begin{description}
    \item[Metadata Servers] --- there are existing models for meta-data servers,
        such as Egeria~\footnote{\url{https://github.com/odpi/egeria}} or
        Amundsen~\footnote{\url{https://github.com/amundsen-io/amundsen}} (a
        demonstrative, but not definitive list) that might be sufficient. If
        not, using a key-value store such as
        WiredTiger~\footnote{\url{https://github.com/wiredtiger/wiredtiger}} to
        construct a metadata server is also viable. \tm{I'd rather not build
            anything I don't have to but my concern is that this is going to be a
            core bit of the work; the right thing to do is to do a more thorough
            search for these technologies and figure out which one looks like a good
            fit. \emph{Building} one is possible but seems like a big project on its
            own.}

    \item[Namespace Servers] --- my expectation is that this will need to be
        constructed; it must interoperate with the metadata servers.  Work here will
        include defining the interface to constructing a new namespace or retrieving
        a previously constructed namespace. While the ultimate goal is to have a
        federated namespace service, this initial effort need only provide a single,
        non-federated namespace even though that is more limited than the actual
        design.

    \item[Activity Monitors] --- these create the activity context information I
        described previously.  I would expect that only a single activity monitor
        will be built for this initial system prototype and the details of this will
        be driven by the chosen definition of the activity context and relevant
        activity providers.

    \item[Attribute Services] --- because \system is a naming system, we need a
        mechanism that retrieves storage attributes. Initially, this will consist of
        a source for scanning current content and then a monitoring mechanism
        (likely similar to the activity provider) that notifies the attribute
        service to update the attributes of files that have changed.

    \item[Update Notification Server] --- existing systems provide change
        notifications.  In a cross-silo system this sort of dynamic state monitoring
        will also be useful.  An initial implementation for a local system could
        consist simply of using the existing change notification mechanism(s) for
        watching changes.  A more robust implementation could use information from
        the meta-data service(s) to determine if a given change is of interest.
        This dynamic change tracking could then be federated as part of the later
        planned work.
\end{description}


This proposal is quite broad and consists of no working software at the present
time.  To achieve this I propose taking the initial architecture and
constructing\MIS{I was expecting this system to do something like, "I need to
    widget that does X. I can do that by starting with existing widget A and
    extending it in the following ways."
}
a design based upon this architecture.  To evaluate the design, I
propose choosing at least two distinct storage silos and at least one target
operating system.\MIS{You must have thought about what you'd pick -- why not say
    that?
}  To the extent possible, initial implementation would focus on
combining existing components as much as possible.  For example, Using MinIo and
Sparkle Share as storage silos, with Windows Cloud Sync would provide documented
and generally well-understood technologies on which to construct these
services.\MIS{OK, so you have -- then you need to talk a bit more about these
    technologies, exactly what they bring to the table and what they don't -- i.e.,
    what you will have to do (not at the code level, but the conceptual level) to
    leverage them.}
The Meta-data server can be constructed using one of the available key-value
stores (e.g., WiredTiger).  The Notification service could start with Emitter.
Initial activity context work can be built using eBPF or other inbuilt tracing
mechanisms.  This leaves how to build the attribute services as an open area to
be further refined.

\tm{TBD}

\section{Federated Naming/Meta-data Security}
\label{ch:plan:sec:federated-naming-security}

\tm{TBD}

\section{Cross-Silo Naming/Meta-data Benefits}
\label{ch:plan:sec:cross-silo-naming-benefits}

\tm{TBD}

\section{Timeline}
\label{ch:plan:sec:timeline}

% Table generated by Excel2LaTeX from sheet 'Sheet1'
\begin{table}[htbp]
  \centering
  \caption{Proposed Schedule for competion of Doctoral Thesis}
  \begin{tabular}{p{0.15\columnwidth}p{0.85\columnwidth}}
    \textbf{Month} & \textbf{Activity}                                          \\
    \rowcolor[HTML]{C0C0C0} \multicolumn{2}{c}{2021}                            \\
    September      & Draft thesis proposal to supervisors                       \\
    October        & Updated draft thesis proposal to supervisors               \\
    November       & Final thesis proposal to committee                         \\
    December       & Defend thesis proposal - reach candidacy                   \\
    \rowcolor[HTML]{C0C0C0} \multicolumn{2}{c}{2022}                            \\
    January        & Begin proposed research Part 1 (Storage Silo)              \\
    April          & Milestone 1 Storage Silo (Alpha)                           \\
    June           & Milestone 2: Storage Silo (Beta)                           \\
    August         & Milestone 3: Storage Silo (Complete)                       \\
    September      & Storage Silo paper submission ready                        \\
    September      & Begin proposed research Part 2 (Federated Naming/Security) \\
    December       & Milestone 1: Federated Naming/Security (Alpha)             \\
    \rowcolor[HTML]{C0C0C0} \multicolumn{2}{c}{2023}                            \\
    March          & Milestone 2: Federated Naming/Security (Beta)              \\
    June           & Milestone 3: Federated Naming/Security (Complete)          \\
    July           & Federated Naming/Security Paper submission ready.          \\
    September      & Milestone 1: Cross-Silo Naming (Alpha)                     \\
    December       & Milestone 2: Cross-Silo Naming (Beta)                      \\
    \rowcolor[HTML]{C0C0C0} \multicolumn{2}{c}{2024}                            \\
    March          & Milestone 3: Cross-Silo Naming (Complete)                  \\
    April          & Cross-silo Naming Paper submission ready                   \\
    June           & Draft thesis.                                              \\
    August         & Final thesis.                                              \\
    October        & Defend thesis.                                             \\
  \end{tabular}%
  \label{ch:plan:tab:schedule}%
\end{table}%


The purpose of the schedule shown in \autoref{ch:plan:tab:schedule} is subject to adjustment as the research
progresses.

