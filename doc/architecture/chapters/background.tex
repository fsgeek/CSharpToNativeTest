\chapter{Background}
\label{ch:background}

\tm{Please note that this is a \textbf{work in progress} and I am providing this
    draft with this section partially written because I seek feedback on the overall
    structure and model that I present.  While this section will change in
    subsequent drafts, I do not expect its content to materially change the model
    that I present in \autoref{ch:model}.}

The prior work of Saltzer and Watson establish that the purpose of file systems
is to serve the naming needs of
\emph{users}~\cite{Saltzer1978,watson1981identifiers}.   However, the background
literature on file systems can be broadly broken up into two categories:

\begin{enumerate}
    \item \textbf{Storage Management} --- much of the prior work related to file
          systems focuses on the management of file data.  For physical media file
          systems --- those file systems that manage some sort of media, whether it is
          magnetic, optical, or solid state --- the concerns are about efficient use
          of the media itself.  For distributed (``network'') file systems, much of
          the prior work focuses on efficient protocols for providing access to file
          system data over a network.

    \item \textbf{Naming} --- a small amount of the prior work related to file
          systems focuses on the organization of naming.  There is some overlap
          between early storage management and naming literature.  Similarly,
          distributed file systems had to consider naming as well.

\end{enumerate}

I review this prior work in the subsequent sections
(\autoref{ch:background:sec:storage} and \autoref{ch:background:sec:naming}) and
then analyze these in the context of \system.  I then provide a brief review of
relevant linguistics literature pertinent to my thesis by reviewing the
linguistic field of \emph{semantics} and \emph{pragmatics}
(\autoref{ch:background:sec:linguistics}). Finally, I look at
the challenges that are not addressed in prior work that are necessary for me to
evaluate my thesis (\autoref{ch:background:sec:challenges}).

\section{Storage Management}
\label{ch:background:sec:storage}

Most of the research literature about media file systems involves management of how
data should be organized on the given media. While not central to my own thesis,
it is useful to understand that file systems research continues to be an active
and ongoing research area, but it is \emph{not} related to naming, and that work
really has little direct bearing on the naming aspects of file systems.

\tm{Multics}
\tm{TENEX}
\tm{BSD FFS}
\tm{Cedar}
\tm{LFS}
\tm{}


One of the earliest papers on file systems, ERMA,
The Multics file system was described by Daley in 1965~\cite{daley1965general},
and while it clearly lays out the hierarchical name space, it also provides
insight into very early file systems development, with important considerations
like parity checking, overflow sending, and the "printer bit and column
counter".

By 1965 media file systems development


\section{Naming}
\label{ch:background:sec:naming}

% \tm{1945 - Memex}

In 1945 Vannevar Bush described the challenges to humans of finding things in a
codified system of records, including those used by early computers:

\begin{quotation}
    Our ineptitude in getting at the record is largely
    caused by the artificiality of systems of indexing. When data of any
    sort are placed in storage, they are filed alphabetically or numerically, and
    information is found (when it is) by tracing it down from subclass to
    subclass. It can be in only one place, unless duplicates are used; one
    has to have rules as to which path will locate it, and the rules are
    cumbersome. Having found one item, moreover, one has to emerge from the
    system and re-enter on a new path.

    The human mind does not work that way. It operates by association. With one
    item in its grasp, it snaps instantly to the next that is suggested by the
    association of thoughts, in accordance with some intricate web of trails
    carried by the cells of the brain. It has other characteristics, of course;
    trails that are not frequently followed are prone to fade, items are not
    fully permanent, memory is transitory. Yet the speed of action, the
    intricacy of trails, the detail of mental pictures, is awe-inspiring beyond
    all else in nature.
\end{quotation}

This is as true in 2021 as it was in 1945.  While preparing this proposal I
spent time looking at the guides many libraries provided about the naming of
files. I found a body of recommendations about file naming
standards\footnote{Data Management for
    Researchers~\cite{briney2015data}}
\footnote{Smithsonian: \url{https://library.si.edu/sites/default/files/tutorial/pdf/filenamingorganizing20180227.pdf}}
\footnote{Stanford: \url{https://library.stanford.edu/research/data-management-services/data-best-practices/best-practices-file-naming}}
\footnote{NIST: \url{https://www.nist.gov/system/files/documents/pml/wmd/labmetrology/ElectronicFileOrganizationTips-2016-03.pdf}}.
Harvard Data Management suggests~\footnote{\url{https://datamanagement.hms.harvard.edu/collect/file-naming-conventions}}:

\begin{itemize}
    \item Think about your files
    \item Identify metadata (e.g., date, sample, experiment)
    \item Abbreviate or encode metadata
    \item Use versioning
    \item Think about how you will search for your files
    \item Deliberately separate metadata elements
    \item Write down your naming conventions
\end{itemize}

\tm{Note that the footnotes aren't rendering cleanly.  I will need to clean this up.}

Throughout these examples there are common themes: a name provides context for
\emph{what} the given object represents. Uniformity of information is also
important --- the ``naming convention'' permits not only identifying similarity
but key elements of \emph{difference} between any two named things.

\tm{It seems that this Harvard list is a serious indictment of the existing
    system: it pushes the cognitive load onto the users, talks about meta-data ,
    versioning, encoding, \emph{and} capturing the naming convention.}




Bush's observation here was about
the importance of developing storage systems that enabled human cognition by
forming an extended memory.  What is remarkable about this early work is how
well Bush captured the basic human need for such a storage system and how it
related to human cognition and associative thinking.  I note that we still do
not have any system that is as functional as Bush's Memex, though there are
aspects of the modern world wide web that are reminiscent of some of the
concepts he first described.

While I do not claim Bush's 1945 Atlantic article was file systems
research, it is useful to understand the perspective of work that
followed~\cite{bush1945we}.

% \tm{1956 - ERMA}

One of the earliest file system papers of which I am aware was
ERMA~\cite{barnard1958}.  This paper was more focused on documenting their
design process, but as part of this work they described the hierarchical
structure of files.  The diagrams clearly show the "folder/file" metaphor that
survives to this day.

\tm{1964 - A File Structure for the Complex, the Changing and the Indeterminate}

Nelson builds upon Bush's Memex description as he explores ``[t]he kinds of file
structures required if we are to use the computer for personal files and ad an
adjunct to creativity\ldots''  He points out the challenges inherent in the
task: ``They need to provide the capacity for intricate and idiosyncratic
arrangements, totally modifiable, undecided alternatives, and thorough internal
documentations.''

Nelson's work is usually cited as inspiration for the world wide web and its use
of \emph{hyperlinks} to associate related content together, yet this work was
primarily focused on how personal file information should be organized within a
computer system.  Key ideas here include the \emph{dynamic} nature of such
organization (``[i]t was also intended that the system would allow index
manipulations which we may call \underline{dynamci outlining} (or
\underline{dynamic indexing}).'')  Another important aspect of this work was his
repudiation of hierarchical organization: ``[n]o hierarchical file relations
were to be built in; the system would hold any shape imposed on
it.''~\cite{nelson19654}

\tm{1965 - Multics}

In 1965 the ``Fall Joint Computer Conference'' was dominated by papers about
General Electric's new Multics operating system.  One of those papers was
specific to the file system~\cite{daley1965general}.  From a naming perspective
the key observation was the use of the hierarchical file system structure,
clearly described and highly reminiscent of the hierarchical file system that
has been adopted first by UNIX and then enshrined in the POSIX interface.  One
observation I have with respect to this work is that while the initially propose
the hierarchical file system, the paper admits that this isn't sufficiently
descriptive and introduces the additional concept of a \emph{link}. This serves
as a good example of how the issues of storage efficiency (e.g., not wanting to
have duplicate copies of the file stored on disk) impacted naming (where the
fact that an object has multiple names need not be tied to the actual manner in
which this is implemented.)

\tm{This raises the question about mutability.  If you change the \emph{file} do
    you preserve the name or do you break it?  That's a really interesting question
    and I don't think anyone will be happy with pretty much any answer.  It's also
    the source of one of the real frustrations in modern computing, namely the use
    of shared libraries with the same \emph{name} but subtly different
    \emph{implementation}.  Hence we end up with \emph{libfoo.so} and
    \emph{libfoo.6.so} and \emph{libfoo.6.4.so}.  In Windows this became ``DLL
    hell'' and has lead to solutions like "side-by-side" installations of commonly
    used shared libraries.  There's a really difficult question lurking beneath this
    for naming: does the name specify something specific or something general?  This
    is the \emph{binding} problem that Saltzer refers to as well.
}

\tm{TENEX}

At the Fall Joint Computer Conference in 1972 Murphy presented a paper about the
storage organization of TENEX, including its naming scheme.  ``A powerful and versatile
directory and file naming facility is provided in which
a particular file is identified by a fixed-depth path which
includes device, directory name, file name, extension,
and version.''~\cite{murphy1972storage}

\tm{UNIX}

The 1974 Symposium on Operating Systems Principles included papers that
ultimately led to both of the primary ``families'' of operating systems that we
use today.  One of those introduced the UNIX operating system and that work
presented the model of ``everything is a file'' and the hierarchical name space
organization of those files~\cite{unix}.  Intriguingly, the need to add
additional contextual information was recognized even then: ``Besides the system
proper, the major programs available under UNIX are: ... and permuted index program.''

\tm{CAP}

The Cambridge Capability Protection operating system included a novel model for
using capabilities to protect files.  The namespace that CAP presented is
interesting in that it did consider non-hierarchical name spaces~\cite{needham1977cap}: ``By handling
directory capabilities in the same way as store capabilities we allow not only
hierarchical directories but also shared directories --- the structure can form
an arbitrary directed graph, which may even be cyclic.''  They also described
the concept of \emph{dynamic} directories: ``By allowing dynamic creation of
directories, we allow the manufacture of directories or of complex structures
of directories separate from the main filing system, accessible only to certain
programs (this is used for the password file, for example).''

\tm{NFS}

As we build communications networks to share information between computers the
idea of sharing files between computers quickly emerged.  Sun Microsystems'
Network File System (NFS) was not the first network file system but it is
certainly one of the most widely used and it continues to be used
today.~\cite{nfs}  It extended the hierarchical file system so that it
incorporated the name spaces of a remote computer system into the local file
system. While NFS provided subtly different behavior than a local file system,
it was ``close enough'' that in the absence of failures it ``just worked.''

\tm{AFS}

The Information Technology Center at Carnegie-Mellon University (CMU) was responsible
for the realization of a system of network attached computer workstations for
use in the CMU environment. Mahadev Satyanarayanan and his resarch group were
responsible for developing the Andrew File System (AFS), which was subsequently
commercialized by Transarc in the late 1990s.  AFS remains in use today.  AFS
was contemporaneous with NFS, but offered significantly different functionality,
including a \emph{global} name space, federated security (via MIT's Kerberos),
and a separate service that was responsible for resolving the global namespace
across various clients~\cite{howard1988scale}.

\tm{Universal Directory Service}

The concept of distributed name or directory services is one that is not new.
In 1985 Lantz provided a description of a ``Universal Directory Service'' that
is similar to some of the work that I anticipate while evaluating my own thesis.
This paper describes their services as offering the following
capabilities~\cite{10.1145/12481.12483}:
\begin{itemize}
    \item ``can span a heterogenous internetwork of existing naming domains;
    \item ``allows us to name, locate, and discover how to manipulate objects
          (including files, processes, mailboxes, people, and services);
    \item ``provides dynamic binding and context mechanisms; and
    \item ``can be integrated into most existing systems as a ``value-added''
          feature.''
\end{itemize}

While similar conceptually to the domain name service (DNS) that is used for
naming in the internet, it is considerably more general.  Intriguingly, while
there is subsequent work that cites to Lantz's work, I could
not find any later work that extended it.

\tm{Properties}
The idea of extended attribute information being associated with a file is one
that emerged in the 1980s.  Mogul's 1986 treatise ``Representing Information
About Files'' does an excellent job of summarizing the research up to that
point, including an explanation of the history regarding ``Leaf'' a protocol
that while developed at XEROX PARC for the Alto was only memorialized by him and
Brian Reid in 1981.  That appears to have been motivation for the creation of
what we now call ``extended attributes'', though Mogul refers to them as ``file
properties''~\cite{mogul1986representing}. Mogul's work provides considerable
insight into how file properties could be implemented, both by applications (as
part of the file itself) as well as within file system meta-data.  We actually
supported properties in Episode~\cite{chutani1992episode}.  An important argument Mogul
makes is that \emph{naming} should be separated from \emph{storage}:

\begin{quotation}

    The old monolithic system model usually implied a tight coupling between file system and directory
    system; in many cases the two are indistinguishable. While this approach might improve performance
    slightly, one of the recognized virtues of the client-server model is that it encourages separation of function.
    Separating file system and directory system into two distinct services (which might nevertheless interact as
    clients of each other) provides several benefits:
    \begin{itemize}
        \item \textbf{Modularity}: with attendant benefits of a cleaner service model, cleaner implementations, and
              the flexibility to substitute new implementations of one service without affecting the other.

              Modularity has additional value: it is easier to distribute specialized services than to build a
              generalized distributed system. For example, LOCUS uses its knowledge of the special
              semantics of directories to recover them after a partition; this cannot be done for files in
              general [Popek 81].

        \item \textbf{Crossing file system boundaries}: an integrated directory system can only store references to
              files in its associated file system. In a heterogeneous distributed system, we would like to
              construct a unified name space covering all file servers in the environment. Why should we
              prohibit users from storing in the same directory references to files stored by two distinct file
              servers? The Universal Directory System [Lantz 85] is an example of an approach to this
              question.

        \item \textbf{Non-file referents}: Directory systems can and should be used as general name-binding agents;
              they need not only refer to files. In fact, directories have been used to name non-file objects
              even in Unix (where devices appear in the file system name space). By separating directory
              from file service and giving non-file referents full citizenship in directory bindings, we can
              gain useful generality.
    \end{itemize}

    Not everyone agrees that directory service and file service should be separate; for example, the V-system
    takes the opposite point of view [Cheriton 84]. We prefer separation because it
    leads to a simpler model.

\end{quotation}

His definition of a file is useful as well: ``\emph{a \emph{file} is a named
    object that stores an arbitrary amount of data for an arbitrarily long time.}''

\begin{quotation}
    The Alto and WFS were research projects. Other groups within Xerox were working on commercial
    products, the Star professional workstation in particular. The original design for the Star file system [Dalal
            86] was tightly integrated with application-level functions. It was then realized that clearer separation of
    file service and application would yield a more open, flexible system, but the resulting file system retained
    some features of Star that were found to be generally useful. One of these features was support for
    extensible attributes, including attribute-based search.
\end{quotation}

My general sense remains that Mogul's work is the most closely related model of
the work I propose doing as part of this thesis.

\tm{Semantic}

Gifford introduced the concept of a \emph{semantic} file system at SOSP in
1991~\cite{gifford1991semantic}.  I reproduce the abstract from his paper to
provide the basic idea of his work.  While I \emph{started} from considering
semantic file systems as part of my PhD, my own work has moved beyond the model
of Gifford and is complementary to Gifford's own work.

\begin{quotation}
    A semantic file system is an information storage system that
    provides flexible associative access to the system's contents
    by automatically extracting attributes from files with file
    type specific transducers. Associative access is provided by a
    conservative extension to existing tree-structured file system
    protocols, and by protocols that are designed specifically for
    content based access. Compatibility with existing file system
    protocols is provided by introducing the concept of a
    virtual directory. Virtual directory names are interpreted as
    queries, and thus provide flexible associative access to files
    and directories in a manner compatible with existing software.
    Rapid attribute-based access to file system contents
    is implemented by automatic extraction and indexing of key
    properties of file system objects. The automatic indexing of
    files and directories is called "semantic" because user programmable
    transducers use information about the semantics
    of updated file system objects to extract the properties for
    indexing. Experimental results from a semantic file system
    implementation support the thesis that semantic file systems
    present a more effective storage abstraction than do traditional
    tree structured file systems for information sharing
    and command level programming.
\end{quotation}

A key point here is that semantics are defined based upon the \emph{content} of
the file itself. This is certainly useful but different than the model I
propose, which would introduce \emph{pragmatics}.  Pragmatics are defined based
upon the \emph{context} of how the given file is used.

Another insightul source of information about semantic file systems is from
Martin's PhD thesis~\cite{martin2008}.  He utilizes \emph{Formal Concept
    Analysis} as a mechanism for improving semantic file systems; his work continued
until a few years ago~\footnote{http://www.libferris.com}.  Some of this serves
as a caution about how challenging it is to build useful systems that augment
the file system namespace.

\tm{Tags}

The concept of using \emph{tags} on files is related to the earlier work on
semantics.  The distinction appears to be that tags were initially designed to
be added by users, though the possibility of combining transducer generated
tags, as in the semantic file systems work, with user created tags is certainly
not excluded~\cite{tagfs}.  Indeed, the Human-Computer Interface (HCI) community
has observed that ``[U]sers are unlikely to use metadata.  Even when the tagging
is somethign as convenient as voice tagging, users are not likely to make the
time investment required to assign metadata to
files.''~\cite{10.1145/642611.642682}  Thus, while it might be useful to allow
human driven tagging, it is unrealistic to consider it to be generally useful.

\tm{Layered}

In my own past work, we implemented a layered log-structured technique within a
single file that permitted us to add additional meta-data to files.  For
example, we were able to add NTFS features to FAT32 files on Windows, with a
file system filter driver responsible for interpreting the additional meta-data
and presenting it to the native operating system.

\tm{Graph}

\tm{360}

\tm{DelveFS}

Key take-away here (for me) was the use of pub/sub systems to provide
notification.  This dovetails nicely into the strong dynamic nature of naming
systems, since they have to be responsive.

\section{Linguistics}
\label{ch:background:sec:linguistics}

\tm{This is where I explain how linguistics describes semantics and pragmatics
    and tie them together with the relevant portions of my thesis.
}

\section{Human-Computer Interface (HCI)}

There is a rich body of literature about various ways of presenting file
information to human users.  Various approaches to this problem have been
considered, including: temporal organization, virtual directories, attribute/tag
organization, and even a table-top model~\cite{collins2007tabletop}.  The
literature seeems to be rather clear on the limitations of the hierarchical file
system for human usage.  Intriguingly, HCI researchers have argued that the
evolution of file systems up to the point of creating hiearchical name spaces
with links with the evolution of databases and to expand their abilities the
\emph{file browser} should evolve to support dynamic interaction on par with
dynamic data interaction in a relational database
works~\cite{marsden2003improving}. This body of prior work has identified that
one key challenge is the sheer
magnitude of how much data is being presented.  Thus, it is not useful: ``If we
were to visualoise all files in this list, as in the email client, the list
would be too long to afford rapid scanning...''

My reading of this prior work is that we need to support more dynamic naming and
naming support, the ability to use transducers to add key meta-data based upon
content and the ability to allow humans to add tags.

At the same time, the importance of hierarchical organization is increasingly
less useful.  While search is useful, particularly when combined with filters
(e.g., faceted search) it also is insufficient to the task.  The associative
data model, such as used in the table-top models~\cite{collins2007tabletop}, is
surprisingly powerful for collaborative work.  No single mechanism satisfies all
these needs; a robust system must enable multiple usage modalities.

\section{Challenges}
\label{ch:background:sec:challenges}

\tm{My concern with this section is that it may be premature --- until I
    introduce the model, identify the research questions to be explored, and the
    artifact(s) I propose to build to explore those research questions this section
    may be ``too soon''.
}


\section{Related Work}
\tm{Note: this text was moved from ``related work'' and needs to be incorporated into this section.}

Ways to organize data within a storage silo have been extensively
studied from multiple perspectives.  That there are so many different approaches
to evaluating the optimal way to organize things is a testament to both the
importance and complexity involved in approaching this topic.  This section will
consider the following related work:
\begin{itemize}
    \item The \textit{storage} perspective, which is primarily rooted in the
          systems perspective.  Storage in this context includes file systems and
          databases, each of which then has multiple different manifestations.

    \item The \textit{provenance} perspective, which is related to systems but
          focuses on creating a context of explainability that is distinct from
          storage.

    \item The \textit{human-computer interface} (HCI) perspective, which is related to
          how humans organize and find information within storage systems. There are a
          number of distinct perspectives to this work including: hieararchical
          data organization, personal information management, enhanced search, and
          cognitive data organization.
\end{itemize}

Each of these perspectives is complementary and assist in better understanding
the problem: systems focuses on being able to efficiently and reliably store and
recover information though often it is agnostic to the specifics of the
information; provenance focuses on accountability and explainability; and HCI
focuses on the human facing problems that need to be solved and tends to ignore
how those solutions are implemented.

\cite{mashwani2019360,9229638,vef2020delvefs,dourish2003the,harrison1996re,barreau1995finding,dourish1999getting,placeless-tois,giffordSFS,plan9,inversion,smartstore,tagfs,gfs,provsearch,uprove2,pindex,page1999pagerank,nfs,metastorage,howard1988scale,afs,Adya:2003:Farsite,unix,scfs,federatedMetaData,federatedACL,benet2014ipfs,guo2012burrito,mazurek2014toward,li2013horus,adya2002farsite,provprimer,camflow}

\nocite{*}

\section{Storage}
\label{ch:related-work:sec:storage}

Much of storage work is dominated by the realities of how media and networks
function, with a goal towards increasing both capacity and performance.  The
basic model of data organization within storage systems tends to separate into
\emph{structured} data --- typically what is found in databases --- and
\emph{unstrutured} data --- typically what is found in file systems or object
stores.

\tm{Describe structured data: why do people use databases, what are the
    strengths and weaknesses.  How does this relate to data organiztion?
}

\tm{Describe unstructured data: why do people use file systems or object stores,
    what are the strengths and weaknesses?  How does this relate to data
    organization?
}

\tm{Describe the work that has been done in semantic file systems, metadata
    augmentation, and non-hierarchical organization structure (e.g., Ground and
    Placeless).
}

\tm{This is a fairly large quotation from the Placeless Documents Project
    Archive that should probably be edited down, but for the moment I want to leave
    it here because I think the insight provides is quite useful.
}

From the Placeless Documents Project
Archive~\footnote{https://web.archive.org/web/20020210170621/http://www.parc.xerox.com/csl/projects/placeless/}:

\begin{quotation}

    Placeless Documents are documents that are organized and managed according
    to their properties, rather than according to their location. Document
    properties can be things you already know about your documents, like that
    they're published, or notes, or about the budget,or drafts, or source code,
    or important, or shared with your colleagues, or from your manager, or big,
    or from the Web, or... whatever suits you. Document properties can also be
    things that you want to be true about your documents, like that they are
    backed up, or replicated on your laptop, or can be purchased for a small
    fee. These latter properties carry the code to implement or interface with
    the desired functionality.  Document properties are statements about your
    documents that make sense to you, and affect what you're going to do with
    the documents.

    What does it mean?

    We live and work in an information-filled world. The project focuses on helping people cope with the large and diverse information spaces that are part of life in the networked world. Our approach is to provide information consumers with a unifying model for organizing and manipulating their information.

    Much of the information that we commonly use is in the form of documents, both physical and electronic. In fact, the use of electronic documents on our computer desktops is pervasive, and even though they unify much of desktop computing, the document metaphor falls short of providing an accessible and readily understood way to interact with all forms of information, whether electronic or physical. Instead, we resort to specialized applications for much of our computer-based work. Electronic documents are managed through different systems like mail, WWW, and file systems. Similarly, the many devices we use in the course of our work (including fax machines, scanners, televisions, VCRs, telephones) manipulate and store information, yet they do not integrate seamlessly with the rest of our on-line information.

    Furthermore, the means available for individuals or groups to organize and customize their information spaces are extremely poor and driven mostly by storage and distribution models, not user needs. The most common model for information organization is hierarchical. The use of folders as a fundamental organizing principle, and the restriction that documents (mail messages, files, URLs, etc.) appear in only one folder at a time, force users to create strict categorizations, resulting in inflexible organizations that tend to persist over time even though their needs evolve. Similarly, customization of the organization and the behavior of information according to individual requirements is cumbersome, if not impossible. Access to a shared document that one individual deems as corresponding to the budget and another project cannot be easily tailored for both specialized functional requirements are equally hard to achieve. If a user can express that a document is read-only, and that it is a Word document, why cant heexpress that updated copies should be faxed to a colleague once a week?

    This project is about removing these hurdles by using a novel infrastructure and proving its benefits through applications that exploit its capabilities. We plan not only to change the way people interact with their currently segregated world of documents, but we plan to exploit a single concept the document and its properties to allow users to interact with arbitrary information.

    Our vision is one of customizable, context-aware management of integrated information spaces, which:

    \begin{itemize}

        \item integrate information components from many sources: repositories (WWW, mail, file systems), devices (scanners, video-cameras, television, phone), and dynamic processes (workflow, source code management systems, search engines, and dynamic document content),
        \item allow customizable organization of the information based on properties of that information, e.g., budget related, read at home, shared with John, and From: petersen@parc.xerox.com,
        \item allow information properties to be arbitrary objects specified through many different mechanisms: explicitly by the users themselves, captured by physical context sensors, inferred from usage, automatically generated by content analysis, etc.
        \item allow information properties to be active and carry behaviors to automate information work, enabling functionality like fax to John at 5pm each day, translate to English, notarized, backed-up in Utah for safety, etc.
        \item scale to sizes anywhere between an individual and the enterprise,
        \item are available at all locations required by the users, and
        \item protect the privacy and intellectual property of users.

              In this world the focus is on information, customization, and functionality that
              extends beyond the abilities of monolithic applications. Essentially,
              information carries the behaviors and semantics needed to operate on it.
              Information is independent of location and becomes responsive to the
              environments it is used in and the contexts of individual users, and it is
              managed independently by both its consumers and providers.
    \end{itemize}
\end{quotation}

\tm{The big hole that I can see is their comment that they want to prove ``its
    benefits through applications that exploit its capabilities.''  The
}

\section{Provenance}
\label{ch:related-work:sec:provenance}

\tm{Explain provenance.  It is a more recent area of study, but it also captures
    more of the kinds of information that will be useful to me.
}

\section{Human-Computer Interface}
\label{ch:related-work:sec:hci}

The traditional primary organizational model provided to users has been the
file/folder metaphor\tm{Need citation to 1965 Daley and the 1956 storage
    papers}, which was itself derived from physical filing cabinets.  However,
electronic storage is not bound by the same restrictions as a physical filing
cabinet, as can be seen even in early work \tm{Again, this is Daley} where the
model added \emph{links}; the closest equivalent to this in filing cabinets
would be to make duplicate copies of a document --- I have seen exactly this
routinely practiced by bookkeeping and accounting professional, whether using
physical or electronic filing systems.  This works because the objects are
generally \emph{immutable}, but for electronic storage systems without
deduplication it is not particularly space efficient.

The Human-Computer Interface community has been studying human-centered data
organization models for decades.  For example, the HCI community observed that
hierarchical file structure is challenging for users with low spatial
abilities~\cite{vicente1988accommodating}. This suggests why storage developers
would not even see there is a problem here: the study of computer science
correlates well with the development of spatial
abilities~\cite{parkinson2018spatial}. In my own discussions with even senior
computer scientists it is the introduction of \emph{silos} that seems to make
the hierarchical abstraction break. ``Did I store that in Dropbox, or Google
Drive, or was it on my laptop or my desktop computer?''

The wealth of research here is astonishing, yet does not appear to influence the
design of storage systems to exploit the results of that research.  Indeed, the
systems community seems to be focused on a \emph{search based} solution while
the HCI community research suggests that search is \emph{not} preferred by users
--- instead, users want to \emph{navigate}:

\begin{quotation}
    \emph{When retrieving a file the user needs to choose between folder
        based navigation and query based search. There are obvious
        intuitive advantages of search for both retrieval and organization.
        Search seems to be more flexible and efficient for
        retrieval. It is flexible because it does not depend on users
        remembering the correct storage location; instead, in their
        query users can specify any file attribute they happen to
        remember...}

    \emph{In fact, regardless of search engine quality,
        people consistently use search only as a ‘last resort’ for that
        minority of cases where they cannot remember file
        locations...}~\cite{bergman2019search}

\end{quotation}

Thus, from an HCI perspective it would seem that one potential research
direction would be to consider potential interfaces that mimic navigation over a
search interface.

For example, recent work around data curation explores the idea of a ``data
dashboard'' since the first step of curation is finding specific data to
curate~\cite{Vitale_2020}. This work builds upon prior research showing that
when presenting data to users it is important to ignore storage silo boundaries.
Indeed, my reading of this work is that it presents what seems to be navigation
even though it is implemented using a search mechanism.  Equally important, this
work also points to the benefits of not changing the \emph{location} of data,
but rather allowing construction of useful relationships via metadata. Thus,
this work supports my ideas of ignoring silo boundaries and providing metadata
for use by a similar tool.  What it does not explore is a way for data storage
to provide enhanced metadata, dynamic update, and notification of changes, which
are important elements to making such a dashboard more useful for data
visualization.




\endinput

\reto{I feel like a lot of Challenges goes here... }

\begin{epigraph}
    \textit{Paradigm paralysis refers to the refusal or inability to think or see
        outside or beyond the current framework or way of thinking or seeing or
        perceiving things.  Paradigm paralysis is often used to indicate a general
        lack of cognitive flexibility and adaptability of thinking.} --- The Oxford
    Review Encyclopedia of Terms (2021).
\end{epigraph}

Computer storage systems continue to evolve and change at an increasing rate.
The difference between file systems, which provide the basic abstraction of
unstructured data, and database systems, which provide the basic abstraction of
structured data, is now filled with a growing array of \emph{semi-structured}
mechanisms including: key-value stores, object stores, document stores, no-SQL
databases, data warehouses, and data lakes.

Each new mechanism invented for storing data creates a new ``silo'' of
that specific storage system.  Each new storage system comes with its own
semantics and meta-data, and seldom with any explicit way of tying related
information together across such silos. Storage silos are often designed with
specific usage patterns in mind.  For example:

\reto{I'd go with categories here instead of products. as there are often
    multiple different solutions within the same category.
}

\begin{description}
    \item[\ac{HDFS}] --- Hadoop was created to solve the problem of large,
        dynamically written write-once files that are typical of machine data
        captured from multiple sources for additional analysis.

    \item[Google Drive] --- Google uses a common storage mechanism for its various
        services including e-mail, documents, spreadsheets, photos, and videos. It
        allows collaborate work, both by sharing access to the object as well as
        providing tools for simultaneous editing between human collaborators.

    \item[Intel DAOS]
        --- Intel's solution for large parallel \ac{HPC} storage
        needs that splits meta-data and data, with meta-data stored in persistent
        memory and data stored on \ac{NVMe} devices~\footnote{\url{https://www.intel.com/content/www/us/en/high-performance-computing/daos-high-performance-storage-brief.html}}.

    \item[Qumulo] --- Seattle-based start-up company that provides a large data
        storage management product that is used in specialized big-data industries.
        For example, Qumulo combines both local and cloud storage to provide video
        post-production work in which file sizes are quite large due to the size of
        high resolution video files. They claim to support secure storage of
        petabytes of file data across multiple tiers of storage.
\end{description}

\tm{I'm not sure these examples are specific enough.}

Each of these storage products attempts to address specific storage needs; each
new product creates a new storage silo.  A human user that can limit themselves
to using a single storage silo can exploit the functionality of that specific
silo and simplifies the challenging task of finding disparate but related
content spread across storage silos.

The proliferation of storage silos exacerbates the easy ability of humans to
navigate their own storage to find a specific file.  One of our own \ac{HCI}
researchers told me that they found their ability to find things on their own
computer broke when they added multiple distinct hard disks~\footnote{Private
    conversation with Joanna McGrenere at UBC CS-50 reception.} The challenge of
finding a specific file or set of files is one that consistently resonates with
many of those with whom I have discussed my research, both inside and outside
the computer science field.

The most commonly presented model of naming files (or objects) within a storage
silo is the hierarchical namespace.  The hierarchical namespace confounds the
storage \emph{location} with the information \emph{relationship}.  Note that
location here is not relative to the logical namespace of the storage silo, not
the physical storage device.

\reto{the challenges section provides quite a broad overview of how things are right now.
    I'm not sure if those are the challenges you will be facing with your thesis, or whether these are general challenges.
    Given your thesis statement,
    shouldn't there be something like a problem statement (or challenges that makes the thesis statement hard)
    I'd recommend to identify say 3-5 challenges and try to express them in one sentence each.
    each sentence then is the \subsection{} heading. where there is maybe a paragraph or two worth of information backing up the challenge/problem.
    ideally, your wock packages will correspond to one challenge/probleme each.
    (maybe those 3-5 challenges are in fact the research questions you plan to answer)
}

Ordinary applications familiar with using hierarchical file systems expect
related objects, of whatever type, to be in or close to the same directory.  The
hierarchical file system model has been highly successful for more than a
half-century, though that success is from the perspective of the storage
community.  The \ac{HCI} community has been pointing out that this model is
deficient for many users.

Storage silos are not a new invention.  The problem of cross-silo management,
including naming is not a new problem.  UNIX addressed multiple silos using
``mount points''.  A \emph{mount point} is a location in an hierarchical name
space where another namespace is logically connected to the existing namespace.
For example, UNIX mounts a new file system instance on top of an existing
directory.  On the Linux system where I am writing this document there are 33
distinct namespaces mounted on top of the base file system namespace --- the
``root'' namespace that starts at "/".  This model has served well as evidenced
by the fact that all mainstream consumer operating systems at present (Linux,
MacOS X, and Windows) support mount points.

NFS on UNIX is implemented using explicit mount points, much like media file
system instances while AFS used a global namespace mechanism to connect its
silos (``volumes'') together so that users were given a single consistent
namespace. Converting the hierarchical name of internet services has been used
to create internet-based namespaces, such as using \ac{WebDAV}, which converts
the internet \texttt{GET/PUT} model into an
hierarchical file systems namespace model.  Amazon's AWS S3 service provides an
object store model that is accessed using HTTP GET/PUT operations as well. This
can be accessed using the hierarchical storage model using any one of the
numerous "S3" FUSE file system implementations on \url{github.com}.  These FUSE
file system implementations of S3 exist not because they are fast or efficient
but because supporting the hierarchical model understood by existing
applications makes them \emph{usable} by existing applications.

The hierarchical name space is definitely powerful: it is a specific
construction of individual name space silos that are composed into a single
uniform namespace. However, the benefits of this single namespace have always
been limited: they tend to follow the \emph{computer} and not the \emph{user}.
Plan 9 was one of the few systems that focused on personal name spaces,
introducing key ideas such as context defined names --- such as the platform
architecture of the computer on which a program is currently running and
\emph{unions} in which multiple namespaces are not joined via a mount point but
rather via a merge of multiple namespaces mounted at the same location.
Xerox's Placeless research project was a system where documents ``are organized
and managed according to their properties, rather than according to their
location.''~\footnote{https://web.archive.org/web/20020210170621/http://www.parc.xerox.com/csl/projects/placeless/}

In the past two decades the complex multi-silo storage environment has exploded:
\begin{description}
    \item[WebDAV] --- the ability to map internet web servers into the
        hierarchical file system;
    \item[Dropbox] --- maintaining a mapping between a public internet remote
        storage solution and a portion of the local hierarchical namespace on
        existing storage across multiple devices;

\end{description}


\tm{This should come from the existing text.  It should explain \textit{how} we
    got to this point.
}

\section{Linguistics}
\label{ch:background:sec:linguistics}

\tm{An interesting way to look at this is from the \emph{linguistic}
    perspective.  Much of the prior work relates to semantics but in fact what I'm
    suggesting is that we also consider \emph{pragmatics}. Wikipedia defines this
    as: ``In linguistics and related fields, pragmatics is the study of how context
    contributes to meaning. Pragmatics encompasses phenomena including implicature,
    speech acts, relevance and conversation. Theories of pragmatics go
    hand-in-hand with theories of semantics, which studies aspects of meaning which
    are grammatically or lexically encoded. The ability to understand another
    speaker's intended meaning is called pragmatic competence. Pragmatics
    emerged as its own subfield in the 1950s after the pioneering work of J.L.
    Austin and Paul Grice.''  This actually seems to capture some of what we are
    trying to do quite well: semantics have been studied in terms of computer
    storage, but not pragmatic linguistics.
}

\begin{quotation}
    Pragmatics is a field of linguistics concerned with what a speaker implies
    and a listener infers based on contributing factors like the situational
    context, the individuals’ mental states, the preceding dialogue, and other
    elements.
\end{quotation}

\url{https://www.masterclass.com/articles/pragmatics-in-linguistics-guide}

\tm{Important take-away here is that Pragmatics is distinct from Semantics
    because it relates to how context affects the meaning of language.  This is
    actually quite close to what we are trying to do: to capture \emph{context} so
    we understand the meaning of the language we are using to describe things
    relative to that context.  However, what we are doing isn't a good mesh with
    pragmatics or computational pragmatics (yes, it exists) because that seems
    more focused on the question of using computational mechanisms for
    understanding pragmatics.
    \url{https://www.oxfordbibliographies.com/view/document/obo-9780199772810/obo-9780199772810-0264.xml}
    is interesting in that it really hits home on the dynamic nature of meaning.
}

\begin{quotation}
    Pragmatics as a branch of linguistics can be characterized as the study of
    the relations between linguistic properties of utterances on the one hand,
    and aspects of the context in which a given utterance is used on the other.
    Computational pragmatics is pragmatics with computational means, which
    include models of dialogue management processes, collections of language use
    data, annotation schemes and standards, software tools for corpus creation,
    annotation and exploration, process models of language generation and
    interpretation, context representations, and inference methods for
    context-dependent utterance generation and interpretation processes. The
    linguistic side of the relations that are studied in pragmatics is formed
    primarily by utterances in a conversation or sentences in a written text. In
    the case of written text the context side consists of the surrounding text
    and the setting in which the text is meant to function. In spoken or
    multimodal dialogue, the context of an utterance is formed by what has been
    said before and the interactive setting, but additionally by other
    perceptual, social, and mutual epistemic information (see Context Modeling).
    Much of this information is dynamic, as it changes during a dialogue and,
    more importantly, as a result of the dialogue, since the participants in a
    conversation influence each other’s state of information when they
    understand each other. Dialogue contexts are thus updated continuously as an
    effect of communication. Central to computational pragmatics is the
    development and use of computational tools and models for studying the
    relations between utterances and their context of use. Essential for
    understanding these relations are the use of inference and the description
    of language in terms of actions that are inspired by the context and that
    are intended to change the context. This bibliography therefore focuses on
    publications concerned with the computational modeling of dialogue in terms
    of communicative actions including the use of inference for utterance
    interpretation. It also considers the more static analysis of discourse
    coherence and semantic relations in text, and concludes with references to
    recent activities concerning the construction and use of resources in
    computational pragmatics, in particular annotation schemes, annotated
    corpora, and tools for corpus construction and use. The popularity of
    probabilistic approaches to natural language processing can also be seen in
    studies of pragmatic aspects of language use, although these approaches are
    so far not as important as in some other areas of language processing. The
    so-called rational speech acts (RSA) model treats language use as a
    recursive process in which probabilistic speaker and listener agents reason
    about each other’s intentions to enrich the literal semantics of their
    language along broadly Gricean lines. The core references for this approach
    are also included in this biography under Inference in Language Processing.
\end{quotation}

\tm{Good stuff here, which resonates with what I've been talking about.  Context
    isn't static, though semantics are static.
}



Computer storage focuses on \emph{content}, which is one form of
\emph{identity} such as is found in ``content addressing'' where the specific
identity of a file is described using a hash value computed from the content of
an object.  Some storage systems provide the ability to encode a limited amount
of information about relationships and properties.  \emph{Relationships} in
file systems are typically expressed using directories (or folders) and
\emph{context} is captured via a name.\reto{does it make sense to talk about an ever growing pile of files?
    somehow I feel like this is where the systems aspect comes into place:
    - humans have  a limited working set.
    - crawling through tons of files across different silos is impractical.
    - search is fuzzy and may not provide  the right results, nor work cross silo.
}
\reto{context is another thing. also: I wouldn't say that context is the
    name. }\tm{I disagree with Reto on this point: when we have hierarchical
    names, we embed the context in the hierarchical name \emph{because} that's
    the only way we have of capturing it.  Indeed, as Reto pointed out earlier,
    names have meaning within a particular context.  So... where do we
    \emph{get} this context.}


\reto{does it make sense to talk about an ever growing pile of files?
    somehow I feel like this is where the systems aspect comes into place:
    - humans have  a limited working set.
    - crawling through tons of files across different silos is impractical.
    - search is fuzzy and may not provide  the right results, nor work cross silo.
}\tm{I had text in at some point about the growing size and the problems
    with it.  The proliferation of silos is yet another issue, such as what is
    happening with in-memory compute, each of which ends up looking like its own
    silo.
}

The goal of making computer storage more flexible is not a new one.  Memex was
first proposed in 1945 by describing how computers might help human users by
serving as ``augmented memory''~\cite{bush1945we}.\reto{what does it do?}  Computer storage has evolved
dramatically in the 76 years since yet our storage systems have failed to
progress towards, let alone realize Memex's associative relationship
model.\reto{maybe say what the assoc relation ship model actually provides....}

\reto{mention the "search" and "lookup" of the right files again here? it appears in the abstract, but maybe could be a little bit expanded here?
    distinction between mete-data / content}
\tm{In fact, my original goal was to \emph{lead} with the thesis statement and
    then build on that.  Then I tried to add a small body of text to introduce the
    topic.  If that body of text is going to grow, I should move all of that after
    the thesis statement.}

\section{Naming}
\label{ch:background:sec:naming}
%\section{Why naming is a Computer Systems problem}
%\label{ch:introduction:sec:systems-problem}


\MIS{This feels like a classic related work paragraph, but I don't know where it
    fits in your story line.
}
Media file systems have long been organized using a simple internal key-value
store.  The BSD UNIX Fast File System used an ``index node'' (inode) as a
description of a data object, such as a file or directory~\cite{mckusick1984a}.  These nodes were
indexed using an identifier, which acts as a simple key.  The NTFS file system
is structured similarly~\cite{custer1994inside}.  Recent work explored the idea
of separating the namespace from storage with an emphasis on high performance
solid-state storage (SSD) devices~\cite{koo2021modernizing}.  This echos earlier
work suggesting using object storage devices (OSD) for file systems
implementations~\cite{seltzer2009hierarchical}.

Beyond separating naming from storage, there is also a trend to separate
meta-data from storage as well.  One early example of this is
Lustre~\cite{braam2019lustre}, with more recent work including Ceph and
Gluster~\cite{noronha2008imca,weil2006ceph}.  Each of these distributed file
systems separates the service of data from meta-data, which allows data paths to
focus on performance, including the use of parallel data paths.  Meta-data does
not benefit from high-performance parallel I/O paths and instead benefits from
low latency random access. This separation does not improve naming support
because that is not the goal of any of these systems.

Intel's DAOS system uses Intel DC Persistent Memory for meta-data and \ac{NVMe}
for data. DAOS supports two naming models: one is a POSIX-compatible
hierarchical name space, and the other is a key-value store interface. Storage
pools are accessed using one or the other of these interfaces but not both.

In a hierarchical name space the fully qualified path corresponds to the
\emph{logical} location of the file, that is the location of this file in the
namespace itself. In most file systems that logical location corresponds to a
specific file system instance and thus defines the storage device(s) on which
that file is stored.  Meta-data associated with that file is interpreted by the
file system to determine how to retrieve the data.  For a media file system that
usually corresponds to the address of the location(s) of the file data on the
corresponding media.  For a network file system that usually corresponds to the
network address used to request the correct file data from the computer storage
system where the data is actually stored.

Separating the file system meta-data from the file data is known to be
beneficial.~\cite{kawai2011a}  This approach logically makes sense when
considered in the multi-silo context as well, since a human user looking for
something actually does not care \textit{where} that the
thing is located they care \emph{what} the thing is.

This raises an intriguing question, one that is at the heart of my own research:
\emph{what is the purpose of naming?}  Applications are quite happy to use names
that are meaningless to humans, such as content hashes, UUIDs, or randomly
generated string names, such as are used for temporary application files.
Indeed, if an application knows that a name has a fixed format, the knowledge of
that fixed format can be used to simplify the implementation or to separate
application named files from other files.

Humans use file names to provide \emph{meaning} as to what is in the named
object.  Librarians assisting people in organizing data routinely suggest
embedding contextual information in the file names; directories are then used to
create additional organizational structure (or ``context''). There are numerous
limitations to this approach, however, because humans do not organize things in
the same fashion and even the same human does not organize things the same over
time. This is such a common phenomenon even the popular press pokes fun at it
(\autoref{fig:xkcd:1459}.)

The inability to find specific things is \textit{not} unique to computer data storage.
Indeed the current situation for organizing digital data seems to be descended
from the organizational structure of a library or a file cabinet, despite the
physical limitations of organizing paper files that are not constraints for
computer storage.  For example, when I am organizing my long play record
collection I have to chose one primary ``key'' to use, such as the time when the
album was released.  However, digital storage does not have this same
restriction - I can easily ask that they be presented to me by genre, artist,
album title, or even instrument type. Further, I can change my organizational
scheme dynamically.  Both Bell Lab's Plan 9~\cite{plan9} and Xerox Placeless
Documents~\cite{dourish1999getting}
observed that there is a dynamic nature to how information is organized.  The
hierarchical name space is basically static, like my album collection is static:
I can choose a different organizational scheme but to do so I have to physically
move them around.

Indeed, the system we have evolved works \emph{in spite} of the obvious
artificial limitations imposed by decades old decisions~\cite{dourish2003the}.  Rather than being
a deep insight that the current system works because it is the best of all
possible models, it is a testament to human ingenuity and a tolerance for
primitive, sub-standard systems.

There are a body of recommendations about file naming
standards\footnote{Data Management for
    Researchers~\cite{briney2015data}}
\footnote{Smithsonian: \url{https://library.si.edu/sites/default/files/tutorial/pdf/filenamingorganizing20180227.pdf}}
\footnote{Stanford: \url{https://library.stanford.edu/research/data-management-services/data-best-practices/best-practices-file-naming}}
\footnote{NIST: \url{https://www.nist.gov/system/files/documents/pml/wmd/labmetrology/ElectronicFileOrganizationTips-2016-03.pdf}}.
Harvard Data Management suggests~\footnote{\url{https://datamanagement.hms.harvard.edu/collect/file-naming-conventions}}:

\begin{itemize}
    \item Think about your files
    \item Identify metadata (e.g., date, sample, experiment)
    \item Abbreviate or encode metadata
    \item Use versioning
    \item Think about how you will search for your files
    \item Deliberately separate metadata elements
    \item Write down your naming conventions
\end{itemize}

Thus, there are common themes: a name provides context for \emph{what} the given
thing represents. Uniformity of information is also important --- the ``naming
convention'' permits not only identifying similarity but key elements of
\emph{difference} between any two named things.  Thus, to return to the original
question: ``what is the purpose of naming?''

\tm{It seems that this Harvard list is a serious indictment of the existing
    system: it pushes the cognitive load onto the users, talks about meta-data ,
    versioning, encoding, \emph{and} capturing the naming convention.}

It seems clear that human naming is
not the same as data storage naming, though we often treat them as the
same.  This can be seen in the data storage tendency to either use names that
are entirely about locating a specific object, such as is the case with a
key-value store, for example, or it combines location with identity, such as is
typified by the Uniform Resource Identifier (URI)~\cite{berners-lee1998uniform}.

Thus, it would seem that naming is about:

\begin{description}
    \item[Identity] --- the name of a thing should be sufficient to verify that
        it is the specific thing we seek.  An example of this is biometric data for
        humans or a cryptographic hash for a storage object;
    \item[Location] --- the name of a thing could provide information that
        directly or indirectly specifies where it is located.  A human example is
        how an identity card might specify a current residential address.  A data
        storage example is the U.S. Library of Congress Classification System, which
        can be used to find a particular work within very large bodies of work even
        though books are typically not thought of as a storage media;
    \item[Relationships]  --- the concept that one thing can be derived from
        another, whether in a direct form, such as a version of the same logical
        thing, or in a more indirect form, such as a provenance relationship showing
        that one thing was derived from another thing by applying some set of
        transformations to it;
    \item[Characteristics] --- this relates to something that is a property of
        the thing.  For humans this might be the date of birth (``creation date''),
        height, weight, eye color, etc.  For storage objects this might include
        timestamps, size, data type;
    \item[Context] --- at some some level, we often rely upon names to provide
        us with \emph{context}.  For humans, we assume that people with the same
        family name are likely to be members of the same family, even though
        ``\persa Smith'' is distinguished from ``\persb Smith''.
\end{description}

When considered from this perspective, it seems clear why ``naming is hard'' ---
it plays multiple distinct roles, sometimes overlapping, sometimes interfering.
These challenges are not well-served by existing naming support in computer
storage systems.

\tm{I feel that I should capture the dynamic nature of naming, which really does
    differ from the traditional static model of it.
}

\reto{Overall, I feel like the challenges are quite an overview of related work. I would try to focus on what challenges/problems arise when you try to desgin and implement your system?
    Challenge a) Multiple silos with different characteristics; and Challenge b) capturing activities, ... (or is that gone)? }
