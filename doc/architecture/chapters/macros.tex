% This file provides examples of some useful macros for typesetting
% dissertations.  None of the macros defined here are necessary beyond
% for the template documentation, so feel free to change, remove, and add
% your own definitions.
%
% We recommend that you define macros to separate the semantics
% of the things you write from how they are presented.  For example,
% you'll see definitions below for a macro \file{}: by using
% \file{} consistently in the text, we can change how filenames
% are typeset simply by changing the definition of \file{} in
% this file.
%
%% The following is a directive for TeXShop to indicate the main file
%%!TEX root = ../diss.tex

\newcommand{\NA}{\textsc{n/a}}	% for "not applicable"
\newcommand{\eg}{e.g.,\ }	% proper form of examples (\eg a, b, c)
\newcommand{\ie}{i.e.,\ }	% proper form for that is (\ie a, b, c)
\newcommand{\etal}{\emph{et al}}

% Some useful macros for typesetting terms.
\newcommand{\file}[1]{\texttt{#1}}
\newcommand{\class}[1]{\texttt{#1}}
\newcommand{\latexpackage}[1]{\href{http://www.ctan.org/macros/latex/contrib/#1}{\texttt{#1}}}
\newcommand{\latexmiscpackage}[1]{\href{http://www.ctan.org/macros/latex/contrib/misc/#1.sty}{\texttt{#1}}}
\newcommand{\env}[1]{\texttt{#1}}
\newcommand{\BibTeX}{Bib\TeX}

% Define a command \doi{} to typeset a digital object identifier (DOI).
% Note: if the following definition raise an error, then you likely
% have an ancient version of url.sty.  Either find a more recent version
% (3.1 or later work fine) and simply copy it into this directory,  or
% comment out the following two lines and uncomment the third.
\DeclareUrlCommand\DOI{}
\newcommand{\doi}[1]{\href{http://dx.doi.org/#1}{\DOI{doi:#1}}}
%\newcommand{\doi}[1]{\href{http://dx.doi.org/#1}{doi:#1}}

% Useful macro to reference an online document with a hyperlink
% as well with the URL explicitly listed in a footnote
% #1: the URL
% #2: the anchoring text
\newcommand{\webref}[2]{\href{#1}{#2}\footnote{\url{#1}}}

% epigraph is a nice environment for typesetting quotations
\makeatletter
\newenvironment{epigraph}{%%
	\begin{flushright}
		\begin{minipage}{\columnwidth-0.75in}
			\begin{flushright}
				\@ifundefined{singlespacing}{}{\singlespacing}%
				}{
			\end{flushright}
		\end{minipage}
	\end{flushright}}
\makeatother

% \FIXME{} is a useful macro for noting things needing to be changed.
% The following definition will also output a warning to the console
\newcommand{\FIXME}[1]{\typeout{**FIXME** #1}\textbf{[FIXME: #1]}}

% Replaceable names
%\newcommand{\system}[0]{\emph{Kwishut}\xspace}
\newcommand{\system}[0]{\emph{Indaleko}\xspace}
% QI'tu'naS - Klingon for "pragmatics" - I'm saving that for the final version.
% qaywI' - Klingon for "finding" - another option for final version.
% Talal - Vulcan for "finding" - yet another option.

\newcommand{\systemone}[0]{\emph{Topish}\xspace}
% Topish is Uzbek for "Finding"
%\newcommand{\systemtwo}[0]{\emph{\textmacedonian{Наоѓање}}\xspace}
% Pronunciation: Naoǵanje.
%\newcommand{\systemtwo}[0]{\emph{\textmacedonian{Naīdi}}\xspace}
% Naīdi ("Find") in Macedonian.
\newcommand{\systemtwo}[0]{\emph{\textmacedonian{Находка}}\xspace}
% Pronunciation: Nakhodka

% Gender neutral names (though very European)
%\newcommand{\persa}[0]{Addison\xspace}
\newcommand{\persa}[0]{Aki\xspace} % Japanese, gender neutral, means "autumn"

%\newcommand{\persb}[0]{Bailey\xspace}
\newcommand{\persb}[0]{Dagon\xspace} % Biblical name meaning "fish"

%\newcommand{\persc}[0]{Cameron\xspace}
\newcommand{\persc}[0]{Fenix\xspace} % Greek Isles, means "dark red"

%\newcommand{\persd}[0]{Dana\xspace}
\newcommand{\persd}[0]{Hao\xspace} % Vietnamese, means "good/perfect"

%\newcommand{\perse}[0]{Evan\xspace}
\newcommand{\perse}[0]{Waneta\xspace} % Native American roots, "shape shifter, charger"

%\newcommand{\persf}[0]{Quinn\xspace}
\newcommand{\persf}[0]{Skylar\xspace} % allegedly American roots, meaning "scholar"

%\newcommand{\persg}[0]{Reese\xspace}
\newcommand{\persg}[0]{Zene\xspace} % "beautiful" based on African culture supposedly.

% use cases
\newcommand{\usecaseactivitycontext}[0]{\textsc{ac\-tiv\-ity con\-text}\xspace}
\newcommand{\usecasedatarelationship}[0]{\textsc{data re\-la\-tion\-ships}\xspace}
\newcommand{\usecasecrosssilosearch}[0]{\textsc{cross-silo search}\xspace}
\newcommand{\usecasenotifications}[0]{\textsc{no\-ti\-fi\-ca\-tions}\xspace}
\newcommand{\usecasepersnamespace}[0]{\textsc{per\-son\-al\-ized name\-space}\xspace}

% terminology
%Copy: bit-for-bit identical
\newcommand{\doccopy}[0]{copy\xspace}
%  derivation: the semantics change
\newcommand{\docderivation}[0]{derivation\xspace}
% conversion: semantically identical but not Bit-for-Bit
\newcommand{\docconversion}[0]{conversion\xspace}

% END
