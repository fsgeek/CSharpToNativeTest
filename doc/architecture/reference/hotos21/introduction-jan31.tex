\section{Introduction}\label{sec:intro}
Modern file systems continue to visualize their data in the way they have done for more than a half-century: 
combining directories in a hierarchical file organization. 
Directories are used to group related files but this organization's purpose is to simplify search. 
Directory names provide a common property used to locate these related files. 
An hierarchical name based organization works for human users to perform search and navigation when there is
a small number of files per directory, use unique names with contextual meaning, and can recall those names.

Much has changed since we adopted this design: most data now does not reside on local file systems.
Instead data is distributed across a myriad of services, e.g., Dropbox, Google docs, Amazon S3, Microsoft OneDrive, github, 
as well as attached to communication mechanisms and applications (e.g., email, chat, Slack, overleaf). 
Today, just like local file systems, each of these storage solutions provides its own organization and naming mechanisms, 
most with the classic hierarchical structure, either as part of their model, or via a layered namespace provider.
Local namespaces have now grown to an extent that they are not realistically searchable.  
When we consider non-local namespaces, we amplify the need to perform multiple searches, usually done
iteratively, in serial fashion by the human user trying to find a specific file or set of files.
Storage silos that do not provide an hierarchical namespace either are not easily searched or provide their own
unique search interface, further exacerbating the problem of \textit{finding} a specific item~\cite{bergman2019factors}. 
Thus accurately remembering and associating complete paths with the contents of the files becomes humanly
impossible, making search a repetitive, time consuming, and frustrating affair.

Apart from a frustrating search, organizing large collections of files is equally hard. The burden
of organizing these vast file collections uniquely in a hierarchical organization across 
disconnected storage silos lies with the humans that stored the files. \textit{Furthermore, the only
reason for this organization burden is to enable prospective future search and navigation for specific
files that are important amidst a sea of files best forgotten.}  Of course, the challenge is knowing
\textit{now} which files will be important \textit{in the future}.
Thus, we need to evolve our storage systems to permit search and navigation across all storage silos: 
local, disconnected, and distributed.

In an hierarchical organization, directories provide organization of files and thus become the basis of file search.
Without directories, files need explicit attributes that can be searched for file retrieval. Such attributes
can be stored as metadata where the files reside, created explicitly by users or by applications built
to add useful information based upon provenance and file content.
Such a system can interface with a file data visualizer that presents personalized user name spaces and
responds to user behavior.  The goal is to enable the user to focus on the \textit{context} and 
\textit{meaning} of files, while the system enables the file data visualizer to present the customized
user namespace.

An ecosystem of novel data visualizers and tools to augment existing file attribute information, combined
with a global namespace service supporting disparate storage silos and enabling storage of existing and
novel file attributes makes it possible to address the human needs for unified search and navigation.
This is a natural progression to meet user needs in a future where the number of storage silos and 
the number of files that are stored keeping growing. 

We propose building Nirvana, a system focused on providing file namespace services that support a personal global namespace and enable
novel new visualizations of file data that relies upon Nirvana's capabilities.
As part of our early validation of this we build a proof-of-concept navigation tool.
This tool relied on using existing and potentially new file properties to present a personalized user-based namespace.
On prelimnary examination, we found them a promising solution to today's multi-silo reality.\footnote{We have omitted a link to our demonstration video for purposes of maintaining the blind nature of this submission; we will include that link in a final paper.} 

% (TM) I think that this is a bit weak, so I've omitted it at this point.
% We hope to conduct future research to see it implemented.
    
In the next section, we highlight prior work that either addresses similar issues or presents building blocks upon which a solution might be developed. Then, we propose an architecture for the construction and maintenance of personal, multi-silo namespaces. Based on this proposed architecture, we then identify and discuss the research challenges that must be addressed to realize such a vision. Finally, we wrap up with some suggested next steps.













