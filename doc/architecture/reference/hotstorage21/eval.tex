\section{Future Directions}
\label{sec:future}
We now explore a few research directions that \system suggests.

\noindent\textbf{Attribute Security and Integrity.}
\system decouples naming and attributes from the storage object. This opens up a research direction on the security model of attributes themselves. Are the permissions on the attributes similar to the ones on the storage objects themselves? Can a user change an attribute in its local namespace, but not in the company wide one? This segues into the question of attribute integrity/quality: not all attribute sources have the same trust-level. For instance, a user might label an image ``dog,'' while the image recognition AS might label it ``cat''.

\noindent\textbf{Privacy.}
\system collects a lot of metadata across multiple communicating channels and storage silos, including activity data. This raises the question of how to manage these metadata in a privacy-preserving manner.

\noindent\textbf{Interface Design.}
We presented a system architecture that provides a rich context to search and organize storage objects. We envision that this will provide the foundation for new directions in HCI research: By using individual namespaces, we can dynamically organize and visualize documents and other storage objects, and seamlessly navigate and locate related documents providing a new user
experience.

\noindent\textbf{Relationship-based Queries.}
\system tracks relationships between storage objects. These relationships provide minimal data provenance~\cite{provprimer} allowin users to locate the chain of related documents originating in a specific activity context.
These relationships are most naturally expressed as graphs, where nodes are objects, and edges are the relationships between objects. Edges could have weighted-labels, indicating the type and importance of their relationship. 
This enables more sophisticated data-analysis beyond pure content-based indexing by using graph queries. 
For instance, lineage queries (i.e., tracing the history of an object) are path traversals, which are challenging to implement efficiently in conventional storage systems. This suggests that the NS and/or MS require sophisticated storage and query mechanisms.