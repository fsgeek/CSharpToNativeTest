
\section{Introduction}
\label{sec:intro}

Today's file systems provide two primary functions: a way to store chunks of data and names that provide users with the familiar
metaphor of a file cabinet (i.e., folders and documents).
Much has changed since we adopted this design.
Now most data resides not on local file systems but on myriad services such as Dropbox, Google docs, Amazon S3, Microsoft OneDrive, and Github, as well as attached to communication mechanisms and applications such as email, chat, and collaborative communication platforms (e.g., Slack, Discord).
Today, just like local file systems, each of these storage solutions provides its own storage and naming mechanisms. Pity the user Alice who wants to find the file that was sent to her by Bob while they were having a slack conversation about cool papers in HotStorage 2020, if she remembers neither the name of the file nor whether she stored it in Dropbox, Google drive or her local file system.

The \emph{Placeless} architecture~\cite{placeless-tois} provided an elegant solution to this problem by enabling search across different storage silos using semantically meaningful names. Each file was annotated with a rich set of attributes, determined by the user or generated by software, and a naming service, spanning silos, searched the entire collection of user files using these semantically meaningful attributes. \emph{Placeless} was a huge improvement over isolated storage silos and semantically-poor names, but it did not solve Alice’s problem. Finding Alice's document requires that we track files across storage silos and \emph{activity contexts}. An activity context describes the other activities that were happening at the same time a document was accessed. This context might include applications, such as Slack, email or a web browser, which are not file systems in any traditional sense, but can be sources and destinations for data and for its semantically meaningful context.

The only solution of which we are aware that captures activity context is Burrito~\cite{guo2012burrito}, which used data provenance to keep track of a user’s activity context, i.e., the applications they were running and actions they were taking while examining a particular file. Unfortunately, Burrito is a desktop application that was intended neither to work across multiple devices nor to span multiple, remote storage silos. While it introduced the idea of activity context sensitive search, it did not address any of the semantic searching issues of Placeless and it did not consider the privacy consequences of storing user context in a distributed environment.

We posit that \textbf{1) user naming should be entirely decoupled from local naming and 2) users need customizable and personal namespaces.}
We present \system\footnote{\system means ``naming'' in a native North American language.}, an architecture embodying this position, leveraging existing infrastructure where possible and extending it where necessary.
\system uses separate metadata and naming services coupled with user activity monitors.
\system is designed to allow incorporation of existing storage, metadata, and naming services without modification, while providing enhanced functionality when services support \system features.
We limit discussion to the systems infrastructure required to realize this vision; current storage management interfaces (e.g., file browser) can use \system namespaces directly, while the availability of rich metadata in \system enables HCI research on better ways for users to identify and find their data.

We begin with use cases motivating the need for \system (\S\ref{sec:use-cases}), highlighting specific features missing from today's storage and naming services. Next, we present the \system architecture (\S\ref{sec:arch}) and future research directions it enables (\S\ref{sec:future}).
We then discuss how \system builds upon prior work
(\S\ref{sec:background}) and conclude summarizing our position (\S\ref{sec:conclusion}).
