\section{Use Cases}

We consider the following potential use cases for our naming system:

\begin{description}

    \item[Finding historical documents] - in this common usage scenario, we are
    looking for an object that we identify by what we were doing and when we
    were doing it as well as potential subjects.

    \item[Related documents in distinct storage locations] --- in this common
    usage scenario, we are looking for \textit{related} objects that, for
    whatever reason, are stored in different ``storage silos''.  For example, we
    received a document via e-mail and then saved it on the ``Dropbox folder''
    of our local system.  The e-mail and the document are related, but we have
    no obvious way to traverse back from the document to the original e-mail.

    \item[Ability to search non-traditional storage locations] --- in this usage
    scenario, we are looking for \textit{objects} that are not stored in a
    traditional storage silo but instead reside inside some other object storage
    domain.  For example, objects in an arbitrary object store, such as the
    ubiquitous key-value store.

    \item[Cross-silo versions / Document Identity]  ---
    draft.doc that you got from your coworker via e-mail, that version you stored
    on your local drive, then the one you uploaded to Office365. Another one you've
    got from another co-worker via Slack. Nirvana will show these as different
    versions from the same logical document, possibly even the temporal relationships
    between then

    \item[Notifications] -- Allowing a user to "subscribe" to change
    notifications for critical documents; e.g., an active collaborative project
    to ensure users are notified when documents are updated (and of course
    cancel notifications when they are no longer needed/useful).

    \item[Search Results] -- What kind of searches happened during the writing
    of of another document (e.g., you are interested in the statistics of a
    search, but not particular results)

    \item[Compliance] --- Identifying related documents, including references,
    to specific material that needs to be located as part of compliance with
    legal mandates, e.g., discovery notices, GDPR data removal requests (``right
    to be forgotten'').
\end{description}


\subsection{Prior Usecase Examples}

\reto{OLD `usecases follow`}

% locating documents in general
\noindent\textbf{Locating Documents.}
Users want to search for particular documents (use-cases \\usecasehistcontext, \\usecasereldocuments).
Indexing services (e.g., Spotlight), cloud-based platforms, or tools like \texttt{grep} or \texttt{find} provide mechanisms to locate documents based on their name, date of creation or modification, or even the content of the files.
However, users often need to search different storage silos independently. This poses a burden to the user.
Worse, searching over locally mounted network attached storage (e.g., NFS or Dropbox) could mean transferring gigabytes worth of data for doing the search.
Finally, current solutions do not capture activities well such as while on a call with a customer, or while attending a webinar or conference.
Not only fail existing architecture to capture these aspects, they are also not capturing seemingly simple relationships between a file stored on disk and the e-mail exchange through which it was received and thus losing important contextual information.

% integration with non-traditional storage
\noindent\textbf{Alternative Storage.}
Traditional files are not the only way to store documents.
Today's applications may use object stores, key-value stores and data bases, or even implement their own storage container holding multiple objects (use case \\usecasealtstorage).
Moreover, applications want to protect their data from unauthorized accesses using various methods such as encryption, for example.
This makes searching hard and the user is forced to use the interface provided by the application to locate its documents, resulting in yet another manual cross-silo search.
While point solutions exist, e.g., Android search integration or export as WebDav-based file system, they are not ubiquitously available or do not allow users to search using the full range of attributes.

% document identity / cross silo version
\noindent\textbf{Document Identity.}
Documents have an identity. Writing an additional paragraph in this paper, does not change its identity. It rather creates a new version of the same document and both versions are related.
Likewise, uploading a document to cloud storage, renaming it, or converting it to a PDF also does not change its identity (use-cases \\usecasereldocuments, \\usecasedocidentity).
In contrast, taking last year's HotStorage paper as a template for this year's paper will change its identity.
While versioning is available on various storage solutions or VCS systems, it fails to capture cross-silo versions and relation ships. For example,
the document that was just received via e-mail is in fact a version of the one you have been editing yesterday, and that you have just uploaded to cloud storage for sharing with your co-worker who just sent it to you via e-mail.

% multi-device
\noindent\textbf{Multi-Device.}
Users may use various devices to access their files, each of which having different compute and connection capabilities (use-case \\usecasedevices).
For instance, a desktop machine is powerful and always connected, while a smartphone is low power and can often be disconnected. Cloud-storage or e-mail client apps on the devices provide a search interfaces that may offload the actual search to the server.
A user with two devices cannot easily look for files on the local disks and has now to do the search manually across multiple devices (similar to \\usecasereldocuments).

% attribute provenance
\noindent\textbf{Attribute Provenance.}
Traditional file systems tie the permissions to change attributes with the permissions to change the file contents.
Any user with sufficient rights can change any attribute or the file contents.
While there are systems that record the user who changed the file last, it may not capture the entire history of the changes (use-case \\usecaseattrprov).
Moreover, existing systems do not capture the intent or reason of those changes.

% attribute provenance
\noindent\textbf{Multi-View.}
A document cannot be physically present in multiple filing cabinets at the same time.
Hard and soft links provide references to other other storage locations.
However, there is still only a single physical organization of files making it impossible to organize files by year-customer, and customer-year at the same time.
File managers may offer tags, and media libraries to notion of albums and grouping by year to provide specific ways to sort, filter or organize files.
This approach, however, is not generally available and users cannot freely choose and adapt their \emph{view} of their files (use-case \\usecaseviews)

\section{Notes from the meeting (remove as pleased)}

Focus on the what the system does, with a bit of how..

\begin{itemize}
    \item Placeless + Burrito?~\footnote{\url{https://www.usenix.org/system/files/conference/tapp12/tapp12-final10.pdf}}
    \item Attributes that are not traditionally a file.
    \item GIS information is useful
    \item don't open files for extracting meta-data
\end{itemize}

Taxonomy

\begin{itemize}
 \item table what can be done what can't. 3 columns. Use-case, what can be done, what's hard.
 \item here's a structure of them. (what's easy and hard to do with today's technology) --> references to architecture section
 \item evaluation: show that we are supporting the things that are hard.
\end{itemize}

 \begin{description}
\item[Section 1] introduction
\item[Section 2] Background
\item[Section 3] Use-cases + table
\item[Section 4] Architecture
\item[Section 5] Show that use-cases are solved
 \end{description}

Different clients:

\begin{itemize}
 \item Desktop: powerful + always connected
 \item Laptop: powerful + can be offline
 \item Smartphone: low power + mostly connected
 \item Smartphone w/o data: low power + mostly disconnected
\end{itemize}
