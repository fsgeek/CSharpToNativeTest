\section{Random Notes}

So this is where things get interesting.  One thought I'd had was that I'd get rid of directories entirely - they don't serve any real \textit{point}
other than to capture a relationship; in another embodiment the directories are simply a form of vertex.  I still like the model without explicit
directories, since we could have a \textit{sibling} relationship.  So, for example, you just give everything in the same directory the same
\textit{is contained by} label.  Then you find everything in that directory by finding all the files that have the same label (ah, because THAT is a
relationship!)

This needs more fleshing out, for sure.  It's quite radical but that is what is needed here.  There are tons of questions here, though.



Graphs are a generalization of trees (or trees are a limited type of graph).  Trees are easier to search than graphs because they are by definition without cycles.

Note: cycles are in fact a real issue in real operating systems already and are detected in a primitive fashion (“how many links have we traversed already”). 

Note that in the search field, graphs are walked and cycles detected by keeping track of what has already been visited.  [5] This approach didn’t work when memory was limited (e.g., 1965 or 1973) but now memory is considerably more available than it has been in the past.

Question: is it viable for us to keep track of the “what’s been visited” as we do graph walking.

What is a “graph file system”?

Files are our vertices.  They remain containers of data and attributes (we will want to generalize container to abstract storage from name space).
Edges are connections between vertices.  Question: do we want a hypergraph (so one edge per property) or a graph (so multiple properties per edge, with unique edges).  Searching the graph is likely to be an important aspect of this (e.g., the pragmatic aspect).
