\section{Search}\label{sec:search}

The driving force behind our graph file system design is to provide the
infrastructure to make it easier for users to find data.
Users do not navigate to data, they \textit{search} for it, so we
consider more effective search models to further
motivate the graph file system.

We observe that there are two different models of ``search'': application
search and user search.
Applications need to be able to open files quickly 
using a \textit{key}. For example, both NFS~\cite{sandberg1986sun}
and AFS~\cite{sidebotham1986volumes} use the file system \textit{inode number} as their
mechanism for identifying the specific file or directory being accessed,
because it is fast, avoiding a costly namespace traversal.
Similarly, NTFS supports the ability of applications to open a file by 
identifier~\cite{sreenivas2011bypass}.
They did this to support their implementation
of the Apple File Protocol (``service for Macintosh'') but has subsequently been used
for other uses. Indeed, it has been further extended to permit files to be opened by an
application-defined identifier (a UUID); Microsoft continues to support file IDs in 
ReFS~\cite{microsoft:refs:features}. The Google File System~\cite{Ghemawat2003} 
observation was similar: applications can use keys to find their files.

Modern applications tend to either create files that they use internally, often going to great lengths
to hide their location from the user; or maintain a list of recently used items with a full path name,
which breaks when the path changes, even if the file did not change.  A key interface for applications
better fits this usage model. Thus, a ``search by key'' interface is sufficient.

The more challenging problem is user-focused search.
\begin{comment}
Many of the
characteristics in a good human usable search system do not benefit the programs
directly.
\end{comment}
For human users, we want to enable a model like the
\textit{memex}~\cite{bush1945we}: ``A memex is a device in which an individual stores
all his books, records, and communications, and which is mechanized so that it may be
consulted with exceeding speed and flexibility. It is an enlarged intimate supplement
to his memory.''

The HCI community has a long history researching
more effective search, including such efforts as
SIS~\cite{dumais2016stuff} and faceted search~\cite{arenas2016faceted,tunkelang2009faceted,hearst2006design,klungre2018evaluating,walton2017looking,cleverley2015retrieving}.
Critical to this work is the idea that search is most effective
when \textit{not} bound to a specific taxonomic order --- very much the opposite of today's 
hierarchical search model, which enforces a rigid
order on the structure of information.

How does a graph file system then enable modern search?
First, support for a broad and extensible set of attributes and 
relationships brings search engine technology to bear in the
service of file systems.
There is some irony that the success of web search, and in particular the
primacy of relationships in those algorithms~\cite{page1999pagerank}, has had
virtually no impact on how we find our own local data.
Second, the generalized graph structure, which no longer elevates any single
organzation gets rid of the \textit{specific taxnomic order} that HCI
researchers determine to be counterproductive.
Third, although some degree of temporal query is possible using \texttt{find},
its interface is not especially accessible to the typical user, and it requires
a series of manual operations to express natural queries such as
``Show me the documents I wrote last summer after I got back from my Amazon
rafting trip.''

Our goal is not to specify the entire range of searches that can be realized, but rather to
explore file system structures that enable the creation and mining of relationships to help users to find relevant data.

\begin{comment}
To help motivate our work, we consider the \textit{Graph Query 
Language}~\footnote{https://gql.today/wp-content/uploads/2018/05/a-proposal-to-the-database-industry-not-three-but-one-gql.pdf} 
(GQL) as a starting point.  GQL is an emerging language
attuned to the needs of \textit{property graphs}, which happen to be similar to the model that we envision for our new
file system.  It attempts to merge the strengths of three existing graph database query languages into a single, standard,
query language for property graphs~\cite{van2016pgql,francis2018cypher,angles2018g}.

% This choice is motivated by our realization that the new model we propose is a property graph~\cite{rudolf2013graph}.
\end{comment}
