\section{Related Work}\label{sec:background}

The need for better name spaces in file systems is hardly a new
topic, with many solutions being proposed and implemented over
the years.

\textit{Search utilities} are successors to the permuted index program.  They permit us to find
files based on \textit{content} and \textit{attributes}.  MacOS X has \textit{spotlight}, which
provides an extensible, index-driven search service~\cite{apple:spotlight-extensions}.  Similarly,
Windows offer a similar extensible service~\cite{microsoft:data-add-in}.  These enable searching
based upon attributes, e.g., file suffix, date, size, etc., and context-sensitive content, e.g.,
music files by artist, composer, song title, or even \textit{rights},
but limited, if any, ability to search by relationship.

%.  A number of other examples
%of similar search mechanisms
% exist~\cite{Suguna2015,huo2016mbfs,leung2009magellan}.

%\textit{Databases} enable developers willing to pre-define their data's structure to enable searching
%on the specifics of the data.  Databases come in a rich array of models: relational, column, document,
%and graph, for example.  File systems constructed from databases have been extensively 
%explored~\cite{olson1993design,balabine1999file,balabine2002database,kashyap2004file,murphy2002design}.
%As we noted previously (\S \ref{sec:graphfs:model}) such approaches have failed to yield clear
%results.

\textit{Tag Systems} were an early approach to improving hierarchical file systems searchability%
~\cite{Parker-Wood2014,chou2015findfs,ma2009file,laursen2014,nayuki2017,Andrews2012,Up2016,Jones2016,aws:s3:object:tagging,ames2006lifs,leung2009magellan,frieder2012hierarchical}.
Automatic tagging systems have become a more common approach here as manual tagging by users 
has proven to be impractical~\cite{soules2003can,soules2004toward}.
The addition of \textit{semantic} information
~\cite{di2017gfs,hua2016real,martin2004formal,Martin2005,martin2008,martin2014,gifford1991semantic,Faubel2008,harlan2011joinfs,Suguna2015,Andrews2012,ngo2007integrating,Omvlee2009,wang2003managing,gopal1999integrating,Codocedo2015,Jones2016,Mahalingam2003,Parker-Wood2014}
is useful but falls short of addressing the fundamental need to understand
data relationships, because like more conventional systems,
these semantically-aware approaches still focus on the file,
not on the relationships between the files.
As such, they are simply an add-on to
the hierarchical model, not a replacement for it.
Such approaches can provide useful functionality
in our graph file system model as well.
Indeed, we even pointed this out (perhaps subconsciously) in prior
work when we said ``How many of them [files] are \textit{related} to
each other?'' [emphasis added]~\cite{Seltzer2009} .

Files are rich with relationships.  However, these relationships are not limited to what the
file system can ``see''.
Narrowing our vision to the closed pool of file system relationships
hobbles our ability to capture them.
For example, the obvious relationship between
a file and the e-mail from whence it originated is not exploitable in
any system of which we are aware.
The academic papers we generate refer to other papers.  An enlightened document application would provide an identifier that can 
be used to find the corresponding paper - a \textit{refers to} relationship,
whether on our local system or elsewhere.
Of course, in the current model, we likely can't
recall where we stored it when we downloaded it.  As we create new works, we refer to older works --- our own
documents, web pages, Jupyter notebooks, spreadsheets, pictures, etc. Capturing these relationships permits
us to reconstruct the process taken to produce an output.
This is the fundamental problem that the provenance community has been
addressing, but few systems~\cite{pasquier17camflow,reddy06pass} demonstrate
an understanding of the role our file systems play in making this
possible.

Versioning is a feature that continues to reappear in various guises.
This is simply one example of temporal locality; a
relationship that we have not yet deeply mined.
While it is now common practice for
individual applications to ``remember'' the last few files you have accessed,
there are few cross-application examples.
In lieu of the right tools, users invent creative solutions.
For example, one of the authors \textit{attaches} documents to e-mail
immediately after reviewing them
specifically to establish temporal relationship.
The ability to establish temporal relationships across applications should
provide powerful capabilities.

We do not know which relationships are useful. One of the hallmarks of good file systems design over the decades
has been \textit{not} to impose a specific restricted model on what files can be --- we leave that to databases.  
We do not intend to establish a definitive set of relationships any more than we focus on defining
the structure of file contents.
However, we encourage others to explore this area, encourage best practices,
and build tools that produce and use such relationships, leaving the
storage and retrieval of relationships to the file system.

% Relationships between files is not new --- this is the quintessential relationship of the modern Internet, with its
% vast web of content that references across the domains --- but those models are certainly more recent than that
% of the hierarchical file system.  Unlike the internet, where information is shared, we seek to enable
% similar relationships useful to our specific usage and with data we do not necessarily wish to share.

Much of the raw data that applications generate would be better \textit{not} injected into
the hierarchical name space: the location of our personal email database, financial
software files, binaries, temporary files, etc.  Their presence in the name space clutter
it and make our existing brute force search slower yet no more useful.

Application programs benefit being
unfettered from the hierarchical name space~\cite{Ghemawat2003}, both
in terms of their efficiency as well as the benefits of \textit{not} commingling the private files of individual 
applications --- but the hierarchical file system requires
they be stored somewhere within its domain.
Applications routinely hide most of their files in out of the way 
locations, as they are only useful to the application itself.
Thus we end up with ``System Volume Information'' and ``.ssh'' and a myriad of obscure locations where applications 
hide data from us.
This is a side effect of the name space model we have used for the past
50 years.

Prior attempts to address this have done so within the confines of a narrow perspective of what is needed to fully 
enable the ability of us to find our data.  The HCI community have been poking at the
edges of this problem for decades as well --- observing the frailties of the hierarchical model and suggesting
alternatives~\cite{harper2013file,lindley2018exploring,khan2018forgotten,vitale2018hoarding,boardman2003too,nayuki2017,martin2014,Jan2011,Andrews2012,Mander1992,Omvlee2009}.

Our graph file system permits them to escape the existing paradigm \textit{without} giving up support
of existing applications.
