
\section{Conclusion}\label{sec:conclusion}

Hierarchical file systems have served us far longer than we had any right
to expect over the past half century, but the need to find and realize a
new paradigm becomes more critical each minute we spend searching for that
elusive document.
To date, our model for coping with this lack of usability has been to ignore
the file system namespace entirely --- is there any more strident
statement as to the model's inadequacy than \textit{irrelevance}?

There is a place for a modern file system name space in systems;
we do not need to abrogate our responsibility to provide this traditional
service by relying upon our programs and users to compensate for our dereliction
of responsibility.
The graph file system is one approach to address these concerns in
a novel way; we claim that it
better achieves the goals that have been set forth over the years by clearly
identifying the missing element in prior proposals: capturing and exploiting
\textit{data relationships}.
This is not so much a bold new model as it is 
a realization that our existing model is constrained, and we can better
achieve our goals by using the time-honored systems tradition of relaxing
constraints to achieve improved levels of functionality.
By elevating arbitrary and generalized relationships as first class file
system elements, we can provide a better user experience. 

Some will avoid this work, because it touches upon the unpredictable and
messy area of human users.
However, it is time to realize that our systems are no longer our personal
playground, but a critical service to humanity.

% Elevating arbitrary and generalized relationships as first class
% file system elements provides a better user experience and leads to entirely
% different implementation approach --- an approach that we intend to realize
% in the coming years.


