\section*{Abstract}
Rumors of the demise of the hierarchical file system namespace have been
greatly exaggerated.
While there seems to be wide spread agreement that most users make no use
of the hierarchical name space, we continue to use it both as the underlying
storage structure and as the default user interface.
To compensate for this mismatch between the native organization and the
way users interact with their data,
we have produced myriad search tools atop the file system.
This approach, however, has some limitations.
In particular, we focus on the absence of generalized relationships
as first class entities.

The hierarchical namespace elevates the \emph{contains} relationship
above all else.
Applications elevate the \emph{was created by me} relationship.
These are only two particular relationships among many;
items accessed around the same time share a temporal relationship;
attachments to email messages share a relationship;
a collection of documents, email message, notes, etc. form another
relationship.
We claim that elevating arbitrary and generalized relationships as first class
file system elements provides a better user experience and leads to entirely
different implementation approach.

