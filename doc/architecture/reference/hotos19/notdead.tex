\pdfminorversion=7
\documentclass[sigconf,10pt,noacm]{acmart}
\usepackage{geometry}
\geometry{reset, twoside=true, head=13pt,
     paperwidth=8.5in, paperheight=11in,
     includeheadfoot, columnsep=2pc,
     top=57pt, bottom=73pt, inner=72pt, outer=72pt,
     marginparwidth=2pc,heightrounded
     }

\usepackage[utf8]{inputenc}
%\usepackage{fontspec} % This line only for XeLaTeX and LuaLaTeX
\usepackage{pgfplots}
\pgfplotsset{compat=1.16}
\usepackage{tikz}


\usepackage{booktabs} % For formal tables
\usepackage{balance}       % to better equalize the last page
% Load basic packages
\usepackage{balance}       % to better equalize the last page
%\usepackage{graphics}      % for EPS, load graphicx instead 
\usepackage{graphicx}
%\usepackage[T1]{fontenc}   % for umlauts and other diaeresis
%\usepackage{txfonts}
%\usepackage{mathptmx}
%\usepackage[pdflang={en-US}]{hyperref}
\usepackage{hyperref}
%\usepackage{color}
\usepackage{textcomp}
\usepackage{blindtext}
\usepackage{subcaption}
\usepackage{dblfloatfix}
\usepackage{adjustbox}

\usepackage{booktabs}
\usepackage{multirow}
% \usepackage{lscape}
% If you use beamer only pass "xcolor=table" option, i.e. \documentclass[xcolor=table]{beamer}
\usepackage{siunitx}
%\usepackage{pgfplotstable}
% Some optional stuff you might like/need.
\usepackage{microtype}        % Improved Tracking and Kerning
\usepackage[all]{hypcap}    % Fixes bug in hyperref caption linking
\usepackage{ccicons}          % Cite your images correctly!
% \usepackage[utf8]{inputenc} % for a UTF8 editor only

% If you want to use todo notes, marginpars etc. during creation of
% your draft document, you have to enable the "chi_draft" option for
% the document class. To do this, change the very first line to:
% "\documentclass[chi_draft]{sigchi}". You can then place todo notes
% by using the "\todo{...}"  command. Make sure to disable the draft
% option again before submitting your final document.
%\usepackage{todonotes}
%\usepackage{draftwatermark}
%\SetWatermarkScale{5}

\usepackage{graphicx}
\usepackage{type1cm}
\usepackage{eso-pic}
\usepackage{lipsum}

% Paper metadata (use plain text, for PDF inclusion and later
% re-using, if desired).  Use \emtpyauthor when submitting for review
% so you remain anonymous.

% Copyright
\setcopyright{none}
%\setcopyright{acmcopyright}
%\setcopyright{acmlicensed}
%\setcopyright{rightsretained}
%\setcopyright{usgov}
%\setcopyright{usgovmixed}
%\setcopyright{cagov}
%\setcopyright{cagovmixed}


%\copyrightyear{2019} 
%\acmYear{2019} 
%\setcopyright{rightsretained}
%\acmConference[HotOS XVII]{The 17th Workshop on Hot Topics in Operating Systems (HOTOS)}{May 12-15, 2019}{Bertinoro, Italy}
%\acmBooktitle{Proceedings of the 17th Workshop on Hot Topics in Operating Systems (HOTOS '19), May 12-15, 2019, Bertinoro, Italy}
%\acmPrice{15.00}
%\acmDOI{10.475/123_4}
%\acmISBN{123-4567-24-567/08/06}
\acmConference{}
\acmBooktitle{}
\acmPrice{}
\acmDOI{}
\acmISBN{}
\acmYear{}

% These commands are optional
%\acmBooktitle{Transactions of the ACM Woodstock conference}
%\editor{Jennifer B. Sartor}
%\editor{Theo D'Hondt}
%\editor{Wolfgang De Meuter}

\pagestyle{plain}
\settopmatter{printacmref=false}
%\settopmatter{printfolios=false}
%\setlength {\marginparwidth}{2cm}
\renewcommand\footnotetextcopyrightpermission[1]{} % removes footnote with conference information in first column

\setlength{\textfloatsep}{10pt plus 1.0pt minus 2.0pt}
\setlength{\intextsep}{6pt plus 2.0pt minus 2.0pt}

\begin{document}
\title{Not Dead Yet}
\subtitle{Hierarchical File Systems Won't Die}

\author{Tony Mason}
\orcid{0000-0002-0651-5019}
\affiliation{%
  \institution{The University of British Columbia}
}
\email{fsgeek@cs.ubc.ca}

\author{Margo Seltzer}
\affiliation{%
\institution{The University of British Columbia}
%\city{Atlanta}
%\state{Georgia}
%\country{US}
}
\email{mseltzer@cs.ubc.ca}

% The default list of authors is too long for headers.
\renewcommand{\shortauthors}{T. Mason et al.}

%
% Generated this code.  Should review it
%
\begin{comment}
\begin{CCSXML}
  <ccs2012>
  <concept>
  <concept_id>10003456.10003457.10003527.10003531.10003533</concept_id>
  <concept_desc>Social and professional topics~Computer science education</concept_desc>
  <concept_significance>300</concept_significance>
  </concept>
  <concept>
  <concept_id>10003456.10003457.10003527.10003531.10003533.10011595</concept_id>
  <concept_desc>Social and professional topics~CS1</concept_desc>
  <concept_significance>300</concept_significance>
  </concept>
  <concept>
  <concept_id>10003456.10003457.10003527.10003540</concept_id>
  <concept_desc>Social and professional topics~Student assessment</concept_desc>
  <concept_significance>300</concept_significance>
  </concept>
  </ccs2012>
  
\end{CCSXML}

\ccsdesc[300]{Social and professional topics~Computer science education}
\ccsdesc[300]{Social and professional topics~CS1}
\ccsdesc[300]{Social and professional topics~Student assessment}
\end{comment}

\begin{comment}
\keywords{Graph File System}
\end{comment}

\makeatletter
\def\@copyrightspace{\relax}
\makeatother

\maketitle

\balance

% This file provides examples of some useful macros for typesetting
% dissertations.  None of the macros defined here are necessary beyond
% for the template documentation, so feel free to change, remove, and add
% your own definitions.
%
% We recommend that you define macros to separate the semantics
% of the things you write from how they are presented.  For example,
% you'll see definitions below for a macro \file{}: by using
% \file{} consistently in the text, we can change how filenames
% are typeset simply by changing the definition of \file{} in
% this file.
% 
%% The following is a directive for TeXShop to indicate the main file
%%!TEX root = diss.tex

\newcommand{\NA}{\textsc{n/a}}	% for "not applicable"
\newcommand{\eg}{e.g.,\ }	% proper form of examples (\eg a, b, c)
\newcommand{\ie}{i.e.,\ }	% proper form for that is (\ie a, b, c)
\newcommand{\etal}{\emph{et al}}

% Some useful macros for typesetting terms.
\newcommand{\file}[1]{\texttt{#1}}
\newcommand{\class}[1]{\texttt{#1}}
\newcommand{\latexpackage}[1]{\href{http://www.ctan.org/macros/latex/contrib/#1}{\texttt{#1}}}
\newcommand{\latexmiscpackage}[1]{\href{http://www.ctan.org/macros/latex/contrib/misc/#1.sty}{\texttt{#1}}}
\newcommand{\env}[1]{\texttt{#1}}
\newcommand{\BibTeX}{Bib\TeX}

% Define a command \doi{} to typeset a digital object identifier (DOI).
% Note: if the following definition raise an error, then you likely
% have an ancient version of url.sty.  Either find a more recent version
% (3.1 or later work fine) and simply copy it into this directory,  or
% comment out the following two lines and uncomment the third.
\DeclareUrlCommand\DOI{}
\newcommand{\doi}[1]{\href{http://dx.doi.org/#1}{\DOI{doi:#1}}}
%\newcommand{\doi}[1]{\href{http://dx.doi.org/#1}{doi:#1}}

% Useful macro to reference an online document with a hyperlink
% as well with the URL explicitly listed in a footnote
% #1: the URL
% #2: the anchoring text
\newcommand{\webref}[2]{\href{#1}{#2}\footnote{\url{#1}}}

% epigraph is a nice environment for typesetting quotations
\makeatletter
\newenvironment{epigraph}{%
	\begin{flushright}
	\begin{minipage}{\columnwidth-0.75in}
	\begin{flushright}
	\@ifundefined{singlespacing}{}{\singlespacing}%
    }{
	\end{flushright}
	\end{minipage}
	\end{flushright}}
\makeatother

% \FIXME{} is a useful macro for noting things needing to be changed.
% The following definition will also output a warning to the console
\newcommand{\FIXME}[1]{\typeout{**FIXME** #1}\textbf{[FIXME: #1]}}

% END


\section*{Abstract}
Rumors of the demise of the hierarchical file system namespace have been
greatly exaggerated.
While there seems to be wide spread agreement that most users make no use
of the hierarchical name space, we continue to use it both as the underlying
storage structure and as the default user interface.
To compensate for this mismatch between the native organization and the
way users interact with their data,
we have produced myriad search tools atop the file system.
This approach, however, has some limitations.
In particular, we focus on the absence of generalized relationships
as first class entities.

The hierarchical namespace elevates the \emph{contains} relationship
above all else.
Applications elevate the \emph{was created by me} relationship.
These are only two particular relationships among many;
items accessed around the same time share a temporal relationship;
attachments to email messages share a relationship;
a collection of documents, email message, notes, etc. form another
relationship.
We claim that elevating arbitrary and generalized relationships as first class
file system elements provides a better user experience and leads to entirely
different implementation approach.



\begin{comment}
  Margo comments from our January 11 meeting.

  Introduction: we have hieararchy but we need graph storage.  We see this as far back as Multics, where they have hard links,
  and UNIX, where they add symlinks [though I don't think symlinks showed up in 1974, but they did eventually.]

  The goal was to mirror the physical world of how filing systems worked [cabinets, drawers, folders, documents.  But this
  ignores the fact that libraries add cross-references - a separate name space of indices that help us find things.  And
  how documents refer to other documents, creating a "web of references".  This is not a new observation and includes
  memex and intelligent document editing systems.]

  Look at storage versus namespace: how tightly coupled are they?
  Paradigm shift: the web started with a more hierarchical model but moved to a graph model.  It also introduced
  the search based metaphor. As the web has grown, this perspective has permeated all of data storage, as we
  can see by the rampant popularity of key-value stores, object caches, etc.

  Namespace

  * Ignored by most users
  * Evolved from the web perspective
  * Those of us in "the field" still fall back to brute force search with find and grep -R

  File systems search and web search have merged into things like Dropbox, Google Drive, One Drive, and Git.

  "One author spent 20 minutes trying to find a document trying to find data from the previous work.  Had to
  search computer, Google Drive, dropbox, various repositories, etc."  Ultimately found that the data wasn't
  in any of those places and managed to get it from a co-author.

  Question: is this a systems problem or an HCI problem?

  Systems problem: data objects related via an extensible set of complex relationships that the system does
  nothing to facilitate.

  This suggests a graph structure.  Our current data storage and name space layers do not support this well.

  The HCI aspect of this is how to exploit a richer structure, so now we have a chicken and egg problem. Thus,
  we ask the question: "how should we rethink the file system to permit capturing a richer set of relationships?

  Alternatives become Background
    * Semantic file systems
    * Tagging (old, manual)
    * Tagging (new, automated/AI driven)
    * Databases

  We argue that it requires a custom, specialized database for our domain problem (e.g., relationship enhanced
  file systems).

  Limit BG to one page.

  ** maybe put Alternatives/Background AFTER our proposal ***

  Our proposal should be ~ 2 pages.

    * Envision your file system as a social graph for this discussion - a "social network file system"
    * Characteristics of data (what something "is") - one kind of relationship.
    * Social network: has people, groups, organizations)
    * File system: blob store, chunks of data
    * Shared "attributes" are one form of relationship (mapping social network to file system)
    
    What does the facebook architecture look like?  A SQL database (MySQL) with a cache (Tau)

    Why not use this as our architecture?  The analogy is not perfect.  Objects have a fixed schema in
    the database world but files do not.  Files vary in all sorts of ways and typically have an internal
    (private) schema.  They have a potentially broader range of attributes, though often we embed these
    in the name now (extension).  File systems typically store sizes, dates, and lengths.  Thus, we argue
    that we already have a partially mapped schema, but it is a rigid one.

    Proposed architecture:
      * storage - this is SOLVED for our purposes, regardless of whether the storage layer is a traditional
        HFS, a blob store, a document database, local or remote to the device.

      * relationships -
        - how do we store them?
        - how do we represent them?
        - how do we process them?

        PRESENTATION of these to users is an HCI problem.
        IMPLEMENTATION of these features is a systems problem.

    Q: why not a graph database?  Observation: in fact, we're a user of the graph database, but we aren't
       a graph database for the same reason a file system isn't a relational database: our needs are more
       simpler, so we can focus on optimizing performance for our use cases.

       Our blobs (files) are potentially quite large.
       Our relationships are ...

  Need to include the Burrito (Guo) reference.  Margo thought he had a more recent paper as well in the area,
  I need to look.

  Note: provenance is one type of relationship.

  We need a table of relationships.  This is demonstrative, not definitive.

  How do we achieve metadata efficiency?  Clustering?  Storage locality and clustering?

  Review: Plan 9, which had per-user and per-process namespaces.

  We must be able to uniquely identify a file [need to define uniquely identify].

  A "re-find" with a nice API

  A cache of recent files.

  Names may be ephemeral, since users cannot remember that much.

  Build our model from here.

  We could add a container on top that elevates this relationship above all others - the hierarchical model.

  Concrete proposal for graph FS.

\end{comment}

%% The following is a directive for TeXShop to indicate the main file
%%!TEX root = ../diss.tex

\chapter{Introduction}
\label{ch:introduction}

\section{Thesis Statement}
\label{ch:introduction:sec:thesis-statement}

%To improve the usefulness of naming systems within computer storage systems
%software must evolve to provide flexible, scalable, and multi-silo management
%of data object naming.

\input{thesis.tex}

I borrow the term \emph{pragmatics} from linguistics because this thesis is
about \emph{naming context}.  In linguistics, \emph{pragmatics} is distinct from
\emph{semantics}.  Semantics relates to \emph{meaning}.  However, meaning is
often dependent upon the \emph{context} in which a name is being used.  For
example, if I use the term \emph{my brother} then you understand that it refers
to a different person than when \emph{you} say it --- unless we are also
siblings, of course.  Prior work has studied file naming from the context of
\emph{semantics}, looking for the absolute meaning of a given item.  Naming,
however, often does not explicitly include the context necessary to understand
the meaning of a given name.

A simple example that Norm Hutchinson gave me during one of our early
discussions: over the years, he has created a variety of python files called
\texttt{test.py}.  If he wants to find a particular instance of one of those
python files he cannot tell what it is based solely upon the name. What he does
know is that if he uses the Linux \texttt{find} utility to find all of the files
called \texttt{test.py} he may be able to discern the purpose of that file based
upon other hints provided within the path to each instance of \texttt{test.py}
that he finds.  Thus, my observation: our existing system requires humans embed
naming context within the path and file name.  As I explain later, this is
insufficient for our needs.

I have done similar searches myself in the past.  I try hard not to resort to
such searching because it ends up either finding \emph{nothing}, which
frustrates me because I know that what I'm looking for exists, or it finds so
many paths listed that the task of looking through the list is itself daunting
--- a bit like asking Google ``What is the meaning of life?''  Google takes 0.52
seconds to advise me that there are ``about 1,100,000,000'' results.

The earlier work on \emph{semantic} file systems actually described techniques
for winnowing this type of ``virtual directory'' down to something more
manageable using exclusions --- reminiscent to modern day web searching where I
might omit documents with particular terms.

To the best of my knowledge, there is no prior work on \emph{pragmatic} file
systems.  Despite this, the importance of \emph{context} when interpreting names
is hardly a new idea - Saltzer discusses it as part of his treatise on object
naming~\cite{Saltzer1978}, though context is often interpreted quite broadly.

In addition, I am focused not on the organization of data within a single
storage device, or even within a single computer, but rather the much broader
definition proposed by Watson for a name: ``[t]he \emph{name} of a resource
indicates what we seek...''~\cite{watson1981identifiers}.

In \emph{semantic} file systems Gifford's focus was on improving understanding
of what the file is based upon its \emph{contents}.  In \emph{pragmatic} file
systems I expand on this by improving understanding of what the file is based
upon the context in which it is used. That context is based upon observations
\emph{outside} the file.

\section{Motivation}
\label{ch:introduction:sec:motivation}

\begin{epigraph}
    \textit{For every particular thing to have a name is impossible. --- First,
        it is beyond the power of human capacity to frame and retain distinct
        ideas of all the particular things we meet with: every bird and beast men
        saw; every tree and plant that affected the senses, could not find a place
        in the most capacious understanding.} --- \textbf{John Locke}, \textit{An Essay
        Concerning Human Understanding}~\cite{locke1844locke}
\end{epigraph}

\reto{one could say: humans are not good at remembering numbers. Thus we have a telefone book, and lookup the number by name of the person.
similarly: we don't remember the inode / IP, but its filename or dns.
This requires the process of resolution
(name, context) => obj/addr/...  [see Saltzer]
At its core, this provides a level of indirection, decoupling the name and the address (uuid, ...). Serving two convenient purposes:
- we can now use easily rememberable names, as opposed to inode numbers / phone numbers
- Always pointing to the right address. e.g., you go to www.example.com and you will get directed to the service that is least loaded, or you dial "John Doe (home)" which dials the right phone number  (i.e., the object can move but is still accessible)
- we can have multiple names for the same object. The phone number of Joseph and
Joe are the same.
}

\emph{Naming} is a long-standing challenge in computer systems.  In the late
1970s much of this work was summarized effectively by Saltzer in \emph{Naming
    and Binding of Objects}~\cite{Saltzer1978} and expounded upon by Watson
in the 1980s for distributed systems~\cite{watson1981identifiers}.  Much of the
emphasis of this
early work was on the naming required \emph{by} the computing system, though the
need for a human-accessible naming system is also clearly pointed out yet mostly
left for research.  Watson made a useful distinction that I have adopted:

\vspace{0.5cm}

\begin{tabular}{p{1.8cm}p{2cm}l}
    \multicolumn{3}{l}{``The \emph{name} of a resource indicates what we
    seek,''}                                                                                                         \\
     & \multicolumn{2}{l}{an \emph{address} indicates where it is, and}                                              \\
     &                                                                  & a \emph{route} tell us how to get there.''
\end{tabular}

\vspace{0.5cm}

\emph{Naming} is an essential human task.  We use names to
describe identity, relationship, and property.\reto{this one gets thrown in here without clearly saying what those are.
    in my mental model, a name is only valid within a specified context.
}

Thus naming, whether in terms of computer systems or human usage, is quite
broad. Naming can be quite concrete --- such as when we
use a name as a means to establish \emph{identity}.  However, naming can be quite
abstract, incorporating the concept of \emph{relationship}.  Often, naming is
heavily dependent upon the \emph{context} of the name.  When I speak of ``my
brother'' we understand that it is relative to \emph{me} --- the speaker.  This
context is essential to understanding how to interpret the name ``brother''
correctly.

Much of computer naming was initially motivated by the desire to reuse code.
Thus, early discussions about naming relate to how the pieces of a program are
bound together to perform tasks.  This is clearly an important contribution: the
ability to re-use code has enabled the rapid growth in computer use over the
past 60 years. The need for human-usable naming was clearly established early,
as well as the distinct nature of such naming versus the naming required by the
computer system itself.  However, much of the early focus was on ensuring the
needs of the computer system were met.  Thus, Watson's distinction between
\emph{naming} and \emph{addressing} is useful because it separates \emph{what}
something is from \emph{where} it is stored.  Watson identified a number of
goals for naming systems, of which I pick those that are most relevant to my
thesis

\begin{itemize}
    \item ``Support at least two levels of identifiers, \emph{one convenient for
              people} and one convenient for machines.'' Much of the work we have done in
          computer storage focuses on the needs of the computers themselves, not on
          people.

    \item ``Provide a system viewed as a global space of identified objects
          rather than one viewed as a space of identified host computers containing
          locally identified objects.'' Even today, computer storage systems fall down
          in this area. Using ``cloud storage'' such as Dropbox or Google Drive, the
          data can be searched within the confines of that silo. This is the modern
          equivalent of the ``identified host computers'', which in this case I call
          storage ``silos''.

    \item ``Support relocation of objects.''  The point here is that the storage
          location of something really doesn't relate to its name: the distinction
          between \emph{name} and \emph{address}.  If I move to a new house, I do not
          change my own name.

    \item ``Support use of multiple copies of the same object.''  Replication,
          at its finest, yet the ability to know when something is a \emph{copy} of
          something else can be useful.

    \item ``Allow multiple local user defined identifiers for the same object.''
          This idea is an important one to consider: when I file my bank statements I
          end up making copies of them, one I place by the month in which the
          statement was sent, the other in an area for statements from that particular
          bank. This type of multiple-use case is common and the existing model for
          hard or soft links in file systems fail to provide sufficient flexibility
          and ease-of-use.

    \item ``Support two or more resources sharing or including as components
          single instances of other objects...'' \tm{This is the idea of constructing
              compounds I suppose.  I'm not convinced on this point yet.}

\end{itemize}

\reto{Ok, this is interesting: there are now multiple ways to lookup objects:
    by "name" (e.g., path / uri)
    by content (e.g., hash / grep)
    by property (e.g., find -d...)
    those queries are already supported in some form or another.
    you can't really say
    "find me something related to this"
}

In 1990 I attended a talk given at Transarc by Dave Gifford about his novel idea of adding
\emph{semantic} capabilities to file systems.  Dr. Gifford's observation was
that human naming did not fit especially well in the context of hierarchical
organization structures.  Instead, he posited the use of \emph{semantic}
information, generated by external agents (``transducers'') to create
associative access.  At the time, I was deeply skeptical of what he was trying
to accomplish yet over time I have come to appreciate the fundamental problem,
which has only grown much more significant as the sheer amount of unstructured
data that we generate continues to grow exponentially.

``Associative access is designed to make it easier for users
to share information by helping them discover and locate
programs, documents, and other relevant objects. For example,
files can be located based upon transducer generated
attributes such as author, exported or imported procedures,
words contained, type, and title.''~\cite{gifford1991semantic}

Gifford's work has inspired quite a few works regarding variations of the
semantic file system, but none of these ideas has been sufficiently persuasive.
File systems have not evolved to improve the human-centric naming network that
they provide.  The term ``naming network''

In some way, the indexing systems offered on mainstream operating systems (e.g.,
Spotlight on MacOS and Search on Windows, or Cerebro on Linux) attempts to
achieve the goals of the semantic file system on top of the existing
hierarchical file system.  While it offers some improvements that make it
``easier for users to share information by helping them discover and locate
programs, documents, and other relevant objects'' it only does so for a limited
subset of the documents that are present on a user's computer.  If a user also
exploits modern ``cloud storage'' they find that their local indexing solution
does little to solve the real-world problem, in which data is stored and shared
across a collection of distinct storage ``silos'', each of which might offer
some search functionality but none of which provide the \emph{user} focused
naming perspective that is so important for human users.

Certainly part of the problem is that humans typically do not think in the
structured way.  The Human-Computer Interface community was certainly observing
how hierarchical folder structured, which is still the most common structure
file file systems, does not work well for all people.  One of the key
determinants they identified was ``spatial ability.''  This is one reason why
the problem is not obvious to those in Computer Science: prior work has
established that spacial ability and STEM performance are strongly
correlated~\cite{doi:10.1177/0016986217722614}.  Thus, the very people that
construct storage systems will not face the same challenge of using the existing
model.

However, even those with excellent spacial abilities generally seem to find that
the ability to \emph{find what they seek} is increasingly complicated by the use
of disjoint storage mechanisms.  More than once, while conversing with Margo
Seltzer about this research area she would reach out to find a specific file,
only to be forced to perform \emph{ad hoc} searches into various storage
locations: her desktop computer, Google Drive, or e-mail.  Sometimes those
searches were successful, sometimes they were not.

Humans do not naturally conform their thinking to the rigid hierarchical
structure of the initial namespaces created for them by the computer storage
community.  The computer storage community never intended that hierarchical
structure be the only organizational structure created for human-focused file
naming. Saltzer viewed the ``naming network'' to burden human users when he
noted ``One such scheme is to provide that each name that is not to be resolved
in the working catalog carry with it the name of the context in which it should
be resolved.  This approach forces back onto the user the responsibility to
state explicitly, as part of each name, the name of the appropriate context.''
Indeed, Saltzer points out that ``[a] naming network admits any arbitrary
arrangement of catalogs, including what is sometimes called a
\underline{recursive structure}.'' My observation from reading this is that he
saw namespaces that were graphs and the choice of limiting to minimally
connected graphs.

In 1945 Vannevar Bush described the challenges to humans of finding things in a
codified system of records, including those used by early computers:

\begin{quotation}
    Our ineptitude in getting at the record is largely
    caused by the artificiality of systems of indexing. When data of any
    sort are placed in storage, they are filed alphabetically or numerically, and
    information is found (when it is) by tracing it down from subclass to
    subclass. It can be in only one place, unless duplicates are used; one
    has to have rules as to which path will locate it, and the rules are
    cumbersome. Having found one item, moreover, one has to emerge from the
    system and re-enter on a new path.

    The human mind does not work that way. It operates by association. With one
    item in its grasp, it snaps instantly to the next that is suggested by the
    association of thoughts, in accordance with some intricate web of trails
    carried by the cells of the brain. It has other characteristics, of course;
    trails that are not frequently followed are prone to fade, items are not
    fully permanent, memory is transitory. Yet the speed of action, the
    intricacy of trails, the detail of mental pictures, is awe-inspiring beyond
    all else in nature.
\end{quotation}

This is as true in 2021 as it was in 1945.  While preparing this proposal I
spent time looking at the guides many libraries provided about the naming of
files. I found a body of recommendations about file naming
standards\footnote{Data Management for
    Researchers~\cite{briney2015data}}
\footnote{Smithsonian: \url{https://library.si.edu/sites/default/files/tutorial/pdf/filenamingorganizing20180227.pdf}}
\footnote{Stanford: \url{https://library.stanford.edu/research/data-management-services/data-best-practices/best-practices-file-naming}}
\footnote{NIST: \url{https://www.nist.gov/system/files/documents/pml/wmd/labmetrology/ElectronicFileOrganizationTips-2016-03.pdf}}.
Harvard Data Management suggests~\footnote{\url{https://datamanagement.hms.harvard.edu/collect/file-naming-conventions}}:

\begin{itemize}
    \item Think about your files
    \item Identify metadata (e.g., date, sample, experiment)
    \item Abbreviate or encode metadata
    \item Use versioning
    \item Think about how you will search for your files
    \item Deliberately separate metadata elements
    \item Write down your naming conventions
\end{itemize}

\tm{Note that the footnotes aren't rendering cleanly.  I will need to clean this up.}

Throughout these examples there are common themes: a name provides context for
\emph{what} the given object represents. Uniformity of information is also
important --- the ``naming convention'' permits not only identifying similarity
but key elements of \emph{difference} between any two named things.

\tm{It seems that this Harvard list is a serious indictment of the existing
    system: it pushes the cognitive load onto the users, talks about meta-data ,
    versioning, encoding, \emph{and} capturing the naming convention.}


Recall Saltzer's comment: ``This approach forces back onto the user the
responsibility to state explicitly, as part of each name, the name of the
appropriate context.''  What we see reflected in these ``naming conventions'' is
the result of how naming is limited in computer storage.

Forcing human users to ensure they provide the ``appropriate context'' in the
naming system was an important limitation.  Note that Saltzer says that this is
\emph{one} such scheme.  He did not argue that it was the only scheme, nor even
the best scheme.  Instead, it was the scheme ``good enough'' at the time.  He
points out that additional research needs to be done to find better schemes.

The \emph{purpose} of the file system was to serve as the provider of
``human-oriented names''~\cite[Table III]{Saltzer1978}. Mogul observed that
``Better file systems allow us to manage our files more effectively, solve
problems that cannot now be efficiently solved, and build better software.''
Gifford observed: ``a semantic file system can provide associative
attribute-based access to the contents of an information storage system with the
help of file type specific transducers... The results to date are consistent
with our thesis that semantic file systems present a more effectrive storage
abstraction than do traditional tree structured file systems for information
sharing...''

Saltzer challenged us with general naming but set it aside for future research.
Mogul captured the idea that \emph{properties} could be used to store additional
meta-data about files.  Gifford explored the idea that information about what a
file represents could be extracted and used to dynamically organize files based
upon semantic content of the files themselves.

Thus the point of my thesis: adding the \emph{activity context} to the
body of information that we use for organizing files will yield more effective
management, which in turn will improve human productivity and collaboration.
While I discuss more carefully defining ``activity context'' later in this
document, for purposes of this discussion it is sufficient to consider it to be
information about what else was going on at the time a file was being created or
used: the programs running, other files that were being created or consumed, the
location of the computer at the time, etc.

\endinput

%% This is older text.  There is still considerable useful information here, but
%% I am thinking it needs to go elsewhere.

\endinput


One of the most important abilities of modern computers is their ability to
augment human memory.  One of the greatest challenges in using modern computers
to augment human memory is the propensity to save essentially everything.
Certainly there is an aspect of the ``fear of missing out'' if material is not
saved, but there is also the reality that technology makes the cost of saving
information very close to zero.  This is quite different than written documents:
books, letters, mementos, records, compact discs, and eight track tapes all
require physical storage.  Further, they tend to degrade over time and become
lost. Humans have fought against this natural tendency to forget in a variety of
ways: iambic pentameter to record the heroic efforts of greek heroes, monks
toiling away to create duplicate copies of important texts, the creation of
ink rubbings in Asia, turning to movable type and then automation into the
printing press.  Modern computer systems provide rapid local storage, combined
with automatic replication of information in a resilient way, including
geo-replication and even recording in highly durable media such as quartz glass.

% It is interesting to note that printing was an Eastern invention - China 206
% BC to 220 AD Han dynasty.  It was Western augmentation that moved to using
% metal and then mechanization.

As our augmented memory has grown, the challenge of being able to locate
specific information within that memory has similarly grown.  Small collections
gave way to great libraries, which required robust indexing systems to permit
finding useful information.  The relatively recent invention of the Internet has
also brought with it a new indexing model of this information.  Now, we
routinely use the verb "to google" to mean to search for information on the
Internet.

Searching for information on the internet is related to a specific problem:
finding \emph{an} answer to a particular search query or question.  For many
problems this is sufficient, but for those seeing specific information, such as
a human user attempting to find a document she created six months prior, the
search engine falls woefully short.  In this case it is \emph{the} answer that
is being sought.

Storage systems mostly organize data in a hierarchical fashion that dates from
the 1950s and became part of the modern computer operating systems model in the
1960s as seen in the publications about Multics from 1965 \tm{It might be time
    to start citing things here - this is the Daley I/O paper, not the more famous
    overview paper.} Data sharing over a network --- the ``network file system'' or
``distributed file system'' began appearing in the 1970s.  Total data storage
capacity continues to expand rapidly.  In 2011 the estimated data
storage capacity in the \emph{world} was
295EB\footnote{\url{https://www.zdnet.com/article/what-is-the-worlds-data-storage-capacity/}}.
At the end of 2020, Forbes estimated that 1ZB of
hard disk drive storage alone was shipped in
2020\footnote{\url{https://www.forbes.com/sites/tomcoughlin/2020/12/18/digital-storage-projections-for-2021-part-1/}}.
New storage technologies continue to emerge and promise greater amounts of
information storage:  DNA data storage
technology\footnote{\url{https://www.scientificamerican.com/article/dna-data-storage-is-closer-than-you-think/}},
carbon nanotube persistent
memory\footnote{\url{http://people.eecs.berkeley.edu/~kubitron/cs262/handouts/papers/Gervasi-Nantero.pdf}},
and
glass\footnote{\url{https://searchstorage.techtarget.com/feature/An-overview-of-Microsoft-Project-Silica-and-its-archive-use}}.
Thus, it is unlikely the trend towards storing ever greater amounts of
information will diminish in the forseeable future.

Storage systems are challenged by the need to meet a broad range of needs. Large
data sets typically involve using highly specialized storage systems.
According to CERN, the experiments they conduct include 90PB of data per year
from the Large Hadron Collector (LHC) and an additional 25PB of data from other
non-LHC experiments. Similarly, genomics research routinely must handle numerous
multi-gigabyte files as part of typical genomic
research~\cite{todesco2020massive}. Intel's DAOS architecture is specifically
designed for these types of large, complex storage
situations~\footnote{\url{https://daos-stack.github.io/}}.

Large datasets are often used collaboratively: CERN projects often involved
dozens of researchers from all over the world; NASA data that was ``big'' when
collected in previous decades is still used today to find novel astronomical
activity, plus the collection of much larger modern data sets, with a need
to preserve that data so it can be used for future
research~\footnote{\url{https://bigdata-madesimple.com/how-does-nasa-use-big-data/}}.

Storage systems, however, are organized as distinct \textit{silos}.  This
organizational scheme makes sense to the storage developer: Google Drive and
Microsoft OneDrive are distinct ``storage silos'' because they are developed by
distinct entities.  This, however, does not reflect the real way in which those
using the data \emph{use} these storage silos.  Often, the choice of storage
silos is driven by external requirements: large data sets are stored in storage
silos optimized for those data sets, for example. Papers written by researchers
using those data sets would \textit{not} be stored in the same storage silo.
This raises one of the fundamental questions motivating this thesis proposal:
how do we associate related data stored in unrelated storage silos?

One challenge here is that the commonly used model for ``finding a specific
file'' is based upon a name. Names, in turn, often embed \emph{location}
and \emph{semantic} information.  Thus, the data notebook that someone builds to
extract information may end up stored inside the same silo as the data itself
because that is required to maintain the locality of the analysis to the data
itself.  When the results of tha analysis are written up, it may be stored in an
online collaborative system, such as
Overleaf~\footnote{\url{https://overleaf.com}}, because it is geared towards
collaborate creation of work intended for publication.

This problem is compounded by the passage of time.  Governance requests for
information, such as under the General Data Protection
Regulation~\footnote{\url{https://gdpr-info.eu/}}, often do not appear until
months or years after the original information was created.  Information spread
across multiple storage systems presents a challenge in locating required data
and may involve considerable ``brute force'' analysis to find relevant
information.  This in turn can amplify costs associated with using some storage
silos that charge based upon data access.  Archivists seeking to understand data
analysis done years or decades earlier similarly must struggle to find the
pieces tying disparate work together.

In this thesis I propose separating storage \emph{location} from \emph{naming},
with an explicit goal of using naming in a human-usage focused model rather than
a storage optimized model.  I explore the specifics of both the human-focused
naming model as well as the storage focused naming model more thoroughly in
\autoref{ch:model} as part of formulating my own proposed model.

A clear separation of location and name mitigates the conflation of location and
naming in current storage systems; supporting a richer human-usage focused naming model
yields a more effective human-usable system; and separating meta-data from
storage permits greater flexibility in identifying dynamic context-sensitive
association between data objects:
paper, notebook, spreadsheet, e-mail, object, or file.

\tm{Some of this text should move to \autoref{ch:model}, since
    it explains the rationale for the model.}

\endinput

Any text after an \endinput is ignored.



\section{A Modest Proposal}\label{sec:graphfs}

% Because our focus is on the \textit{naming} system and not the storage system,
% for the present time we will not consider the issues that will arise in the
% storage management layer to support the new model, though we admit that there 
% are likely to be concerns that will need to be addressed in future work.

Our proposed file system focuses on \textit{relationships} between our files.
We use an analogy between social graphs and file systems to explore this 
approach.
Facebook's graph is a collection of typed objects
(e.g., users, actions, places) and associations (e.g., friend, authored, tagged).
File system objects map to users and files; \textit{contains} is, perhaps, the only
association captured in a file system.
For the rest of this discussion, we treat directories as the embodiment of the
\emph{contains} relationship, not objects.

We consider two strawman implementations for elevating relationships to
first class file system objects.

\subsection{File System as a Graph Database}

In considering a graph based file system, first we consider
implementing a file system in a graph database, of which there are
many (\S \ref{sec:background}). Their primary focus is the storage of 
graph-structured \textit{data}.  Our focus is in the use of graph-structured
data as critical meta-data inside of a storage system.
As such, there is a mismatch in design targets between a file system and
existing graph databases: nodes in graph databases are small; nodes in
file systems are large. Graph databases tend to favor a navigation-based API;
file systems need a point query and search API. Graph databases assume that
attributes and relationships are provided; file systems will frequently derive
attributes and relationships.
These differences suggest to us that existing graph databases are not suitable
as the basis for file systems.
Nonetheless, we encourage others to consider such an arrangement, should they
have compelling reason to do so.

\subsection{File System as Social Network}

Next, we consider implementing a file system in the same way Facebook
implements their social network graph.
Facebook's original implementation stored the social graph in MySQL, queried
it from PHP, and cached the results in memcache.
More recently, Facebook introduced Tao, which is a service that more directly
implements the fundamental objects and relationships that comprise the
social graph~\cite{bronson2013tao}.
While Tao is specifically designed to support the widely distributed,
replicated, and rapidly changing social network scenario, it provides the
starting point for conceptualizing a data model premised on the primality of
relationships.
Tao stores both objects and associations in a MySQL database and presents
the graph abstraction via Association and Query APIs in the caching layer.
Is this a viable structure for a file system?

% in fact, facebook has a separate storage system for videos and pictures 
% because they are large.
Unlike objects in Facebook, files are large.
Although prior work has considered using relational
database~\cite{olson1993design} and other index-based
structures~\cite{spillane2013vttree}  to store files,
the community seems to have
concluded that such storage is not ideal. We agree, suggesting that
an RDBMS is not the desired storage system.

What about relationships? Is it appropriate
to use one persistent representation (e.g., a relational one) and a second
memory representation (e.g., a graph-structured on) or
should we use a single graph-structured representation both in persistent store
and in-memory.
We propose the latter for two reasons.
First, the rumored era of non-volatile main memory seems to be around the
corner, so a modern file system design should embrace a single
representation.
Second, while it is reasonable for Facebook to construct the entire graph in
a distributed pool of main memory, file systems must work on a more limited
scale and therefore cannot ensure that the realized graph structure will fit
in main memory.

As neither strawman design seems suitable for our relationship-centric file
system, we present a new model and file system design.

\subsection{Graph FS Model}
\label{sec:graphfs:model}

We set out a basic description of our core objects in Table \ref{table:graphfs:terminology}
%\footnote{We took inspiration for this model from https://github.com/opencypher.}
and a demonstrative set of example relationships in Table \ref{table:relationship-examples}.
We do not consider either of these to be exhaustive, but rather propose them as an initial
basis for discussion.
The presented model can encompass
the functionality of the existing hierarchical file system model.

% https://github.com/opencypher/openCypher/blob/master/docs/property-graph-model.adoc

\begin{table}[h]
    \captionsetup{justification=centering}
    \begin{tabular}{p{2cm}p{5cm}}
        Term                          & Definition\tabularnewline\hline
        \multirow{1}{*}{file}         &
        \multirow{1}{*}{\parbox{4.8cm}{Uniquely identified storage unit}}
        \tabularnewline
        \multirow{1}{*}{relationship} &
        \multirow{1}{*}{\parbox{4.8cm}{Directional file association}}
        \tabularnewline
        \multirow{1}{*}{labels}       &
        \multirow{1}{*}{\parbox{4.8cm}{A binary attribute, e.g., executable}}
        \tabularnewline
        \multirow{1}{*}{property}     &
        \multirow{1}{*}{\parbox{4.8cm}{Key/Value attribute}}
        \tabularnewline
    \end{tabular}
    \caption{Graph File Systems Terminology}\label{hotos19:table:graphfs:terminology}
    %    \Description{Graph File Systems Terminology}
\end{table}



\begin{comment}
% Omit because I don't think it is central to our thesis.
Our relationship-centric file system consists of a \textit{union} of distinct
name spaces, which permits a name space that is not
fully connected, yet can still be merged via the existing union model. This permits an
application to have a private name space for files,
similar to both CAP and Plan 9~\cite{needham1977cap,pike1992use}.  Isolation also
enhances security, e.g., the POSIX \textbf{mkstemp} function, which was introduced to
address the need of applications for temporary files not visible to other applications.
Such namespaces are like unlinked but open files in UNIX, where such files are
reclaimed when the process terminates.  By creating a process private ephemeral namespace
we augment the file system's ability to provide enhanced isolation.
\end{comment}

Every file has a unique identifier, such as a \textbf{UUID}, similar to
an inode number or object ID.
We do not rely upon \textit{names}
as they are simply mutable properties.

\begin{table}[h]
    \begin{tabular}{p{1.9cm}p{5.5cm}}
        Relationship                           & Description\tabularnewline
        \hline
        %        \multirow{1}{*}{\textit{is}} &
        %        \multirow{1}{*}{\parbox{5.4cm}{Attribute of a file, e.g. size or timestamp}}
        %        \tabularnewline
        \multirow{1}{*}{\textit{similar}}      &
        \multirow{1}{*}{\parbox{5.4cm}{Similarity measure, e.g., \cite{masci2014multimodal}}}
        \tabularnewline
        \multirow{1}{*}{\textit{precedes}}     &
        \multirow{1}{*}{\parbox{5.4cm}{temporal relationship (e.g., versioning)}}
        \tabularnewline
        \multirow{1}{*}{\textit{succeeds}}     &
        \multirow{1}{*}{\parbox{5.4cm}{temporal relationship (e.g., versioning)}}
        %        \tabularnewline
        %        \multirow{1}{*}{\textit{located}} &
        %        \multirow{1}{*}{\parbox{5.4cm}{link or url}}
        \tabularnewline
        \multirow{1}{*}{\textit{contains}}     &
        \multirow{1}{*}{\parbox{5.4cm}{directory/file relationship}}
        \tabularnewline
        \multirow{1}{*}{\textit{contained by}} &
        \multirow{1}{*}{\parbox{5.4cm}{directory/file relationship}}
        \tabularnewline
        \multirow{1}{*}{\textit{derived from}} &
        \multirow{1}{*}{\parbox{5.4cm}{provenance (e.g., .o to .c)}}
        \tabularnewline
    \end{tabular}
    \caption{Graph File System Relationship Examples}\label{hotos19:table:relationship-examples}
    %    \Description{Graph File System Relationship Examples}
\end{table}



A \textit{relationship} is a directional association between two files.  We expect there
to be far fewer relationships than files, though many more \textit{instances} of
relationships (i.e., the number of edges in our graph exceeds the number of vertices).
Relationships may be either uni-directional (e.g., derived from) or
bi-directional (e.g., similar).
Table \ref{table:relationship-examples}
provides a set of sample relationships; the universe of relationships
is extensible.
As in RDF, relationships are triples: two files and the relationship.

As files have attributes in a conventional file system, both files and
relationships have attributes in a graph file system.
A \textit{label} is a simple binary attribute (e.g., executable),
while a \textit{property} is an arbitrary name/value pair, much like
an extended attribute, but they are native to the model, not
an afterthought.

% We will extend our terminology as needed, using the existing terminology of the
% relationship graph as inspiration for usable models.

\subsection{Interface}\label{sec:graphfs:interface}

\begin{table}[b]
    \small
    \captionsetup{justification=centering}
    \begin{tabular}{p{2cm}p{5cm}}
        Operation & Description\tabularnewline\hline
        \multirow{1}{*}{create} &
        \multirow{1}{*}{\parbox{4.8cm}{Insert new file into graph}}
        \tabularnewline
        \multirow{1}{*}{relate} &
        \multirow{1}{*}{\parbox{4.8cm}{Insert new edge into graph}}
        \tabularnewline
        \multirow{1}{*}{label} &
        \multirow{1}{*}{\parbox{4.8cm}{Insert new labels}}
        \tabularnewline
        \multirow{1}{*}{set} &
        \multirow{1}{*}{\parbox{4.8cm}{Insert new properties}}
        \tabularnewline
        \multirow{1}{*}{remove} &
        \multirow{1}{*}{\parbox{4.8cm}{Remove something from graph}}
        \tabularnewline
    \end{tabular}
    \caption{Graph File Systems Operations}\label{table:graphfs:operations}
%    \Description{Graph File Systems Operation Examples}
\end{table}


One of the lessons from Plan 9 is that everything can be represented as a file~\cite{pike1992use};
we expect to
continue with this paradigm as it has served us well over the years.  While we generally
think of files as a blob of \textit{persistent} data, in fact it is useful to
think of them as abstract \textit{generators} of byte stream data.  This fits well
with our model of separating namespace from storage; how the storage
providers return data to us is orthogonal to the namespace we use to retrieve it.
For example, the \textit{procfs} file system creates a synthetic namespace and supports
I/O operations for reading and modifying data contents of the pseudo-files.

From the namespace perspective, our file system must support operations that manipulate that
namespace. This includes the ability to create files, relationships,
relationship instances \textit{between} files, labels, and properties.
Similarly, we need the ability to remove each of these.

Our model is simple, yet powerful.  It captures interesting concepts such as versioning, using relationships such as
\textit{precedes} and \textit{succeeds},
and provenance, using relationships around derivation and use,
and application specific relationships, such as \textit{indexes} so a database
system can expose the relationship between its primary data and the
corresponding index files.
Although relationships are binary,
we can create clusters of related files by asking for all the vertices connected
by a specific relationship.

Where do relationships, labels, and properties come from?
We identify at least the following five sources:
1) the system itself will generate traditional
attributes (e.g., \textit{size}, \textit{read time}) and some
relationships (e.g., contains);
2) tools that extract meta-data from different file
types~\cite{soules2004toward,bloehdorn2006tagfs} will produce more attributes;
3) applications will generate both attributes and relationships;
4) users may generate attributes and relationships, although history
suggests that asking users to annotate data is a losing
proposition~\cite{soules2003can}; and
5) kernel extensions, e.g., provenance tracking systems~\cite{pasquier17camflow}
will generate attributes and relationships.

Several interesting possibilities emerge from this design.
Hard links are multiple \textit{name} properties attached to the
same file, potentially in different namespaces.
Soft links are a relationship between two names.
The system can capture relationships that extend beyond the file system.
For example, the \textit{derived from} relationship from
Table \ref{table:relationship-examples} might describe a file that came
from a particular email or web site.

\begin{figure}[bt]
    \captionsetup{justification=centering}
\includegraphics[width=0.9\linewidth]{figures/model-graph.eps}
\caption{Graph File System}\label{fig:graphfs-example}
%\Description{Simplistic Graph File System Picture}
\end{figure}

Figure \ref{fig:graphfs-example} provides a simplified visualization of our graph file
system model.
Our inclusion of disjoint graphs captures the notion that the system
naturally supports multiple namespaces, implemented as disconnected graph
components.

\begin{comment}

\subsection{Search Functionality}


Motivation for our graph file system was to provide a rich set of search semantics
to enable more effective human-usable search.  Search starts from an arbitrary 
file and radiates out from that point: what we seek is often ``close'' in terms of
our relationship graph. We can return results quickly without doing
an exhaustive search of the entire graph.  This is similar to how a modern internet search 
engine, such as Google, works. Only the first few hundred results returned from millions
of entries returned are usually analyzed.

Our file sytem can support an iterative interface in which it asynchronously constructs
the search results, returning initial results quickly even as further results are constructed.
Bounding search depth combined with resuming search as needed permits the iterative process
to continue without overwhelming the system. We seek to balance performance and completeness. 
Indeed, our expectation is that --- much like your search engine never returns the 7,345th
page of search results, the file system will seldom be asked to search the entire graph.  
This is a far cry from the brute force search that some of us now do when we use \textbf{find} 
or \textbf{grep -R} to scan through the hundreds of thousands of files that we have managed to
accumulate over years of work.

There is considerable work already done in the area of graph searching; we expect to utilize
that work to construct a viable search 
interface~\cite{angles2018g,,rudolf2013graph,francis2018cypher,van2016pgql}.  
This aspect of our graph file system is likely to evolve the most as we better understand the
needs of applications, particularly those seeking to expand the human level searchability, and
adapt the search interface to meet their needs.
\end{comment}

\begin{comment}

Current work in labeled property graphs has recently moved to introducing \textit{paths} to the model
as well as the labels and properties that have already been added to the basic graph model~\cite{angles2018g}.
This prior work identifies five key features that are heavily utilized in searching such graphs: 
\textit{reachability}, \textit{construction}, \textit{pattern matching}, \textit{shortest path search},
and \textit{graph clustering}.  For file systems, we expect that some of these will be of less
interest than others, but we think it premature to exclude any of these from a useful design, though we expect to 
focus on those we view as providing the greatest utility for our prototype implementation work. Thus, we expect
the greatest utility for file systems will be \textit{pattern matching} initially, though we find both
\textit{reachability} and \textit{shortest path} potentially intriguing.

Pattern matching makes the most sense in terms of searching for specific labels and properties of files --- this most closely
matches with the types of searching that indexers are already performing.  Thus, from the perspective of novel new
search models it is the least satisfying yet likely one that will still provide substantial utility.

Reachability is intriguing because it meshes well with our model for temporal relationships, in which we define ``reachability''
with respect temporal reachability.  POSIX co-location (e.g., part of the same directory) is another potential area, though
we suspect there are further areas to explore, including similarity.  For example, similarity preserving hashing~\cite{masci2014multimodal}
could be used to search for files that are similar, but not identical to one another.  This could be used to trace the evolution
of files as they are copied and modified from one location to another as well as to identify files that are identical.  Similarly,
finding the set of paths between two vertices could yield useful information about the linkage of files, something that we
cannot easily reconstruct from current file systems.

Shortest path is a classic search problem (e.g., Dijstra's Algorithm~\cite{dijkstra1959note}) as a means of finding the closest relationship
between two files.  This could be combined with reachability to identify a \textit{sequence} of iterative steps as we move between
similar files.  Another use would be to find a path between a directory and a file that provides access, e.g., for which the security label
on all the edges permits access.  This works well with the ``open by inode number' problem, for example and may help us address
the interesting question of how to perform POSIX-compatible security checks on files that are being opened by their identifier.
\end{comment}

\begin{comment}
\subsection{POSIX}

Our model permits support for POSIX file system semantics with respect to the file system
name space.  While our model does not require an hierarchical layout, we can support
such a system within this model.

First, we assume there is an external mechanism for identifying a distinguished file that
corresponds to the \textit{root} directory.  We added the \textit{contains} relationship
to capture this form of relationship.  While we suggested the \textit{contained by} relationship
(Table \ref{table:relationship-examples}), this is not a requirement for POSIX compatibility,
though one of the authors has been asked for this feature numerous times in the past for
existing POSIX file systems.

Enumerating the contents of one of these \textit{directory} type files would enable
enumeration of the directory in keeping with the POSIX model, by simply querying the
edges capturing the \textit{contains} relationship.  A hard link from two directories
to the same file is trivially supported.  Soft links are already a part of our model, so
we can provide symbolic links as well.

POXIS APIs that retrieve file properties, such as \textbf{stat}, are similarly converted
into a call to retrieve properties of the file and then converting them into the
expected native format.

While our own graph file system does not inherently place any importance on names, this
POSIX layer can store names as a property of the edge.  This is consistent with the way
that file names now function --- they are an attribute of the directory entry that
references the file, rather than an attribute of the file itself.
\end{comment}

\begin{comment}
\subsection{Interactions}

Eventually, we expect there will be challenges associated with the POSIX interface and the richer interface we offer.  We have already mentioned
one such issue: security.  The POSIX specification expects a security check at each node along the path as part of an attempt to open or create
a file.  Windows will enforce the same path walking security model but provides a work-around that allows skipping this~\cite{conover2006analysis}
check in favor of performance.  The ability to perform such a security check at open time could be an interesting systems use of enhanced
searchability.

One area in which we can envision challenges is in preserving the labels and properties of files within our graph because standard utilities
such as \textbf{cp} are ignorant of them and would simply copy the data contents of a file without copying the labels and properties.  We can
certainly enhance our system utilities to be enlightened about these new features, but it will not resolve the problem for cases in which
existing applications use standard POSIX systems calls to copy a file.

In addition, the ability to create disconnected name spaces does not fit well within the POSIX model, though it is more like distinct file systems and thus
likely does not break existing semantics.  Still, we expect that as we gain further experience with our graph file system and the graph
features are exploited we will need to handle more interaction cases.
\end{comment}

\begin{comment}

Our proposal is that file system name spaces evolve from \textit{trees} to \textit{graphs}.  Of course, a tree is simply a minimally connected graph.  This leads
to our first question for our new file system model: \textit{should it be a connected graph?}  In fact, we observe that even our personal name space is a set of
disjoint graphs --- there is one for each of our devices.  Thus, we observe that this question is already answered.  However, having answered it, we also
observe the utility of actually creating connections across those name spaces.

This becomes easier when we divorce the concept of \textit{storage} from the concept of \textit{name space}.  Our ideal name space can thus refer to files
that are not part of the current device's name space.  We are not (yet) proposing how we add that information to the namespace, merely noting that it
is a desired feature.

One argument for allowing disjoint namespaces on our individual devices is to provide a simple model for namespace isolation --- files that need not be
made visible outside some limited scope (e.g., the temporary working files of a particular application) can be in what is a localized name space; in some
ways this is reminiscent of capabilities based file systems~\cite{needham1977cap}.  While an interesting area to explore further, we leave it open
at the present time for our design.

Our graph is essentially a \textit{relationship} graph, in which vertices represent what we usually think of as \textit{files}, and edges represent
the \textit{relationships} between our vertices.  In an attempt to emphasize the generality of our vertices, we note they may be able to supply a byte
stream of data --- a \textit{generator}, and they may be able to consume a byte stream of data --- a \textit{processor}; we leave it open if we wish
to consider richer interfaces, such as \textit{insert}, such as we previously proposed~\cite{Seltzer2009}.  An individual vertex may be any subset
of generator and processor and thus have read-write, read-only, and write-only vertices as suits the needs of the specific vertex.

Vertices may have a set of \textit{attributes}.  Since our goal is to provide generality here, we would establish that each attribute consists of a
\textit{domain} which identifies how to interpret the attribute, the \textit{attribute-name}, which specifies the name of the specific attribute
within the given domain, and a \textit{value}.  Our system can then define a core set of attributes relevant to the file system domain; other domains
can then be added by agreement of the domain creators and users without us enforcing any given structure.  While individual applications can then
create labels within their own domain, our hope is that over time related applications will come to an agreement as to the meanings of attribute-names
to enable sharing of properties.

Our edges are labeled, weighted directional edges in a hypergraph: each label consists of a \textit{relationship} and a \textit{weight}.  Because our
goal is not to strictly restrict the relationships, we suggest that a relationship itself consists of a \textit{domain} identifying how to interpret
the relationship (presumably by agreement for those creating those relationships) and a domain-specific \textit{relationship-label}.  We can define
a set of relationships that we will initially support: \textit{contains}, \textit{is contained by}, \textit{is related to}, etc.

Thus, we now have enough structure for us to create a mapping from the existing POSIX file system interface, including its hierarchical name structure
to be compatible with existing systems. 
\end{comment}

\subsection{Aspects of Implementation}

In the absence of space to provide a full implementation, we offer a few
strategies that make a graph file system both feasible and novel.
The underlying storage structure for files is essentially an object store~\cite{factor2005object} and
attribute storage is largely a solved problem
(although the last time
one of the authors said that, her colleague disagreed~\cite{mao2012cache}), so we
focus on fast and efficient graph storage and query.

Today's systems either provide graph storage~\cite{rudolf2013graph,webber2018programmatic,microsoft:cosmosdb} 
or graph processing~\cite{shun2013ligra,gonzalez2014graphx,malewicz2010pregel,salihoglu2013gps,nguyen2013lightweight,low2014graphlab,kyrola2012graphchi}, 
but a graph file system needs a high performance, space-efficient, mutable and queryable
native graph representation.
We have found that mutable compressed sparse row representations~\cite{macko2015llama}
meet all these requirements (we used them as the query and storage mechanism in the
SHEEP graph partitioner~\cite{margo2015scalable}).
Just as high performance key/value stores are considered reasonable implementation
strategies for attribute storage and management in file systems, similarly efficient
structures supporting graph storage and management should be adopted in file systems
as well.


\section{Search}\label{sec:search}

The driving force behind our graph file system design is to provide the
infrastructure to make it easier for users to find data.
Users do not navigate to data, they \textit{search} for it, so we
consider more effective search models to further
motivate the graph file system.

We observe that there are two different models of ``search'': application
search and user search.
Applications need to be able to open files quickly 
using a \textit{key}. For example, both NFS~\cite{sandberg1986sun}
and AFS~\cite{sidebotham1986volumes} use the file system \textit{inode number} as their
mechanism for identifying the specific file or directory being accessed,
because it is fast, avoiding a costly namespace traversal.
Similarly, NTFS supports the ability of applications to open a file by 
identifier~\cite{sreenivas2011bypass}.
They did this to support their implementation
of the Apple File Protocol (``service for Macintosh'') but has subsequently been used
for other uses. Indeed, it has been further extended to permit files to be opened by an
application-defined identifier (a UUID); Microsoft continues to support file IDs in 
ReFS~\cite{microsoft:refs:features}. The Google File System~\cite{Ghemawat2003} 
observation was similar: applications can use keys to find their files.

Modern applications tend to either create files that they use internally, often going to great lengths
to hide their location from the user; or maintain a list of recently used items with a full path name,
which breaks when the path changes, even if the file did not change.  A key interface for applications
better fits this usage model. Thus, a ``search by key'' interface is sufficient.

The more challenging problem is user-focused search.
\begin{comment}
Many of the
characteristics in a good human usable search system do not benefit the programs
directly.
\end{comment}
For human users, we want to enable a model like the
\textit{memex}~\cite{bush1945we}: ``A memex is a device in which an individual stores
all his books, records, and communications, and which is mechanized so that it may be
consulted with exceeding speed and flexibility. It is an enlarged intimate supplement
to his memory.''

The HCI community has a long history researching
more effective search, including such efforts as
SIS~\cite{dumais2016stuff} and faceted search~\cite{arenas2016faceted,tunkelang2009faceted,hearst2006design,klungre2018evaluating,walton2017looking,cleverley2015retrieving}.
Critical to this work is the idea that search is most effective
when \textit{not} bound to a specific taxonomic order --- very much the opposite of today's 
hierarchical search model, which enforces a rigid
order on the structure of information.

How does a graph file system then enable modern search?
First, support for a broad and extensible set of attributes and 
relationships brings search engine technology to bear in the
service of file systems.
There is some irony that the success of web search, and in particular the
primacy of relationships in those algorithms~\cite{page1999pagerank}, has had
virtually no impact on how we find our own local data.
Second, the generalized graph structure, which no longer elevates any single
organzation gets rid of the \textit{specific taxnomic order} that HCI
researchers determine to be counterproductive.
Third, although some degree of temporal query is possible using \texttt{find},
its interface is not especially accessible to the typical user, and it requires
a series of manual operations to express natural queries such as
``Show me the documents I wrote last summer after I got back from my Amazon
rafting trip.''

Our goal is not to specify the entire range of searches that can be realized, but rather to
explore file system structures that enable the creation and mining of relationships to help users to find relevant data.

\begin{comment}
To help motivate our work, we consider the \textit{Graph Query 
Language}~\footnote{https://gql.today/wp-content/uploads/2018/05/a-proposal-to-the-database-industry-not-three-but-one-gql.pdf} 
(GQL) as a starting point.  GQL is an emerging language
attuned to the needs of \textit{property graphs}, which happen to be similar to the model that we envision for our new
file system.  It attempts to merge the strengths of three existing graph database query languages into a single, standard,
query language for property graphs~\cite{van2016pgql,francis2018cypher,angles2018g}.

% This choice is motivated by our realization that the new model we propose is a property graph~\cite{rudolf2013graph}.
\end{comment}


\section{Related Work}\label{sec:background}

The need for better name spaces in file systems is hardly a new
topic, with many solutions being proposed and implemented over
the years.

\textit{Search utilities} are successors to the permuted index program.  They permit us to find
files based on \textit{content} and \textit{attributes}.  MacOS X has \textit{spotlight}, which
provides an extensible, index-driven search service~\cite{apple:spotlight-extensions}.  Similarly,
Windows offer a similar extensible service~\cite{microsoft:data-add-in}.  These enable searching
based upon attributes, e.g., file suffix, date, size, etc., and context-sensitive content, e.g.,
music files by artist, composer, song title, or even \textit{rights},
but limited, if any, ability to search by relationship.

%.  A number of other examples
%of similar search mechanisms
% exist~\cite{Suguna2015,huo2016mbfs,leung2009magellan}.

%\textit{Databases} enable developers willing to pre-define their data's structure to enable searching
%on the specifics of the data.  Databases come in a rich array of models: relational, column, document,
%and graph, for example.  File systems constructed from databases have been extensively 
%explored~\cite{olson1993design,balabine1999file,balabine2002database,kashyap2004file,murphy2002design}.
%As we noted previously (\S \ref{sec:graphfs:model}) such approaches have failed to yield clear
%results.

\textit{Tag Systems} were an early approach to improving hierarchical file systems searchability%
~\cite{Parker-Wood2014,chou2015findfs,ma2009file,laursen2014,nayuki2017,Andrews2012,Up2016,Jones2016,aws:s3:object:tagging,ames2006lifs,leung2009magellan,frieder2012hierarchical}.
Automatic tagging systems have become a more common approach here as manual tagging by users 
has proven to be impractical~\cite{soules2003can,soules2004toward}.
The addition of \textit{semantic} information
~\cite{di2017gfs,hua2016real,martin2004formal,Martin2005,martin2008,martin2014,gifford1991semantic,Faubel2008,harlan2011joinfs,Suguna2015,Andrews2012,ngo2007integrating,Omvlee2009,wang2003managing,gopal1999integrating,Codocedo2015,Jones2016,Mahalingam2003,Parker-Wood2014}
is useful but falls short of addressing the fundamental need to understand
data relationships, because like more conventional systems,
these semantically-aware approaches still focus on the file,
not on the relationships between the files.
As such, they are simply an add-on to
the hierarchical model, not a replacement for it.
Such approaches can provide useful functionality
in our graph file system model as well.
Indeed, we even pointed this out (perhaps subconsciously) in prior
work when we said ``How many of them [files] are \textit{related} to
each other?'' [emphasis added]~\cite{Seltzer2009} .

Files are rich with relationships.  However, these relationships are not limited to what the
file system can ``see''.
Narrowing our vision to the closed pool of file system relationships
hobbles our ability to capture them.
For example, the obvious relationship between
a file and the e-mail from whence it originated is not exploitable in
any system of which we are aware.
The academic papers we generate refer to other papers.  An enlightened document application would provide an identifier that can 
be used to find the corresponding paper - a \textit{refers to} relationship,
whether on our local system or elsewhere.
Of course, in the current model, we likely can't
recall where we stored it when we downloaded it.  As we create new works, we refer to older works --- our own
documents, web pages, Jupyter notebooks, spreadsheets, pictures, etc. Capturing these relationships permits
us to reconstruct the process taken to produce an output.
This is the fundamental problem that the provenance community has been
addressing, but few systems~\cite{pasquier17camflow,reddy06pass} demonstrate
an understanding of the role our file systems play in making this
possible.

Versioning is a feature that continues to reappear in various guises.
This is simply one example of temporal locality; a
relationship that we have not yet deeply mined.
While it is now common practice for
individual applications to ``remember'' the last few files you have accessed,
there are few cross-application examples.
In lieu of the right tools, users invent creative solutions.
For example, one of the authors \textit{attaches} documents to e-mail
immediately after reviewing them
specifically to establish temporal relationship.
The ability to establish temporal relationships across applications should
provide powerful capabilities.

We do not know which relationships are useful. One of the hallmarks of good file systems design over the decades
has been \textit{not} to impose a specific restricted model on what files can be --- we leave that to databases.  
We do not intend to establish a definitive set of relationships any more than we focus on defining
the structure of file contents.
However, we encourage others to explore this area, encourage best practices,
and build tools that produce and use such relationships, leaving the
storage and retrieval of relationships to the file system.

% Relationships between files is not new --- this is the quintessential relationship of the modern Internet, with its
% vast web of content that references across the domains --- but those models are certainly more recent than that
% of the hierarchical file system.  Unlike the internet, where information is shared, we seek to enable
% similar relationships useful to our specific usage and with data we do not necessarily wish to share.

Much of the raw data that applications generate would be better \textit{not} injected into
the hierarchical name space: the location of our personal email database, financial
software files, binaries, temporary files, etc.  Their presence in the name space clutter
it and make our existing brute force search slower yet no more useful.

Application programs benefit being
unfettered from the hierarchical name space~\cite{Ghemawat2003}, both
in terms of their efficiency as well as the benefits of \textit{not} commingling the private files of individual 
applications --- but the hierarchical file system requires
they be stored somewhere within its domain.
Applications routinely hide most of their files in out of the way 
locations, as they are only useful to the application itself.
Thus we end up with ``System Volume Information'' and ``.ssh'' and a myriad of obscure locations where applications 
hide data from us.
This is a side effect of the name space model we have used for the past
50 years.

Prior attempts to address this have done so within the confines of a narrow perspective of what is needed to fully 
enable the ability of us to find our data.  The HCI community have been poking at the
edges of this problem for decades as well --- observing the frailties of the hierarchical model and suggesting
alternatives~\cite{harper2013file,lindley2018exploring,khan2018forgotten,vitale2018hoarding,boardman2003too,nayuki2017,martin2014,Jan2011,Andrews2012,Mander1992,Omvlee2009}.

Our graph file system permits them to escape the existing paradigm \textit{without} giving up support
of existing applications.


\section{Conclusion}
\label{sec:conclusion}

We have presented our position that we need \system, a storage architecture that decouples naming from the storage location of documents and data objects, provides customizable and personalized namespaces, and that makes relationships between documents a first class citizen. With \system, users will be able to organize, share and find their data conveniently across multiple storage silos using a rich set of attributes breaking away from the rigid, hierarchical organization.

We expect \system will enable a broad area of research in HCI exploring new ways to visualize and interact with data using the mechanism's provided by \system. Moreover, we expect \system to provide interesting scenarios for security and privacy research in storage systems.



%\section{Use Cases}

We consider the following potential use cases for our naming system:

\begin{description}

    \item[Finding historical documents] - in this common usage scenario, we are
    looking for an object that we identify by what we were doing and when we
    were doing it as well as potential subjects.

    \item[Related documents in distinct storage locations] --- in this common
    usage scenario, we are looking for \textit{related} objects that, for
    whatever reason, are stored in different ``storage silos''.  For example, we
    received a document via e-mail and then saved it on the ``Dropbox folder''
    of our local system.  The e-mail and the document are related, but we have
    no obvious way to traverse back from the document to the original e-mail.

    \item[Ability to search non-traditional storage locations] --- in this usage
    scenario, we are looking for \textit{objects} that are not stored in a
    traditional storage silo but instead reside inside some other object storage
    domain.  For example, objects in an arbitrary object store, such as the
    ubiquitous key-value store.

    \item[Cross-silo versions / Document Identity]  ---
    draft.doc that you got from your coworker via e-mail, that version you stored
    on your local drive, then the one you uploaded to Office365. Another one you've
    got from another co-worker via Slack. Nirvana will show these as different
    versions from the same logical document, possibly even the temporal relationships
    between then

    \item[Notifications] -- Allowing a user to "subscribe" to change
    notifications for critical documents; e.g., an active collaborative project
    to ensure users are notified when documents are updated (and of course
    cancel notifications when they are no longer needed/useful).

    \item[Search Results] -- What kind of searches happened during the writing
    of of another document (e.g., you are interested in the statistics of a
    search, but not particular results)

    \item[Compliance] --- Identifying related documents, including references,
    to specific material that needs to be located as part of compliance with
    legal mandates, e.g., discovery notices, GDPR data removal requests (``right
    to be forgotten'').
\end{description}


\subsection{Prior Usecase Examples}

\reto{OLD `usecases follow`}

% locating documents in general
\noindent\textbf{Locating Documents.}
Users want to search for particular documents (use-cases \\usecasehistcontext, \\usecasereldocuments).
Indexing services (e.g., Spotlight), cloud-based platforms, or tools like \texttt{grep} or \texttt{find} provide mechanisms to locate documents based on their name, date of creation or modification, or even the content of the files.
However, users often need to search different storage silos independently. This poses a burden to the user.
Worse, searching over locally mounted network attached storage (e.g., NFS or Dropbox) could mean transferring gigabytes worth of data for doing the search.
Finally, current solutions do not capture activities well such as while on a call with a customer, or while attending a webinar or conference.
Not only fail existing architecture to capture these aspects, they are also not capturing seemingly simple relationships between a file stored on disk and the e-mail exchange through which it was received and thus losing important contextual information.

% integration with non-traditional storage
\noindent\textbf{Alternative Storage.}
Traditional files are not the only way to store documents.
Today's applications may use object stores, key-value stores and data bases, or even implement their own storage container holding multiple objects (use case \\usecasealtstorage).
Moreover, applications want to protect their data from unauthorized accesses using various methods such as encryption, for example.
This makes searching hard and the user is forced to use the interface provided by the application to locate its documents, resulting in yet another manual cross-silo search.
While point solutions exist, e.g., Android search integration or export as WebDav-based file system, they are not ubiquitously available or do not allow users to search using the full range of attributes.

% document identity / cross silo version
\noindent\textbf{Document Identity.}
Documents have an identity. Writing an additional paragraph in this paper, does not change its identity. It rather creates a new version of the same document and both versions are related.
Likewise, uploading a document to cloud storage, renaming it, or converting it to a PDF also does not change its identity (use-cases \\usecasereldocuments, \\usecasedocidentity).
In contrast, taking last year's HotStorage paper as a template for this year's paper will change its identity.
While versioning is available on various storage solutions or VCS systems, it fails to capture cross-silo versions and relation ships. For example,
the document that was just received via e-mail is in fact a version of the one you have been editing yesterday, and that you have just uploaded to cloud storage for sharing with your co-worker who just sent it to you via e-mail.

% multi-device
\noindent\textbf{Multi-Device.}
Users may use various devices to access their files, each of which having different compute and connection capabilities (use-case \\usecasedevices).
For instance, a desktop machine is powerful and always connected, while a smartphone is low power and can often be disconnected. Cloud-storage or e-mail client apps on the devices provide a search interfaces that may offload the actual search to the server.
A user with two devices cannot easily look for files on the local disks and has now to do the search manually across multiple devices (similar to \\usecasereldocuments).

% attribute provenance
\noindent\textbf{Attribute Provenance.}
Traditional file systems tie the permissions to change attributes with the permissions to change the file contents.
Any user with sufficient rights can change any attribute or the file contents.
While there are systems that record the user who changed the file last, it may not capture the entire history of the changes (use-case \\usecaseattrprov).
Moreover, existing systems do not capture the intent or reason of those changes.

% attribute provenance
\noindent\textbf{Multi-View.}
A document cannot be physically present in multiple filing cabinets at the same time.
Hard and soft links provide references to other other storage locations.
However, there is still only a single physical organization of files making it impossible to organize files by year-customer, and customer-year at the same time.
File managers may offer tags, and media libraries to notion of albums and grouping by year to provide specific ways to sort, filter or organize files.
This approach, however, is not generally available and users cannot freely choose and adapt their \emph{view} of their files (use-case \\usecaseviews)

\section{Notes from the meeting (remove as pleased)}

Focus on the what the system does, with a bit of how..

\begin{itemize}
    \item Placeless + Burrito?~\footnote{\url{https://www.usenix.org/system/files/conference/tapp12/tapp12-final10.pdf}}
    \item Attributes that are not traditionally a file.
    \item GIS information is useful
    \item don't open files for extracting meta-data
\end{itemize}

Taxonomy

\begin{itemize}
 \item table what can be done what can't. 3 columns. Use-case, what can be done, what's hard.
 \item here's a structure of them. (what's easy and hard to do with today's technology) --> references to architecture section
 \item evaluation: show that we are supporting the things that are hard.
\end{itemize}

 \begin{description}
\item[Section 1] introduction
\item[Section 2] Background
\item[Section 3] Use-cases + table
\item[Section 4] Architecture
\item[Section 5] Show that use-cases are solved
 \end{description}

Different clients:

\begin{itemize}
 \item Desktop: powerful + always connected
 \item Laptop: powerful + can be offline
 \item Smartphone: low power + mostly connected
 \item Smartphone w/o data: low power + mostly disconnected
\end{itemize}


\clearpage
\pagebreak

% I add a table of nocites here to get a list of all the papers in the bibtex
% comment out to only use the references in the paper itself.
\begin{comment}
\nocite{daley1965general}
\nocite{koetsier2013digg}
\nocite{ritchie1973unix}
\nocite{wilkes1964programmer}
\nocite{reinsel2018data}
\nocite{macdonald1956datafile}
\nocite{halasz1982analogy}
\nocite{shapiro1964extracting}
\nocite{microsoft:data-add-in}
\nocite{apple:spotlight-extensions}
\nocite{chou2015findfs}
\nocite{vicente1988accommodating}
\nocite{marsden2003improving}
\nocite{jones2005don}
\nocite{karger2006data}
\nocite{ma2009file}
\nocite{jensen2010life}
\nocite{karlson2011version}
\nocite{odom2012lost}
\nocite{Thereska2013}
\nocite{harper2013file}
\nocite{mendez2012linked}
\nocite{bothorel2015clustering}
\nocite{huo2016mbfs}
\nocite{jayalakshmi2016semantic}
\nocite{carvalho2016finding}
\nocite{gao2016implicit}
\nocite{hua2016real}
\nocite{di2017gfs}
\nocite{lindley2018exploring}
\nocite{khan2018forgotten}
\nocite{vitale2018hoarding}
\nocite{laursen2014}
\nocite{boardman2003too}
\nocite{nayuki2017}
\nocite{martin2004formal}
\nocite{martin2008}
\nocite{martin2014}
\nocite{soules2004toward}
\nocite{gifford1991semantic}
\nocite{corbato1965introduction}
\nocite{Faubel2008}
\nocite{harlan2011joinfs}
\nocite{Jan2011}
\nocite{Soules2004}
\nocite{Suguna2015}
\nocite{Andrews2012}
\nocite{Eck2011}
\nocite{Hua2010}
\nocite{Joshi}
\nocite{Mander1992}
\nocite{Martin2004}
\nocite{ngo2007integrating}
\nocite{Omvlee2009}
\nocite{wang2003managing}
\nocite{Seltzer2009}
\nocite{Shah2007}
\nocite{Up2016}
\nocite{gopal1999integrating}
\nocite{Martin2005}
\nocite{Reiser2016Name}
\nocite{Benefits2016}
\nocite{Codocedo2015}
\nocite{Ghemawat2003}
\nocite{Jones2016}
\nocite{Mahalingam2003}
\nocite{Os2016}
\nocite{Parker-Wood2014}
\nocite{Prabhakaran2005}
\nocite{Schandl2009}
\nocite{cellier2008formal}
\nocite{strong2013semantic}
\nocite{Xu2014}
\nocite{olson1993design}
\nocite{Leung2009}
\nocite{Schandl2009a}
\nocite{Shah2006}
\nocite{Wiki2016}
\nocite{vicente1987assaying}
\nocite{malone1983people}
\nocite{terrizzano2015data}
\nocite{watson2017exploring}
\nocite{mogul1986representing}
\nocite{GuoBurrito2012}
\nocite{macko2015llama}
\nocite{margo2015scalable}
\nocite{macko2013performance}
\nocite{aws:s3:object:tagging}
\nocite{balabine1999file}
\nocite{stonebraker1994mariposa}
\nocite{balabine2002database}
\nocite{xu2003towards}
\nocite{kashyap2004file}
\nocite{ames2006lifs}
\nocite{koren2007searching}
\nocite{huang2012just}
\nocite{xu2009semantic}
\nocite{murphy2002design}
\nocite{leung2009magellan}
\nocite{frieder2012hierarchical}
\nocite{bush1945we}
\nocite{whang1991multilevel}
\nocite{kumar1997browsing}
\nocite{van2003beamtrees}
\nocite{dumais2016stuff}
\nocite{soules2003can}
\nocite{maccormick2004boxwood}
\nocite{marchionini2006exploratory}
\nocite{brandt2009fusing}
\nocite{leung2009spyglass}
\nocite{ren2013tablefs}
\nocite{niazi2017hopsfs}
\nocite{van2011efficient}
\nocite{215e74710c2245a59b2ce117f9e74cd7}
\nocite{appuswamy2014building}
\nocite{Harlan2011}
\nocite{Min2015}
\nocite{Open2016}
%\nocite{SemFS2016}
\nocite{Chou2015}
\nocite{Ngo2015}
%\nocite{Regatta2016}
\nocite{Gopal}
\nocite{Beydoun2009}
\nocite{Gifford2006}
\nocite{Hyvonen2004}
\nocite{Siekmann}
\nocite{Strong}
%\nocite{Dantalian2016}
\nocite{Escriva2015}
\nocite{Agee2016}
\nocite{Gifford1991}
\nocite{Guide2016}
\nocite{Liu2006}
\nocite{needham1977cap}
\nocite{arenas2016faceted}
\nocite{tunkelang2009faceted}
\nocite{hearst2006design}
\nocite{arenas2016faceted}
\nocite{klungre2018evaluating}
\nocite{walton2015searching}
\nocite{walton2017looking}
\nocite{cleverley2015retrieving}
\nocite{huurdeman2016active}
\end{comment}

\bibliographystyle{IEEEtranSN}
\bibliography{notdead} 

\end{document}
