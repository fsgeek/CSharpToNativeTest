\pdfminorversion=7
\documentclass[sigconf,10pt,noacm]{acmart}
\usepackage{geometry}
\geometry{reset, twoside=true, head=13pt,
     paperwidth=8.5in, paperheight=11in,
     includeheadfoot, columnsep=2pc,
     top=57pt, bottom=73pt, inner=72pt, outer=72pt,
     marginparwidth=2pc,heightrounded
     }

\usepackage[utf8]{inputenc}
%\usepackage{fontspec} % This line only for XeLaTeX and LuaLaTeX
\usepackage{pgfplots}
\pgfplotsset{compat=1.16}
\usepackage{tikz}


\usepackage{booktabs} % For formal tables
\usepackage{balance}       % to better equalize the last page
% Load basic packages
\usepackage{balance}       % to better equalize the last page
%\usepackage{graphics}      % for EPS, load graphicx instead 
\usepackage{graphicx}
%\usepackage[T1]{fontenc}   % for umlauts and other diaeresis
%\usepackage{txfonts}
%\usepackage{mathptmx}
%\usepackage[pdflang={en-US}]{hyperref}
\usepackage{hyperref}
%\usepackage{color}
\usepackage{textcomp}
\usepackage{blindtext}
\usepackage{subcaption}
\usepackage{dblfloatfix}
\usepackage{adjustbox}

\usepackage{booktabs}
\usepackage{multirow}
% \usepackage{lscape}
% If you use beamer only pass "xcolor=table" option, i.e. \documentclass[xcolor=table]{beamer}
\usepackage{siunitx}
%\usepackage{pgfplotstable}
% Some optional stuff you might like/need.
\usepackage{microtype}        % Improved Tracking and Kerning
\usepackage[all]{hypcap}    % Fixes bug in hyperref caption linking
\usepackage{ccicons}          % Cite your images correctly!
% \usepackage[utf8]{inputenc} % for a UTF8 editor only

% If you want to use todo notes, marginpars etc. during creation of
% your draft document, you have to enable the "chi_draft" option for
% the document class. To do this, change the very first line to:
% "\documentclass[chi_draft]{sigchi}". You can then place todo notes
% by using the "\todo{...}"  command. Make sure to disable the draft
% option again before submitting your final document.
%\usepackage{todonotes}
%\usepackage{draftwatermark}
%\SetWatermarkScale{5}

\usepackage{graphicx}
\usepackage{type1cm}
\usepackage{eso-pic}
\usepackage{lipsum}

% Paper metadata (use plain text, for PDF inclusion and later
% re-using, if desired).  Use \emtpyauthor when submitting for review
% so you remain anonymous.

% Copyright
\setcopyright{none}
%\setcopyright{acmcopyright}
%\setcopyright{acmlicensed}
%\setcopyright{rightsretained}
%\setcopyright{usgov}
%\setcopyright{usgovmixed}
%\setcopyright{cagov}
%\setcopyright{cagovmixed}


%\copyrightyear{2019} 
%\acmYear{2019} 
%\setcopyright{rightsretained}
%\acmConference[HotOS XVII]{The 17th Workshop on Hot Topics in Operating Systems (HOTOS)}{May 12-15, 2019}{Bertinoro, Italy}
%\acmBooktitle{Proceedings of the 17th Workshop on Hot Topics in Operating Systems (HOTOS '19), May 12-15, 2019, Bertinoro, Italy}
%\acmPrice{15.00}
%\acmDOI{10.475/123_4}
%\acmISBN{123-4567-24-567/08/06}
\acmConference{}
\acmBooktitle{}
\acmPrice{}
\acmDOI{}
\acmISBN{}
\acmYear{}

% These commands are optional
%\acmBooktitle{Transactions of the ACM Woodstock conference}
%\editor{Jennifer B. Sartor}
%\editor{Theo D'Hondt}
%\editor{Wolfgang De Meuter}

\pagestyle{plain}
\settopmatter{printacmref=false}
%\settopmatter{printfolios=false}
%\setlength {\marginparwidth}{2cm}
\renewcommand\footnotetextcopyrightpermission[1]{} % removes footnote with conference information in first column

\setlength{\textfloatsep}{10pt plus 1.0pt minus 2.0pt}
\setlength{\intextsep}{6pt plus 2.0pt minus 2.0pt}

\begin{document}
\title{Not Dead Yet}
\subtitle{Hierarchical File Systems Won't Die}

\author{Tony Mason}
\orcid{0000-0002-0651-5019}
\affiliation{%
  \institution{The University of British Columbia}
}
\email{fsgeek@cs.ubc.ca}

\author{Margo Seltzer}
\affiliation{%
\institution{The University of British Columbia}
%\city{Atlanta}
%\state{Georgia}
%\country{US}
}
\email{mseltzer@cs.ubc.ca}

% The default list of authors is too long for headers.
\renewcommand{\shortauthors}{T. Mason et al.}

%
% Generated this code.  Should review it
%
\begin{comment}
\begin{CCSXML}
  <ccs2012>
  <concept>
  <concept_id>10003456.10003457.10003527.10003531.10003533</concept_id>
  <concept_desc>Social and professional topics~Computer science education</concept_desc>
  <concept_significance>300</concept_significance>
  </concept>
  <concept>
  <concept_id>10003456.10003457.10003527.10003531.10003533.10011595</concept_id>
  <concept_desc>Social and professional topics~CS1</concept_desc>
  <concept_significance>300</concept_significance>
  </concept>
  <concept>
  <concept_id>10003456.10003457.10003527.10003540</concept_id>
  <concept_desc>Social and professional topics~Student assessment</concept_desc>
  <concept_significance>300</concept_significance>
  </concept>
  </ccs2012>
  
\end{CCSXML}

\ccsdesc[300]{Social and professional topics~Computer science education}
\ccsdesc[300]{Social and professional topics~CS1}
\ccsdesc[300]{Social and professional topics~Student assessment}
\end{comment}

\begin{comment}
\keywords{Graph File System}
\end{comment}

\makeatletter
\def\@copyrightspace{\relax}
\makeatother

\maketitle

\balance

% This file provides examples of some useful macros for typesetting
% dissertations.  None of the macros defined here are necessary beyond
% for the template documentation, so feel free to change, remove, and add
% your own definitions.
%
% We recommend that you define macros to separate the semantics
% of the things you write from how they are presented.  For example,
% you'll see definitions below for a macro \file{}: by using
% \file{} consistently in the text, we can change how filenames
% are typeset simply by changing the definition of \file{} in
% this file.
%
%% The following is a directive for TeXShop to indicate the main file
%%!TEX root = ../diss.tex

\newcommand{\NA}{\textsc{n/a}}	% for "not applicable"
\newcommand{\eg}{e.g.,\ }	% proper form of examples (\eg a, b, c)
\newcommand{\ie}{i.e.,\ }	% proper form for that is (\ie a, b, c)
\newcommand{\etal}{\emph{et al}}

% Some useful macros for typesetting terms.
\newcommand{\file}[1]{\texttt{#1}}
\newcommand{\class}[1]{\texttt{#1}}
\newcommand{\latexpackage}[1]{\href{http://www.ctan.org/macros/latex/contrib/#1}{\texttt{#1}}}
\newcommand{\latexmiscpackage}[1]{\href{http://www.ctan.org/macros/latex/contrib/misc/#1.sty}{\texttt{#1}}}
\newcommand{\env}[1]{\texttt{#1}}
\newcommand{\BibTeX}{Bib\TeX}

% Define a command \doi{} to typeset a digital object identifier (DOI).
% Note: if the following definition raise an error, then you likely
% have an ancient version of url.sty.  Either find a more recent version
% (3.1 or later work fine) and simply copy it into this directory,  or
% comment out the following two lines and uncomment the third.
\DeclareUrlCommand\DOI{}
\newcommand{\doi}[1]{\href{http://dx.doi.org/#1}{\DOI{doi:#1}}}
%\newcommand{\doi}[1]{\href{http://dx.doi.org/#1}{doi:#1}}

% Useful macro to reference an online document with a hyperlink
% as well with the URL explicitly listed in a footnote
% #1: the URL
% #2: the anchoring text
\newcommand{\webref}[2]{\href{#1}{#2}\footnote{\url{#1}}}

% epigraph is a nice environment for typesetting quotations
\makeatletter
\newenvironment{epigraph}{%%
	\begin{flushright}
		\begin{minipage}{\columnwidth-0.75in}
			\begin{flushright}
				\@ifundefined{singlespacing}{}{\singlespacing}%
				}{
			\end{flushright}
		\end{minipage}
	\end{flushright}}
\makeatother

% \FIXME{} is a useful macro for noting things needing to be changed.
% The following definition will also output a warning to the console
\newcommand{\FIXME}[1]{\typeout{**FIXME** #1}\textbf{[FIXME: #1]}}

% Replaceable names
%\newcommand{\system}[0]{\emph{Kwishut}\xspace}
\newcommand{\system}[0]{\emph{Indaleko}\xspace}
% QI'tu'naS - Klingon for "pragmatics" - I'm saving that for the final version.
% qaywI' - Klingon for "finding" - another option for final version.
% Talal - Vulcan for "finding" - yet another option.

\newcommand{\systemone}[0]{\emph{Topish}\xspace}
% Topish is Uzbek for "Finding"
%\newcommand{\systemtwo}[0]{\emph{\textmacedonian{Наоѓање}}\xspace}
% Pronunciation: Naoǵanje.
%\newcommand{\systemtwo}[0]{\emph{\textmacedonian{Naīdi}}\xspace}
% Naīdi ("Find") in Macedonian.
\newcommand{\systemtwo}[0]{\emph{\textmacedonian{Находка}}\xspace}
% Pronunciation: Nakhodka

% Gender neutral names (though very European)
%\newcommand{\persa}[0]{Addison\xspace}
\newcommand{\persa}[0]{Aki\xspace} % Japanese, gender neutral, means "autumn"

%\newcommand{\persb}[0]{Bailey\xspace}
\newcommand{\persb}[0]{Dagon\xspace} % Biblical name meaning "fish"

%\newcommand{\persc}[0]{Cameron\xspace}
\newcommand{\persc}[0]{Fenix\xspace} % Greek Isles, means "dark red"

%\newcommand{\persd}[0]{Dana\xspace}
\newcommand{\persd}[0]{Hao\xspace} % Vietnamese, means "good/perfect"

%\newcommand{\perse}[0]{Evan\xspace}
\newcommand{\perse}[0]{Waneta\xspace} % Native American roots, "shape shifter, charger"

%\newcommand{\persf}[0]{Quinn\xspace}
\newcommand{\persf}[0]{Skylar\xspace} % allegedly American roots, meaning "scholar"

%\newcommand{\persg}[0]{Reese\xspace}
\newcommand{\persg}[0]{Zene\xspace} % "beautiful" based on African culture supposedly.

% use cases
\newcommand{\usecaseactivitycontext}[0]{\textsc{ac\-tiv\-ity con\-text}\xspace}
\newcommand{\usecasedatarelationship}[0]{\textsc{data re\-la\-tion\-ships}\xspace}
\newcommand{\usecasecrosssilosearch}[0]{\textsc{cross-silo search}\xspace}
\newcommand{\usecasenotifications}[0]{\textsc{no\-ti\-fi\-ca\-tions}\xspace}
\newcommand{\usecasepersnamespace}[0]{\textsc{per\-son\-al\-ized name\-space}\xspace}

% terminology
%Copy: bit-for-bit identical
\newcommand{\doccopy}[0]{copy\xspace}
%  derivation: the semantics change
\newcommand{\docderivation}[0]{derivation\xspace}
% conversion: semantically identical but not Bit-for-Bit
\newcommand{\docconversion}[0]{conversion\xspace}

% END


\chapter{Abstract}

Human society is collecting data at an alarming rate: per-capita data generation
is now over 1.7MB \emph{per second}. We expect to send 361 billion e-mails
\emph{per day} by 2024. Rapid data growth, combined with increasing ways to
store and present data to users creates a frustrating challenge finding specific
documents a few days old, let alone those created months or years earlier.

Our data is scattered across physical locations.  Existing storage is presented
to us with old and new interfaces that blur the lines between file system and
application. For example, my Outlook mailbox resides on my local disk
drive in both databases and discrete files.  Yet I use Outlook, not my local
file system to find documents that were attached to e-mails.

On-demand cloud storage systems provide strong benefits yet also make it
impractical to search locally because not all the content is resident to be
indexed. Currently none of the cloud storage systems offer the rich extensible
search tools found on modern desktop operating systems.  Even if they were to
provide such search, it would require querying each one in turn to find relevant
files.

To address these challenges I propose \emph{Finding as a Service}, which provides two
important capabilities.  First, it explicitly decouples \emph{finding} objects from
\emph{storing} and \emph{presenting} objects.  Second, it exploits the
observation that users' mental associations with objects are more complex than
the arbitrary name, type, dates, or attributes on which users search today.

\emph{Finding as a Service} requires two things: (1) a mechanism for exploiting
the information that modern devices already capture and for capturing
additional useful information that relates to interactions with digital data and
the environment in which that data is used; and (2) the ability to collect,
store, and query both existing and new usage context pertinent to digital
information. Both of these requirements enable building powerful tools that
helps users in \emph{finding} digital objects efficiently.


%\vfill
%\begin{center}
%    \begin{sf}
%        \fbox{Revision: \DTMnow}
%    \end{sf}
%\end{center}



\begin{comment}
  Margo comments from our January 11 meeting.

  Introduction: we have hieararchy but we need graph storage.  We see this as far back as Multics, where they have hard links,
  and UNIX, where they add symlinks [though I don't think symlinks showed up in 1974, but they did eventually.]

  The goal was to mirror the physical world of how filing systems worked [cabinets, drawers, folders, documents.  But this
  ignores the fact that libraries add cross-references - a separate name space of indices that help us find things.  And
  how documents refer to other documents, creating a "web of references".  This is not a new observation and includes
  memex and intelligent document editing systems.]

  Look at storage versus namespace: how tightly coupled are they?
  Paradigm shift: the web started with a more hierarchical model but moved to a graph model.  It also introduced
  the search based metaphor. As the web has grown, this perspective has permeated all of data storage, as we
  can see by the rampant popularity of key-value stores, object caches, etc.

  Namespace

  * Ignored by most users
  * Evolved from the web perspective
  * Those of us in "the field" still fall back to brute force search with find and grep -R

  File systems search and web search have merged into things like Dropbox, Google Drive, One Drive, and Git.

  "One author spent 20 minutes trying to find a document trying to find data from the previous work.  Had to
  search computer, Google Drive, dropbox, various repositories, etc."  Ultimately found that the data wasn't
  in any of those places and managed to get it from a co-author.

  Question: is this a systems problem or an HCI problem?

  Systems problem: data objects related via an extensible set of complex relationships that the system does
  nothing to facilitate.

  This suggests a graph structure.  Our current data storage and name space layers do not support this well.

  The HCI aspect of this is how to exploit a richer structure, so now we have a chicken and egg problem. Thus,
  we ask the question: "how should we rethink the file system to permit capturing a richer set of relationships?

  Alternatives become Background
    * Semantic file systems
    * Tagging (old, manual)
    * Tagging (new, automated/AI driven)
    * Databases

  We argue that it requires a custom, specialized database for our domain problem (e.g., relationship enhanced
  file systems).

  Limit BG to one page.

  ** maybe put Alternatives/Background AFTER our proposal ***

  Our proposal should be ~ 2 pages.

    * Envision your file system as a social graph for this discussion - a "social network file system"
    * Characteristics of data (what something "is") - one kind of relationship.
    * Social network: has people, groups, organizations)
    * File system: blob store, chunks of data
    * Shared "attributes" are one form of relationship (mapping social network to file system)
    
    What does the facebook architecture look like?  A SQL database (MySQL) with a cache (Tau)

    Why not use this as our architecture?  The analogy is not perfect.  Objects have a fixed schema in
    the database world but files do not.  Files vary in all sorts of ways and typically have an internal
    (private) schema.  They have a potentially broader range of attributes, though often we embed these
    in the name now (extension).  File systems typically store sizes, dates, and lengths.  Thus, we argue
    that we already have a partially mapped schema, but it is a rigid one.

    Proposed architecture:
      * storage - this is SOLVED for our purposes, regardless of whether the storage layer is a traditional
        HFS, a blob store, a document database, local or remote to the device.

      * relationships -
        - how do we store them?
        - how do we represent them?
        - how do we process them?

        PRESENTATION of these to users is an HCI problem.
        IMPLEMENTATION of these features is a systems problem.

    Q: why not a graph database?  Observation: in fact, we're a user of the graph database, but we aren't
       a graph database for the same reason a file system isn't a relational database: our needs are more
       simpler, so we can focus on optimizing performance for our use cases.

       Our blobs (files) are potentially quite large.
       Our relationships are ...

  Need to include the Burrito (Guo) reference.  Margo thought he had a more recent paper as well in the area,
  I need to look.

  Note: provenance is one type of relationship.

  We need a table of relationships.  This is demonstrative, not definitive.

  How do we achieve metadata efficiency?  Clustering?  Storage locality and clustering?

  Review: Plan 9, which had per-user and per-process namespaces.

  We must be able to uniquely identify a file [need to define uniquely identify].

  A "re-find" with a nice API

  A cache of recent files.

  Names may be ephemeral, since users cannot remember that much.

  Build our model from here.

  We could add a container on top that elevates this relationship above all others - the hierarchical model.

  Concrete proposal for graph FS.

\end{comment}

%% The following is a directive for TeXShop to indicate the main file
%%!TEX root = ../diss.tex

\chapter{Introduction}
\label{ch:introduction}

\begin{epigraph}
    \emph{
        What should I possibly have to tell you, oh venerable one? Perhaps that you're searching far too much? That in all that searching, you don't find the time for finding?
    } --- Siddhartha (1922), Hermann Hesse.
\end{epigraph}

\section{Activity Context}
\label{ch:introduction:sec:activitycontext}

In 1945 Vannevar Bush described the challenges to humans of
finding things in a codified system of records~\cite{bush1945we}:

\begin{quotation}
    \emph{Our ineptitude in getting at the record is largely
        caused by the artificiality of systems of indexing. When data of any
        sort are placed in storage, they are filed alphabetically or numerically, and
        information is found (when it is) by tracing it down from subclass to
        subclass. It can be in only one place, unless duplicates are used; one
        has to have rules as to which path will locate it, and the rules are
        cumbersome. Having found one item, moreover, one has to emerge from the
        system and re-enter on a new path.}

    \emph{
        The human mind does not work that way. It operates by association. With one
        item in its grasp, it snaps instantly to the next that is suggested by the
        association of thoughts, in accordance with some intricate web of trails
        carried by the cells of the brain. It has other characteristics, of course;
        trails that are not frequently followed are prone to fade, items are not
        fully permanent, memory is transitory. Yet the speed of action, the
        intricacy of trails, the detail of mental pictures, is awe-inspiring beyond
        all else in nature.}
\end{quotation}

This is as true in 2021 as it was in 1945.  Thus, the question that motivates my
research is: ``Can we build systems that get us closer to that ideal?''  My
argument that we can by capturing additional information that is not necessarily
useful to computers, but is useful to humans.

I call this additional captured information \emph{activity context}. While similar
to the ideas proposed in \emph{Burrito}~\cite{guo2012burrito}, I have broadened
that idea beyond just ``what a user is doing'' to incorporate information about
what the user is \emph{experiencing} that corresponds to more human-like context
information because it is useful for constructing \emph{association}.

Thus an ``activity context'' is an answer to the question:
\emph{what is going on in relation to the current event on a digital object?}
The job of the system becomes answering this question in a way that is useful.

More concretely, activity context is concerned with the environment in which a
given digital object is accessed.  Without restricting the abstract concept,
concrete examples of what I consider to be elements of an ``activity context''
might include:

\begin{itemize}
    \item Current weather.
    \item Notable news events.
    \item Focus website opened in a visible browser tab.
    \item User's mood.
    \item User's heart rate.
\end{itemize}

This is distinguished from \emph{what the digital object is}. While
understanding what something \emph{is} has merit, the context in which a digital
object is \emph{used} yields additional understanding about that digital object.

Further examples, in the form of ``use cases,'' can be found in
\autoref{ch:intro:sec:use-cases}.

\section{Thesis}
\label{ch:introduction:sec:thesis-statement}

\textbf{
    \input{thesis.tex}
}

\vspace{0.25cm}

``Intelligent use of files depends on having sufficient knowledge about them: their purposes, structures, and
contexts. Humans have traditionally made do by using their own methods for capturing and manipulating
such knowledge, but this is not available to programs, nor is it necessarily
convenient for humans~\cite{mogul1986representing}.''

Determining \emph{context} is a challenge with modern computer storage systems.
It is unrealistic to expect human users to provide that context.
Context is dynamic, imprecise, and not necessarily obvious, yet humans rely
upon context to create associations. By making environmental context information
available to programs, those same programs are able to present better options,
which \emph{is} convenient for humans.

%To improve the usefulness of naming systems within computer storage systems
%software must evolve to provide flexible, scalable, and multi-silo management
%of data object naming.


\section{Finding}
\label{ch:intro:sec:finding}

The volume of digital data is growing exponentially. Thus, it is not surprising
that solutions that worked when users were grappling with kilobytes (KB) or megabytes
(MB) of data do not work in the face of this growing deluge.

IBM's first magnetic disk drive could store up to 2.5
megabytes(MB)~\footnote{\url{https://www.7dayshop.com/blog/terabyte-evolution/}}
of digital data.  In 2020, we add 1.7MB of data \emph{per second per
    person}~\footnote{\url{https://techjury.net/blog/big-data-statistics/}}.  Today
we have become digital hoarders, collecting and keeping so much data that we often
cannot find specific objects when we need them.  How did we reach this point?

Early persistent storage systems used a simple flat directory structure
that gave a unique name to each object (``file'').  Such a simple structure was sufficient
to  name and identify distinct data objects (``files'').
\emph{Finding} the correct file was just a matter of scanning and picking from
the list.  This simple
structure did not scale well and was replaced by
a model based upon how paper documents were organized.
The hierarchical name
space~\cite{barnard1958,daley1965general,ritchie1973unix,Saltzer1978} is one in
which files (``digital objects'') are grouped into directories (sometimes called
folders).  Directories can also be grouped into other directories.  This model
mirrors how a filing cabinet works: multiple sheets of paper are gathered into a
folder, folders are organized into drawers, drawers into filing cabinets, filing
cabinets into rooms, etc. The directory and file metaphor was in use by
1958~\cite{barnard1958} and persists today as the common model despite the
volume of data being stored by a single computer storage device (``disk drive'') increasing by at least $10^6$~\footnote{Disk
    drives were measured in MB in 1965 and are measured in TB today.}.

In addition to the challenges of scaling, the file cabinet metaphor imposes
physical file limitations that are not valid for digital data.
Physical file cabinets do not allow a document to be in two folders at the same
time but there is no such restriction on electronic documents. The Multics
researchers addressed this by creating the \emph{link}, an idea that is
still used in many modern file systems~\cite{daley1965general}.

Computer networking enabled data sharing between users and computers but
complicated naming. Remote data access was typically represented to users either
as a hierarchical file system~\cite{nfs,howard1988scale} or an application program that programmatically
connected users to remote data~\cite{levin1979transport,10.1145/800216.806594,birrell1982grapevine}.

By 1990 the volume of data with which users interacted was so
large that researchers questioned the utility of the hierarchical name space~\cite{vicente1987assaying}.
The Semantic File System (SFS)~\cite{gifford1991semantic} suggested that
organization of digital objects be more fluid so that
users could group items together in ways that were semantically meaningful. The
meta-data generated from semantic information generated from file contents
permitted powerful query-based dynamic file organization.

While semantic file systems have not been widely adopted, the concept of
extracting semantic information from file content is present in modern file
indexing systems.  These indexing services employ ``transducers'' to extract
semantic information from files. Desktop search utilities (Windows
Search, Apple Spotlight, Station for Linux) rely upon indexing services to
provide their functionality.

\begin{figure}
    \centering
    \caption{First Wayback capture of google.com (google.stanford.edu) in 1998}
    \label{fig:google}
    \includegraphics[width=0.95\textwidth]{figures/1998-11-11-Google-Screenshot-showing-early-index-size.png}
\end{figure}

There are parallels between the challenges of indexing files and the challenges
of indexing Internet web pages. Early search engines used a curation model in which
humans decided what websites were of interest based upon the information within
the web page itself --- similar to the way that semantic information was
extracted from files in the Semantic File
System~\footnote{\url{https://www.hpe.com/us/en/insights/articles/how-search-worked-before-google-1703.html}}.

Internet web page indexing changed profoundly when two Stanford graduate
students proposed a novel way to exploit the structure of Internet web pages to
extract \emph{usage} information from web pages that did not depend upon
semantic content~\cite{page1999pagerank}.

Google's website used to state the number of pages that they indexed.  In 1998
the first capture of Google's website by archive.org shows they claimed to index
more than 25 million pages (\autoref{fig:google}). Google no longer publishes
that number but industry estimates indicate the number is at least $10^3$ more
now than it was then, and this only covers a few percent of the entire content
stored on the
Internet~\cite{bosch2016estimating}~\footnote{\url{https://www.worldwidewebsize.com/}}.

Could we utilize a similar technique for finding information within our own
trove of files? While there are similarities between the Internet and our file
collections, there are also significant differences. Files lack the level of common
structure present in web pages, preventing simple extraction of references
between files. Files (or digital objects) are stored in myriad
locations with different access mechanisms: local storage, cloud storage,
database, collaboration applications, e-mail programs, etc.  Sometimes these
overlap: your e-mail program stores some or all of your data on your local
computer, within your local storage. However, you do not expect to use the tools
for searching your local storage to find things within your e-mail software.
Thus, we should consider them to be distinct storage locations. I refer to these
distinct storage locations as \emph{storage silos} (or just \emph{silos}) to
emphasize their inherently separated nature.

Network storage is presented in many different formats: an inexpensive disk
drive attached to the local network represents ``Network Attached Storage''
(NAS) or a specialized parallel data cluster such as HDFS~\footnote{\url{https://hadoop.apache.org/docs/r1.2.1/hdfs_design.html}},
DAOS~\footnote{\url{https://www.intel.ca/content/www/ca/en/high-performance-computing/daos-high-performance-storage-brief.html}},
Lustre~\footnote{\url{https://www.lustre.org/}}, or
Ceph~\footnote{\url{https://docs.ceph.com/en/pacific/cephfs/index.html}}.
They typically support one or
more common data sharing protocols, such as NFS~\cite{sandberg1986sun} or
CIFS~\footnote{\url{https://docs.microsoft.com/en-us/openspecs/windows_protocols/ms-cifs/d416ff7c-c536-406e-a951-4f04b2fd1d2b}}.
They vary dramatically in how they are managed, accessed, and searched.  In most
cases there is no common interface --- each represents a unique ``storage
silo.''

Cloud storage is one specific type of network storage that is popular because it
allows you to access your data from any of your devices, provides a reliable
backup mechanism, and permits selective download to any given device. However,
these benefits are paired with challenges when it comes to finding specific
digital objects.  If files are not present on your local device, the indexing
services on those devices cannot assist you.  You could download all of the
content from the cloud storage providers to enable indexing, but that consumes
considerably more bandwidth and storage and is impractical for devices that have
resource constraints.  While we can use the cloud providers' search services, that
requires iteration over each of those services, using different interfaces with
variable results.

Our files come from multiple sources including websites, e-mails, databases,
and collaboration tools.  Those documents are stored both locally and remotely.
We create, access, and modify documents and then send them onwards using any of
the variety of silos and collaboration tools at our disposal.  Just a few days
after we last accessed them we struggle to find those
documents.

Given the diffusion of files across storage silos that occurs because of our
sharing and use, we often find versions and related files scattered across
multiple silos. We struggle to find these versions and determine when we have
found the ``right'' one.

Returning to the question of using contextual information for improving
\emph{finding}, the research community has observed that adding
contextual information, such as the current weather, to existing file
collections materially improves human ability to find the relevant digital
object~\cite{vianna2019thesis,10.1145/1559845.1559992,dumais2016stuff}. Thus, it
seems the answer is ``Yes, using contextual information improves
\emph{finding}.''

Google solves a simpler problem: an Internet web search need merely find
\emph{an} answer to the search query.  A personal file search needs provide
\emph{the} answer. Thus, it may not be possible to provide a definitive answer,
but narrowing the potential list of plausible answers leads users to the
relevant file, which is the goal of \emph{finding}.

The data that we need for creating activity context is already being collected
by our computers. Modern computers collect vast amounts of information about us:
what we do, where we are, with whom we communicate, the applications we use, the
files we access, the music we play, the web pages we visit, even how we
feel~\cite{chakriswaran2019emotion,8933554}.  We \emph{know} this
data exists because our own devices provide this information to third parties.
Given this data is already being collected, we know the additional cost will be
to store and make it accessible to applications.

\emph{Finding as a Service} (FaaS) will use this existing information to solve our data
finding problem. FaaS decouples \emph{finding} objects from \emph{storing}
objects.  FaaS facilitates \emph{finding} by exploiting contextual information
beyond the basic object characteristics widely available for searching today:
names, types, and dates. By relating information we already have with how our
digital objects are used, we provide the \emph{activity context} to enable
\emph{Finding as a Service} (FaaS).

\emph{Activity context} is important because it captures useful
information about the environment in which files are created,
consumed, and updated.  \emph{Activity context} need not be something the system
ordinarily relates to the digital object.  \emph{Activity context} captures key
information about the the user's wholistic environment.

\section{Use Cases}
\label{ch:intro:sec:use-cases}

The following use cases provide specific scenarios that cannot be
achieved using current systems.  I maintain that \emph{Finding as a Service}
addresses these use cases, which supports my thesis.

\begin{itemize}
    \item \label{use-case:e-mail}\textbf{The lost original.} Imagine that you received a spreadsheet
          from someone.  You begin to edit it, add information, sift through it.  At
          some point you realize that you sorted a subset of the columns, hopelessly
          scrambling the original information and your edits. You try to find the
          original source of the spreadsheet, only to realize that you cannot do so,
          even when you search in your e-mail program using the \emph{name} of the
          file that was saved on your local drive as part of your editing process. The
          system should permit you to find the original source of the information,
          even though the related digital objects are in different \emph{silos}.

    \item \label{use-case:misplaced-presentation}\textbf{The misplaced presentation.} You arrive at a meeting with your
          client after a long trip, only to realize the laptop computer you were
          using will not boot.  You have your smart phone and you \emph{think}
          you saved it to a cloud service.  How do you find it so you can share
          it with a colleague at the meeting? The system should permit you to
          find your own digital data in your cloud storage regardless of which
          device you used to create it.

    \item \label{use-case:multi-silo-problem}\textbf{The multi-silo relationship problem.} A colleague shares their
          experimental data with you, which was stored in NREL's
          High-Performance Computing Data
          Center~\footnote{\url{https://www.nrel.gov/computational-science/hpc-data-center.html}}.
          You then use that data as part of your own work, which you wish to
          share with a broader audience using Compute
          Canada~\footnote{\url{https://www.computecanada.ca/techrenewal/rdm/}}.
          You also shared your computational notebooks using your organization's
          account with Microsoft~\footnote{\url{https://visualstudio.microsoft.com/vs/features/notebooks-at-microsoft/}}.
          You created your slides in
          Prezi~\footnote{prezi.com} and presented them to a different research
          group on their Discord server~\footnote{\url{discord.com}}. One of the
          people that attended downloaded two of your notebooks and created a
          new notebook from them.  They then shared that notebook publicly via
          Google Colab~\footnote{\url{https://colab.research.google.com/}}.
          Your colleague knows that she shared her data with you and wants to be
          able to quickly find the documents that you and others have shared
          based upon that original data. Collaborative work like this is a
          modern reality and it is unlikely that all work product will be co-located.
          The system should permit your colleagues to find the work you and
          others have shared with her \emph{without} your intervention.

    \item \label{use-case:multi-silo-finding}\textbf{Multi-silo finding.} \persd is a visiting student from the
          country of Lemuria doing an internship with you in Camelot.  While arranging
          for this internship, \persd required \emph{numerous} different data objects:
          email messages with the host, offer letters, academic forms, a visa,
          boarding passes, project proposals, and more. The system should be able to
          provide you with a set of related files, regardless of their storage
          silo.

    \item \label{use-case:info-source}\textbf{The where did I get this information conundrum.}  In our modern
          world we often have one (or more) web pages open when we are authoring one
          (or more) documents.  A reasonable question to ask would then be ``what web
          pages did I look at while writing this document?'' Often it is not just that
          you looked at a given web page, but also if it was the last web page you
          looked at and how long you looked at it.  The system should be able to
          provide you with a list of that web activity.

    \item \label{use-case:privacy}\textbf{How do I share information while
              preserving privacy?} \persg, an investigative journalist who routinely receives
          sensitive information
          from third parties, is investigating the company from the prior use cases.
          \persg needs to be able store and access sensitive information, including
          information about the activity context of various e-mails, documents, pictures,
          and audio and video files. While \persg ensures that these data are
          encrypted, they need to also ensure
          that they can both find information and ensure that meta-data associated with
          those files is both usable and properly protected across silos.
          While \persg must protect their sources, they must also be able to associate
          evidence with those sources to make judgement calls about their validity.
          The system should support security and privacy policies for attributes that
          accomplish both.
\end{itemize}

This list of use cases is not exhaustive and is intended to provide cases that
resonate with readers.  They are use cases that are not addressed by existing
systems.  I review these existing systems and how they fail to address these use
cases in \autoref{ch:intro:sec:existing-solutions}.

\section{Existing Solutions Fall Short}
\label{ch:intro:sec:existing-solutions}

Prior work has addressed some of the challenges that I identified in the use
cases (\autoref{ch:intro:sec:use-cases}).  I briefly introduce key
aspects of how they fall short here and provide greater detail in
\autoref{ch:background}.

A simple solution to the multi-silo namespace challenge is to graft those
namespaces together. UNIX mount points~\cite{unix} are perhaps the first
instance of such federating namespaces. Distributed federation, as provided by
distributed file systems such as NFS~\cite{nfs} and AFS~\cite{howard1988scale}
emerged in the 1980s soon after adoption of high speed networks such as
Ethernet~\cite{digital1980ethernet}.

There is some work in cloud storage federated
namespaces~\cite{scfs,federatedMetaData}.
Nextcloud~\footnote{\url{https://nextcloud.com}} allows users to connect
multiple Nextcloud instances and integrate with FTP, CIFS, NFS and object
stores. This yields a classic hierarchical namespace structure with its known
limitations~\cite{vicente1987assaying,vicente1988accommodating}. It does nothing
to facilitate \emph{finding}. Peer-to-peer sharing networks (e.g., IPFS \cite{benet2014ipfs}) implement a
distributed file system where nodes advertise their files to users.
MetaStorage~\cite{metastorage} implements a highly available, distributed hash
table, similar to Amazon's DynamoDB~\cite{10.1145/2213836.2213945},
but with its data replicated and distributed across different cloud providers.
MetaStore offers a key-value store interface~\footnote{\url{https://cwiki.apache.org/confluence/display/hive/design}}.
Farsite~\cite{Adya:2003:Farsite} organizes multiple machines into virtual file
servers, each of which acts as the root of a distributed file system. Comet
describes a cloud oriented federated metadata service~\cite{federatedMetaData}.

None of these address the scaling problem that arises when data is analyzed and indexed
away from where it is stored.  Similarly, none of these address the integration
of sensitive locally stored information about personal usage of these objects.
Thus, one key benefit of better \emph{finding} is that
it should improve the efficiency of retrieving data stored across non-local
storage silos.

Some prior work explored using extrinsic usage information for \emph{finding}.
Placeless~\cite{placeless-tois} focused on using process level information extracted from their document processing
system to associate files together. Similarly,
Burrito~\cite{guo2012burrito} proposed \emph{activity context}, which they define as ``the
user's actions at a particular time.'' Both look at narrow instances of the
larger \emph{finding} problem.

Provenance uses observable information about construction of a file to augment file
search, which in turn improves findability~\cite{provsearch}. Provenance search
takes a narrow view of the activities of interest and are all largely
\emph{causality} focused.  However, humans tend to think
associatively~\cite{10.1145/1559845.1559992}, focusing on what else was
happening --- the cleaning crew came by their desk while they were writing that
document, the discussion at a meeting with others, their location when they
wrote a given document, or some other event that was happening around the time
they interacted with a given document.

While environmental information is not as obviously related as causal
relationships, prior work related to using statistical inference to establish
relationships within the storage domain has demonstrated such mechanisms can be
more efficient at identifying patterns that lead to higher
efficiency~\cite{10.1145/3035918.3064029}.

\section{Contributions}
\label{ch:intro:sec:contributions}

The research to support my thesis will contribute the following:

\begin{enumerate}
    \item Production of the \emph{Finding as a Service} (FaaS) dataset, a
          collection of meta-data and activity context from a local system, that
          will enable me to explore potentially useful information for informing
          activity context data collection. In addition, I will publicly
          share this data set to enable other researchers to develop new
          techniques for users to find digital items.

    \item \system~\footnote{\system is Xhosa for pragmatics.  In Linguistics,
              pragmatics is the study of meaning within a given context}, my
          architecture for a system that captures, stores, and
          disseminates \emph{activity context} without imposing excessive resource
          demand.

    \item \systemone~\footnote{\systemone is the Uzbek word for finding; the most
              recent graduate student from our research group is from Uzbekistan and
              has always been supportive of my research}, a single node
          implementation consistent with \system that provides
          \emph{FaaS} across multiple storage silos on a single
          system.

    \item \systemtwo~\footnote{\systemtwo is the Russian word for finding in
              recognition of the support for my research that I have received from both Ada Gavrilovska and
              Alexandra Fedorova.}, a distributed implementation of \system that provides
          \emph{Finding as a Service} (FaaS) across multiple systems using a
          combination of device private and cross-device shared storage
          silos.

    \item An evaluation demonstrating that it is possible to capture activity
          context without imposing excessive overhead, in either space or time.

\end{enumerate}

These projects focus on improving \emph{finding}.

The remainder of this document provides more specific insight into these
contributions and how I propose creating and disseminating them.
In \autoref{ch:background} I review the prior work that
underlies my thesis: what types of storage silos exist, what information we
already have available and why these are not sufficient to meet these use cases.
In \autoref{ch:research-questions} I set out the research questions that I
seek to answer to fully explore my thesis.
In  \autoref{ch:architecture} I describe the structure of the system I
propose building in order to support my thesis and how it addresses these use
cases.  In \autoref{ch:evaluation} I discuss how I propose evaluating
my system.  Specifically I attempt to address key questions, such as: ``how well does it address these use cases?'', ``what are the performance
and resource implications of using my system?'', and ``how well does it enable other
communities to construct more effective finding tools?''

\endinput

\section{Meeting Notes}
\tm{This section is to be removed.  It is my recording of notes from the meeting of October 19, 2021 with Margo and Sasha.}

The vast majority of prior work has focused on using properties that are
intrinsic to the file to facilitate search and naming and in large part this
project is about looking outside the file and including environmental
information which we call ``activity context.''

That leads to the story line that says if we look at how we have both

Differentiate \emph{naming} from \emph{finding}. We've always been talking about
this as \emph{naming} and I'm no longer convinced that we are solving a naming
problem, I feel that we are solving a \emph{finding} problem.

There's this nice linear history.  We start with the hierarchical name space.
It was a way to organize digital information in a manner that reflected physical
data organization.  Gifford points out that the digital world offers us a richer
space and therefore when we consider these intrinsic attributes we can
facilitate even better finding.  There have been \emph{hints} that expanding to
looking at factors outside the file might provide an improved experience.  That
is Guo [Burrito] and Soules [Provenance Search].  We are taking that hint of an
observation and broadening it into something we all ``activity context'' which
is about the environment in which an object exists and here are examples of the
kinds of queries that we want such a system to be able to satisfy.

\item The prior work suggests context helps (cite Soules).  We want to
enable HCI researchers to ask the broader question of whether broadening the
definition of environment [activity context] also helps.  We do not want to
\emph{answer} that question --- we want to enable a system that permits the
HCI research community to explore it.  To do that, we need to identify how
you get this information.

That then begs several research questions:

\begin{itemize}

    \item What events are useful in establishing relationships between objects?
          I might suggest the question you want to ask is ``What is the universe of
          relationships that I can extract and how can I do that?''  Then we can ask
          to the HCI researchers to ask them which would be of interest to them.
          ``What is the universe of relationships that we could extract?'' and ``what
          are the techniques that we need to develop to extract them.''  At some point
          we have to introduce this multi-silo world, but let's hold off.

    \item We would like to build an extensible framework that allows you to add
          all these things.  What we have discussed in the past is another mode for
          getting this internal information, which is ``I pipe it through the ML
          service, which indicates if there is a cat in it, or not.'' [Note that in
          the extrinsic view, this isn't interesting because the ML classifier is just
          another form of transducer].

          Your description of what one needs to do is largely spot on (from the
          spreadsheet).  I would hesitate to call this ``the thesis question''; as a
          systems researcher identify all the things that I can collect already and
          then develop a framework that makes it easy to add other things and then
          another piece where we bring them together in the multiple silo thing.

          Notice you can in fact do a bunch of this in a single silo; you can do this
          for a single system by broadening Guo and Soules work and saying ``ok we
          really want to capture''.  I'd go back and look at Guo carefully that even
          in the single systems world I think his work captures all the things we're
          talking about but it is worth going over it carefully and making sure of
          that.

              [Sasha typed] ``The current silos are conventional ones can we think about
          something that one might not think of as a silo?''  Discord, slack, e-mail
          are all silos.  \emph{Those are frustrating silos}. ``What I was browsing at
          a particular time.''  Two different dimensions here: the different kinds of
          silos and

          In the background chapter it is worth enumerating all these different silos
          (the ones we can think of) and going from obvious to crazy.

          The other dimension, the kinds of environmental factors we might want to
          think about.  What was I browsing, what was I listening to, what other
          documents were open, did I move the browser tab to a different window [how
                  long did I spend looking at the given tab - e.g., ``how long did this tab
                  have focus'' for example].

          Do we try to capture everything?  Winnowing process.

          Sasha: I have a thought on how this could easily turn into a systems
          problem.  We could try to categorize alll kinds of crazy silos,
          applications, activity contexts with respect to how we can collect their
          information.  For example, conventional things like chat and discord we just
          go into their meta-data.  Things like file systems we go into their files
          and file names.  More crazy things like ``what I was browsing'' we might
          need to capture screen or capture charactters.  Have a categorization of
          techniques for collecting activity context such as browser search and meta
          data versus capturing the screen versus something else. The categorization
          of silos and see how they match.  Which silos work with which methods.
          Where do we need to invent new methods.  This then leads to the question
          like the amount of data and parsing it and storing it and that quickly turns
          into a systems problem.

          ``What is the information?''
          ``How can we capture it?''
          ``What is the service that unifies this information and makes it
          available/usable?''



\end{itemize}

This is the diagram from the white board:

\begin{itemize}
\item Introduction
\begin{itemize}
    \item Linear history
          \begin{itemize}
              \item HNS
              \item SFS
              \item TFS [Tony FS].
          \end{itemize}

    \item Background
          \begin{itemize}
              \item Silos (Taxonomy).  Classification of silos that let you collect data.
                    \begin{itemize}
                        \item What can you collect?
                        \item How can you collect?
                    \end{itemize}

                    We need HCI feedback on this information.  Helps us focus our initial
                    data collection effort to focus on selecting data that is useful to
                    them.

                        [Margo] Asking them what they want means we are limited to the things
                    they can imagine.  Thus, we need to have a universe of things that we
                    can offer them. [TM: there are integration components already for quite
                            a range of applications. E.g., discord integrates with games, music
                            players integrate with other stuff as well]
          \end{itemize}

    \item Architecture

          How do you integrate info into a service?  ``Context as a Service'' (CaaS).
          This is Kwishut so start by lifting that work.

    \item Privacy/Security

          We agreed that \emph{security} is not really the core issue for this system:
          we aren't subverting the existing system, we aren't proposing some bypass
          for the existing system.  Thus, the security consideration is largely about
          protecting our own meta-data.  There are standard techniques for doing
          authentication that can be used for connecting from a client to a trusted
          server, there is no need for us to re-invent the wheel.  By deliberately
          designing the system so that activity context information is maintained on a
          trusted machine, whether local or remote, we can assert that it is \emph{by
              design} secure.  Similarly, by restricting access to only a trusted machine
          the system ensures privacy as well since only authorized agents can access
          the activity context.  This does not exclude the possibility of having
          different architectures for data sharing.

          Worrying about securing the system at this point is premature: a generalized
          multi-tenant activity context system could be useful for correlating
          activities across users and systems, but is premature until we have
          established that activity context is itself useful.

              [Shasha] I wonder if we are inventing a new type of privacy paradigm on a
          local device.  What we have right now is when I'm setting up my new
          device each app asks me what is allowed or not allowed.  Here, I want
          something else: I want to allow it as long as the information stays on
          the device and is for my personal use, or allowed by certain
          applications, but I don't allow it to leave the device, or to allow it
          to be exfiltrated but only if it is encrypted.

          Kosta in ECE usable privacy/security.  He will probably be able to
          point to related work and decide what to do with it.  Use it to
          prepare for responding to the issue.

              [Margo] extract from the draft, get it to him before the presentation
          could have more information to share at the presentation.

    \item Evaluation

          We show this data is useful because the HCI researchers have confirmed it is
          useful.  Now the evaluation will fall out: lots of data, difficult to gather
          it, how you store it, model it, query it.  How to build this thing and make
          it useful.  Each one of these items will have an evaluation piece.

          How do I enable query?  Here's the interface I'm going to do and here
          is how that addresses these N use cases that we presented that you
          can't do in other systems.  These are use cases that we hypothesize
          are useful and you can't do them today and here's how you do them with
          this proposed system.  The overhead: massive amounts of data, how do
          we manage it.  What is the size, what is the performance impact.  How
          easy is it to add another silo?  These metrics can be used to evaluate
          the system we're proposing to build.

          The efficacy comes down to ``assuming that we have got some HCI
          researchers off looking at the data we collected initially, by the
          time we have built a system we ought to be able to just re-run that
          evaluation across multiple silos with this distributed data.''

          The storyline for the thesis has to be this is about building a system
          to facilitate this other world.  We are hoping that we will have
          collaborators to do that and are actively pursuing them.  My question
          is ``assuming it is useful, can we even build such a system that will
          enable that?''

\end{itemize}

[Sasha]  Write the intro and background from scratch, watch the recording,
transcribe what you said and you'll have most of the background.




\section{A Modest Proposal}\label{sec:graphfs}

% Because our focus is on the \textit{naming} system and not the storage system,
% for the present time we will not consider the issues that will arise in the
% storage management layer to support the new model, though we admit that there 
% are likely to be concerns that will need to be addressed in future work.

Our proposed file system focuses on \textit{relationships} between our files.
We use an analogy between social graphs and file systems to explore this 
approach.
Facebook's graph is a collection of typed objects
(e.g., users, actions, places) and associations (e.g., friend, authored, tagged).
File system objects map to users and files; \textit{contains} is, perhaps, the only
association captured in a file system.
For the rest of this discussion, we treat directories as the embodiment of the
\emph{contains} relationship, not objects.

We consider two strawman implementations for elevating relationships to
first class file system objects.

\subsection{File System as a Graph Database}

In considering a graph based file system, first we consider
implementing a file system in a graph database, of which there are
many (\S \ref{sec:background}). Their primary focus is the storage of 
graph-structured \textit{data}.  Our focus is in the use of graph-structured
data as critical meta-data inside of a storage system.
As such, there is a mismatch in design targets between a file system and
existing graph databases: nodes in graph databases are small; nodes in
file systems are large. Graph databases tend to favor a navigation-based API;
file systems need a point query and search API. Graph databases assume that
attributes and relationships are provided; file systems will frequently derive
attributes and relationships.
These differences suggest to us that existing graph databases are not suitable
as the basis for file systems.
Nonetheless, we encourage others to consider such an arrangement, should they
have compelling reason to do so.

\subsection{File System as Social Network}

Next, we consider implementing a file system in the same way Facebook
implements their social network graph.
Facebook's original implementation stored the social graph in MySQL, queried
it from PHP, and cached the results in memcache.
More recently, Facebook introduced Tao, which is a service that more directly
implements the fundamental objects and relationships that comprise the
social graph~\cite{bronson2013tao}.
While Tao is specifically designed to support the widely distributed,
replicated, and rapidly changing social network scenario, it provides the
starting point for conceptualizing a data model premised on the primality of
relationships.
Tao stores both objects and associations in a MySQL database and presents
the graph abstraction via Association and Query APIs in the caching layer.
Is this a viable structure for a file system?

% in fact, facebook has a separate storage system for videos and pictures 
% because they are large.
Unlike objects in Facebook, files are large.
Although prior work has considered using relational
database~\cite{olson1993design} and other index-based
structures~\cite{spillane2013vttree}  to store files,
the community seems to have
concluded that such storage is not ideal. We agree, suggesting that
an RDBMS is not the desired storage system.

What about relationships? Is it appropriate
to use one persistent representation (e.g., a relational one) and a second
memory representation (e.g., a graph-structured on) or
should we use a single graph-structured representation both in persistent store
and in-memory.
We propose the latter for two reasons.
First, the rumored era of non-volatile main memory seems to be around the
corner, so a modern file system design should embrace a single
representation.
Second, while it is reasonable for Facebook to construct the entire graph in
a distributed pool of main memory, file systems must work on a more limited
scale and therefore cannot ensure that the realized graph structure will fit
in main memory.

As neither strawman design seems suitable for our relationship-centric file
system, we present a new model and file system design.

\subsection{Graph FS Model}
\label{sec:graphfs:model}

We set out a basic description of our core objects in Table \ref{table:graphfs:terminology}
%\footnote{We took inspiration for this model from https://github.com/opencypher.}
and a demonstrative set of example relationships in Table \ref{table:relationship-examples}.
We do not consider either of these to be exhaustive, but rather propose them as an initial
basis for discussion.
The presented model can encompass
the functionality of the existing hierarchical file system model.

% https://github.com/opencypher/openCypher/blob/master/docs/property-graph-model.adoc

\begin{table}[h]
    \captionsetup{justification=centering}
    \begin{tabular}{p{2cm}p{5cm}}
        Term                          & Definition\tabularnewline\hline
        \multirow{1}{*}{file}         &
        \multirow{1}{*}{\parbox{4.8cm}{Uniquely identified storage unit}}
        \tabularnewline
        \multirow{1}{*}{relationship} &
        \multirow{1}{*}{\parbox{4.8cm}{Directional file association}}
        \tabularnewline
        \multirow{1}{*}{labels}       &
        \multirow{1}{*}{\parbox{4.8cm}{A binary attribute, e.g., executable}}
        \tabularnewline
        \multirow{1}{*}{property}     &
        \multirow{1}{*}{\parbox{4.8cm}{Key/Value attribute}}
        \tabularnewline
    \end{tabular}
    \caption{Graph File Systems Terminology}\label{hotos19:table:graphfs:terminology}
    %    \Description{Graph File Systems Terminology}
\end{table}



\begin{comment}
% Omit because I don't think it is central to our thesis.
Our relationship-centric file system consists of a \textit{union} of distinct
name spaces, which permits a name space that is not
fully connected, yet can still be merged via the existing union model. This permits an
application to have a private name space for files,
similar to both CAP and Plan 9~\cite{needham1977cap,pike1992use}.  Isolation also
enhances security, e.g., the POSIX \textbf{mkstemp} function, which was introduced to
address the need of applications for temporary files not visible to other applications.
Such namespaces are like unlinked but open files in UNIX, where such files are
reclaimed when the process terminates.  By creating a process private ephemeral namespace
we augment the file system's ability to provide enhanced isolation.
\end{comment}

Every file has a unique identifier, such as a \textbf{UUID}, similar to
an inode number or object ID.
We do not rely upon \textit{names}
as they are simply mutable properties.

\begin{table}[h]
    \begin{tabular}{p{1.9cm}p{5.5cm}}
        Relationship                           & Description\tabularnewline
        \hline
        %        \multirow{1}{*}{\textit{is}} &
        %        \multirow{1}{*}{\parbox{5.4cm}{Attribute of a file, e.g. size or timestamp}}
        %        \tabularnewline
        \multirow{1}{*}{\textit{similar}}      &
        \multirow{1}{*}{\parbox{5.4cm}{Similarity measure, e.g., \cite{masci2014multimodal}}}
        \tabularnewline
        \multirow{1}{*}{\textit{precedes}}     &
        \multirow{1}{*}{\parbox{5.4cm}{temporal relationship (e.g., versioning)}}
        \tabularnewline
        \multirow{1}{*}{\textit{succeeds}}     &
        \multirow{1}{*}{\parbox{5.4cm}{temporal relationship (e.g., versioning)}}
        %        \tabularnewline
        %        \multirow{1}{*}{\textit{located}} &
        %        \multirow{1}{*}{\parbox{5.4cm}{link or url}}
        \tabularnewline
        \multirow{1}{*}{\textit{contains}}     &
        \multirow{1}{*}{\parbox{5.4cm}{directory/file relationship}}
        \tabularnewline
        \multirow{1}{*}{\textit{contained by}} &
        \multirow{1}{*}{\parbox{5.4cm}{directory/file relationship}}
        \tabularnewline
        \multirow{1}{*}{\textit{derived from}} &
        \multirow{1}{*}{\parbox{5.4cm}{provenance (e.g., .o to .c)}}
        \tabularnewline
    \end{tabular}
    \caption{Graph File System Relationship Examples}\label{hotos19:table:relationship-examples}
    %    \Description{Graph File System Relationship Examples}
\end{table}



A \textit{relationship} is a directional association between two files.  We expect there
to be far fewer relationships than files, though many more \textit{instances} of
relationships (i.e., the number of edges in our graph exceeds the number of vertices).
Relationships may be either uni-directional (e.g., derived from) or
bi-directional (e.g., similar).
Table \ref{table:relationship-examples}
provides a set of sample relationships; the universe of relationships
is extensible.
As in RDF, relationships are triples: two files and the relationship.

As files have attributes in a conventional file system, both files and
relationships have attributes in a graph file system.
A \textit{label} is a simple binary attribute (e.g., executable),
while a \textit{property} is an arbitrary name/value pair, much like
an extended attribute, but they are native to the model, not
an afterthought.

% We will extend our terminology as needed, using the existing terminology of the
% relationship graph as inspiration for usable models.

\subsection{Interface}\label{sec:graphfs:interface}

\begin{table}[b]
    \small
    \captionsetup{justification=centering}
    \begin{tabular}{p{2cm}p{5cm}}
        Operation & Description\tabularnewline\hline
        \multirow{1}{*}{create} &
        \multirow{1}{*}{\parbox{4.8cm}{Insert new file into graph}}
        \tabularnewline
        \multirow{1}{*}{relate} &
        \multirow{1}{*}{\parbox{4.8cm}{Insert new edge into graph}}
        \tabularnewline
        \multirow{1}{*}{label} &
        \multirow{1}{*}{\parbox{4.8cm}{Insert new labels}}
        \tabularnewline
        \multirow{1}{*}{set} &
        \multirow{1}{*}{\parbox{4.8cm}{Insert new properties}}
        \tabularnewline
        \multirow{1}{*}{remove} &
        \multirow{1}{*}{\parbox{4.8cm}{Remove something from graph}}
        \tabularnewline
    \end{tabular}
    \caption{Graph File Systems Operations}\label{table:graphfs:operations}
%    \Description{Graph File Systems Operation Examples}
\end{table}


One of the lessons from Plan 9 is that everything can be represented as a file~\cite{pike1992use};
we expect to
continue with this paradigm as it has served us well over the years.  While we generally
think of files as a blob of \textit{persistent} data, in fact it is useful to
think of them as abstract \textit{generators} of byte stream data.  This fits well
with our model of separating namespace from storage; how the storage
providers return data to us is orthogonal to the namespace we use to retrieve it.
For example, the \textit{procfs} file system creates a synthetic namespace and supports
I/O operations for reading and modifying data contents of the pseudo-files.

From the namespace perspective, our file system must support operations that manipulate that
namespace. This includes the ability to create files, relationships,
relationship instances \textit{between} files, labels, and properties.
Similarly, we need the ability to remove each of these.

Our model is simple, yet powerful.  It captures interesting concepts such as versioning, using relationships such as
\textit{precedes} and \textit{succeeds},
and provenance, using relationships around derivation and use,
and application specific relationships, such as \textit{indexes} so a database
system can expose the relationship between its primary data and the
corresponding index files.
Although relationships are binary,
we can create clusters of related files by asking for all the vertices connected
by a specific relationship.

Where do relationships, labels, and properties come from?
We identify at least the following five sources:
1) the system itself will generate traditional
attributes (e.g., \textit{size}, \textit{read time}) and some
relationships (e.g., contains);
2) tools that extract meta-data from different file
types~\cite{soules2004toward,bloehdorn2006tagfs} will produce more attributes;
3) applications will generate both attributes and relationships;
4) users may generate attributes and relationships, although history
suggests that asking users to annotate data is a losing
proposition~\cite{soules2003can}; and
5) kernel extensions, e.g., provenance tracking systems~\cite{pasquier17camflow}
will generate attributes and relationships.

Several interesting possibilities emerge from this design.
Hard links are multiple \textit{name} properties attached to the
same file, potentially in different namespaces.
Soft links are a relationship between two names.
The system can capture relationships that extend beyond the file system.
For example, the \textit{derived from} relationship from
Table \ref{table:relationship-examples} might describe a file that came
from a particular email or web site.

\begin{figure}[bt]
    \captionsetup{justification=centering}
\includegraphics[width=0.9\linewidth]{figures/model-graph.eps}
\caption{Graph File System}\label{fig:graphfs-example}
%\Description{Simplistic Graph File System Picture}
\end{figure}

Figure \ref{fig:graphfs-example} provides a simplified visualization of our graph file
system model.
Our inclusion of disjoint graphs captures the notion that the system
naturally supports multiple namespaces, implemented as disconnected graph
components.

\begin{comment}

\subsection{Search Functionality}


Motivation for our graph file system was to provide a rich set of search semantics
to enable more effective human-usable search.  Search starts from an arbitrary 
file and radiates out from that point: what we seek is often ``close'' in terms of
our relationship graph. We can return results quickly without doing
an exhaustive search of the entire graph.  This is similar to how a modern internet search 
engine, such as Google, works. Only the first few hundred results returned from millions
of entries returned are usually analyzed.

Our file sytem can support an iterative interface in which it asynchronously constructs
the search results, returning initial results quickly even as further results are constructed.
Bounding search depth combined with resuming search as needed permits the iterative process
to continue without overwhelming the system. We seek to balance performance and completeness. 
Indeed, our expectation is that --- much like your search engine never returns the 7,345th
page of search results, the file system will seldom be asked to search the entire graph.  
This is a far cry from the brute force search that some of us now do when we use \textbf{find} 
or \textbf{grep -R} to scan through the hundreds of thousands of files that we have managed to
accumulate over years of work.

There is considerable work already done in the area of graph searching; we expect to utilize
that work to construct a viable search 
interface~\cite{angles2018g,,rudolf2013graph,francis2018cypher,van2016pgql}.  
This aspect of our graph file system is likely to evolve the most as we better understand the
needs of applications, particularly those seeking to expand the human level searchability, and
adapt the search interface to meet their needs.
\end{comment}

\begin{comment}

Current work in labeled property graphs has recently moved to introducing \textit{paths} to the model
as well as the labels and properties that have already been added to the basic graph model~\cite{angles2018g}.
This prior work identifies five key features that are heavily utilized in searching such graphs: 
\textit{reachability}, \textit{construction}, \textit{pattern matching}, \textit{shortest path search},
and \textit{graph clustering}.  For file systems, we expect that some of these will be of less
interest than others, but we think it premature to exclude any of these from a useful design, though we expect to 
focus on those we view as providing the greatest utility for our prototype implementation work. Thus, we expect
the greatest utility for file systems will be \textit{pattern matching} initially, though we find both
\textit{reachability} and \textit{shortest path} potentially intriguing.

Pattern matching makes the most sense in terms of searching for specific labels and properties of files --- this most closely
matches with the types of searching that indexers are already performing.  Thus, from the perspective of novel new
search models it is the least satisfying yet likely one that will still provide substantial utility.

Reachability is intriguing because it meshes well with our model for temporal relationships, in which we define ``reachability''
with respect temporal reachability.  POSIX co-location (e.g., part of the same directory) is another potential area, though
we suspect there are further areas to explore, including similarity.  For example, similarity preserving hashing~\cite{masci2014multimodal}
could be used to search for files that are similar, but not identical to one another.  This could be used to trace the evolution
of files as they are copied and modified from one location to another as well as to identify files that are identical.  Similarly,
finding the set of paths between two vertices could yield useful information about the linkage of files, something that we
cannot easily reconstruct from current file systems.

Shortest path is a classic search problem (e.g., Dijstra's Algorithm~\cite{dijkstra1959note}) as a means of finding the closest relationship
between two files.  This could be combined with reachability to identify a \textit{sequence} of iterative steps as we move between
similar files.  Another use would be to find a path between a directory and a file that provides access, e.g., for which the security label
on all the edges permits access.  This works well with the ``open by inode number' problem, for example and may help us address
the interesting question of how to perform POSIX-compatible security checks on files that are being opened by their identifier.
\end{comment}

\begin{comment}
\subsection{POSIX}

Our model permits support for POSIX file system semantics with respect to the file system
name space.  While our model does not require an hierarchical layout, we can support
such a system within this model.

First, we assume there is an external mechanism for identifying a distinguished file that
corresponds to the \textit{root} directory.  We added the \textit{contains} relationship
to capture this form of relationship.  While we suggested the \textit{contained by} relationship
(Table \ref{table:relationship-examples}), this is not a requirement for POSIX compatibility,
though one of the authors has been asked for this feature numerous times in the past for
existing POSIX file systems.

Enumerating the contents of one of these \textit{directory} type files would enable
enumeration of the directory in keeping with the POSIX model, by simply querying the
edges capturing the \textit{contains} relationship.  A hard link from two directories
to the same file is trivially supported.  Soft links are already a part of our model, so
we can provide symbolic links as well.

POXIS APIs that retrieve file properties, such as \textbf{stat}, are similarly converted
into a call to retrieve properties of the file and then converting them into the
expected native format.

While our own graph file system does not inherently place any importance on names, this
POSIX layer can store names as a property of the edge.  This is consistent with the way
that file names now function --- they are an attribute of the directory entry that
references the file, rather than an attribute of the file itself.
\end{comment}

\begin{comment}
\subsection{Interactions}

Eventually, we expect there will be challenges associated with the POSIX interface and the richer interface we offer.  We have already mentioned
one such issue: security.  The POSIX specification expects a security check at each node along the path as part of an attempt to open or create
a file.  Windows will enforce the same path walking security model but provides a work-around that allows skipping this~\cite{conover2006analysis}
check in favor of performance.  The ability to perform such a security check at open time could be an interesting systems use of enhanced
searchability.

One area in which we can envision challenges is in preserving the labels and properties of files within our graph because standard utilities
such as \textbf{cp} are ignorant of them and would simply copy the data contents of a file without copying the labels and properties.  We can
certainly enhance our system utilities to be enlightened about these new features, but it will not resolve the problem for cases in which
existing applications use standard POSIX systems calls to copy a file.

In addition, the ability to create disconnected name spaces does not fit well within the POSIX model, though it is more like distinct file systems and thus
likely does not break existing semantics.  Still, we expect that as we gain further experience with our graph file system and the graph
features are exploited we will need to handle more interaction cases.
\end{comment}

\begin{comment}

Our proposal is that file system name spaces evolve from \textit{trees} to \textit{graphs}.  Of course, a tree is simply a minimally connected graph.  This leads
to our first question for our new file system model: \textit{should it be a connected graph?}  In fact, we observe that even our personal name space is a set of
disjoint graphs --- there is one for each of our devices.  Thus, we observe that this question is already answered.  However, having answered it, we also
observe the utility of actually creating connections across those name spaces.

This becomes easier when we divorce the concept of \textit{storage} from the concept of \textit{name space}.  Our ideal name space can thus refer to files
that are not part of the current device's name space.  We are not (yet) proposing how we add that information to the namespace, merely noting that it
is a desired feature.

One argument for allowing disjoint namespaces on our individual devices is to provide a simple model for namespace isolation --- files that need not be
made visible outside some limited scope (e.g., the temporary working files of a particular application) can be in what is a localized name space; in some
ways this is reminiscent of capabilities based file systems~\cite{needham1977cap}.  While an interesting area to explore further, we leave it open
at the present time for our design.

Our graph is essentially a \textit{relationship} graph, in which vertices represent what we usually think of as \textit{files}, and edges represent
the \textit{relationships} between our vertices.  In an attempt to emphasize the generality of our vertices, we note they may be able to supply a byte
stream of data --- a \textit{generator}, and they may be able to consume a byte stream of data --- a \textit{processor}; we leave it open if we wish
to consider richer interfaces, such as \textit{insert}, such as we previously proposed~\cite{Seltzer2009}.  An individual vertex may be any subset
of generator and processor and thus have read-write, read-only, and write-only vertices as suits the needs of the specific vertex.

Vertices may have a set of \textit{attributes}.  Since our goal is to provide generality here, we would establish that each attribute consists of a
\textit{domain} which identifies how to interpret the attribute, the \textit{attribute-name}, which specifies the name of the specific attribute
within the given domain, and a \textit{value}.  Our system can then define a core set of attributes relevant to the file system domain; other domains
can then be added by agreement of the domain creators and users without us enforcing any given structure.  While individual applications can then
create labels within their own domain, our hope is that over time related applications will come to an agreement as to the meanings of attribute-names
to enable sharing of properties.

Our edges are labeled, weighted directional edges in a hypergraph: each label consists of a \textit{relationship} and a \textit{weight}.  Because our
goal is not to strictly restrict the relationships, we suggest that a relationship itself consists of a \textit{domain} identifying how to interpret
the relationship (presumably by agreement for those creating those relationships) and a domain-specific \textit{relationship-label}.  We can define
a set of relationships that we will initially support: \textit{contains}, \textit{is contained by}, \textit{is related to}, etc.

Thus, we now have enough structure for us to create a mapping from the existing POSIX file system interface, including its hierarchical name structure
to be compatible with existing systems. 
\end{comment}

\subsection{Aspects of Implementation}

In the absence of space to provide a full implementation, we offer a few
strategies that make a graph file system both feasible and novel.
The underlying storage structure for files is essentially an object store~\cite{factor2005object} and
attribute storage is largely a solved problem
(although the last time
one of the authors said that, her colleague disagreed~\cite{mao2012cache}), so we
focus on fast and efficient graph storage and query.

Today's systems either provide graph storage~\cite{rudolf2013graph,webber2018programmatic,microsoft:cosmosdb} 
or graph processing~\cite{shun2013ligra,gonzalez2014graphx,malewicz2010pregel,salihoglu2013gps,nguyen2013lightweight,low2014graphlab,kyrola2012graphchi}, 
but a graph file system needs a high performance, space-efficient, mutable and queryable
native graph representation.
We have found that mutable compressed sparse row representations~\cite{macko2015llama}
meet all these requirements (we used them as the query and storage mechanism in the
SHEEP graph partitioner~\cite{margo2015scalable}).
Just as high performance key/value stores are considered reasonable implementation
strategies for attribute storage and management in file systems, similarly efficient
structures supporting graph storage and management should be adopted in file systems
as well.


\section{Search}\label{sec:search}

The driving force behind our graph file system design is to provide the
infrastructure to make it easier for users to find data.
Users do not navigate to data, they \textit{search} for it, so we
consider more effective search models to further
motivate the graph file system.

We observe that there are two different models of ``search'': application
search and user search.
Applications need to be able to open files quickly 
using a \textit{key}. For example, both NFS~\cite{sandberg1986sun}
and AFS~\cite{sidebotham1986volumes} use the file system \textit{inode number} as their
mechanism for identifying the specific file or directory being accessed,
because it is fast, avoiding a costly namespace traversal.
Similarly, NTFS supports the ability of applications to open a file by 
identifier~\cite{sreenivas2011bypass}.
They did this to support their implementation
of the Apple File Protocol (``service for Macintosh'') but has subsequently been used
for other uses. Indeed, it has been further extended to permit files to be opened by an
application-defined identifier (a UUID); Microsoft continues to support file IDs in 
ReFS~\cite{microsoft:refs:features}. The Google File System~\cite{Ghemawat2003} 
observation was similar: applications can use keys to find their files.

Modern applications tend to either create files that they use internally, often going to great lengths
to hide their location from the user; or maintain a list of recently used items with a full path name,
which breaks when the path changes, even if the file did not change.  A key interface for applications
better fits this usage model. Thus, a ``search by key'' interface is sufficient.

The more challenging problem is user-focused search.
\begin{comment}
Many of the
characteristics in a good human usable search system do not benefit the programs
directly.
\end{comment}
For human users, we want to enable a model like the
\textit{memex}~\cite{bush1945we}: ``A memex is a device in which an individual stores
all his books, records, and communications, and which is mechanized so that it may be
consulted with exceeding speed and flexibility. It is an enlarged intimate supplement
to his memory.''

The HCI community has a long history researching
more effective search, including such efforts as
SIS~\cite{dumais2016stuff} and faceted search~\cite{arenas2016faceted,tunkelang2009faceted,hearst2006design,klungre2018evaluating,walton2017looking,cleverley2015retrieving}.
Critical to this work is the idea that search is most effective
when \textit{not} bound to a specific taxonomic order --- very much the opposite of today's 
hierarchical search model, which enforces a rigid
order on the structure of information.

How does a graph file system then enable modern search?
First, support for a broad and extensible set of attributes and 
relationships brings search engine technology to bear in the
service of file systems.
There is some irony that the success of web search, and in particular the
primacy of relationships in those algorithms~\cite{page1999pagerank}, has had
virtually no impact on how we find our own local data.
Second, the generalized graph structure, which no longer elevates any single
organzation gets rid of the \textit{specific taxnomic order} that HCI
researchers determine to be counterproductive.
Third, although some degree of temporal query is possible using \texttt{find},
its interface is not especially accessible to the typical user, and it requires
a series of manual operations to express natural queries such as
``Show me the documents I wrote last summer after I got back from my Amazon
rafting trip.''

Our goal is not to specify the entire range of searches that can be realized, but rather to
explore file system structures that enable the creation and mining of relationships to help users to find relevant data.

\begin{comment}
To help motivate our work, we consider the \textit{Graph Query 
Language}~\footnote{https://gql.today/wp-content/uploads/2018/05/a-proposal-to-the-database-industry-not-three-but-one-gql.pdf} 
(GQL) as a starting point.  GQL is an emerging language
attuned to the needs of \textit{property graphs}, which happen to be similar to the model that we envision for our new
file system.  It attempts to merge the strengths of three existing graph database query languages into a single, standard,
query language for property graphs~\cite{van2016pgql,francis2018cypher,angles2018g}.

% This choice is motivated by our realization that the new model we propose is a property graph~\cite{rudolf2013graph}.
\end{comment}


\section{Related Work}\label{sec:background}

The need for better name spaces in file systems is hardly a new
topic, with many solutions being proposed and implemented over
the years.

\textit{Search utilities} are successors to the permuted index program.  They permit us to find
files based on \textit{content} and \textit{attributes}.  MacOS X has \textit{spotlight}, which
provides an extensible, index-driven search service~\cite{apple:spotlight-extensions}.  Similarly,
Windows offer a similar extensible service~\cite{microsoft:data-add-in}.  These enable searching
based upon attributes, e.g., file suffix, date, size, etc., and context-sensitive content, e.g.,
music files by artist, composer, song title, or even \textit{rights},
but limited, if any, ability to search by relationship.

%.  A number of other examples
%of similar search mechanisms
% exist~\cite{Suguna2015,huo2016mbfs,leung2009magellan}.

%\textit{Databases} enable developers willing to pre-define their data's structure to enable searching
%on the specifics of the data.  Databases come in a rich array of models: relational, column, document,
%and graph, for example.  File systems constructed from databases have been extensively 
%explored~\cite{olson1993design,balabine1999file,balabine2002database,kashyap2004file,murphy2002design}.
%As we noted previously (\S \ref{sec:graphfs:model}) such approaches have failed to yield clear
%results.

\textit{Tag Systems} were an early approach to improving hierarchical file systems searchability%
~\cite{Parker-Wood2014,chou2015findfs,ma2009file,laursen2014,nayuki2017,Andrews2012,Up2016,Jones2016,aws:s3:object:tagging,ames2006lifs,leung2009magellan,frieder2012hierarchical}.
Automatic tagging systems have become a more common approach here as manual tagging by users 
has proven to be impractical~\cite{soules2003can,soules2004toward}.
The addition of \textit{semantic} information
~\cite{di2017gfs,hua2016real,martin2004formal,Martin2005,martin2008,martin2014,gifford1991semantic,Faubel2008,harlan2011joinfs,Suguna2015,Andrews2012,ngo2007integrating,Omvlee2009,wang2003managing,gopal1999integrating,Codocedo2015,Jones2016,Mahalingam2003,Parker-Wood2014}
is useful but falls short of addressing the fundamental need to understand
data relationships, because like more conventional systems,
these semantically-aware approaches still focus on the file,
not on the relationships between the files.
As such, they are simply an add-on to
the hierarchical model, not a replacement for it.
Such approaches can provide useful functionality
in our graph file system model as well.
Indeed, we even pointed this out (perhaps subconsciously) in prior
work when we said ``How many of them [files] are \textit{related} to
each other?'' [emphasis added]~\cite{Seltzer2009} .

Files are rich with relationships.  However, these relationships are not limited to what the
file system can ``see''.
Narrowing our vision to the closed pool of file system relationships
hobbles our ability to capture them.
For example, the obvious relationship between
a file and the e-mail from whence it originated is not exploitable in
any system of which we are aware.
The academic papers we generate refer to other papers.  An enlightened document application would provide an identifier that can 
be used to find the corresponding paper - a \textit{refers to} relationship,
whether on our local system or elsewhere.
Of course, in the current model, we likely can't
recall where we stored it when we downloaded it.  As we create new works, we refer to older works --- our own
documents, web pages, Jupyter notebooks, spreadsheets, pictures, etc. Capturing these relationships permits
us to reconstruct the process taken to produce an output.
This is the fundamental problem that the provenance community has been
addressing, but few systems~\cite{pasquier17camflow,reddy06pass} demonstrate
an understanding of the role our file systems play in making this
possible.

Versioning is a feature that continues to reappear in various guises.
This is simply one example of temporal locality; a
relationship that we have not yet deeply mined.
While it is now common practice for
individual applications to ``remember'' the last few files you have accessed,
there are few cross-application examples.
In lieu of the right tools, users invent creative solutions.
For example, one of the authors \textit{attaches} documents to e-mail
immediately after reviewing them
specifically to establish temporal relationship.
The ability to establish temporal relationships across applications should
provide powerful capabilities.

We do not know which relationships are useful. One of the hallmarks of good file systems design over the decades
has been \textit{not} to impose a specific restricted model on what files can be --- we leave that to databases.  
We do not intend to establish a definitive set of relationships any more than we focus on defining
the structure of file contents.
However, we encourage others to explore this area, encourage best practices,
and build tools that produce and use such relationships, leaving the
storage and retrieval of relationships to the file system.

% Relationships between files is not new --- this is the quintessential relationship of the modern Internet, with its
% vast web of content that references across the domains --- but those models are certainly more recent than that
% of the hierarchical file system.  Unlike the internet, where information is shared, we seek to enable
% similar relationships useful to our specific usage and with data we do not necessarily wish to share.

Much of the raw data that applications generate would be better \textit{not} injected into
the hierarchical name space: the location of our personal email database, financial
software files, binaries, temporary files, etc.  Their presence in the name space clutter
it and make our existing brute force search slower yet no more useful.

Application programs benefit being
unfettered from the hierarchical name space~\cite{Ghemawat2003}, both
in terms of their efficiency as well as the benefits of \textit{not} commingling the private files of individual 
applications --- but the hierarchical file system requires
they be stored somewhere within its domain.
Applications routinely hide most of their files in out of the way 
locations, as they are only useful to the application itself.
Thus we end up with ``System Volume Information'' and ``.ssh'' and a myriad of obscure locations where applications 
hide data from us.
This is a side effect of the name space model we have used for the past
50 years.

Prior attempts to address this have done so within the confines of a narrow perspective of what is needed to fully 
enable the ability of us to find our data.  The HCI community have been poking at the
edges of this problem for decades as well --- observing the frailties of the hierarchical model and suggesting
alternatives~\cite{harper2013file,lindley2018exploring,khan2018forgotten,vitale2018hoarding,boardman2003too,nayuki2017,martin2014,Jan2011,Andrews2012,Mander1992,Omvlee2009}.

Our graph file system permits them to escape the existing paradigm \textit{without} giving up support
of existing applications.


\section{Conclusion}
\label{sec:conclusion}

We have presented our position that we need \system, a storage architecture that decouples naming from the storage location of documents and data objects, provides customizable and personalized namespaces, and that makes relationships between documents a first class citizen. With \system, users will be able to organize, share and find their data conveniently across multiple storage silos using a rich set of attributes breaking away from the rigid, hierarchical organization.

We expect \system will enable a broad area of research in HCI exploring new ways to visualize and interact with data using the mechanism's provided by \system. Moreover, we expect \system to provide interesting scenarios for security and privacy research in storage systems.



%\section{Use Cases}

We consider the following potential use cases for our naming system:

\begin{description}

    \item[Finding historical documents] - in this common usage scenario, we are
    looking for an object that we identify by what we were doing and when we
    were doing it as well as potential subjects.

    \item[Related documents in distinct storage locations] --- in this common
    usage scenario, we are looking for \textit{related} objects that, for
    whatever reason, are stored in different ``storage silos''.  For example, we
    received a document via e-mail and then saved it on the ``Dropbox folder''
    of our local system.  The e-mail and the document are related, but we have
    no obvious way to traverse back from the document to the original e-mail.

    \item[Ability to search non-traditional storage locations] --- in this usage
    scenario, we are looking for \textit{objects} that are not stored in a
    traditional storage silo but instead reside inside some other object storage
    domain.  For example, objects in an arbitrary object store, such as the
    ubiquitous key-value store.

    \item[Cross-silo versions / Document Identity]  ---
    draft.doc that you got from your coworker via e-mail, that version you stored
    on your local drive, then the one you uploaded to Office365. Another one you've
    got from another co-worker via Slack. Nirvana will show these as different
    versions from the same logical document, possibly even the temporal relationships
    between then

    \item[Notifications] -- Allowing a user to "subscribe" to change
    notifications for critical documents; e.g., an active collaborative project
    to ensure users are notified when documents are updated (and of course
    cancel notifications when they are no longer needed/useful).

    \item[Search Results] -- What kind of searches happened during the writing
    of of another document (e.g., you are interested in the statistics of a
    search, but not particular results)

    \item[Compliance] --- Identifying related documents, including references,
    to specific material that needs to be located as part of compliance with
    legal mandates, e.g., discovery notices, GDPR data removal requests (``right
    to be forgotten'').
\end{description}


\subsection{Prior Usecase Examples}

\reto{OLD `usecases follow`}

% locating documents in general
\noindent\textbf{Locating Documents.}
Users want to search for particular documents (use-cases \\usecasehistcontext, \\usecasereldocuments).
Indexing services (e.g., Spotlight), cloud-based platforms, or tools like \texttt{grep} or \texttt{find} provide mechanisms to locate documents based on their name, date of creation or modification, or even the content of the files.
However, users often need to search different storage silos independently. This poses a burden to the user.
Worse, searching over locally mounted network attached storage (e.g., NFS or Dropbox) could mean transferring gigabytes worth of data for doing the search.
Finally, current solutions do not capture activities well such as while on a call with a customer, or while attending a webinar or conference.
Not only fail existing architecture to capture these aspects, they are also not capturing seemingly simple relationships between a file stored on disk and the e-mail exchange through which it was received and thus losing important contextual information.

% integration with non-traditional storage
\noindent\textbf{Alternative Storage.}
Traditional files are not the only way to store documents.
Today's applications may use object stores, key-value stores and data bases, or even implement their own storage container holding multiple objects (use case \\usecasealtstorage).
Moreover, applications want to protect their data from unauthorized accesses using various methods such as encryption, for example.
This makes searching hard and the user is forced to use the interface provided by the application to locate its documents, resulting in yet another manual cross-silo search.
While point solutions exist, e.g., Android search integration or export as WebDav-based file system, they are not ubiquitously available or do not allow users to search using the full range of attributes.

% document identity / cross silo version
\noindent\textbf{Document Identity.}
Documents have an identity. Writing an additional paragraph in this paper, does not change its identity. It rather creates a new version of the same document and both versions are related.
Likewise, uploading a document to cloud storage, renaming it, or converting it to a PDF also does not change its identity (use-cases \\usecasereldocuments, \\usecasedocidentity).
In contrast, taking last year's HotStorage paper as a template for this year's paper will change its identity.
While versioning is available on various storage solutions or VCS systems, it fails to capture cross-silo versions and relation ships. For example,
the document that was just received via e-mail is in fact a version of the one you have been editing yesterday, and that you have just uploaded to cloud storage for sharing with your co-worker who just sent it to you via e-mail.

% multi-device
\noindent\textbf{Multi-Device.}
Users may use various devices to access their files, each of which having different compute and connection capabilities (use-case \\usecasedevices).
For instance, a desktop machine is powerful and always connected, while a smartphone is low power and can often be disconnected. Cloud-storage or e-mail client apps on the devices provide a search interfaces that may offload the actual search to the server.
A user with two devices cannot easily look for files on the local disks and has now to do the search manually across multiple devices (similar to \\usecasereldocuments).

% attribute provenance
\noindent\textbf{Attribute Provenance.}
Traditional file systems tie the permissions to change attributes with the permissions to change the file contents.
Any user with sufficient rights can change any attribute or the file contents.
While there are systems that record the user who changed the file last, it may not capture the entire history of the changes (use-case \\usecaseattrprov).
Moreover, existing systems do not capture the intent or reason of those changes.

% attribute provenance
\noindent\textbf{Multi-View.}
A document cannot be physically present in multiple filing cabinets at the same time.
Hard and soft links provide references to other other storage locations.
However, there is still only a single physical organization of files making it impossible to organize files by year-customer, and customer-year at the same time.
File managers may offer tags, and media libraries to notion of albums and grouping by year to provide specific ways to sort, filter or organize files.
This approach, however, is not generally available and users cannot freely choose and adapt their \emph{view} of their files (use-case \\usecaseviews)

\section{Notes from the meeting (remove as pleased)}

Focus on the what the system does, with a bit of how..

\begin{itemize}
    \item Placeless + Burrito?~\footnote{\url{https://www.usenix.org/system/files/conference/tapp12/tapp12-final10.pdf}}
    \item Attributes that are not traditionally a file.
    \item GIS information is useful
    \item don't open files for extracting meta-data
\end{itemize}

Taxonomy

\begin{itemize}
 \item table what can be done what can't. 3 columns. Use-case, what can be done, what's hard.
 \item here's a structure of them. (what's easy and hard to do with today's technology) --> references to architecture section
 \item evaluation: show that we are supporting the things that are hard.
\end{itemize}

 \begin{description}
\item[Section 1] introduction
\item[Section 2] Background
\item[Section 3] Use-cases + table
\item[Section 4] Architecture
\item[Section 5] Show that use-cases are solved
 \end{description}

Different clients:

\begin{itemize}
 \item Desktop: powerful + always connected
 \item Laptop: powerful + can be offline
 \item Smartphone: low power + mostly connected
 \item Smartphone w/o data: low power + mostly disconnected
\end{itemize}


\clearpage
\pagebreak

% I add a table of nocites here to get a list of all the papers in the bibtex
% comment out to only use the references in the paper itself.
\begin{comment}
\nocite{daley1965general}
\nocite{koetsier2013digg}
\nocite{ritchie1973unix}
\nocite{wilkes1964programmer}
\nocite{reinsel2018data}
\nocite{macdonald1956datafile}
\nocite{halasz1982analogy}
\nocite{shapiro1964extracting}
\nocite{microsoft:data-add-in}
\nocite{apple:spotlight-extensions}
\nocite{chou2015findfs}
\nocite{vicente1988accommodating}
\nocite{marsden2003improving}
\nocite{jones2005don}
\nocite{karger2006data}
\nocite{ma2009file}
\nocite{jensen2010life}
\nocite{karlson2011version}
\nocite{odom2012lost}
\nocite{Thereska2013}
\nocite{harper2013file}
\nocite{mendez2012linked}
\nocite{bothorel2015clustering}
\nocite{huo2016mbfs}
\nocite{jayalakshmi2016semantic}
\nocite{carvalho2016finding}
\nocite{gao2016implicit}
\nocite{hua2016real}
\nocite{di2017gfs}
\nocite{lindley2018exploring}
\nocite{khan2018forgotten}
\nocite{vitale2018hoarding}
\nocite{laursen2014}
\nocite{boardman2003too}
\nocite{nayuki2017}
\nocite{martin2004formal}
\nocite{martin2008}
\nocite{martin2014}
\nocite{soules2004toward}
\nocite{gifford1991semantic}
\nocite{corbato1965introduction}
\nocite{Faubel2008}
\nocite{harlan2011joinfs}
\nocite{Jan2011}
\nocite{Soules2004}
\nocite{Suguna2015}
\nocite{Andrews2012}
\nocite{Eck2011}
\nocite{Hua2010}
\nocite{Joshi}
\nocite{Mander1992}
\nocite{Martin2004}
\nocite{ngo2007integrating}
\nocite{Omvlee2009}
\nocite{wang2003managing}
\nocite{Seltzer2009}
\nocite{Shah2007}
\nocite{Up2016}
\nocite{gopal1999integrating}
\nocite{Martin2005}
\nocite{Reiser2016Name}
\nocite{Benefits2016}
\nocite{Codocedo2015}
\nocite{Ghemawat2003}
\nocite{Jones2016}
\nocite{Mahalingam2003}
\nocite{Os2016}
\nocite{Parker-Wood2014}
\nocite{Prabhakaran2005}
\nocite{Schandl2009}
\nocite{cellier2008formal}
\nocite{strong2013semantic}
\nocite{Xu2014}
\nocite{olson1993design}
\nocite{Leung2009}
\nocite{Schandl2009a}
\nocite{Shah2006}
\nocite{Wiki2016}
\nocite{vicente1987assaying}
\nocite{malone1983people}
\nocite{terrizzano2015data}
\nocite{watson2017exploring}
\nocite{mogul1986representing}
\nocite{GuoBurrito2012}
\nocite{macko2015llama}
\nocite{margo2015scalable}
\nocite{macko2013performance}
\nocite{aws:s3:object:tagging}
\nocite{balabine1999file}
\nocite{stonebraker1994mariposa}
\nocite{balabine2002database}
\nocite{xu2003towards}
\nocite{kashyap2004file}
\nocite{ames2006lifs}
\nocite{koren2007searching}
\nocite{huang2012just}
\nocite{xu2009semantic}
\nocite{murphy2002design}
\nocite{leung2009magellan}
\nocite{frieder2012hierarchical}
\nocite{bush1945we}
\nocite{whang1991multilevel}
\nocite{kumar1997browsing}
\nocite{van2003beamtrees}
\nocite{dumais2016stuff}
\nocite{soules2003can}
\nocite{maccormick2004boxwood}
\nocite{marchionini2006exploratory}
\nocite{brandt2009fusing}
\nocite{leung2009spyglass}
\nocite{ren2013tablefs}
\nocite{niazi2017hopsfs}
\nocite{van2011efficient}
\nocite{215e74710c2245a59b2ce117f9e74cd7}
\nocite{appuswamy2014building}
\nocite{Harlan2011}
\nocite{Min2015}
\nocite{Open2016}
%\nocite{SemFS2016}
\nocite{Chou2015}
\nocite{Ngo2015}
%\nocite{Regatta2016}
\nocite{Gopal}
\nocite{Beydoun2009}
\nocite{Gifford2006}
\nocite{Hyvonen2004}
\nocite{Siekmann}
\nocite{Strong}
%\nocite{Dantalian2016}
\nocite{Escriva2015}
\nocite{Agee2016}
\nocite{Gifford1991}
\nocite{Guide2016}
\nocite{Liu2006}
\nocite{needham1977cap}
\nocite{arenas2016faceted}
\nocite{tunkelang2009faceted}
\nocite{hearst2006design}
\nocite{arenas2016faceted}
\nocite{klungre2018evaluating}
\nocite{walton2015searching}
\nocite{walton2017looking}
\nocite{cleverley2015retrieving}
\nocite{huurdeman2016active}
\end{comment}

\bibliographystyle{IEEEtranSN}
\bibliography{notdead} 

\end{document}
