
\section{A Modest Proposal}\label{sec:graphfs}

% Because our focus is on the \textit{naming} system and not the storage system,
% for the present time we will not consider the issues that will arise in the
% storage management layer to support the new model, though we admit that there 
% are likely to be concerns that will need to be addressed in future work.

Our proposed file system focuses on \textit{relationships} between our files.
We use an analogy between social graphs and file systems to explore this 
approach.
Facebook's graph is a collection of typed objects
(e.g., users, actions, places) and associations (e.g., friend, authored, tagged).
File system objects map to users and files; \textit{contains} is, perhaps, the only
association captured in a file system.
For the rest of this discussion, we treat directories as the embodiment of the
\emph{contains} relationship, not objects.

We consider two strawman implementations for elevating relationships to
first class file system objects.

\subsection{File System as a Graph Database}

In considering a graph based file system, first we consider
implementing a file system in a graph database, of which there are
many (\S \ref{sec:background}). Their primary focus is the storage of 
graph-structured \textit{data}.  Our focus is in the use of graph-structured
data as critical meta-data inside of a storage system.
As such, there is a mismatch in design targets between a file system and
existing graph databases: nodes in graph databases are small; nodes in
file systems are large. Graph databases tend to favor a navigation-based API;
file systems need a point query and search API. Graph databases assume that
attributes and relationships are provided; file systems will frequently derive
attributes and relationships.
These differences suggest to us that existing graph databases are not suitable
as the basis for file systems.
Nonetheless, we encourage others to consider such an arrangement, should they
have compelling reason to do so.

\subsection{File System as Social Network}

Next, we consider implementing a file system in the same way Facebook
implements their social network graph.
Facebook's original implementation stored the social graph in MySQL, queried
it from PHP, and cached the results in memcache.
More recently, Facebook introduced Tao, which is a service that more directly
implements the fundamental objects and relationships that comprise the
social graph~\cite{bronson2013tao}.
While Tao is specifically designed to support the widely distributed,
replicated, and rapidly changing social network scenario, it provides the
starting point for conceptualizing a data model premised on the primality of
relationships.
Tao stores both objects and associations in a MySQL database and presents
the graph abstraction via Association and Query APIs in the caching layer.
Is this a viable structure for a file system?

% in fact, facebook has a separate storage system for videos and pictures 
% because they are large.
Unlike objects in Facebook, files are large.
Although prior work has considered using relational
database~\cite{olson1993design} and other index-based
structures~\cite{spillane2013vttree}  to store files,
the community seems to have
concluded that such storage is not ideal. We agree, suggesting that
an RDBMS is not the desired storage system.

What about relationships? Is it appropriate
to use one persistent representation (e.g., a relational one) and a second
memory representation (e.g., a graph-structured on) or
should we use a single graph-structured representation both in persistent store
and in-memory.
We propose the latter for two reasons.
First, the rumored era of non-volatile main memory seems to be around the
corner, so a modern file system design should embrace a single
representation.
Second, while it is reasonable for Facebook to construct the entire graph in
a distributed pool of main memory, file systems must work on a more limited
scale and therefore cannot ensure that the realized graph structure will fit
in main memory.

As neither strawman design seems suitable for our relationship-centric file
system, we present a new model and file system design.

\subsection{Graph FS Model}
\label{sec:graphfs:model}

We set out a basic description of our core objects in Table \ref{table:graphfs:terminology}
%\footnote{We took inspiration for this model from https://github.com/opencypher.}
and a demonstrative set of example relationships in Table \ref{table:relationship-examples}.
We do not consider either of these to be exhaustive, but rather propose them as an initial
basis for discussion.
The presented model can encompass
the functionality of the existing hierarchical file system model.

% https://github.com/opencypher/openCypher/blob/master/docs/property-graph-model.adoc

\begin{table}[h]
    \captionsetup{justification=centering}
    \begin{tabular}{p{2cm}p{5cm}}
        Term                          & Definition\tabularnewline\hline
        \multirow{1}{*}{file}         &
        \multirow{1}{*}{\parbox{4.8cm}{Uniquely identified storage unit}}
        \tabularnewline
        \multirow{1}{*}{relationship} &
        \multirow{1}{*}{\parbox{4.8cm}{Directional file association}}
        \tabularnewline
        \multirow{1}{*}{labels}       &
        \multirow{1}{*}{\parbox{4.8cm}{A binary attribute, e.g., executable}}
        \tabularnewline
        \multirow{1}{*}{property}     &
        \multirow{1}{*}{\parbox{4.8cm}{Key/Value attribute}}
        \tabularnewline
    \end{tabular}
    \caption{Graph File Systems Terminology}\label{hotos19:table:graphfs:terminology}
    %    \Description{Graph File Systems Terminology}
\end{table}



\begin{comment}
% Omit because I don't think it is central to our thesis.
Our relationship-centric file system consists of a \textit{union} of distinct
name spaces, which permits a name space that is not
fully connected, yet can still be merged via the existing union model. This permits an
application to have a private name space for files,
similar to both CAP and Plan 9~\cite{needham1977cap,pike1992use}.  Isolation also
enhances security, e.g., the POSIX \textbf{mkstemp} function, which was introduced to
address the need of applications for temporary files not visible to other applications.
Such namespaces are like unlinked but open files in UNIX, where such files are
reclaimed when the process terminates.  By creating a process private ephemeral namespace
we augment the file system's ability to provide enhanced isolation.
\end{comment}

Every file has a unique identifier, such as a \textbf{UUID}, similar to
an inode number or object ID.
We do not rely upon \textit{names}
as they are simply mutable properties.

\begin{table}[h]
    \begin{tabular}{p{1.9cm}p{5.5cm}}
        Relationship                           & Description\tabularnewline
        \hline
        %        \multirow{1}{*}{\textit{is}} &
        %        \multirow{1}{*}{\parbox{5.4cm}{Attribute of a file, e.g. size or timestamp}}
        %        \tabularnewline
        \multirow{1}{*}{\textit{similar}}      &
        \multirow{1}{*}{\parbox{5.4cm}{Similarity measure, e.g., \cite{masci2014multimodal}}}
        \tabularnewline
        \multirow{1}{*}{\textit{precedes}}     &
        \multirow{1}{*}{\parbox{5.4cm}{temporal relationship (e.g., versioning)}}
        \tabularnewline
        \multirow{1}{*}{\textit{succeeds}}     &
        \multirow{1}{*}{\parbox{5.4cm}{temporal relationship (e.g., versioning)}}
        %        \tabularnewline
        %        \multirow{1}{*}{\textit{located}} &
        %        \multirow{1}{*}{\parbox{5.4cm}{link or url}}
        \tabularnewline
        \multirow{1}{*}{\textit{contains}}     &
        \multirow{1}{*}{\parbox{5.4cm}{directory/file relationship}}
        \tabularnewline
        \multirow{1}{*}{\textit{contained by}} &
        \multirow{1}{*}{\parbox{5.4cm}{directory/file relationship}}
        \tabularnewline
        \multirow{1}{*}{\textit{derived from}} &
        \multirow{1}{*}{\parbox{5.4cm}{provenance (e.g., .o to .c)}}
        \tabularnewline
    \end{tabular}
    \caption{Graph File System Relationship Examples}\label{hotos19:table:relationship-examples}
    %    \Description{Graph File System Relationship Examples}
\end{table}



A \textit{relationship} is a directional association between two files.  We expect there
to be far fewer relationships than files, though many more \textit{instances} of
relationships (i.e., the number of edges in our graph exceeds the number of vertices).
Relationships may be either uni-directional (e.g., derived from) or
bi-directional (e.g., similar).
Table \ref{table:relationship-examples}
provides a set of sample relationships; the universe of relationships
is extensible.
As in RDF, relationships are triples: two files and the relationship.

As files have attributes in a conventional file system, both files and
relationships have attributes in a graph file system.
A \textit{label} is a simple binary attribute (e.g., executable),
while a \textit{property} is an arbitrary name/value pair, much like
an extended attribute, but they are native to the model, not
an afterthought.

% We will extend our terminology as needed, using the existing terminology of the
% relationship graph as inspiration for usable models.

\subsection{Interface}\label{sec:graphfs:interface}

\begin{table}[b]
    \small
    \captionsetup{justification=centering}
    \begin{tabular}{p{2cm}p{5cm}}
        Operation & Description\tabularnewline\hline
        \multirow{1}{*}{create} &
        \multirow{1}{*}{\parbox{4.8cm}{Insert new file into graph}}
        \tabularnewline
        \multirow{1}{*}{relate} &
        \multirow{1}{*}{\parbox{4.8cm}{Insert new edge into graph}}
        \tabularnewline
        \multirow{1}{*}{label} &
        \multirow{1}{*}{\parbox{4.8cm}{Insert new labels}}
        \tabularnewline
        \multirow{1}{*}{set} &
        \multirow{1}{*}{\parbox{4.8cm}{Insert new properties}}
        \tabularnewline
        \multirow{1}{*}{remove} &
        \multirow{1}{*}{\parbox{4.8cm}{Remove something from graph}}
        \tabularnewline
    \end{tabular}
    \caption{Graph File Systems Operations}\label{table:graphfs:operations}
%    \Description{Graph File Systems Operation Examples}
\end{table}


One of the lessons from Plan 9 is that everything can be represented as a file~\cite{pike1992use};
we expect to
continue with this paradigm as it has served us well over the years.  While we generally
think of files as a blob of \textit{persistent} data, in fact it is useful to
think of them as abstract \textit{generators} of byte stream data.  This fits well
with our model of separating namespace from storage; how the storage
providers return data to us is orthogonal to the namespace we use to retrieve it.
For example, the \textit{procfs} file system creates a synthetic namespace and supports
I/O operations for reading and modifying data contents of the pseudo-files.

From the namespace perspective, our file system must support operations that manipulate that
namespace. This includes the ability to create files, relationships,
relationship instances \textit{between} files, labels, and properties.
Similarly, we need the ability to remove each of these.

Our model is simple, yet powerful.  It captures interesting concepts such as versioning, using relationships such as
\textit{precedes} and \textit{succeeds},
and provenance, using relationships around derivation and use,
and application specific relationships, such as \textit{indexes} so a database
system can expose the relationship between its primary data and the
corresponding index files.
Although relationships are binary,
we can create clusters of related files by asking for all the vertices connected
by a specific relationship.

Where do relationships, labels, and properties come from?
We identify at least the following five sources:
1) the system itself will generate traditional
attributes (e.g., \textit{size}, \textit{read time}) and some
relationships (e.g., contains);
2) tools that extract meta-data from different file
types~\cite{soules2004toward,bloehdorn2006tagfs} will produce more attributes;
3) applications will generate both attributes and relationships;
4) users may generate attributes and relationships, although history
suggests that asking users to annotate data is a losing
proposition~\cite{soules2003can}; and
5) kernel extensions, e.g., provenance tracking systems~\cite{pasquier17camflow}
will generate attributes and relationships.

Several interesting possibilities emerge from this design.
Hard links are multiple \textit{name} properties attached to the
same file, potentially in different namespaces.
Soft links are a relationship between two names.
The system can capture relationships that extend beyond the file system.
For example, the \textit{derived from} relationship from
Table \ref{table:relationship-examples} might describe a file that came
from a particular email or web site.

\begin{figure}[bt]
    \captionsetup{justification=centering}
\includegraphics[width=0.9\linewidth]{figures/model-graph.eps}
\caption{Graph File System}\label{fig:graphfs-example}
%\Description{Simplistic Graph File System Picture}
\end{figure}

Figure \ref{fig:graphfs-example} provides a simplified visualization of our graph file
system model.
Our inclusion of disjoint graphs captures the notion that the system
naturally supports multiple namespaces, implemented as disconnected graph
components.

\begin{comment}

\subsection{Search Functionality}


Motivation for our graph file system was to provide a rich set of search semantics
to enable more effective human-usable search.  Search starts from an arbitrary 
file and radiates out from that point: what we seek is often ``close'' in terms of
our relationship graph. We can return results quickly without doing
an exhaustive search of the entire graph.  This is similar to how a modern internet search 
engine, such as Google, works. Only the first few hundred results returned from millions
of entries returned are usually analyzed.

Our file sytem can support an iterative interface in which it asynchronously constructs
the search results, returning initial results quickly even as further results are constructed.
Bounding search depth combined with resuming search as needed permits the iterative process
to continue without overwhelming the system. We seek to balance performance and completeness. 
Indeed, our expectation is that --- much like your search engine never returns the 7,345th
page of search results, the file system will seldom be asked to search the entire graph.  
This is a far cry from the brute force search that some of us now do when we use \textbf{find} 
or \textbf{grep -R} to scan through the hundreds of thousands of files that we have managed to
accumulate over years of work.

There is considerable work already done in the area of graph searching; we expect to utilize
that work to construct a viable search 
interface~\cite{angles2018g,,rudolf2013graph,francis2018cypher,van2016pgql}.  
This aspect of our graph file system is likely to evolve the most as we better understand the
needs of applications, particularly those seeking to expand the human level searchability, and
adapt the search interface to meet their needs.
\end{comment}

\begin{comment}

Current work in labeled property graphs has recently moved to introducing \textit{paths} to the model
as well as the labels and properties that have already been added to the basic graph model~\cite{angles2018g}.
This prior work identifies five key features that are heavily utilized in searching such graphs: 
\textit{reachability}, \textit{construction}, \textit{pattern matching}, \textit{shortest path search},
and \textit{graph clustering}.  For file systems, we expect that some of these will be of less
interest than others, but we think it premature to exclude any of these from a useful design, though we expect to 
focus on those we view as providing the greatest utility for our prototype implementation work. Thus, we expect
the greatest utility for file systems will be \textit{pattern matching} initially, though we find both
\textit{reachability} and \textit{shortest path} potentially intriguing.

Pattern matching makes the most sense in terms of searching for specific labels and properties of files --- this most closely
matches with the types of searching that indexers are already performing.  Thus, from the perspective of novel new
search models it is the least satisfying yet likely one that will still provide substantial utility.

Reachability is intriguing because it meshes well with our model for temporal relationships, in which we define ``reachability''
with respect temporal reachability.  POSIX co-location (e.g., part of the same directory) is another potential area, though
we suspect there are further areas to explore, including similarity.  For example, similarity preserving hashing~\cite{masci2014multimodal}
could be used to search for files that are similar, but not identical to one another.  This could be used to trace the evolution
of files as they are copied and modified from one location to another as well as to identify files that are identical.  Similarly,
finding the set of paths between two vertices could yield useful information about the linkage of files, something that we
cannot easily reconstruct from current file systems.

Shortest path is a classic search problem (e.g., Dijstra's Algorithm~\cite{dijkstra1959note}) as a means of finding the closest relationship
between two files.  This could be combined with reachability to identify a \textit{sequence} of iterative steps as we move between
similar files.  Another use would be to find a path between a directory and a file that provides access, e.g., for which the security label
on all the edges permits access.  This works well with the ``open by inode number' problem, for example and may help us address
the interesting question of how to perform POSIX-compatible security checks on files that are being opened by their identifier.
\end{comment}

\begin{comment}
\subsection{POSIX}

Our model permits support for POSIX file system semantics with respect to the file system
name space.  While our model does not require an hierarchical layout, we can support
such a system within this model.

First, we assume there is an external mechanism for identifying a distinguished file that
corresponds to the \textit{root} directory.  We added the \textit{contains} relationship
to capture this form of relationship.  While we suggested the \textit{contained by} relationship
(Table \ref{table:relationship-examples}), this is not a requirement for POSIX compatibility,
though one of the authors has been asked for this feature numerous times in the past for
existing POSIX file systems.

Enumerating the contents of one of these \textit{directory} type files would enable
enumeration of the directory in keeping with the POSIX model, by simply querying the
edges capturing the \textit{contains} relationship.  A hard link from two directories
to the same file is trivially supported.  Soft links are already a part of our model, so
we can provide symbolic links as well.

POXIS APIs that retrieve file properties, such as \textbf{stat}, are similarly converted
into a call to retrieve properties of the file and then converting them into the
expected native format.

While our own graph file system does not inherently place any importance on names, this
POSIX layer can store names as a property of the edge.  This is consistent with the way
that file names now function --- they are an attribute of the directory entry that
references the file, rather than an attribute of the file itself.
\end{comment}

\begin{comment}
\subsection{Interactions}

Eventually, we expect there will be challenges associated with the POSIX interface and the richer interface we offer.  We have already mentioned
one such issue: security.  The POSIX specification expects a security check at each node along the path as part of an attempt to open or create
a file.  Windows will enforce the same path walking security model but provides a work-around that allows skipping this~\cite{conover2006analysis}
check in favor of performance.  The ability to perform such a security check at open time could be an interesting systems use of enhanced
searchability.

One area in which we can envision challenges is in preserving the labels and properties of files within our graph because standard utilities
such as \textbf{cp} are ignorant of them and would simply copy the data contents of a file without copying the labels and properties.  We can
certainly enhance our system utilities to be enlightened about these new features, but it will not resolve the problem for cases in which
existing applications use standard POSIX systems calls to copy a file.

In addition, the ability to create disconnected name spaces does not fit well within the POSIX model, though it is more like distinct file systems and thus
likely does not break existing semantics.  Still, we expect that as we gain further experience with our graph file system and the graph
features are exploited we will need to handle more interaction cases.
\end{comment}

\begin{comment}

Our proposal is that file system name spaces evolve from \textit{trees} to \textit{graphs}.  Of course, a tree is simply a minimally connected graph.  This leads
to our first question for our new file system model: \textit{should it be a connected graph?}  In fact, we observe that even our personal name space is a set of
disjoint graphs --- there is one for each of our devices.  Thus, we observe that this question is already answered.  However, having answered it, we also
observe the utility of actually creating connections across those name spaces.

This becomes easier when we divorce the concept of \textit{storage} from the concept of \textit{name space}.  Our ideal name space can thus refer to files
that are not part of the current device's name space.  We are not (yet) proposing how we add that information to the namespace, merely noting that it
is a desired feature.

One argument for allowing disjoint namespaces on our individual devices is to provide a simple model for namespace isolation --- files that need not be
made visible outside some limited scope (e.g., the temporary working files of a particular application) can be in what is a localized name space; in some
ways this is reminiscent of capabilities based file systems~\cite{needham1977cap}.  While an interesting area to explore further, we leave it open
at the present time for our design.

Our graph is essentially a \textit{relationship} graph, in which vertices represent what we usually think of as \textit{files}, and edges represent
the \textit{relationships} between our vertices.  In an attempt to emphasize the generality of our vertices, we note they may be able to supply a byte
stream of data --- a \textit{generator}, and they may be able to consume a byte stream of data --- a \textit{processor}; we leave it open if we wish
to consider richer interfaces, such as \textit{insert}, such as we previously proposed~\cite{Seltzer2009}.  An individual vertex may be any subset
of generator and processor and thus have read-write, read-only, and write-only vertices as suits the needs of the specific vertex.

Vertices may have a set of \textit{attributes}.  Since our goal is to provide generality here, we would establish that each attribute consists of a
\textit{domain} which identifies how to interpret the attribute, the \textit{attribute-name}, which specifies the name of the specific attribute
within the given domain, and a \textit{value}.  Our system can then define a core set of attributes relevant to the file system domain; other domains
can then be added by agreement of the domain creators and users without us enforcing any given structure.  While individual applications can then
create labels within their own domain, our hope is that over time related applications will come to an agreement as to the meanings of attribute-names
to enable sharing of properties.

Our edges are labeled, weighted directional edges in a hypergraph: each label consists of a \textit{relationship} and a \textit{weight}.  Because our
goal is not to strictly restrict the relationships, we suggest that a relationship itself consists of a \textit{domain} identifying how to interpret
the relationship (presumably by agreement for those creating those relationships) and a domain-specific \textit{relationship-label}.  We can define
a set of relationships that we will initially support: \textit{contains}, \textit{is contained by}, \textit{is related to}, etc.

Thus, we now have enough structure for us to create a mapping from the existing POSIX file system interface, including its hierarchical name structure
to be compatible with existing systems. 
\end{comment}

\subsection{Aspects of Implementation}

In the absence of space to provide a full implementation, we offer a few
strategies that make a graph file system both feasible and novel.
The underlying storage structure for files is essentially an object store~\cite{factor2005object} and
attribute storage is largely a solved problem
(although the last time
one of the authors said that, her colleague disagreed~\cite{mao2012cache}), so we
focus on fast and efficient graph storage and query.

Today's systems either provide graph storage~\cite{rudolf2013graph,webber2018programmatic,microsoft:cosmosdb} 
or graph processing~\cite{shun2013ligra,gonzalez2014graphx,malewicz2010pregel,salihoglu2013gps,nguyen2013lightweight,low2014graphlab,kyrola2012graphchi}, 
but a graph file system needs a high performance, space-efficient, mutable and queryable
native graph representation.
We have found that mutable compressed sparse row representations~\cite{macko2015llama}
meet all these requirements (we used them as the query and storage mechanism in the
SHEEP graph partitioner~\cite{margo2015scalable}).
Just as high performance key/value stores are considered reasonable implementation
strategies for attribute storage and management in file systems, similarly efficient
structures supporting graph storage and management should be adopted in file systems
as well.
