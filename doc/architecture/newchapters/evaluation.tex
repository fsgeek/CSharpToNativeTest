\chapter{Evaluation}
\label{ch:evaluation}

\begin{epigraph}
    \emph{For there is nothing lost, that may be found, if sought.} --- Edmund
    Spenser, \emph{Finding the Faerie Queene}, 1590.
\end{epigraph}

An important aspect of supporting my thesis is to evaluate the system that I
have proposed in \autoref{ch:architecture} and ensure the proposed system
provides the information necessary to answer my research questions
(\autoref{ch:research-questions}.)

\section{Useful Events}
\label{ch:evaluation:sec:events}

Research question \ref{rq:define-ac} asks what constitutes an activity
context. Research question \ref{rq:capture-ac} asks how to capture this
information.  Research question \ref{rq:leverage-ac} asks how applications can
leverage activity context.  These three questions all rely upon identifying
which events are both useful and practical to collect and aggregate. While prior
work has identified events that are of interest I expect to find additional
potentially useful events to collect.  Thus, evaluating the overhead of adding
new events to the activity context is useful.  Such an evaluation of the
overhead associated with adding activity context would include: how difficult is
it to add a new activity context provider to the model and how difficult is it
to add support for the new activity context
in an existing tool.

I suggest these metrics because they reflect upon the performance of the
architectural model that I set out in \autoref{ch:architecture}.

The prior work, notably the personal information trace
work~\cite{vianna2019thesis}, has publicly available tools that could ease the
collection of data. I can then use my implementation against my own architecture
to ensure that resource cost of collection demonstrates the low overhead that I
expect for collecting such data.

Once I have demonstrated that my own tools implemented against my architecture
do not have substantial overhead, I can then look at the complexity of adding
additional data collection, using the suggestions in
\autoref{table:useful-information} as well as additional information that I can
identify as potentially useful.  Identifying such potentially useful activity
context data is an area in which I would expect collaboration could be quite
beneficial but I am not relying upon such collaboration to conduct my own
research.

An important metric in evaluating my architecture will be to consider both the
performance cost of adding additional data collection (measured in performance
impact) as well as the development effort (measured in code size).  Thus, I
propose collecting that information while I develop the tools and build
extensions to collect additional data.

\section{Usefuless of Activity Context}
\label{ch:evaluation:sec:activity-context}

Research question \ref{rq:leverage-ac} asks a critical question underlying my
thesis: that \emph{activity context} is itself useful.  There is at least one
prior work outside the systems field that indicates it
is~\cite{vianna2019searching} and thus I reasonably expect that I will be able
to reproduce their results.  It seems logical to consider their evaluation
methodology as one way to measure the effectiveness of our activity context
driven model.  This, however, is not an ideal fit as the personal digital traces
work was evaluated against synthetic existing benchmarks.  Thus, other prior
work that suggests other potential metrics including the time it takes for a
user to perform their search and whether or not the search itself was successful
(using the \emph{abandonment rate}).

Assuming that activity context is useful, a more traditional systems evaluation
seems justified: what is the cost of collecting and disseminating the activity
context, what is the time to process queries~\cite{ames2013qmds}, how difficult
is it to add additional activity context providers, and what is the potential
added complexity for applications to utilize \emph{activity context}.

\textbf{Note:} While I expect there will be substantial benefit to collaborating
with others interested in the human-computer interface (HCI) and information
retrieval (IR) potential for using my work, based upon consultation with my
supervisors I do not assume that this will be the case.  Thus, the possibility of
collaboration has the potential to provide considerable impact if my research
supports my thesis, my thesis proposal is not dependent upon such collaborative
work. Thus, I intend on being sufficiently flexible to take advantage of
collaborative opportunities that do arise, but also realistic in completing my
own work in order to complete my thesis.

\section{Backwards Compatibility}
\label{ch:evaluation:sec:existing-apps}

Prior work, such as with semantic file systems~\cite{gifford1991semantic}, has
been realized by using indexing services.  Similarly, personal digital traces
have been used to augment indexing services~\cite{vianna2019searching,Xu2014}.
These works have evaluations for the effectiveness of their solutions. Thus,
virtual directory solutions and indexing solutions have prior evaluations that
can be leveraged to develop a more extensive evaluation model.

With respect to my own thesis related work, the availability of this type of
indexing and/or virtual directory mechanism would be helpful in understanding
the costs and performance of my own architecture to ensure that it can
adequately meet the needs of my target tool-building community.

\section{Access to Activity Context}
\label{ch:evaluation:sec:providing-ac}


Research question \ref{rq:meta-data-query} asks how to provide activity context to
enable users and application developers to exploit the enhanced rich meta-data
of \system.  Much of this work
relates to performing efficient meta-data queries against a potentially large
collection of such data. There is a strong body of prior work regarding
meta-data queries of both static
and dynamic
sources~\cite{Strong,revol2011universal,smartstore,pindex,federatedMetaData,huo2016mbfs,Suguna2015,Parker-Wood2014,watson2017exploring,leung2009magellan,leung2009spyglass,niazi2017hopsfs,van2011efficient}.
The prior work includes a model for providing evaluation.  In addition,
file system meta-data query specific research has also created a framework for
evaluation that relates to the performance speed of meta-data
queries~\cite{ames2013qmds}.

In addition to the performance of such queries, I expect it will also be useful
to determine the generality and ability to form specific queries.  My
expectation is that these queries will not ordinarily be initiated directly by
users but it may be useful, as part of evaluating the interface, to determine if
human-provided queries are viable and the level of ease with which they can be
constructed.  I expect further refinement on how to evaluate the flexibility of
the query mechanism may be avoided by adopting an existing query
language~\cite{francis2018cypher,van2016pgql}.

\section{Privacy}
\label{ch:evaluation:sec:privacy}

Research question \ref{rq:privacy} asks about how to ensure the privacy of users
is preserved when capturing detailed personal information in the \emph{activity
    context} I propose recording.  Because I have explicitly stated that for my own
thesis work I will be assuming the user maintains complete control of their
activity context, I do not propose any specific model for evaluating this
security because it is \emph{by design} as secure as the user's own data.

Ensuring privacy of meta-data is an active research area, in terms of
extraction, dissemination, and
sharing~\cite{10.1007/978-3-030-72465-8_14,eskandarian2021express,budzko2019architecture}.
Thus, future work that is not envisioned as part of my thesis should be done in
a context where emerging work is used to evaluate privacy concerns of a more
general meta-data service.



