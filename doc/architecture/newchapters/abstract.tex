\chapter{Abstract}

Human society is collecting data at an alarming rate: per-capita data generation
is now over 1.7MB \emph{per second}. We expect to send 361 billion e-mails
\emph{per day} by 2024. Rapid data growth, combined with increasing ways to
store and present data to users creates a frustrating challenge finding specific
documents a few days old, let alone those created months or years earlier.

Our data is scattered across physical locations.  Existing storage is presented
to us with old and new interfaces that blur the lines between file system and
application. For example, my Outlook mailbox resides on my local disk
drive in both databases and discrete files.  Yet I use Outlook, not my local
file system to find documents that were attached to e-mails.

On-demand cloud storage systems provide strong benefits yet also make it
impractical to search locally because not all the content is resident to be
indexed. Currently none of the cloud storage systems offer the rich extensible
search tools found on modern desktop operating systems.  Even if they were to
provide such search, it would require querying each one in turn to find relevant
files.

To address these challenges I propose \emph{Finding as a Service}, which provides two
important capabilities.  First, it explicitly decouples \emph{finding} objects from
\emph{storing} and \emph{presenting} objects.  Second, it exploits the
observation that users' mental associations with objects are more complex than
the arbitrary name, type, dates, or attributes on which users search today.

\emph{Finding as a Service} requires two things: (1) a mechanism for exploiting
the information that modern devices already capture and for capturing
additional useful information that relates to interactions with digital data and
the environment in which that data is used; and (2) the ability to collect,
store, and query both existing and new usage context pertinent to digital
information. Both of these requirements enable building powerful tools that
helps users in \emph{finding} digital objects efficiently.


%\vfill
%\begin{center}
%    \begin{sf}
%        \fbox{Revision: \DTMnow}
%    \end{sf}
%\end{center}

