\chapter{Background}
\label{ch:background}

\begin{epigraph}
    \textit{Paradigm paralysis refers to the refusal or inability to think or see
        outside or beyond the current framework or way of thinking or seeing or
        perceiving things.  Paradigm paralysis is often used to indicate a general
        lack of cognitive flexibility and adaptability of thinking.} --- The Oxford
    Review Encyclopedia of Terms (2021).
\end{epigraph}


Key topics crucial to understanding my proposal are:

\begin{enumerate}
    \item Why finding is important. I discuss the related background in
          \autoref{ch:background:sec:importance-of-finding}.

    \item How to help people find things.  I discuss the background related to
          finding things in \autoref{ch:background:sec:useful-information-for-finding}.

    \item Storage Access mechanisms.  I discuss the APIs that are available to
          various services in \autoref{ch:background:sec:storage-silos}.

    \item Existing Meta-data.  I discuss existing meta-data that are already
          known to exist and can be extracted in \autoref{ch:background:sec:meta-data}.

\end{enumerate}

This information is useful in better understanding the architecture proposed in
\autoref{ch:architecture}.

\section{The Importance of Finding}
\label{ch:background:sec:importance-of-finding}

In \autoref{ch:introduction:sec:activitycontext} I provided a basic definition
of activity context and described Vannevar Bush's 1945 work observing the
difference between computer index systems and human associative
memory~\cite{bush1945we}.

While preparing this proposal I spent time looking at the guides many libraries
provided about the naming of files. I found a body of recommendations about file naming
standards from significant academic and governmental sources and summarize that
in \autoref{table:data-management}.  These recommendations are consistent with
prior work~\cite{briney2015data}.

\begin{sidewaystable}
    \centering
    \caption{Sample Academic and Governmental Naming Conventions}
    \label{table:data-management}
    {\renewcommand{\arraystretch}{2.0} %<- modify value to suit your needs
        \begin{tabular}{p{5cm}p{15cm}}
            University of Cambridge                                   &
            \url{https://www.data.cam.ac.uk/data-management-guide/organising-your-data}
            \\
            Harvard University                                        &
            \url{https://datamanagement.hms.harvard.edu/collect/file-naming-conventions}
            \\
            Smithsonian Institution                                   &
            \url{https://library.si.edu/sites/default/files/tutorial/pdf/filenamingorganizing20180227.pdf}
            \\
            Leland Stanford, Jr. University                           &
            \url{https://library.stanford.edu/research/data-management-services/data-best-practices/best-practices-file-naming}
            \\
            United States National Institute of Science \& Technology &
            \url{https://www.nist.gov/system/files/documents/pml/wmd/labmetrology/ElectronicFileOrganizationTips-2016-03.pdf}
            \\
            University of British Columbia                            &
            \url{https://researchdata.library.ubc.ca/files/2019/01/FileName_Guidelines_20140410_v03.pdf}
            \\
            University of Chicago                                     &
            \url{https://guides.lib.uchicago.edu/c.php?g=565143       & p=3892706}
            \\
            University of Toronto                                     &
            \url{https://onesearch.library.utoronto.ca/researchdata/file-management}
            \\
            University of Washington                                  &
            \url{https://itconnect.uw.edu/learn/workshops/online-tutorials/web-publishing/web-publishing-at-the-uw/internet-file-management/4}
            \\
        \end{tabular}
    }%
\end{sidewaystable}

In \autoref{table:data-management} I provide links to a number of organizations
that publish recommendations for ``naming files.'' While they vary somewhat,
there is far more similarity than difference.  Harvard's list is:

\begin{itemize}
    \item Think about your files
    \item Identify metadata (e.g., date, sample, experiment)
    \item Abbreviate or encode metadata
    \item Use versioning
    \item Think about how you will search for your files
    \item Deliberately separate metadata elements
    \item Write down your naming conventions
\end{itemize}

The important observation here is how we rely upon the file name to provide
context for \emph{what} a given file represents.  Uniformity of information is
important --- the ``naming convention'' permits not only identifying similarity
but key elements of \emph{difference} between any two named things.

It is difficult not to look at this as a modern indictment of a system that is
fundamentally broken: requiring users understand meta-data, versioning,
encoding, \emph{and} capturing the naming represents a significant cognitive
burden.

Saltzer pointed this out as well: ``This approach forces back onto the user the
responsibility to state explicitly, as part of each name, the name of the
appropriate context~\cite{Saltzer1978}.''

The \emph{purpose} of the file system was to serve as the provider of
``human-oriented names''~\cite[Table III]{Saltzer1978}. Mogul observed that
``Better file systems allow us to manage our files more effectively, solve
problems that cannot now be efficiently solved, and build better
software~\cite[p. 1]{mogul1986representing}.''
Gifford observed: ``[A] semantic file system can provide associative
attribute-based access to the contents of an information storage system with the
help of file type specific transducers... The results to date are consistent
with our thesis that semantic file systems present a more effective storage
abstraction than do traditional tree structured file systems for information
sharing...~[p. 22]\cite{gifford1991semantic}''

Saltzer challenged us with general naming but set it aside for future research.
Mogul captured the idea that \emph{properties} could be used to store additional
meta-data about files.  Gifford explored the idea that information about \emph{what} a
file represents could be extracted and used to dynamically organize files based
upon semantic content of the files themselves.

Despite these insights, the tools we have for \emph{finding} remain
primitive. Forcing users to embed context is a \emph{naming} solution, but it
creates cognitively challenging requirements such as ``naming conventions.''
Similarly, creating tagging mechanisms using extended attributes or properties
has not provided sufficient benefit to be broadly used.  Semantic file systems
have been implemented in modern indexing services and are somewhat useful but
have clearly \emph{not} solved the problem.

All of these solutions are ``inward focused.'' That is, they focus on the
\emph{file} (or object): its attributes, which includes its name and its
contents. They fail to understand the file's usage context: it's relationship to
other files and \emph{to other events in the user's environment}. Thus, the
relevant prior systems work falls short of evaluating my thesis.

\section{Useful Information for Finding}
\label{ch:background:sec:useful-information-for-finding}

\begin{table}[tbph]
    \centering
    \caption{Useful Context Information (See
        \autoref{ch:background:sec:useful-information-for-finding})}\label{table:useful-information}
    \begin{minipage}{\textwidth}
        \begin{tabular}{p{3cm}ccccccl}
            \multicolumn{1}{c}{\multirow{2}[3]{*}{Information}}          & \multicolumn{6}{c}{Dimensions}                        & \multicolumn{1}{c}{\multirow{2}[3]{*}{Reference}}                                                                                                                                                                                                                                                                                                                                                                                                                                               \\
            \cmidrule{2-7}                                               & \multicolumn{1}{c}{\begin{sideways}Who\end{sideways}} & \multicolumn{1}{c}{\cellcolor[rgb]{ .906,  .902,  .902}\begin{sideways}When\end{sideways}} & \multicolumn{1}{c}{\begin{sideways}Where\end{sideways}} & \multicolumn{1}{c}{\cellcolor[rgb]{ .906,  .902,  .902}\begin{sideways}What\end{sideways}} & \multicolumn{1}{c}{\begin{sideways}Why\end{sideways}} & \multicolumn{1}{c}{\cellcolor[rgb]{ .906,  .902,  .902}\begin{sideways}How\end{sideways}} &                                                                                         \\
            \midrule
            Social Media                                                 & X                                                                                                                                                                                                                                                                                                                                                                                                                                                                                                                                                       %who
                                                                         & \cellcolor[rgb]{.906, .902, .902} X                                                                                                                                                                                                                                                                                                                                                                                                                                                                                                                     %when
                                                                         & X                                                                                                                                                                                                                                                                                                                                                                                                                                                                                                                                                       %where
                                                                         &
            \cellcolor[rgb]{
                .906,  .902,
                .902} X %what
                                                                         &                                                                                                                                                                                                                                                                                                                                                                                                                                                                                                                                                         %what
                                                                         &
            \cellcolor[rgb]{ .906,  .902,  .902} X %How
                                                                         & \cite{vianna2014a,vianna2019searching}                                                                                                                                                                                                                                                                                                                                                                                                                                                                                                                  \\
            \rowcolor[rgb]
            { .851,  .851,  .851}  E-mail                                & X
                                                                         & X
                                                                         &

                                                                         &
            X

                                                                         &

                                                                         & X
                                                                         &
            \cite{kim2009retrieval}
            \tabularnewline
            Webpage                                                      & X
                                                                         & X
                                                                         & X                                                     & X
                                                                         &                                                       & X
                                                                         & \cite{kim2009retrieval,soules2005connections}
            \tabularnewline

            \rowcolor[rgb]
            { .851,  .851,  .851} Document~\footnote{PDF,RTF,Word, etc.} & X                                                     & X                                                                                          & X                                                       & X                                                                                          &                                                       & X                                                                                         & \cite{kim2009retrieval,soules2005connections,hellerstein2017ground,vianna2019searching}
            \tabularnewline

            Audio                                                        &                                                       & X                                                                                          &                                                         &                                                                                            &                                                       & X                                                                                         & \cite{gemmell2002mylifebits}
            \tabularnewline

            \rowcolor[rgb]{ .851,  .851,  .851}
            Image                                                        &                                                       & X                                                                                          & X                                                       &                                                                                            &                                                       & X                                                                                         & \cite{gemmell2002mylifebits}
            \tabularnewline

            Video                                                        &
                                                                         & X
                                                                         & X                                                     &
                                                                         & X                                                     &
                                                                         & \cite{gemmell2002mylifebits,}
            \tabularnewline

            \rowcolor[rgb]{ .851,  .851,  .851}
            Applications                                                 & X                                                     & X                                                                                          & X                                                       & X                                                                                          &                                                       & X                                                                                         & \cite{hellerstein2017ground,provsearch}
            \tabularnewline

            Behavior                                                     & X                                                     & X                                                                                          & X                                                       & X                                                                                          & X                                                     & X                                                                                         & \cite{hellerstein2017ground,provsearch}
            \tabularnewline

            \rowcolor[rgb]{ .851,  .851,  .851}
            Change                                                       & X                                                     & X                                                                                          & X                                                       & X                                                                                          & X                                                     & X                                                                                         & \cite{hellerstein2017ground,provsearch}
            \tabularnewline

            Calendar                                                     & X                                                     & X                                                                                          & X                                                       & X                                                                                          & X                                                     & X                                                                                         & \cite{kalokyri2017integration,vianna2014a,vianna2019searching}
            \tabularnewline

            \rowcolor[rgb]{ .851,  .851,  .851}
            Paths                                                        & X                                                     & X                                                                                          & X                                                       & X                                                                                          & X                                                     & X                                                                                         & \cite{vianna2019searching,provsearch}
            \tabularnewline

            GPS                                                          &                                                       & X                                                                                          & X                                                       &                                                                                            &                                                       & X                                                                                         & \cite{vianna2014a}
            \tabularnewline

            \rowcolor[rgb]{ .851,  .851,  .851}
            File System~\footnote{The distinction between \emph{paths} which
                involve understanding application behavior and file system activity,
                which is about specific files or directories demonstrates potential
                complexity due to data collection from multiple sources in the
            same system.}                                                & X
                                                                         & X
                                                                         & X                                                     & X
                                                                         &                                                       & X                                                                                          & \cite{vianna2014a,provsearch}
            \tabularnewline

            Weather                                                      &                                                       & X                                                                                          & X                                                       & X                                                                                          &                                                       & X                                                                                         & \cite{vianna2014a}
            \tabularnewline

            \rowcolor[rgb]{ .851,  .851,  .851}
            Financial Data                                               & X                                                     & X                                                                                          & X                                                       & X                                                                                          &                                                       & X                                                                                         & \cite{vianna2019thesis}
            \tabularnewline

            % Blank line before new header
            \multicolumn{8}{c}{}
            \tabularnewline


            \rowcolor[rgb]{ .751,  .751,  .751} \multicolumn{8}{c}{New Suggestions}
            \tabularnewline
            Voice/Video Calls                                            & X                                                     & X                                                                                          &                                                         &                                                                                            &                                                       & X                                                                                         &
            \tabularnewline

            \rowcolor[rgb]{ .851,  .851,  .851}
            Collaboration~\footnote{Discord, Slack, Teams}               & X                                                     & X                                                                                          & X                                                       & X                                                                                          &                                                       & X                                                                                         &
            \tabularnewline

            Music                                                        & X                                                     & X                                                                                          &                                                         & X                                                                                          &                                                       & X                                                                                         &
            \tabularnewline

            \rowcolor[rgb]{ .851,  .851,  .851}
            Medical Info~\footnote{Wearable Monitor, Insulin Pump}       & X                                                     & X                                                                                          &                                                         & X                                                                                          & X                                                     & X                                                                                         &
            \tabularnewline
        \end{tabular}%
    \end{minipage}
\end{table}

% What is this table going to tell me? Ideally, I want to be able to identify
% information that has been identified as useful.

I summarize prior work useful to this proposal in
\autoref{table:useful-information}.  In some cases the precise information about
what data was used is not clear from the available materials and thus is not
included in \autoref{table:useful-information}.  The table identifies the
specific information, what \emph{type} this represents, and the reference that
provides this information. My classification of \emph{type} is based upon
previously proposed types~\cite{vianna2019searching} of \emph{who}, \emph{when},
\emph{where}, \emph{what}, \emph{why}, and \emph{how}.

The remainder of this section reviews the literature on which
\autoref{table:useful-information} is based and explains how it
relates to my broader thesis.

\emph{Search} is one tool that assists in \emph{finding} but it is not the first
choice for many human users~\cite{bergman2019search}.  Thus, improvements in search
are only beneficial at the point that the existing organizational system has
already failed.

The Human-Computer Interface community has provided an insightful definition of
good design:

\begin{quotation}
    \emph{
        A well designed computer system permits use of the tools it offers without
        requiring users to dedicate extensive mental processing to operations
        inherent in the system design rather than the task.  Furthermore, its
        tools are designed to also reduce task-specific mental processing,
        especially those types of processing that are performed more effectively
        by computers than by people, such as calculations and \textbf{accurate storage and
            recall of large amounts of pre-specified
            information}~\cite{brown1999human}\emph{[bold face is my addition]}.
    }
\end{quotation}

Information retrieval researchers are thus more interested in evaluating the
outcome of a human finding the object they seek, rather than the efficiency of
the underlying search infrastructure~\cite{bergman2019factors}. The need to find
things is pervasive: clearly there are obvious things like the need to access a
particular document, but there are other important needs, such as the need to
find objects that must be removed, such as when it is required to comply with
legal obligations, such as the ``right to be forgotten'' under the
GDPR~\footnote{https://gdpr.eu/right-to-be-forgotten/}~\cite{ritzdorf2014assisted}.
Some users prefer to delete content they know they will no longer
need~\cite{Vitale_2020}.  While the reasons for finding and removing content
vary, the ability to do so efficiently is directly related to \emph{finding}
that information.

Prior work has suggested a number of different factors that can be used to find
content:

\begin{itemize}
    \item Naming similarity; similarly names files are often related.
    \item Content similarity: files with similar content are often related.
    \item Temporal similarity: files that were created and/or accessed around
          the same time are more likely to be related~\cite{soules2003can}.
    \item Causality: files that are created using the same tool
          are more likely to be related.
\end{itemize}

Prior work has also observed that using contextual information from ``personal
digital traces'' clearly assists in finding relevant personal data: ``Work in
Cognitive Psychology has shown that contextual cues are strong triggers for
autobiographical memories~\cite{vianna2019searching}.''  The source code for the
data collection tools used by the authors is still available, albeit
dated~\footnote{\url{https://github.com/ameliemarian/DigitalSelf}}.

The \emph{finding} problem is one at the heart of the personal information
management research efforts, whose work suggests that one reason search is not
preferred is that it \emph{takes longer} than
navigation~\cite{bergman2019search}.  The semantic file system work fits well
with this observation, as one of its primary contributions was the observation
that search results can be represented as ``virtual directories,'' which
provides a means of presenting search results as a form of
navigation~\cite{gifford1991semantic}.

Similarly, prior work established the need to be able to support a broad
range of storage locations --- the ``storage silos'' that I have previously
mentioned.  In \emph{Stuff I've Seen} researchers found that the ability to
search across silos in a uniform fashion led to increased
utilization of such tools~\cite{dumais2003stuff}. Indeed, the observation in the
personal information management community repeatedly stresses the importance of
supporting cross-silo management of digital data.

This makes sense: when we are looking for a specific object that we know exists,
we do not particularly care \emph{where} it is stored.

The research on ``table top interfaces'' is another useful example of how the
HCI field is exploring alternative interfaces.  Their findings include the fact
that hierarchical structures do not work well in collaborative table top
systems~\cite{collins2007tabletop}.  There are similarities between table
top interfaces and mobile interfaces in terms of their interaction models, e.g.,
no keyboards or mice.  The mobile device solution initially was to create silos
for each application's files. While it freed the applications and users from the
underlying hierarchical name space, it led to an explosion in the number of
silos on a single device.  Some applications (e.g., cloud storage on mobile
devices) still expose hierarchical interfaces but even they tend to demote the
hierarchical name space in favor of other presentation models.  In essence, the
virtual directories of the semantic file system have become the primary
interface.

The \emph{MyLifeBits} project followed one person's goal of organizing their
own data: ``We hoped to substantially improve the ability to organize, search,
annotate, and utilize content. Also, we wanted to obtain a unified database in
contrast to the many data “islands” being created including mail, contacts, and
meetings, finances, health records, photos, etc. Frustration with the file
system led to testing the suitability of databases for personal storage, and
ultimately into research about next generation storage
systems~\cite{gemmell2002mylifebits}.'' Note that the authors
identified many of the same problems that still exist today, including the
multi-silo problem, yet those problems remain unsolved.

The use of richer contextual information to better search and organize data is
one that seems to be a perennial favorite for greater exploration.  The database
community has observed specific types of context that are useful: applications,
behavior, and change:  ``In decoupled systems, behavioral context spans multiple
services, applications and formats and often originates from high volume
sources~\cite{hellerstein2017ground}\ldots''

The use of ``personal digital traces'' is close to the work that I have proposed
as part of my thesis.  The authors observe: ``Search of personal data is usually focused on retrieving information
that users know exists in their own data set, even though most
of the time they do not know in which source or device they have
seen the desired information. Current search tools such as Spotlight
and Gmail search are not adequate to deal with this scenario where
the user has to perform the same search multiple times on different
services or/and devices rather than search over just a single service.
Besides, traditional searches are often inefficient as they typically
identify too many matching documents.~\cite{vianna2019searching}''

Finally, I note that the importance of context spans disciplines.  The study of
\emph{pragmatics} in Linguistics relates to the understanding of meaning within
the context in which it is used: ``Pragmatics is a field of linguistics
concerned with what a speaker implies and a listener infers based on
contributing factors like the situational context, the individuals’ mental
states, the preceding dialogue, and other
elements.~\footnote{\url{https://www.masterclass.com/articles/pragmatics-in-linguistics-guide}}''
Closer to home, the database community has explored the importance of
context in terms of human understanding: ``Context has often a significant
impact on the way humans (or machines) act and on how they
interpret things; furthermore, a change in context causes a
transformation in the experience that is going to be lived.
The word itself, derived from the Latin \emph{con} (with or together) and
\emph{texere} (to weave), describes a context not just
as a profile, but as \emph{an active process dealing with the way
    humans weave their experience within their whole environment, to give it
    meaning}.~\cite{bolchini2007data}''

\section{Storage Silo Access}
\label{ch:background:sec:storage-silos}

The number of novel storage implementations is large.  For example there are dozens of file
systems actively in use~\footnote{See
    \url{https://en.wikipedia.org/wiki/Comparison_of_file_systems}}. In addition,
there is a constant stream of new proposed
variants~\cite{10.1145/3477113.3487265,kadekodi2021winefs}.

Thus, rather than review the myriad of file systems that exist and contribute
novel storage silos, I instead focus on classifying storage silos by the
interface that is used to access them.

\begin{comment}
\begin{table}[tbhp]
    \centering
    \caption{Cloud Storage API reference}
    \label{table:cloud-apis}
    \begin{adjustbox}{max width=\textwidth}
        \begin{tabular}{ll}
            Cloud Storage & Reference                                                                 \\
            Dropbox       & \url{https://www.dropbox.com/developers/documentation/http/documentation} \\
            Amazon S3     & \url{https://docs.aws.amazon.com/AmazonS3/latest/API/Welcome.html}        \\
            OneDrive      &
            \url{https://docs.microsoft.com/en-us/graph/onedrive-concept-overview}
            \\
            Google Drive  & \url{https://developers.google.com/drive}                                 \\
            \\
        \end{tabular}
    \end{adjustbox}
\end{table}
\end{comment}

\begin{itemize}
    \item File system APIs.  Most file systems use or support a POSIX like
          interface, which typically includes create, open, close, read, write, delete,
          rename, and read directory.  In addition, most local file
          systems provide monitoring interfaces, which permits monitoring
          state change.

    \item Object Store. The OpenStack Object Store interface provides a
          useful definition of a robust implementation that maps to the HTTP
          protocol quite
          closely~\footnote{\url{https://docs.openstack.org/api-ref/object-store/}}.
          Object stores are commonly used because of their simplicity and
          sufficiency for a range of uses in both device local and internet
          enabled applications. Thus, object stores normally support an
          authentication protocol and object access protocol using get
          (retrieve object contents and meta-data), put (object create or
          update), copy, delete, head (retrieve object meta-data), and
          post (update object meta-data).

    \item Cloud storage.  There is a greater range of APIs for cloud storage,
          with each implementation typically providing documentation. Many of these
          consist of ``Web APIs'' which are implemented using the HTTP or HTTPS
          protocols. Areas of common functionality, albeit varying
          implementation, are authentication, file access, and file change
          notifications. Google Drive, Dropbox, Amazon S3, and Microsoft
          OneDrive all support mechanisms for authentication, file access
          (including meta-data access) and file change notifications. They do
          not use a common API for doing this so that an
          importation/interaction layer must be written for each one of them;
          ideally I expect to be able to produce a common event format that I
          can use with all of them.

    \item Databases. There are several types of common databases,
          including relational databases such as Oracle, MySQL, Microsoft SQL,
          PostgresSQL, IBM DB2, Sybase, and Teradata. They all support some
          variant of the common structured query language which is typically
          accessed via programming libraries that simplify the various
          differences. Non-relational databases have become increasingly
          popular in recent years.  Examples of commonly-used non-relational
          databases include MongoDB, Cassandra, Redis, and Neo4j and can be
          classified as document stores, column store, key-value stores, and graph stores.

    \item Applications. While applications \emph{also} tend to consume services from other
          storage layers, the context of those operations is often not
          visible. A file system has no way of knowing that the file just
          created by the e-mail program was an attachment, rather than a data
          file used by the application program itself --- but the
          application is aware of this important contextual understanding.
          There is far less structure or
          regularity of applications.  These applications tend to have their
          own unique interfaces. For example, graphical user interface based
          ``file browsers'' often have ``hooking''
          interfaces that permit intercepting higher level operations such as ``copy
          a file.'' Collaboration applications such as Slack, Discord, and Teams
          also have extension interfaces for interacting with them but these
          interfaces do tend to be specific to the application.  Commonly used
          non-web based e-mail programs such as Thunderbird and Outlook have public
          APIs for building extensions.

\end{itemize}

One of the challenges in trying to create a classification system for ``storage
silos'' is that the dividing line is often not clear.  For example, if one
accesses an Oracle database via a REST API, is that a database or cloud storage?
For the purposes of understanding the range of APIs this general breakdown is
sufficient and allows me to identify broad categories of silos to consider using
in my work.

\section{Existing Meta-Data}
\label{ch:background:sec:meta-data}

The volume of available meta-data is high enough that one of the challenges I
face in conducting experiments to evaluate my hypothesis is coping with the
volume of information.

My summary of useful information in \autoref{table:useful-information} includes
references to prior work that draws upon existing meta-data. The work regarding
collection of personal digital traces is particularly germane, because not only
did the authors identify useful information, they made their own tools publicly
available.

Beyond this, I can point to existing meta-data sources that I know exist and can
be used as part of my own work.  One is the extended Berkeley Packet Filter
(eBPF) support that is available on Linux and being added to Windows.  eBPF is
an extensible framework for collecting data from the running operating system by
injecting ``hooks'' that allow detailed monitoring.  There are already existing
eBPF filters that provide extensive meta-data.  For example, the Linux OSQuery
interface has an eBPF alternative backend for data collection.  The community
developing and extending eBPF is quite active and the scope of information
already available is
extensive~\footnote{https://ebpf.io/blog/ebpf-updates-2021-02}.  From the
perspective of testing my thesis, I expect it will not require extensive
development of new software.

Windows has an operating system level introspection package known as
Event Tracing for
Windows~\footnote{\url{https://docs.microsoft.com/en-us/windows/win32/etw/about-event-tracing}}
(ETW)
as well as Microsoft's recent work on supporting eBPF on Windows (likely by
leveraging their ETW
work)~\footnote{\url{https://microsoft.github.io/ebpf-for-windows/}}.

Both ETW and eBPF should provide ample existing meta-data from which to draw
upon.  Combined with the Personal Digital Tracing tools~\cite{vianna2014a}, there
is a rich set of existing meta-data from which to draw upon, making the design
and implementation of the tools I need to test my thesis simpler.

\endinput

\begin{comment}
There is an extensive amount of relevant prior work that has been done, not only within
the computer systems literature but also other related fields such as Human-Computer
Interface, Information Retrieval, Data Visualization, Data Analytics, Machine
Learning, Linguistics, Personal Information Management, and Security.

Fundamentally, the topic of \emph{finding} relevant data is one that is
important, particularly as the size of data collections continues to grow.  The
key related prior work falls into one of three categories:

\begin{enumerate}

    \item Important information explaining why the systems community must
          provide services for improving \emph{finding}.  Improving \emph{finding}
          is the fundamental reason I propose building \emph{Finding as a Service}.

    \item Insight into the types of information that are useful for enhancing
          \emph{finding}. This information comes from a diverse range of fields. My
          review of this literature is focused on identifying what information is
          already known to be useful.

    \item Background within the systems field about prior work upon which I am
          building.  I focus on adding usage information --- what I refer to as
          \emph{activity context} --- in combination with other information about
          digital objects.

\end{enumerate}



\section{Known Useful Usage Information}
\label{ch:background:sec:known-useful-usage-info}

\section{Prior Systems Work}
\label{ch:background:sec:prior-systems-work}

\begin{comment}

\tm{I have added a number of crucial papers here that need to be added to the
    bibliography and referenced here. They help establish why all of this is
    important.
}

\tm{It suddenly occurred to me \emph{why} the table top interface is important:
    it's more or less what we have with mobile devices.  No mouse/keyboard like a
    desktop/laptop and a need to be able to visually limit things.  MOST
    applications completely hide the file system beneath them and the few that don't
    (e.g., Dropbox) provide a mostly hierarchical view.  That makes me wonder if
    there's an interesting project in taking the work from the table top space,
    combining it with Connections, and MyDigitalBits (or whatever it is called),
    etc. and building an interesting interface to this stuff.  I don't want to build
    that, but if I have to I'll contract with someone to build it once I have a
    better idea of what I think should be built.  I know I found a graph style file
    browser at one point.  I need to find that paper again because that was
    interesting.
}

\tm{I moved this material from the Intro to here, since it seems more applicable to here.}
\section{Naming versus Finding}
\label{sec:intro:sec:naming-vs-finding}

Early work in describing information focused on the problem of \emph{naming} as
a means of identifying a specific digital object. Thus, very early systems used
single level name systems, which were appropriate for the systems at that time.

In 1956 IBM introduced its first fixed disk drive (IBM 350 RAMAC Disk,) which
stored up to 5MB of data.  That IBM disk drive weighed approximately
1,000kg~\footnote{https://www.7dayshop.com/blog/terabyte-evolution/}.
From this is is clear that the total universe of data was quite small by modern
considerations.

Of course, a single name space certainly did not scale well.  Fortunately, the
nascent field of computer science had an obvious metaphor to use: the physical
filing cabinet.  By 1958 the hierarchical name space was described as a logical
organization scheme for files~\cite{barnard1958}.  Hierarchical name systems
were quickly and broadly adopted in a number of areas, including file systems,
which provide logical computer storage organization. The idea of hierarchical
name space makes its way into modern operating systems through
Multics~\cite{daley1965general} and TENEX~\cite{murphy1972storage}.  The
research communit did explore non-hierarchical file organization models at least
by the 1970s in the CAP file system~\cite{needham1977cap}.

In the 1970s work on naming extended to rich ``naming networks'' with the
hierarchical name space used as a simplified solution required by resource
constraints of the time.  Saltzer defines a \emph{naming network} as ``a catalog
system in which a catalog may contain the name of any object, including another
catalog.  An object is located by a multi-component path name (q.v.) relative to
some working catalog (q.v)~\cite{Saltzer1978}.''

Usually these ``catalogs'' are tied to an underlying unit of storage, which I
refer to as a ``silo''.  While this includes traditional storage devices
such as ``disk drives'' which are managed by file systems, it extends to any
logical storage mechanism for maintaining digital data: object stores and
databases, which are typically presented to users via a rich interface,
such as through an e-mail server or collaborative messaging service.

Thus, it is logical to think of this as a \emph{naming} problem, which has led
to formal recommendations on naming \emph{schemes} that should be adopted to
simplify location of relevant information.  Beneath this is the key observation
that it is \emph{finding} that is the fundamental problem, not \emph{naming}.

Indeed, the systems community has long admitted the importance of
``findability''.  For example, the original UNIX developers explicitly called
out the presence of their indexing service as being an essential service when
presenting their research to the systems community~\cite{ritchie1973unix}. The
community proposed adding additional meta-data to
files~\cite{mogul1986representing}: this work gave rise to \emph{extended
    attributes} or \emph{properties} which are found in a number of file systems in
active use today~\cite{cao2007ext4,chutani1992episode}. Semantic file systems
were proposed to extract additional indexing information from the contents of
the files~\cite{gifford1991semantic}.  While semantic file systems never really
caught on, the idea of using file contents as part of an indexing system is
present in some form on all modern operating systems. More recent work has
suggested adding ``tag'' information, either manually or
automatically~\cite{tagfs,bloehdorn2006tagfs,Soules2004}, non-hierarchical name spaces over object
stores~\cite{seltzer2009hierarchical}, and graph file systems~\cite{di2017gfs}.

This exploration of presenting digital data to human users is not restricted to
the systems community.  The Information Retrieval and the Human Computer
Interface communities have pointed out the deficiencies of hierarchical name
spaces since at least the 1980s~\cite{malone1983how,vicente1987assaying} and the
research literature is rich with suggested alternatives~\cite{vicente1988accommodating,gemmell2002mylifebits,dumais2003stuff,collins2007tabletop}.

The bulk of this prior work focuses on characteristics of the digital objects we
seek to find.  Thus, prior work focuses on using file data and meta-data to
improve findability.  However, this inward looking focus does not allow us to
answer questions that are logical to a human but are related to the extrinsic
characteristics of digital objects: \emph{what else was happening?}

Some prior work has explored considering extrinsic factors. \emph{Placeless}
focused on using process level information extracted from their document processing
system to associate files together~\cite{placeless-tois}. \emph{Burrito} proposed \emph{activity context},
which they define as ``the user's actions at a particular time~\cite{guo2012burrito}.''
Provenance information relates to directly observable information about the
construction of a given digital object and it can be used to augment file
search, which in turn improves findability~\cite{provsearch}.

Thus, the goal of my thesis is to expand upon this idea of looking at extrinsic
information to better understand the use of digital artifacts, which will enable
other communities, such as the HCI community, to build powerful new tools to
enable finding relevant digital content.

There is an extensive history of work in storage.  Thus, rather than attempt to
review all potentially related work I have tried to curate it in focusing on key
aspects of my own thesis. Thus, I attempt to explore the broad range of things
that might be considered a ``storage silo''.  Then I review the information that
prior research has identified is useful in improving the \emph{finding} process.
I also review the information that we already collect on our systems, both
information that has previously been applied to the finding problem as well as
information that is collected but does not appear to have been used.  Throughout
this process I strive to try and point back to the use cases
(\autoref{ch:intro:sec:use-cases}) and explain why this prior work does not
satisfy these use cases.


\section{Storage Silos}
\label{ch:background:sec:storage-silos}

I use the term \emph{Storage Silo} in a very broad fashion.  My motivation for
doing so is to admit that many tools exist that \emph{de facto} serve as a form
of storage provider, even if we do not ordinarily think of them as such.  For
example, if one uses an e-mail program it combines some form of internal
database with existing storage.  To the user, this information is presented in a
name space that enables finding using ordering and search.  Associated
information, such as a spreadsheet, executable program, or image is associated
with a given e-mail message --- we often say it is \emph{attached}.  Even though
the e-mail program might use other storage services, we normally do not expect
users to look directly through the database and storage location of the e-mails
and attachments.  Thus, it is logical to think of this as a distinct storage
silo.

To simplify this process, I propose the following taxonomy of storage
silos:

\begin{itemize}
    \item \textbf{Local device storage}.  This is the storage provided by the
          local system.  Typically this will have an hierarchical name space
          representation and while its contents may be shared with other computers,
          normally it is not shared.  The usual expectation then is that there is a
          software component which manages the namespace, the allocation and freeing
          of storage space, and the enforcement of any access security provided by the
          storage system.  Typically, this would be what we call a \emph{file
              system}. Examples of this would include FFS~\cite{mckusick1984a},
          NTFS~\cite{custer1994inside}, EXT4~\cite{cao2007ext4}, and Apple's
          proprietary
          APFS~\footnote{\url{https://developer.apple.com/documentation/foundation/file_system/about_apple_file_system}}.
          All of these file systems provide at least one notification mechanism
          for when the contents of the file system change. These file systems
          provide varying level of support for ``extended attributes'' but none
          provide a way to efficiently search those extended attributes. None of
          them have a way of associating activity context between local files
          and other storage silos.


    \item \textbf{Remote storage}.  This is storage accessed by the local system
          but usually provided by a remote system.  Typically this will have an
          hierarchical name space representation that the remote system maintains and shares with
          the local system. While local storage is typically always available, remote
          storage usually requires network access to the remote storage facility.
          Further, the limited bandwidth and increased latency may make it impractical
          to perform brute force operations such as search on the remote storage in
          many instances.  Remote storage does not have a common model for
          associating activity context with files.

    \item \textbf{Cloud storage}.  In recent years a number of different cloud
          storage solutions have entered common use.  This includes IDrive, Google Drive,
          Nextcloud, iCloud, pCloud, Box, Spider Oak, Microsoft OneDrive, Adobe
          Cloud, and Dropbox.  Each of these cloud storage mechanisms
          does provide search tools, but they do not use a common interface and
          searching across them requires looking at each one.  Increasingly common is
          to use a sparse file technique where the entire namespace is stored on the
          local computer system but the data contents of the files are retrieved in an
          on-demand fashion. As I noted with remote storage, this limits the ability
          to perform brute-force searches because it may require transfering large
          amounts of data from the cloud storage to the local system. The examples
          that I have provided all generally have some sort of hierarchical name
          space for organizing content.  None of these systems provide a mechanism for
          associating activity context with cloud stored files or querying such
          state.

    \item \textbf{Database}.  The line between databases and file systems has
          always been a fluid one, where the capabilities of one are sometimes adopted
          by the other.  Applications use databases because they offer the ability to
          perform certain data operations rapidly, such as finding relevant records
          within the database.  Indeed, the idea of presenting a database as being a
          file system with an hierarchical name has been previously
          explored~\cite{inversion}.  Examples of commonly used local databases
          include Microsoft SQL, MySQL, PostreSQL, MongoDB, OrientDB, MariaDB,
          SQLite. Each database supports a mechanism for accessing specific
          content within the database externally, suggesting that information
          could be constructed externally.  None of these databases provide a mechanism
          for associating activity context with database entries, though they
          normally provide a mechanism for identifying specific content stored by the
          database externally.

    \item \textbf{Object Stores}.  Object stores have become a common mechanism
          for providing flexibile file systems-like storage without the requirement of
          an hierarchical name space.  Such stores can be local --- a simple key-value
          store such as BerkelyDB used for rapid access to application data, for  example, or remote ---
          Amazon S3, Google Cloud Storage, or Azure Blob Storage.  None of these
          systems provide a mechanism for associating activity context with objects.


    \item \textbf{Collaboration Tools}.  Increasingly, a common way to share
          data is through a collaboration tool.  Examples of this include: Slack,
          Discord, Teams, Asana, Trello, and Ryver.  These act more like
          non-traditional storage silos, yet they are used precisely that way:
          we share files using them, download them to our local system, and then
          use them.  Intriguingly, at least some of these allow extracting a
          link to the message in which a file has been shared, which suggests
          that we can create an external linkage. To the best of my knowledge,
          none of these systems provide a mechanism for associating activity
          context with their objects.

    \item \textbf{E-Mail programs}. While early e-mail programs did not have a
          mechanism for ``attaching documents'' we gained the ability to do so more than
          two decades ago~\cite{rfc2045}. While there is a set of standards about the
          encoding, most e-mail programs provide the ability to independently access
          attached documents.  As a result, our very e-mail boxes have become yet
          another commonly used storage silo. While e-mail providers can be local
          programs on our computers, web interfaces, or some amalgam of these, they all
          provide mechanisms for accessing attached documents. Examples of common e-mail
          providers that support attached documents includes: Gmail, Outlook,
          Thunderbird, Edison, Spike, Proton, Front, and Mimecast.  While search
          mechanisms are common, none of these e-mail programs appear to provide a
          mechanism for associating activity context with their e-mails and attached
          documents.  Note that this is such a common scenario, I described it as one of
          my use cases (\autoref{use-case:e-mail}).

\end{itemize}

\section{Useful Information for Finding}
\label{ch:background:sec:useful-information-for-finding}

Much of systems research is focused on \emph{search} but for those using search
their primary goal is \emph{finding}.  Other communities have a much broader
sense of utility:

\begin{quotation}
    \emph{
        A well designed computer system permits use of the tools it offers without
        requiring users to dedicate extensive mental processing to operations
        inherent in the system design rather than the task.  Furthermore, its
        tools are designed to also reduce task-specific metnal processing,
        especially those types of processing that are performed more effectively
        by computers than by people, such as calculations and accurate storage and
        recall of large amounts of pre-specified
        information~\cite{brown1999human}.
    }
\end{quotation}

Thus, the emphasis for researchers working with data retrieval is not to
evaluate the efficiency of the search infrastructure but rather to determine if
that infrastructure leads to the human user finding the document they
seek~cite{bergman2019factors}. The need for finding is one that the HCI
community continues to explore: one recent example relates to finding documents
that are no longer needed, a specialized, complex, yet also important aspect of
data management~\cite{Vitale_2020}. Finding related content is sometimes also
required for compliance~\cite{ritzdorf2014assisted}.
Fundamentally, though, the most important
example of finding is when a human user is looking for a \emph{specific} set of
files.  Abandonment --- where the user gives up on the search --- is certainly
one viable way of determining success of potential new tools, since users stop
when they are either successful or have exhausted their patience.

Prior work has suggested a number of different factors that can be used to find
content:

\begin{itemize}
    \item Naming similarity; similarly names files are often related.
    \item Content similarity: files with similar content are often related.
    \item Temporal similarity: files that were created and/or accessed around
          the same time are more likely to be related~\cite{soules2003can}.
    \item Causality: files that are created using the same tool
          are more likely to be related.
\end{itemize}

In addition, we hypothesize that there are usage factors that can be used to
find content:

\begin{itemize}
    \item Location similarity: files that were accessed or modified in the
          same geographical location may be related.  I suspect this is really more
          a factor for mobile devices.

    \item
\end{itemize}

Some recent findings in the personal information management research community
are quite counter-intuitive, such as finding that search tends to take longer to
find something than navigation~\cite{bergman2019search}. While this suggests to
me that an approach in which search is used by tools to find relevant
information, using the virtual directory approach used with semantic file
systems~\cite{gifford1991semantic} might be more effective.

Similarly, prior work has established the need to be able to support a broad
range of storage locations --- the ``storage silos'' that I have previously
mentioned.  In \emph{Stuff I've Seen} researchers found that the ability to
search across these storage domains in a uniform fashion led to increased
utilization of such tools~\cite{dumais2003stuff}. Indeed, the observation in the
personal information management community repeatedly stresses the importance of
supporting cross-silo management of digital data.

This makes sense: when we are looking for a specific object that we know exists,
we do not particularly care \emph{where} it is stored.

\tm{My sense is I need to do more here but this is quickly becoming a time sink
    and I don't think it fundamentally changes the narrative.  Thus, I defer this.
}

\section{Meta-data Collection}
\label{ch:background:meta-data-collection}

Modern systems collect a dizzying amount of information from our computing
devices: the websites we visit, the programs we run, the files with which we
interact, the music we play, our location, the time of day, etc.  Some prior
work has used this information to improve search, including
Burrito~\cite{guo2012burrito} and provenance directed
search~\cite{provsearch,uprove2,vianna2019searching}.  Burrito focused on the
operations directly related to the user \emph{actions} which is narrower than
considering the user's \emph{environment}.  Similarly, provenance focuses on
causality, which is certainly useful for excluding noise, yet can also miss
information that might prove useful.  Examples that I am considering
include:

\begin{itemize}
    \item Location --- this type of information is routinely used by
          advertisers on a user's computing device to determine what advertisements
          to show, so we know this information is available.  Being able to identify
          content that is linked by \emph{where} the files were created and/or
          accessed yields potentially interesting new data for sifting the data
          presented to users.

    \item Music --- we have posited that some users may associate what they
          were doing with what they were listening to at the time.  While it may be
          possible to capture audio recordings of ambient sound, we also note that
          integration between music players and other applications \emph{already}
          exists, and thus can be captured as well.  Being able to sort through
          content based upon music could be useful.  While prior work has not
          explicitly identified music, it has identified \emph{mood} as a factor
          that impacts file usage and the two can be linked.

    \item News events --- external events do impact human users and create
          associative context that might be generally useful in finding relevant
          content.  Existing digital devices already monitor news events on our
          behalf, so the data is readily available.  Of course, news in this context
          could be fairly general and include the weather, notable local events, or
          events at national or international scale.  Both free and paid
          services are available.  For example \url{https://mediastack.com}.
          Their API is a publicly documented REST API.

    \item Web access --- we know it is common that searching the internet
          using a web browser is a common activity during creative endeavors,
          including writing this document.  Thus, being able to identify web pages
          that might have been accessed and/or used is another logical source of
          information.  This information is already available inside existing
          browsers, so it could be incorporated as one more event in the relevant
          activity context.  There are a range of ways that internet access
          can be identified, including inside the browser via a browser
          extension, using the network activity information from your local
          system, using eBPF filters to monitor network activity, etc. We know
          this can be done because existing products provide such features,
          such as SentryPC (\url{https://www.sentrypc.com}), which works with
          both Windows and Mac computers. Linux makes similar information
          available through eBPF and/or monitoring the proc file system. Note
          that the specific APIs for these are generally platform specific.

    \item Chat --- we often speak with people about projects we are
          working on together.  Thus, knowing that we were working with
          someone on a particular project yields another example of activity
          context that could be used to improve our ability to find the
          relevant content.  I am not certain if this information is readily
          available from existing communications applications.  If it is, this
          could yield another powerful example of activity context that proves
          useful to the broader goal of identifying applicable files.

\end{itemize}

This list is hardly exhaustive; rather I describe these elements as
demonstrative.  One of the interesting aspects of this work will be to
collaborate with researchers from other fields to suggest a more thorough list
of potential information that we can provide and then work with them to identify
the most interesting.  Ideally, this should produce a feedback loop that can be
used to improve activity contexts in the future.

\end{comment}
