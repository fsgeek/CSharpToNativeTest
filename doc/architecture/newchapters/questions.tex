\chapter{Research Questions}
\label{ch:research-questions}

\begin{epigraph}
    \textit{For every particular thing to have a name is impossible. --- First,
        it is beyond the power of human capacity to frame and retain distinct
        ideas of all the particular things we meet with: every bird and beast men
        saw; every tree and plant that affected the senses, could not find a place
        in the most capacious understanding.} --- \textbf{John Locke}, \textit{An Essay
        Concerning Human Understanding}~\cite{locke1844locke}
\end{epigraph}

The goal of my research is to develop a framework for gathering and
disseminating rich data tracking activity across a user's silos and devices to
facilitate end-user \emph{finding}.  While the evaluation of these finding
questions lies in the domain of HCI researchers, my research must enable
the collection, aggregation, storage, and querying the data that these
researchers can use to evaluate different finding approaches.  Thus, my research
will answer the following questions:

\begin{enumerate}
    \item \label{rq:define-ac} \textbf{What data comprises activity context?} As
          previously observed
          (\autoref{ch:background:sec:useful-information-for-finding}) there is
          a wealth of potential information that could be included in the activity
          context.  This question asks: ``what data should be included in the activity
          context?''

    \item \label{rq:capture-ac} \textbf{How do we capture activity context?}
          While there are numerous different potential sources for activity
          information, how do we capture and store it?

    \item \label{rq:rich-data} \textbf{How do we collect rich data (including
              activity context) from multiple storage silos and make that rich data
              accessible efficiently?}
          We know from prior work that one can
          capture \emph{all} state for a single computer system with
          surprisingly moderate cost~\cite{devecsery2014eidetic}, but that
          approach does not make the collected state easily available.
          Thus, collecting and storing the rich data is not sufficient, we
          also need to make it accessible to applications. How do we make it
          available to applications in a manner whereby the benefits of better
          findability outweigh any impact on application performance?

    \item \label{rq:meta-data-query} \textbf{How do we facilitate query of this
              rich meta-data?} It is important that both users and applications have a
          mechanism via which they can exploit this extensive collection of rich
          meta-data.  What does the query interface for this rich meta-data look like?

    \item \label{rq:leverage-ac} \textbf{How can applications leverage this rich
              meta-data?} Applications must programmatically find the documents with which
          they interact.  Most applications use temporal relationships, such as
          ``recently accessed documents,'' as a primary mechanism to present users
          with a set of files they might wish to access.  When that fails, they fall
          back to offering the user files in a directory or list that the application
          deems `likely'.  This question asks: ``if an application has access to
          activity context data, how can it use the information to give users a better
          collection of candidate files from which to select?''

    \item \label{rq:privacy} \textbf{How do we preserve the privacy of sensitive
              meta-data?} In a model that incorporates extensive amounts of personally
          identifiable information there is a very real risk that this information
          will prove to be economically valuable to someone; how do we ensure that the
          user retains ownership and control of that information so that it is only
          released when they approve doing so?

\end{enumerate}

I intend to provide data that will allow other researchers to answer related
questions such as:

\begin{itemize}
    \item What relationships are most valuable in helping users find data?
    \item Does activity context provide better information for user search than
          semantic information alone?
    \item What interfaces best allow users to leverage rich meta-data search
          capabilities?
\end{itemize}

The questions themselves that are core to my proposed thesis work revolve around
exploring and broadening the of an \emph{activity context}. However, activity
context by itself does not replace the other prior work that
has added semantic understanding around files, e.g., semantic file systems,
formal concept analysis, tag file systems, graph file systems, etc. Thus, my
proposed system (\autoref{ch:architecture}) includes support for this prior work
but extends it by augmenting previous mechanisms with this additional rich usage
information that I call ``activity context.''

\endinput

The goal of my research is to explore a number of key research questions.  One
of the complications involved in this work is that defending my thesis will best
involve work by others.  Thus, in this section I identify key research
questions; from these I then identify those that relate to those I propose
constructing as part of my thesis.  The goal of the other questions then is to
provide guidance as to what I envision as the usage model for this work.

% Table generated by Excel2LaTeX from sheet 'Research Questions'

\begin{enumerate}
    \item\label{rq:events} \textbf{What events are useful in establishing relationships between
        objects?}  Prior work has explored a few relationships.  For example, the
    provenance guided search work identified \emph{causality} as being a better
    predictor of a relationship than temporal locality.  However, modern
    computer systems capture a vast array of information.  This provides us with
    intriguing opportunities to consider what of just the currently captured
    information is useful.  Beyond that, I propose there may be information that
    we do not currently capture that would also be useful.  The function of this
    research question then is to identify what information would be useful.
    My expectation is that answering this research question is likely best done
    in collaboration with researchers from other communities, such as HCI, in
    order to identify specific events that turn out to be useful.

    \item\label{rq:improve} \textbf{Does adding activity context improve associative access to
        files?}  More specifically, prior work has identified that using extracted
    semantic and specific tag information yield useful insight into file
    associations.  While the purpose of my work is to provide the tools
    necessary to answer this research question, I consider it to be beyond the
    scope of the work I will do in support of my thesis.  Ideally, I would work
    collaboratively with researchers in other fields better suited to evaluating
    this, likely as part of the work involved in answering question
    \ref{rq:events}.

    \item \label{rq:existing-apps} \textbf{How can existing applications benefit from the availability of
              activity context?} There are existing ideas in this space that have been
          explored such as ``virtual directories.'' While interesting, I don't
          consider this to be a core question to answer as part of my thesis.  I
          suggest this would be a good project to work with another systems researcher
          to explore in collaboration.

    \item \label{rq:providing-ac} \textbf{How do we provide activity context to enable others to exploit
              it to build better tools?}  This question presupposes that the answer to
          question \ref{rq:improve} is yes.  Thus, assuming that we find activity
          context does improve associative access to files, how do we provide access
          to that information.  There is a body of existing work upon which to draw
          for inspiration regarding meta-data queries of both static and dynamic
          sources, but it seems likely there will be at least systems-specific issues around
          processing a potentially high volume event stream efficiently.

    \item \label{rq:privacy} \textbf{How can we provide distributed meta-data services while
              preserving privacy?} Prior work in utilizing ``personal digital traces''
          raised similar concerns and the approach they took --- to store the data on
          the user's computer --- seems like the correct solution for the work that I
          do for my thesis~\cite{vianna2019searching}.  However, I note that it seems
          likely that this question is a logical follow-on if this system proves to be
          useful because I can envision benefits of being able to share activity
          context between multiple devices for a single user as well as the ability to
          share activity context with other users with whom the owning user is
          collaborating.  Thus, I do not propose fully addressing this question within the
          context of the work for my thesis.

\end{enumerate}

I do not propose answering all of these research questions.  My goal in
providing them is to lay the groundwork for understanding why the research I
propose doing in support of my thesis could have substantial impact.
