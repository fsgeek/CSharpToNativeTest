%!TEX TS-program = lualatex
%!TEX encoding = UTF-8 Unicode

% I use LuaLaTex by default so that I get full UTF-8 support, which simplifies
% when using mixed language.

\documentclass[sigconf,anonymous,review]{acmart}
\setcopyright{none}  % suppress copyright generation
\usepackage{lipsum}
\usepackage{xspace}

%% Change this to change the name of the system
\newcommand{\system}[0]{\emph{Kwisatz}\xspace}

\author{Tony Mason}

\title{\system}
\subtitle{Enabling Activity Context}

% These are marks for inserting comments. Feel free to edit as needed!
\newcommand{\nb}[2]{{\yellowbox{#1}\triangles{#2}}}
\newcommand{\nbc}[3]{
 {\colorbox{#3}{\bfseries\sffamily\scriptsize\textcolor{white}{#1}}}
 {\textcolor{#3}{\sf\small$\blacktriangleright$\textit{#2}$\blacktriangleleft$}}}
\newcommand{\version}{\emph{\scriptsize\id}}
\newcommand{\ugh}[1]{#1} % please rephrase
\newcommand{\ins}[1]{#1} % please insert
\newcommand{\del}[1]{} % please delete
\newcommand{\chg}[2]{#2} % please change
\renewcommand{\nb}[2]{\nbc{#1}{#2}{orange}}

% Tony
\definecolor{tmcolor}{rgb}{0.5,0,0.5}
\newcommand\tm[1]{\nbc{TM}{#1}{tmcolor}}

% Margo
\definecolor{miscolor}{rgb}{0.4,0.6,0.2}
\newcommand\MIS[1]{\nbc{MIS}{#1}{miscolor}}

% Ada
\definecolor{adacolor}{rgb}{1.0, 0.5, 0.5}
\newcommand\ada[1]{\nbc{AG}{#1}{adacolor}}

% Sasha
\definecolor{sfcolor}{rgb}{0.2,0.0,0.5}
\newcommand\sasha[1]{\nbc{SF}{#1}{sfcolor}}

\begin{document}


\begin{abstract}
    Our ability to find digital data is reaching a tipping point: brute force
    search techniques are inefficient and searching multiple storage locations
    to find related objects is challenging.  Prior research found using
    contextual clues facilitates finding specific digital objects. Despite
    modern systems collecting vast amounts of contextual information, our
    systems do not provide an efficient mechanism for using that information to
    facilitate more efficient \emph{finding} of digital objects.

    \emph{\system} is our system for collecting, storing, and
    disseminating contextual information we call \emph{activity context} to
    facilitate finding groups of related digital objects regardless of where
    those objects are stored.  We find \emph{\system} is a viable way to
    provide \emph{activity context} and enabling its use by other services and
    applications.
\end{abstract}

\maketitle

\section{Introduction}

\tm{TBD}

\section{Background}

\tm{TBD}

\section{Architecture}\label{sec:Architecture}

\system is logically composed of three major components, each of which is an
essential portion of providing the end-to-end systems level support for
capturing, storing, and utilizing \emph{activity data}.  Each of these
components consists of various smaller components assembled together to provide
the necessary services.

In this section, I lay out the basic architecture of these components.  In
subsequent sections, I will drill down into this architecture and identify key
aspects of the system design. In constructing this architecture, I have
attempted to broadly address what I consider to be the key aspects of these
components, including defining terminology and identifying potential use cases
that may be relevant.  Frequently, I will suggest potential implementations that
would fit within this architecture: there is no expectation that I will
implement even a majority of these potential implementations.  Rather, the goal
of using broad considerations is to assist in ensuring the system architecture
is reasonably flexible. Ultimately, my goal is to demonstrate the architecture
is itself viable as a system service.

\begin{figure}
    \caption{\system Architecture Diagram}\label{fig:architecture}
    \textbf{TODO}
\end{figure}

\begin{description}
    \item[Ingestion] --- the activity data that are presented by various
        services within the system needs to support a rich and robust model in which
        captured data may be converted into a common form that permits
        utilization. \S \ref{sec:Architecture:Ingestion}
    \item[Storage] --- raw activity data, along with extrapolated activity
        meta-data, need to be stored in a format that is scalable and efficient as
        well as supporting a robust utilization model. \S \ref{sec:Architecture:Storage}
    \item[Utilization] --- to realize the benefits of activity data, the system
        must have a useful model for using the activity data. \S \ref{sec:Architecture:Utilization}

\end{description}

\subsection{Ingestion}\label{sec:Architecture:Ingestion}

Activity data can arise from a variety of sources.  For example, because my
primary focus is on utilizing activity data to better inform logical data
organization, I view data storage as being a key source of such information.
Indeed, much of the prior work in this area has focused on utilizing information
about data, including:

\begin{description}
    \item[File Names] --- the \emph{file name} is a time honored way to embed
    information about the file itself.  For example, in
    \emph{Burrito}~\cite{guo2012burrito} the observation is that we capture
    parameter information within the file name.  This is because it is the
    \emph{only} way to safely capture this information in a way that is broadly
    viable across file systems.

    \item[Extended Attributes] --- the \emph{extended attribute} is a mechanism
    that provided a way for applications to create additional meta-data using an
    attribute/key model~\cite{mogul1986representing}.  File systems that support extended attributes maintain
    them as meta-data of the file itself.  Unfortunately, extended attributes
    suffer from two limitations: (1) file systems that do support them provide
    no mechanism for associating files based upon the extended attributes; (2)
    there is no uniform support or implementation of extended attributes.

    \item[File Meta-data] --- most file systems support at least a minimal
    subset of meta-data elements, specifically timestamps and size. There is a
    lack of uniformity of other attributes: POSIX file systems typically support
    ``mode'' bits that represent access permissions as well as potentially other
    behaviors, as well as an access time, modification time, and change time.
    Creation timestamps are often maintained by file systems as well.  A number
    of UNIX-like file systems maintain a creation timestamp.  Windows includes
    the \emph{creation} time of the file and that has been adopted by a number
    of UNIX-derived file systems, including ext4, jfs, and btrfs.  Recent
    changes in Linux include a new system call, \textbf{statx} that permits
    applications to retrieve this information programmatically.

    \item[Directories] --- the traditional hierarchical file system provides a
    mechanism for composing groups of files into a \emph{directory}, which is a
    set of files.  Some file systems restrict a file to being a member of a
    single directory, while others allow a file to be a member of multiple
    directories. The Mutics file system included the ability to create links to
    files.  Modern file systems may implement links as either direct references
    from the directory to the file (``hard links'') or indirect references from
    an entry in the directory to the file (``soft links.'')  They have somewhat
    different semantics.

    \item[Views] --- in semantic file systems~\cite{gifford1991semantic} there
    is less emphasis on reference counted relationships (e.g., hard links) or
    even persistence and more emphasis on creating logical groups (``virtual
    directories'') of files based upon some criteria.

\end{description}

\system does not seek to \emph{replace} any of these existing file storage
mechanisms.  Instead, it focuses on providing rich support for meta-data
\emph{about} digital objects, which includes files but should not be limited
strictly to files.

Meta-data is not strictly limited to the data that is available from the file
system itself.  Further, meta-data may be about digital objects that
\emph{existed} at some point in time but no longer exist: this is a reality of
separating the storage from the meta-data service. Instead of focusing on
maintaining a strongly referenced model, I instead adopt the model of the
Internet, which means that meta-data may reference digital objects that no
longer exist.  Applications that use \system should be aware that the underlying
data could cease to exist and act accordingly.

Other potential sources of meta-data are quite broad and include:

\begin{description}
    \item[Semantic transducers] --- the term \emph{transducer} was first
    introduced as part of semantic file systems~\cite{gifford1991semantic}.  The
    basic idea was that active components would \emph{extract} semantic
    information from the contents of a file and then use it for indexing.
    Indeed, modern indexers work on this basic principle, without making any
    changes to the file system.

    \item[Content classifiers] --- one common use for machine learning is to
    identify the content of specific files, such as images or videos, to
    determine if the given file contains specific content: a cat, for example.
    Such classifiers can be more targeted, such as finding pictures of a
    specific person, or containing a \emph{particular} cat.  This information
    can then be used to cluster files together, such as the ``video reels'' that
    some service providers now give us on our personal devices.

    \item[Hashing] --- a \emph{hash} can be computed on a given file to
    determine if the contents of said file has changed.  For example, this can
    be used by ``cloud storage'' providers to determine if a given file has
    already been uploaded.  Hashing can also be used to determine when files
    have changed.

    \item[Metrics] --- one common approach to information is to establish the
    logical proximity of the files, whether based upon the \emph{content} of the
    file, or the \emph{meta-data} of the file.  For example, plagiarism tools
    like MOSS look for structural similarity (versus simpler textual similarity)
    by comparing the abstract syntax trees of code.  This generates a measure of
    similarity.  The \emph{value} of the metric is not material to this project,
    but the \emph{use} of metrics is because it allows us to create a logical
    distance between the objects.  This can, in turn, be used to \emph{cluster}
    objects that are ``close to'' each other.

    \item[Environment] --- our devices maintain multiple sources of
    environmental information.  For example, location information (\emph{GIS})
    identifies where a device is located at a given time.  Increasingly, our
    devices track other aspects of the environment, including the ambient
    temperature, our vital measurements such as heart rate and blood pressure,
    and even more detailed health information including data from pace makers,
    insulin pumps, and menstrual cycle trackers.  The data from these is likely
    useful in creating associations between extrinsic events and human storage
    usage.

    \item[Social] --- our devices routinely track our social activity: with whom
    we are interacting via text, chat, e-mail, and video call, for example.
    Frequently, as part of this we share information --- information that is
    subsequently stored, modified, and re-shared onwards.  This type of activity
    data can be used to help us identify where information came from, what other
    digital objects were accessed as a result, and establishing data
    relationships based upon usage patterns.  Web sites visited, Reddit posts
    liked, Discord messages exchanged, music listened to, purchases made, and
    even games played can all be used to construct associative relationships
    that make sense to human users.

\end{description}

In general, activity data can be \emph{intrinsic} to the digital object, such as
information based upon its semantic content, it's length, and contents as well
as \emph{extrinsic} to the digital object, such as what applications were used
to access it, where things were done, with whom, etc. The \system architecture
is influenced by my desire to ensure support for a broad range of activity data
sources.

In \S \ref{sec:ingestion} I delve into greater detail about handling activity
data.

\subsection{Storage}\label{sec:Architecture:Storage}

The choice of storage models, while important, is unlikely to give rise to
significant research questions at this juncture: as we begin to understand the
nature of the activity data we anticipate collecting, it is distinctly possible
that new challenges will emerge.  However, we have an extensive body of
knowledge on how to handle high data rates (``drinking from the fire hose'') as
well as scaling approaches for managing potentially large bodies of data.

Thus, while the architecture is fairly neutral with respect to the details here,
I anticipate the initial implementation of this will utilize ``reasonable''
limitations on activity data sources (e.g., curation to avoid excessive data
loads).

An important aspect of the architecture is to propose a model for the data
format that I propose using.  This data format must be able to:

\begin{itemize}
    \item Identify the \emph{source} of the activity data.  This permits
    interpretation of the captured data by any transducer familiar with the data
    generated by the given source.

    \item Specify the \emph{version} of the activity data.  My own review of
    numerous data sources suggests that it is common for many of them to change
    the format of the data over time; typically this \emph{extends} the data
    format (common for systems-related activity data sources, for example) but
    sometimes it involves significant restructuring of the data that is
    available (common for web-based activity data sources.)  By including a
    version, a transducer can determine if it understand the format of this data
    and permits evolution.

    \item Provide an ordering of the activity data.  Typically this would be
    a ``timestamp,'' though there is no reason this needs to be a timestamp
    relative to any other data source.  Further, when there are multiple
    providers of information, the interpretation of this Lamport clock is
    ultimately determined by the transducer. This \emph{allows} both system
    relative and universal clocks but does not dictate their presence nor
    disallow clocks that are shared between activity data providers.

    \item A list of \emph{attributes}.  These are in the format of ``extended
    attributes,'' with both an identifier as well as a value.  This permits
    attributes that can be the same across sources, as well as allow the same
    ``attribute'' to have different meanings for different sources.  This is
    neither required nor prohibited within the \system architecture.  This
    ability provides very broad support for activity data, as well as \emph{post
    hoc} supplementation by transducers.

    \item The \emph{raw data} originally captured by the activity data provider.
    This allows the capture of information without requiring interpretation at
    the time it is captured. In cases where there is no additional raw data,
    this can remain empty.  Note that this is \emph{not} anticipated as being an
    area in which to store the digital object's data.

\end{itemize}

As a concrete example, I have implemented an activity data provider that scans
and captures the change data from the NTFS USN Journal on a Windows 11 computer.
The raw data is captured, and then certain elements of the data can be
augmented.  For instance, the raw data provides a \emph{file id} that is used to
obtain the name of the file.  Similarly, the raw data also provides a
\emph{directory id} that is used to obtain the path of the containing directory.
This is relative to the NTFS volume (which does \emph{not} include a drive
letter.)  Thus, the transducer for this can utilize the \emph{volume id} to map
to the current drive letter, making this name available for ordinary Windows
applications (which tend to use drive letters, even though they aren't visible
to the NTFS file system controlling a given volume.)

Subsequent to this, I envision a separate transducer that can be used to compute
the hash value of the file's contents.  That hash value could then be
incorporated into the attributes list of the corresponding change journal record
(assuming the file has not changed by the time the hash value has been computed,
of course.)

This data format can then be easily captured as a JSON expression, which can
then be used to store the relevant data in a database (e.g., DynamoDB or
MongoDB, for example.)

\subsection{Utilization}\label{sec:Architecture:Utilization}

\section{Ingestion}
\label{sec:ingestion}

\subsection{Use Cases}

\subsection{Evaluation}

\section{Storage}
\label{sec:storage}

\subsection{Use Cases}

\subsection{Evaluation}

\section{Utilization}
\label{sec:utilization}

\subsection{Use Cases}

\subsection{Evaluation}

\section{Conclusion}

\nocite{*}
\clearpage

\bibliographystyle{ACM-Reference-Format}
\bibliography{bib/indaleko.bib}

\end{document}

