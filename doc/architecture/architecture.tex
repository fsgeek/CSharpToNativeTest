%!TEX TS-program = lualatex
%!TEX encoding = UTF-8 Unicode

% I use LuaLaTex by default so that I get full UTF-8 support, which simplifies
% when using mixed language.

\documentclass[sigconf,anonymous,review]{acmart}
\setcopyright{none}  % suppress copyright generation
\usepackage{lipsum}


\author{Tony Mason}

\title{Indaleko}
\subtitle{Enabling Activity Context}

% These are marks for inserting comments. Feel free to edit as needed!
\newcommand{\nb}[2]{{\yellowbox{#1}\triangles{#2}}}
\newcommand{\nbc}[3]{
 {\colorbox{#3}{\bfseries\sffamily\scriptsize\textcolor{white}{#1}}}
 {\textcolor{#3}{\sf\small$\blacktriangleright$\textit{#2}$\blacktriangleleft$}}}
\newcommand{\version}{\emph{\scriptsize\id}}
\newcommand{\ugh}[1]{#1} % please rephrase
\newcommand{\ins}[1]{#1} % please insert
\newcommand{\del}[1]{} % please delete
\newcommand{\chg}[2]{#2} % please change
\renewcommand{\nb}[2]{\nbc{#1}{#2}{orange}}

% Tony
\definecolor{tmcolor}{rgb}{0.5,0,0.5}
\newcommand\tm[1]{\nbc{TM}{#1}{tmcolor}}

% Margo
\definecolor{miscolor}{rgb}{0.4,0.6,0.2}
\newcommand\MIS[1]{\nbc{MIS}{#1}{miscolor}}

% Ada
\definecolor{adacolor}{rgb}{1.0, 0.5, 0.5}
\newcommand\ada[1]{\nbc{AG}{#1}{adacolor}}

% Sasha
\definecolor{sfcolor}{rgb}{0.2,0.0,0.5}
\newcommand\sasha[1]{\nbc{SF}{#1}{sfcolor}}

\begin{document}


\begin{abstract}
    Our ability to find digital data is reaching a tipping point: brute force
    search techniques are inefficient and searching multiple storage locations
    to find related objects is challenging.  Prior research found using
    contextual clues facilitates finding specific digital objects. Despite
    modern systems collecting vast amounts of contextual information, our
    systems do not provide an efficient mechanism for using that information to
    facilitate more efficient \emph{finding} of digital objects.

    \emph{Indaleko} is our system for collecting, storing, and
    disseminating contextual information we call \emph{activity context} to
    facilitate finding groups of related digital objects regardless of where
    those objects are stored.  We find \emph{Indaleko} is a viable way to
    provide \emph{activity context} and enabling its use by other services and
    applications.
\end{abstract}

\maketitle

\section{Introduction}
\lipsum[1]

\section{Background}

\lipsum[2]

\section{Architecture}

\lipsum[3]

\section{Evaluation}

\lipsum[4]

\section{Discussion}

\lipsum[5]

\section{Conclusion}

\lipsum[6]

\nocite{*}
\clearpage

\bibliography{bib/Indaleko.bib}

\end{document}

