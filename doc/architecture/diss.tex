%%%%%%%%%%%%%%%%%%%%%%%%%%%%%%%%%%%%%%%%%%%%%%%%%%%%%%%%%%%%%%%%%%%%%%
% Template for a UBC-compliant dissertation
% At the minimum, you will need to change the information found
% after the "Document meta-data"
%
%!TEX TS-program = pdflatex
%!TEX encoding = UTF-8 Unicode

%% The ubcdiss class provides several options:
%%   gpscopy (aka fogscopy)
%%       set parameters to exactly how GPS specifies
%%         * single-sided
%%         * page-numbering starts from title page
%%         * the lists of figures and tables have each entry prefixed
%%           with 'Figure' or 'Table'
%%       This can be tested by `\ifgpscopy ... \else ... \fi'
%%   10pt, 11pt, 12pt
%%       set default font size
%%   oneside, twoside
%%       whether to format for single-sided or double-sided printing
%%   balanced
%%       when double-sided, ensure page content is centred
%%       rather than slightly offset (the default)
%%   singlespacing, onehalfspacing, doublespacing
%%       set default inter-line text spacing; the ubcdiss class
%%       provides \textspacing to revert to this configured spacing
%%   draft
%%       disable more intensive processing, such as including
%%       graphics, etc.
%%

%\RequirePackage{pdf17}
% To force PDF 1.7 version (since some of the inclusion PDFs are 1.7)
%\pdfminorversion=7
\directlua{pdf.setminorversion(7)}

% For submission to GPS
%\documentclass[gpscopy,onehalfspacing,11pt]{ubcdiss}

% For your own copies (looks nicer)
\documentclass[balanced,twoside,11pt]{ubcdiss}

%\usepackage{gitinfo2}
\usepackage{datetime2}

%%%%%%%%%%%%%%%%%%%%%%%%%%%%%%%%%%%%%%%%%%%%%%%%%%%%%%%%%%%%%%%%%%%%%%
%%%%%%%%%%%%%%%%%%%%%%%%%%%%%%%%%%%%%%%%%%%%%%%%%%%%%%%%%%%%%%%%%%%%%%
%%
%% FONTS:
%%
%% The defaults below configures Times Roman for the serif font,
%% Helvetica for the sans serif font, and Courier for the
%% typewriter-style font.  Configuring fonts can be time
%% consuming; we recommend skipping to END FONTS!
%%
%% If you're feeling brave, have lots of time, and wish to use one
%% your platform's native fonts, see the commented out bits below for
%% XeTeX/XeLaTeX.  This is not for the faint at heart.
%% (And shouldn't you be writing? :-)
%%
%\usepackage[T1,T2,T2A]{fontenc}
%\usepackage{}
%\usepackage[lutf8]{luainputenc}
\usepackage[utf8]{inputenc}
%\usepackage[OT2,T1]{fontenc}
%\usepackage[english,russian]{babel}
%\usepackage{babel}
%\newfontfamily\russianfont[Script=Cyrillic]{CMU Serif}
\usepackage{polyglossia}
\setmainlanguage{english}
\setotherlanguage{macedonian}
\newfontfamily\macedonianfont[Script=Cyrillic]{CMU Serif}

%% NFSS font specification (New Font Selection Scheme)
\usepackage{times,mathptmx,courier}
\usepackage[scaled=.92]{helvet}

\usepackage{xspace}

%% Math or theory people may want to include the handy AMS macros
\usepackage{amssymb}
\usepackage{amsmath}
\usepackage{amsfonts}
\usepackage{lmodern} % this font family has italics!

%% epigraphs - already loaded by... something
%\usepackage{epigraph}
%\usepackage{ebgaramond}
%\renewcommand\textflush{flushright}

%\usepackage{etoolbox}
%\makeatletter
%\newlength\epitextskip
%\pretocmd{\@epitext}{\em}{}{}
%\apptocmd{\@epitext}{\em}{}{}
%\patchcmd{\epigraph}{\@epitext{#1}\\}{\@epitext{#1}\\[\epitextskip]}{}{}
%\makeatother

%\setlength\epigraphrule{0pt}
%\setlength\epitextskip{2ex}
%\setlength\epigraphwidth{.8\textwidth}

%% The pifont package provides access to the elements in the dingbat font.
%% Use \ding{##} for a particular dingbat (see p7 of psnfss2e.pdf)
%%   Useful:
%%     51,52 different forms of a checkmark
%%     54,55,56 different forms of a cross (saltyre)
%%     172-181 are 1-10 in open circle (serif)
%%     182-191 are 1-10 black circle (serif)
%%     192-201 are 1-10 in open circle (sans serif)
%%     202-211 are 1-10 in black circle (sans serif)
%% \begin{dinglist}{##}\item... or dingautolist (which auto-increments)
%% to create a bullet list with the provided character.
\usepackage{pifont}

%%%%%%%%%%%%%%%%%%%%%%%%%%%%%%%%%%%%%%%%%%%%%%%%%%%%%%%%%%%%%%%%%%%%%%
%% Configure fonts for XeTeX / XeLaTeX using the fontspec package.
%% Be sure to check out the fontspec documentation.
%\usepackage{fontspec,xltxtra,xunicode}	% required
%\defaultfontfeatures{Mapping=tex-text}	% recommended
%% Minion Pro and Myriad Pro are shipped with some versions of
%% Adobe Reader.  Adobe representatives have commented that these
%% fonts can be used outside of Adobe Reader.
%\setromanfont[Numbers=OldStyle]{Minion Pro}
%\setsansfont[Numbers=OldStyle,Scale=MatchLowercase]{Myriad Pro}
%\setmonofont[Scale=MatchLowercase]{Andale Mono}

%% Other alternatives:
%\setromanfont[Mapping=tex-text]{Adobe Caslon}
%\setsansfont[Scale=MatchLowercase]{Gill Sans}
%\setsansfont[Scale=MatchLowercase,Mapping=tex-text]{Futura}
%\setmonofont[Scale=MatchLowercase]{Andale Mono}
%\newfontfamily{\SYM}[Scale=0.9]{Zapf Dingbats}
%% END FONTS
%%%%%%%%%%%%%%%%%%%%%%%%%%%%%%%%%%%%%%%%%%%%%%%%%%%%%%%%%%%%%%%%%%%%%%
%%%%%%%%%%%%%%%%%%%%%%%%%%%%%%%%%%%%%%%%%%%%%%%%%%%%%%%%%%%%%%%%%%%%%%



%%%%%%%%%%%%%%%%%%%%%%%%%%%%%%%%%%%%%%%%%%%%%%%%%%%%%%%%%%%%%%%%%%%%%%
%%%%%%%%%%%%%%%%%%%%%%%%%%%%%%%%%%%%%%%%%%%%%%%%%%%%%%%%%%%%%%%%%%%%%%
%%
%% Recommended packages
%%
\usepackage{checkend}	% better error messages on left-open environments
\usepackage{graphicx}	% for incorporating external images

%% booktabs: provides some special commands for typesetting tables as used
%% in excellent journals.  Ignore the examples in the Lamport book!
\usepackage{booktabs}
\usepackage{multirow}
\usepackage[table,xcdraw]{xcolor}
\usepackage{adjustbox}

%% listings: useful support for including source code listings, with
%% optional special keyword formatting.  The \lstset{} causes
%% the text to be typeset in a smaller sans serif font, with
%% proportional spacing.
\usepackage{listings}
\lstset{basicstyle=\sffamily\scriptsize,showstringspaces=false,fontadjust}

%% The acronym package provides support for defining acronyms, providing
%% their expansion when first used, and building glossaries.  See the
%% example in glossary.tex and the example usage throughout the example
%% document.
%% NOTE: to use \MakeTextLowercase in the \acsfont command below,
%%   we *must* use the `nohyperlinks' option -- it causes errors with
%%   hyperref otherwise.  See Section 5.2 in the ``LaTeX 2e for Class
%%   and Package Writers Guide'' (clsguide.pdf) for details.
\usepackage[printonlyused,nohyperlinks]{acronym}
%% The ubcdiss.cls loads the `textcase' package which provides commands
%% for upper-casing and lower-casing text.  The following causes
%% the acronym package to typeset acronyms in small-caps
%% as recommended by Bringhurst.
%%
%% Note: this is making it _lower case_ not small caps.  Figure it out at some
%% point
%%\renewcommand{\acsfont}[1]{{\scshape \MakeTextLowercase{#1}}}

%% color: add support for expressing colour models.  Grey can be used
%% to great effect to emphasize other parts of a graphic or text.
%% For an excellent set of examples, see Tufte's "Visual Display of
%% Quantitative Information" or "Envisioning Information".
\usepackage{color}
\definecolor{greytext}{gray}{0.5}

%% comment: provides a new {comment} environment: all text inside the
%% environment is ignored.
%%   \begin{comment} ignored text ... \end{comment}
\usepackage{comment}

%% The natbib package provides more sophisticated citing commands
%% such as \citeauthor{} to provide the author names of a work,
%% \citet{} to produce an author-and-reference citation,
%% \citep{} to produce a parenthetical citation.
%% We use \citeeg{} to provide examples
%\usepackage[numbers,sort&compress]{natbib}
%newcommand{\citeeg}[1]{\citep[e.g.,][]{#1}}

%\usepackage[backend=biber,style=ieee,sortlocale=en_EN,natbib=true]{biblatex}
\usepackage[style=ieee,citetracker=true,natbib=true,hyperref=true,backend=biber,maxbibnames=99,maxcitenames=2,mincitenames=1,firstinits=true,uniquename=init,parentracker=true,backref=true,backrefstyle=three]{biblatex}
\DeclareLanguageMapping{macedonian}{russian}
\addbibresource{bib/indaleko.bib}


%% The titlesec package provides commands to vary how chapter and
%% section titles are typeset.  The following uses more compact
%% spacings above and below the title.  The titleformat that follow
%% ensure chapter/section titles are set in singlespace.
\usepackage[compact]{titlesec}
\titleformat*{\section}{\singlespacing\raggedright\bfseries\Large}
\titleformat*{\subsection}{\singlespacing\raggedright\bfseries\large}
\titleformat*{\subsubsection}{\singlespacing\raggedright\bfseries}
\titleformat*{\paragraph}{\singlespacing\raggedright\itshape}

%% The caption package provides support for varying how table and
%% figure captions are typeset.
\usepackage[format=hang,indention=-1cm,labelfont={bf},margin=1em]{caption}

%% url: for typesetting URLs and smart(er) hyphenation.
%% \url{http://...}
\usepackage{url}
\urlstyle{sf}	% typeset urls in sans-serif


%%%%%%%%%%%%%%%%%%%%%%%%%%%%%%%%%%%%%%%%%%%%%%%%%%%%%%%%%%%%%%%%%%%%%%
%%%%%%%%%%%%%%%%%%%%%%%%%%%%%%%%%%%%%%%%%%%%%%%%%%%%%%%%%%%%%%%%%%%%%%
%%
%% Possibly useful packages: you may need to explicitly install
%% these from CTAN if they aren't part of your distribution;
%% teTeX seems to ship with a smaller base than MikTeX and MacTeX.
%%
%\usepackage{pdfpages}	% insert pages from other PDF files
\usepackage{longtable}	% provide tables spanning multiple pages
%\usepackage{chngpage}	% support changing the page widths on demand
\usepackage{tabularx}	% an enhanced tabular environment

%% enumitem: support pausing and resuming enumerate environments.
%\usepackage{enumitem}
\usepackage{enumerate}

%% rotating: provides two environments, sidewaystable and sidewaysfigure,
%% for typesetting tables and figures in landscape mode.
\usepackage{rotating}
%\usepackage{tablefootnote} % must follow rotating for some reason.

%% subfig: provides for including subfigures within a figure,
%% and includes being able to separately reference the subfigures.
\usepackage{subfig}

%% Better footnote options
\usepackage[multiple]{footmisc}

%% ragged2e: provides several new new commands \Centering, \RaggedLeft,
%% \RaggedRight and \justifying and new environments Center, FlushLeft,
%% FlushRight and justify, which set ragged text and are easily
%% configurable to allow hyphenation.
%\usepackage{ragged2e}

%% The ulem package provides a \sout{} for striking out text and
%% \xout for crossing out text.  The normalem and normalbf are
%% necessary as the package messes with the emphasis and bold fonts
%% otherwise.
%\usepackage[normalem,normalbf]{ulem}    % for \sout

%%%%%%%%%%%%%%%%%%%%%%%%%%%%%%%%%%%%%%%%%%%%%%%%%%%%%%%%%%%%%%%%%%%%%%
%% HYPERREF:
%% The hyperref package provides for embedding hyperlinks into your
%% document.  By default the table of contents, references, citations,
%% and footnotes are hyperlinked.
%%
%% Hyperref provides a very handy command for doing cross-references:
%% \autoref{}.  This is similar to \ref{} and \pageref{} except that
%% it automagically puts in the *type* of reference.  For example,
%% referencing a figure's label will put the text `Figure 3.4'.
%% And the text will be hyperlinked to the appropriate place in the
%% document.
%%
%% Generally hyperref should appear after most other packages

%% The following puts hyperlinks in very faint grey boxes.
%% The `pagebackref' causes the references in the bibliography to have
%% back-references to the citing page; `backref' puts the citing section
%% number.  See further below for other examples of using hyperref.
%% 2009/12/09: now use `linktocpage' (Jacek Kisynski): GPS now prefers
%%   that the ToC, LoF, LoT place the hyperlink on the page number,
%%   rather than the entry text.
\usepackage[bookmarks,bookmarksnumbered,%
    allbordercolors={0.8 0.8 0.8},%
    ]{hyperref}
%% The following change how the the back-references text is typeset in a
%% bibliography when `backref' or `pagebackref' are used
%%
%% Change \nocitations if you'd like some text shown where there
%% are no citations found (e.g., pulled in with \nocite{xxx})
\newcommand{\nocitations}{\relax}
%%\newcommand{\nocitations}{No citations}
%%
%\renewcommand*{\backref}[1]{}% necessary for backref < 1.33
%\renewcommand*{\backrefsep}{,~}%
%\renewcommand*{\backreftwosep}{,~}% ', and~'
%\renewcommand*{\backreflastsep}{,~}% ' and~'
%\renewcommand*{\backrefalt}[4]{%
%\textcolor{greytext}{\ifcase #1%
%\nocitations%
%\or
%\(\rightarrow\) page #2%
%\else
%\(\rightarrow\) pages #2%
%\fi}}


%% The following uses most defaults, which causes hyperlinks to be
%% surrounded by colourful boxes; the colours are only visible in
%% PDFs and don't show up when printed:
%\usepackage[bookmarks,bookmarksnumbered]{hyperref}

%% The following disables the colourful boxes around hyperlinks.
%\usepackage[bookmarks,bookmarksnumbered,pdfborder={0 0 0}]{hyperref}

%% The following disables all hyperlinking, but still enabled use of
%% \autoref{}
%\usepackage[draft]{hyperref}

%% The following commands causes chapter and section references to
%% uppercase the part name.
\renewcommand{\chapterautorefname}{Chapter}
\renewcommand{\sectionautorefname}{Section}
\renewcommand{\subsectionautorefname}{Section}
\renewcommand{\subsubsectionautorefname}{Section}

%% If you have long page numbers (e.g., roman numbers in the
%% preliminary pages for page 28 = xxviii), you might need to
%% uncomment the following and tweak the \@pnumwidth length
%% (default: 1.55em).  See the tocloft documentation at
%% http://www.ctan.org/tex-archive/macros/latex/contrib/tocloft/
% \makeatletter
% \renewcommand{\@pnumwidth}{3em}
% \makeatother

%%%%%%%%%%%%%%%%%%%%%%%%%%%%%%%%%%%%%%%%%%%%%%%%%%%%%%%%%%%%%%%%%%%%%%
%%%%%%%%%%%%%%%%%%%%%%%%%%%%%%%%%%%%%%%%%%%%%%%%%%%%%%%%%%%%%%%%%%%%%%
%%
%% Some special settings that controls how text is typeset
%%
% \raggedbottom		% pages don't have to line up nicely on the last line
% \sloppy		% be a bit more relaxed in inter-word spacing
% \clubpenalty=10000	% try harder to avoid orphans
% \widowpenalty=10000	% try harder to avoid widows
% \tolerance=1000

%% And include some of our own useful macros
% This file provides examples of some useful macros for typesetting
% dissertations.  None of the macros defined here are necessary beyond
% for the template documentation, so feel free to change, remove, and add
% your own definitions.
%
% We recommend that you define macros to separate the semantics
% of the things you write from how they are presented.  For example,
% you'll see definitions below for a macro \file{}: by using
% \file{} consistently in the text, we can change how filenames
% are typeset simply by changing the definition of \file{} in
% this file.
%
%% The following is a directive for TeXShop to indicate the main file
%%!TEX root = ../diss.tex

\newcommand{\NA}{\textsc{n/a}}	% for "not applicable"
\newcommand{\eg}{e.g.,\ }	% proper form of examples (\eg a, b, c)
\newcommand{\ie}{i.e.,\ }	% proper form for that is (\ie a, b, c)
\newcommand{\etal}{\emph{et al}}

% Some useful macros for typesetting terms.
\newcommand{\file}[1]{\texttt{#1}}
\newcommand{\class}[1]{\texttt{#1}}
\newcommand{\latexpackage}[1]{\href{http://www.ctan.org/macros/latex/contrib/#1}{\texttt{#1}}}
\newcommand{\latexmiscpackage}[1]{\href{http://www.ctan.org/macros/latex/contrib/misc/#1.sty}{\texttt{#1}}}
\newcommand{\env}[1]{\texttt{#1}}
\newcommand{\BibTeX}{Bib\TeX}

% Define a command \doi{} to typeset a digital object identifier (DOI).
% Note: if the following definition raise an error, then you likely
% have an ancient version of url.sty.  Either find a more recent version
% (3.1 or later work fine) and simply copy it into this directory,  or
% comment out the following two lines and uncomment the third.
\DeclareUrlCommand\DOI{}
\newcommand{\doi}[1]{\href{http://dx.doi.org/#1}{\DOI{doi:#1}}}
%\newcommand{\doi}[1]{\href{http://dx.doi.org/#1}{doi:#1}}

% Useful macro to reference an online document with a hyperlink
% as well with the URL explicitly listed in a footnote
% #1: the URL
% #2: the anchoring text
\newcommand{\webref}[2]{\href{#1}{#2}\footnote{\url{#1}}}

% epigraph is a nice environment for typesetting quotations
\makeatletter
\newenvironment{epigraph}{%%
	\begin{flushright}
		\begin{minipage}{\columnwidth-0.75in}
			\begin{flushright}
				\@ifundefined{singlespacing}{}{\singlespacing}%
				}{
			\end{flushright}
		\end{minipage}
	\end{flushright}}
\makeatother

% \FIXME{} is a useful macro for noting things needing to be changed.
% The following definition will also output a warning to the console
\newcommand{\FIXME}[1]{\typeout{**FIXME** #1}\textbf{[FIXME: #1]}}

% Replaceable names
%\newcommand{\system}[0]{\emph{Kwishut}\xspace}
\newcommand{\system}[0]{\emph{Indaleko}\xspace}
% QI'tu'naS - Klingon for "pragmatics" - I'm saving that for the final version.
% qaywI' - Klingon for "finding" - another option for final version.
% Talal - Vulcan for "finding" - yet another option.

\newcommand{\systemone}[0]{\emph{Topish}\xspace}
% Topish is Uzbek for "Finding"
%\newcommand{\systemtwo}[0]{\emph{\textmacedonian{Наоѓање}}\xspace}
% Pronunciation: Naoǵanje.
%\newcommand{\systemtwo}[0]{\emph{\textmacedonian{Naīdi}}\xspace}
% Naīdi ("Find") in Macedonian.
\newcommand{\systemtwo}[0]{\emph{\textmacedonian{Находка}}\xspace}
% Pronunciation: Nakhodka

% Gender neutral names (though very European)
%\newcommand{\persa}[0]{Addison\xspace}
\newcommand{\persa}[0]{Aki\xspace} % Japanese, gender neutral, means "autumn"

%\newcommand{\persb}[0]{Bailey\xspace}
\newcommand{\persb}[0]{Dagon\xspace} % Biblical name meaning "fish"

%\newcommand{\persc}[0]{Cameron\xspace}
\newcommand{\persc}[0]{Fenix\xspace} % Greek Isles, means "dark red"

%\newcommand{\persd}[0]{Dana\xspace}
\newcommand{\persd}[0]{Hao\xspace} % Vietnamese, means "good/perfect"

%\newcommand{\perse}[0]{Evan\xspace}
\newcommand{\perse}[0]{Waneta\xspace} % Native American roots, "shape shifter, charger"

%\newcommand{\persf}[0]{Quinn\xspace}
\newcommand{\persf}[0]{Skylar\xspace} % allegedly American roots, meaning "scholar"

%\newcommand{\persg}[0]{Reese\xspace}
\newcommand{\persg}[0]{Zene\xspace} % "beautiful" based on African culture supposedly.

% use cases
\newcommand{\usecaseactivitycontext}[0]{\textsc{ac\-tiv\-ity con\-text}\xspace}
\newcommand{\usecasedatarelationship}[0]{\textsc{data re\-la\-tion\-ships}\xspace}
\newcommand{\usecasecrosssilosearch}[0]{\textsc{cross-silo search}\xspace}
\newcommand{\usecasenotifications}[0]{\textsc{no\-ti\-fi\-ca\-tions}\xspace}
\newcommand{\usecasepersnamespace}[0]{\textsc{per\-son\-al\-ized name\-space}\xspace}

% terminology
%Copy: bit-for-bit identical
\newcommand{\doccopy}[0]{copy\xspace}
%  derivation: the semantics change
\newcommand{\docderivation}[0]{derivation\xspace}
% conversion: semantically identical but not Bit-for-Bit
\newcommand{\docconversion}[0]{conversion\xspace}

% END


%%%%%%%%%%%%%%%%%%%%%%%%%%%%%%%%%%%%%%%%%%%%%%%%%%%%%%%%%%%%%%%%%%%%%%
%%%%%%%%%%%%%%%%%%%%%%%%%%%%%%%%%%%%%%%%%%%%%%%%%%%%%%%%%%%%%%%%%%%%%%
%%
%% Document meta-data: be sure to also change the \hypersetup information
%%

\title{\system}
\subtitle{Using System Activity Context to Improve Finding}

\author{William Anthony Mason}
\previousdegree{S.B. Mathematics, University of Chicago, 1987}
\previousdegree{MSc. Computer Science, Georgia Institute of Technology, 2017}

% What is this dissertation for?
\degreetitle{Doctor of Philosophy}

\institution{The University of British Columbia}
\campus{Vancouver}

\faculty{The Faculty of Science}
\department{Computer Science}
\submissionmonth{November}
\submissionyear{2021}

% details of your examining committee
\examiningcommittee{Joana McGrenere, Computer Science}{Examination Chair}
\examiningcommittee{Margo I. Seltzer, Computer Science}{Co-Supervisor}
\examiningcommittee{Ada Gavrilovska, Georgia Institute of Technology, College of
Computing}%
    {Co-Supervisor}
\examiningcommittee{Sasha Fedorova, Electrical and Computer Engineering}{Supervisory Committee Member}
\examiningcommittee{Norman Hutchinson, Computer Science}{Supervisory Committee Member}
\examiningcommittee{Andrew Warfield, Computer Science}{Supervisory Committee Member}

% details of your supervisory committee
%\supervisorycommittee{Margo I. Seltzer, Computer Science}{Co-Supervisor}
%\supervisorycommittee{Ada Gavrilovska, Georgia Institute of Technology, College of
%Computing}%
%    {Co-Supervisor}
%\supervisorycommittee{Sasha Fedorova, Electrical and Computer Engineering}{Supervisory Committee Member}
%\supervisorycommittee{Norman Hutchinson, Computer Science}{Supervisory Committee Member}
%\supervisorycommittee{Andrew Warfield, Computer Science}{Supervisory Committee Member}

%% hyperref package provides support for embedding meta-data in .PDF
%% files
\hypersetup{
  pdftitle={\system  (DRAFT: \today)},
  pdfauthor={Tony Mason},
  pdfkeywords={Naming}
}


% These are marks for inserting comments. Feel free to edit as needed!
\newcommand{\nb}[2]{{\yellowbox{#1}\triangles{#2}}}
\newcommand{\nbc}[3]{
 {\colorbox{#3}{\bfseries\sffamily\scriptsize\textcolor{white}{#1}}}
 {\textcolor{#3}{\sf\small$\blacktriangleright$\textit{#2}$\blacktriangleleft$}}}
\newcommand{\version}{\emph{\scriptsize\id}}
\newcommand{\ugh}[1]{#1} % please rephrase
\newcommand{\ins}[1]{#1} % please insert
\newcommand{\del}[1]{} % please delete
\newcommand{\chg}[2]{#2} % please change
\renewcommand{\nb}[2]{\nbc{#1}{#2}{orange}}

% Tony
\definecolor{tmcolor}{rgb}{0.5,0,0.5}
\newcommand\tm[1]{\nbc{TM}{#1}{tmcolor}}

% Margo
\definecolor{miscolor}{rgb}{0.4,0.6,0.2}
\newcommand\MIS[1]{\nbc{MIS}{#1}{miscolor}}

% Ada
\definecolor{adacolor}{rgb}{1.0, 0.5, 0.5}
\newcommand\ada[1]{\nbc{AG}{#1}{adacolor}}

% Sasha
\definecolor{sfcolor}{rgb}{0.2,0.0,0.5}
\newcommand\sasha[1]{\nbc{SF}{#1}{sfcolor}}

% Reto
\definecolor{retocolor}{rgb}{1.0,0.49,0.0}
\newcommand\reto[1]{\nbc{RA}{#1}{retocolor}}

% Surbhi
\definecolor{spcolor}{rgb}{0.0,0.4,1.0}
\renewcommand\sp[1]{\nbc{SP}{#1}{spcolor}}

% Swati
\definecolor{jncolor}{rgb}{0.5,0.4,1.0}
\newcommand\jkn[1]{\nbc{JN}{#1}{jncolor}}


%%%%%%%%%%%%%%%%%%%%%%%%%%%%%%%%%%%%%%%%%%%%%%%%%%%%%%%%%%%%%%%%%%%%%%
%%%%%%%%%%%%%%%%%%%%%%%%%%%%%%%%%%%%%%%%%%%%%%%%%%%%%%%%%%%%%%%%%%%%%%
%%
%% The document content
%%

%% LaTeX's \includeonly commands causes any uses of \include{} to only
%% include files that are in the list.  This is helpful to produce
%% subsets of your thesis (e.g., for committee members who want to see
%% the dissertation chapter by chapter).  It also saves time by
%% avoiding reprocessing the entire file.
%\includeonly{intro,conclusions}
%\includeonly{discussion}

\begin{document}

%%%%%%%%%%%%%%%%%%%%%%%%%%%%%%%%%%%%%%%%%%%%%%%%%%
%% From Thesis Components: Tradtional Thesis
%% <http://www.grad.ubc.ca/current-students/dissertation-thesis-preparation/order-components>

% Preliminary Pages (numbered in lower case Roman numerals)
%    1. Title page (mandatory)
\maketitle

%    2. Committee page (mandatory): lists supervisory committee and,
%    if applicable, the examining committee
\makecommitteepage

%    3. Abstract (mandatory - maximum 350 words)
\chapter{Abstract}

Human society is collecting data at an alarming rate: per-capita data generation
is now over 1.7MB \emph{per second}. We expect to send 361 billion e-mails
\emph{per day} by 2024. Rapid data growth, combined with increasing ways to
store and present data to users creates a frustrating challenge finding specific
documents a few days old, let alone those created months or years earlier.

Our data is scattered across physical locations.  Existing storage is presented
to us with old and new interfaces that blur the lines between file system and
application. For example, my Outlook mailbox resides on my local disk
drive in both databases and discrete files.  Yet I use Outlook, not my local
file system to find documents that were attached to e-mails.

On-demand cloud storage systems provide strong benefits yet also make it
impractical to search locally because not all the content is resident to be
indexed. Currently none of the cloud storage systems offer the rich extensible
search tools found on modern desktop operating systems.  Even if they were to
provide such search, it would require querying each one in turn to find relevant
files.

To address these challenges I propose \emph{Finding as a Service}, which provides two
important capabilities.  First, it explicitly decouples \emph{finding} objects from
\emph{storing} and \emph{presenting} objects.  Second, it exploits the
observation that users' mental associations with objects are more complex than
the arbitrary name, type, dates, or attributes on which users search today.

\emph{Finding as a Service} requires two things: (1) a mechanism for exploiting
the information that modern devices already capture and for capturing
additional useful information that relates to interactions with digital data and
the environment in which that data is used; and (2) the ability to collect,
store, and query both existing and new usage context pertinent to digital
information. Both of these requirements enable building powerful tools that
helps users in \emph{finding} digital objects efficiently.


%\vfill
%\begin{center}
%    \begin{sf}
%        \fbox{Revision: \DTMnow}
%    \end{sf}
%\end{center}


\cleardoublepage

%    4. Lay Summary (Effective May 2017, mandatory - maximum 150 words)
%\include{laysummary}
%\cleardoublepage

%    5. Preface
%\include{preface}
%\cleardoublepage

% Margo asked me to omit the extra "stuff" so I have.

%    6. Table of contents (mandatory - list all items in the preliminary pages
%    starting with the abstract, followed by chapter headings and
%    subheadings, bibliographies and appendices)
%\tableofcontents
%\cleardoublepage	% required by tocloft package

%    7. List of tables (mandatory if thesis has tables)
%\listoftables
%\cleardoublepage	% required by tocloft package

%    8. List of figures (mandatory if thesis has figures)
%\listoffigures
%\cleardoublepage	% required by tocloft package

%    9. List of illustrations (mandatory if thesis has illustrations)
%   10. Lists of symbols, abbreviations or other (optional)

%   11. Glossary (optional)
%\input{chapters/glossary}	% always input, since other macros may rely on it

\textspacing		% begin one-half or double spacing

%   12. Acknowledgements (optional)
% \include{chapters/ack}

%   13. Dedication (optional)

% Body of Thesis (not all sections may apply)
\mainmatter

\acresetall	% reset all acronyms used so far

%% https://blogs.ubc.ca/educ500/files/2012/07/researchproposal.pdf

%    1. Introduction
%%% The following is a directive for TeXShop to indicate the main file
%%!TEX root = ../diss.tex

\chapter{Introduction}
\label{ch:introduction}

\begin{epigraph}
    \emph{
        What should I possibly have to tell you, oh venerable one? Perhaps that you're searching far too much? That in all that searching, you don't find the time for finding?
    } --- Siddhartha (1922), Hermann Hesse.
\end{epigraph}

\section{Activity Context}
\label{ch:introduction:sec:activitycontext}

In 1945 Vannevar Bush described the challenges to humans of
finding things in a codified system of records~\cite{bush1945we}:

\begin{quotation}
    \emph{Our ineptitude in getting at the record is largely
        caused by the artificiality of systems of indexing. When data of any
        sort are placed in storage, they are filed alphabetically or numerically, and
        information is found (when it is) by tracing it down from subclass to
        subclass. It can be in only one place, unless duplicates are used; one
        has to have rules as to which path will locate it, and the rules are
        cumbersome. Having found one item, moreover, one has to emerge from the
        system and re-enter on a new path.}

    \emph{
        The human mind does not work that way. It operates by association. With one
        item in its grasp, it snaps instantly to the next that is suggested by the
        association of thoughts, in accordance with some intricate web of trails
        carried by the cells of the brain. It has other characteristics, of course;
        trails that are not frequently followed are prone to fade, items are not
        fully permanent, memory is transitory. Yet the speed of action, the
        intricacy of trails, the detail of mental pictures, is awe-inspiring beyond
        all else in nature.}
\end{quotation}

This is as true in 2021 as it was in 1945.  Thus, the question that motivates my
research is: ``Can we build systems that get us closer to that ideal?''  My
argument that we can by capturing additional information that is not necessarily
useful to computers, but is useful to humans.

I call this additional captured information \emph{activity context}. While similar
to the ideas proposed in \emph{Burrito}~\cite{guo2012burrito}, I have broadened
that idea beyond just ``what a user is doing'' to incorporate information about
what the user is \emph{experiencing} that corresponds to more human-like context
information because it is useful for constructing \emph{association}.

Thus an ``activity context'' is an answer to the question:
\emph{what is going on in relation to the current event on a digital object?}
The job of the system becomes answering this question in a way that is useful.

More concretely, activity context is concerned with the environment in which a
given digital object is accessed.  Without restricting the abstract concept,
concrete examples of what I consider to be elements of an ``activity context''
might include:

\begin{itemize}
    \item Current weather.
    \item Notable news events.
    \item Focus website opened in a visible browser tab.
    \item User's mood.
    \item User's heart rate.
\end{itemize}

This is distinguished from \emph{what the digital object is}. While
understanding what something \emph{is} has merit, the context in which a digital
object is \emph{used} yields additional understanding about that digital object.

Further examples, in the form of ``use cases,'' can be found in
\autoref{ch:intro:sec:use-cases}.

\section{Thesis}
\label{ch:introduction:sec:thesis-statement}

\textbf{
    \input{thesis.tex}
}

\vspace{0.25cm}

``Intelligent use of files depends on having sufficient knowledge about them: their purposes, structures, and
contexts. Humans have traditionally made do by using their own methods for capturing and manipulating
such knowledge, but this is not available to programs, nor is it necessarily
convenient for humans~\cite{mogul1986representing}.''

Determining \emph{context} is a challenge with modern computer storage systems.
It is unrealistic to expect human users to provide that context.
Context is dynamic, imprecise, and not necessarily obvious, yet humans rely
upon context to create associations. By making environmental context information
available to programs, those same programs are able to present better options,
which \emph{is} convenient for humans.

%To improve the usefulness of naming systems within computer storage systems
%software must evolve to provide flexible, scalable, and multi-silo management
%of data object naming.


\section{Finding}
\label{ch:intro:sec:finding}

The volume of digital data is growing exponentially. Thus, it is not surprising
that solutions that worked when users were grappling with kilobytes (KB) or megabytes
(MB) of data do not work in the face of this growing deluge.

IBM's first magnetic disk drive could store up to 2.5
megabytes(MB)~\footnote{\url{https://www.7dayshop.com/blog/terabyte-evolution/}}
of digital data.  In 2020, we add 1.7MB of data \emph{per second per
    person}~\footnote{\url{https://techjury.net/blog/big-data-statistics/}}.  Today
we have become digital hoarders, collecting and keeping so much data that we often
cannot find specific objects when we need them.  How did we reach this point?

Early persistent storage systems used a simple flat directory structure
that gave a unique name to each object (``file'').  Such a simple structure was sufficient
to  name and identify distinct data objects (``files'').
\emph{Finding} the correct file was just a matter of scanning and picking from
the list.  This simple
structure did not scale well and was replaced by
a model based upon how paper documents were organized.
The hierarchical name
space~\cite{barnard1958,daley1965general,ritchie1973unix,Saltzer1978} is one in
which files (``digital objects'') are grouped into directories (sometimes called
folders).  Directories can also be grouped into other directories.  This model
mirrors how a filing cabinet works: multiple sheets of paper are gathered into a
folder, folders are organized into drawers, drawers into filing cabinets, filing
cabinets into rooms, etc. The directory and file metaphor was in use by
1958~\cite{barnard1958} and persists today as the common model despite the
volume of data being stored by a single computer storage device (``disk drive'') increasing by at least $10^6$~\footnote{Disk
    drives were measured in MB in 1965 and are measured in TB today.}.

In addition to the challenges of scaling, the file cabinet metaphor imposes
physical file limitations that are not valid for digital data.
Physical file cabinets do not allow a document to be in two folders at the same
time but there is no such restriction on electronic documents. The Multics
researchers addressed this by creating the \emph{link}, an idea that is
still used in many modern file systems~\cite{daley1965general}.

Computer networking enabled data sharing between users and computers but
complicated naming. Remote data access was typically represented to users either
as a hierarchical file system~\cite{nfs,howard1988scale} or an application program that programmatically
connected users to remote data~\cite{levin1979transport,10.1145/800216.806594,birrell1982grapevine}.

By 1990 the volume of data with which users interacted was so
large that researchers questioned the utility of the hierarchical name space~\cite{vicente1987assaying}.
The Semantic File System (SFS)~\cite{gifford1991semantic} suggested that
organization of digital objects be more fluid so that
users could group items together in ways that were semantically meaningful. The
meta-data generated from semantic information generated from file contents
permitted powerful query-based dynamic file organization.

While semantic file systems have not been widely adopted, the concept of
extracting semantic information from file content is present in modern file
indexing systems.  These indexing services employ ``transducers'' to extract
semantic information from files. Desktop search utilities (Windows
Search, Apple Spotlight, Station for Linux) rely upon indexing services to
provide their functionality.

\begin{figure}
    \centering
    \caption{First Wayback capture of google.com (google.stanford.edu) in 1998}
    \label{fig:google}
    \includegraphics[width=0.95\textwidth]{figures/1998-11-11-Google-Screenshot-showing-early-index-size.png}
\end{figure}

There are parallels between the challenges of indexing files and the challenges
of indexing Internet web pages. Early search engines used a curation model in which
humans decided what websites were of interest based upon the information within
the web page itself --- similar to the way that semantic information was
extracted from files in the Semantic File
System~\footnote{\url{https://www.hpe.com/us/en/insights/articles/how-search-worked-before-google-1703.html}}.

Internet web page indexing changed profoundly when two Stanford graduate
students proposed a novel way to exploit the structure of Internet web pages to
extract \emph{usage} information from web pages that did not depend upon
semantic content~\cite{page1999pagerank}.

Google's website used to state the number of pages that they indexed.  In 1998
the first capture of Google's website by archive.org shows they claimed to index
more than 25 million pages (\autoref{fig:google}). Google no longer publishes
that number but industry estimates indicate the number is at least $10^3$ more
now than it was then, and this only covers a few percent of the entire content
stored on the
Internet~\cite{bosch2016estimating}~\footnote{\url{https://www.worldwidewebsize.com/}}.

Could we utilize a similar technique for finding information within our own
trove of files? While there are similarities between the Internet and our file
collections, there are also significant differences. Files lack the level of common
structure present in web pages, preventing simple extraction of references
between files. Files (or digital objects) are stored in myriad
locations with different access mechanisms: local storage, cloud storage,
database, collaboration applications, e-mail programs, etc.  Sometimes these
overlap: your e-mail program stores some or all of your data on your local
computer, within your local storage. However, you do not expect to use the tools
for searching your local storage to find things within your e-mail software.
Thus, we should consider them to be distinct storage locations. I refer to these
distinct storage locations as \emph{storage silos} (or just \emph{silos}) to
emphasize their inherently separated nature.

Network storage is presented in many different formats: an inexpensive disk
drive attached to the local network represents ``Network Attached Storage''
(NAS) or a specialized parallel data cluster such as HDFS~\footnote{\url{https://hadoop.apache.org/docs/r1.2.1/hdfs_design.html}},
DAOS~\footnote{\url{https://www.intel.ca/content/www/ca/en/high-performance-computing/daos-high-performance-storage-brief.html}},
Lustre~\footnote{\url{https://www.lustre.org/}}, or
Ceph~\footnote{\url{https://docs.ceph.com/en/pacific/cephfs/index.html}}.
They typically support one or
more common data sharing protocols, such as NFS~\cite{sandberg1986sun} or
CIFS~\footnote{\url{https://docs.microsoft.com/en-us/openspecs/windows_protocols/ms-cifs/d416ff7c-c536-406e-a951-4f04b2fd1d2b}}.
They vary dramatically in how they are managed, accessed, and searched.  In most
cases there is no common interface --- each represents a unique ``storage
silo.''

Cloud storage is one specific type of network storage that is popular because it
allows you to access your data from any of your devices, provides a reliable
backup mechanism, and permits selective download to any given device. However,
these benefits are paired with challenges when it comes to finding specific
digital objects.  If files are not present on your local device, the indexing
services on those devices cannot assist you.  You could download all of the
content from the cloud storage providers to enable indexing, but that consumes
considerably more bandwidth and storage and is impractical for devices that have
resource constraints.  While we can use the cloud providers' search services, that
requires iteration over each of those services, using different interfaces with
variable results.

Our files come from multiple sources including websites, e-mails, databases,
and collaboration tools.  Those documents are stored both locally and remotely.
We create, access, and modify documents and then send them onwards using any of
the variety of silos and collaboration tools at our disposal.  Just a few days
after we last accessed them we struggle to find those
documents.

Given the diffusion of files across storage silos that occurs because of our
sharing and use, we often find versions and related files scattered across
multiple silos. We struggle to find these versions and determine when we have
found the ``right'' one.

Returning to the question of using contextual information for improving
\emph{finding}, the research community has observed that adding
contextual information, such as the current weather, to existing file
collections materially improves human ability to find the relevant digital
object~\cite{vianna2019thesis,10.1145/1559845.1559992,dumais2016stuff}. Thus, it
seems the answer is ``Yes, using contextual information improves
\emph{finding}.''

Google solves a simpler problem: an Internet web search need merely find
\emph{an} answer to the search query.  A personal file search needs provide
\emph{the} answer. Thus, it may not be possible to provide a definitive answer,
but narrowing the potential list of plausible answers leads users to the
relevant file, which is the goal of \emph{finding}.

The data that we need for creating activity context is already being collected
by our computers. Modern computers collect vast amounts of information about us:
what we do, where we are, with whom we communicate, the applications we use, the
files we access, the music we play, the web pages we visit, even how we
feel~\cite{chakriswaran2019emotion,8933554}.  We \emph{know} this
data exists because our own devices provide this information to third parties.
Given this data is already being collected, we know the additional cost will be
to store and make it accessible to applications.

\emph{Finding as a Service} (FaaS) will use this existing information to solve our data
finding problem. FaaS decouples \emph{finding} objects from \emph{storing}
objects.  FaaS facilitates \emph{finding} by exploiting contextual information
beyond the basic object characteristics widely available for searching today:
names, types, and dates. By relating information we already have with how our
digital objects are used, we provide the \emph{activity context} to enable
\emph{Finding as a Service} (FaaS).

\emph{Activity context} is important because it captures useful
information about the environment in which files are created,
consumed, and updated.  \emph{Activity context} need not be something the system
ordinarily relates to the digital object.  \emph{Activity context} captures key
information about the the user's wholistic environment.

\section{Use Cases}
\label{ch:intro:sec:use-cases}

The following use cases provide specific scenarios that cannot be
achieved using current systems.  I maintain that \emph{Finding as a Service}
addresses these use cases, which supports my thesis.

\begin{itemize}
    \item \label{use-case:e-mail}\textbf{The lost original.} Imagine that you received a spreadsheet
          from someone.  You begin to edit it, add information, sift through it.  At
          some point you realize that you sorted a subset of the columns, hopelessly
          scrambling the original information and your edits. You try to find the
          original source of the spreadsheet, only to realize that you cannot do so,
          even when you search in your e-mail program using the \emph{name} of the
          file that was saved on your local drive as part of your editing process. The
          system should permit you to find the original source of the information,
          even though the related digital objects are in different \emph{silos}.

    \item \label{use-case:misplaced-presentation}\textbf{The misplaced presentation.} You arrive at a meeting with your
          client after a long trip, only to realize the laptop computer you were
          using will not boot.  You have your smart phone and you \emph{think}
          you saved it to a cloud service.  How do you find it so you can share
          it with a colleague at the meeting? The system should permit you to
          find your own digital data in your cloud storage regardless of which
          device you used to create it.

    \item \label{use-case:multi-silo-problem}\textbf{The multi-silo relationship problem.} A colleague shares their
          experimental data with you, which was stored in NREL's
          High-Performance Computing Data
          Center~\footnote{\url{https://www.nrel.gov/computational-science/hpc-data-center.html}}.
          You then use that data as part of your own work, which you wish to
          share with a broader audience using Compute
          Canada~\footnote{\url{https://www.computecanada.ca/techrenewal/rdm/}}.
          You also shared your computational notebooks using your organization's
          account with Microsoft~\footnote{\url{https://visualstudio.microsoft.com/vs/features/notebooks-at-microsoft/}}.
          You created your slides in
          Prezi~\footnote{prezi.com} and presented them to a different research
          group on their Discord server~\footnote{\url{discord.com}}. One of the
          people that attended downloaded two of your notebooks and created a
          new notebook from them.  They then shared that notebook publicly via
          Google Colab~\footnote{\url{https://colab.research.google.com/}}.
          Your colleague knows that she shared her data with you and wants to be
          able to quickly find the documents that you and others have shared
          based upon that original data. Collaborative work like this is a
          modern reality and it is unlikely that all work product will be co-located.
          The system should permit your colleagues to find the work you and
          others have shared with her \emph{without} your intervention.

    \item \label{use-case:multi-silo-finding}\textbf{Multi-silo finding.} \persd is a visiting student from the
          country of Lemuria doing an internship with you in Camelot.  While arranging
          for this internship, \persd required \emph{numerous} different data objects:
          email messages with the host, offer letters, academic forms, a visa,
          boarding passes, project proposals, and more. The system should be able to
          provide you with a set of related files, regardless of their storage
          silo.

    \item \label{use-case:info-source}\textbf{The where did I get this information conundrum.}  In our modern
          world we often have one (or more) web pages open when we are authoring one
          (or more) documents.  A reasonable question to ask would then be ``what web
          pages did I look at while writing this document?'' Often it is not just that
          you looked at a given web page, but also if it was the last web page you
          looked at and how long you looked at it.  The system should be able to
          provide you with a list of that web activity.

    \item \label{use-case:privacy}\textbf{How do I share information while
              preserving privacy?} \persg, an investigative journalist who routinely receives
          sensitive information
          from third parties, is investigating the company from the prior use cases.
          \persg needs to be able store and access sensitive information, including
          information about the activity context of various e-mails, documents, pictures,
          and audio and video files. While \persg ensures that these data are
          encrypted, they need to also ensure
          that they can both find information and ensure that meta-data associated with
          those files is both usable and properly protected across silos.
          While \persg must protect their sources, they must also be able to associate
          evidence with those sources to make judgement calls about their validity.
          The system should support security and privacy policies for attributes that
          accomplish both.
\end{itemize}

This list of use cases is not exhaustive and is intended to provide cases that
resonate with readers.  They are use cases that are not addressed by existing
systems.  I review these existing systems and how they fail to address these use
cases in \autoref{ch:intro:sec:existing-solutions}.

\section{Existing Solutions Fall Short}
\label{ch:intro:sec:existing-solutions}

Prior work has addressed some of the challenges that I identified in the use
cases (\autoref{ch:intro:sec:use-cases}).  I briefly introduce key
aspects of how they fall short here and provide greater detail in
\autoref{ch:background}.

A simple solution to the multi-silo namespace challenge is to graft those
namespaces together. UNIX mount points~\cite{unix} are perhaps the first
instance of such federating namespaces. Distributed federation, as provided by
distributed file systems such as NFS~\cite{nfs} and AFS~\cite{howard1988scale}
emerged in the 1980s soon after adoption of high speed networks such as
Ethernet~\cite{digital1980ethernet}.

There is some work in cloud storage federated
namespaces~\cite{scfs,federatedMetaData}.
Nextcloud~\footnote{\url{https://nextcloud.com}} allows users to connect
multiple Nextcloud instances and integrate with FTP, CIFS, NFS and object
stores. This yields a classic hierarchical namespace structure with its known
limitations~\cite{vicente1987assaying,vicente1988accommodating}. It does nothing
to facilitate \emph{finding}. Peer-to-peer sharing networks (e.g., IPFS \cite{benet2014ipfs}) implement a
distributed file system where nodes advertise their files to users.
MetaStorage~\cite{metastorage} implements a highly available, distributed hash
table, similar to Amazon's DynamoDB~\cite{10.1145/2213836.2213945},
but with its data replicated and distributed across different cloud providers.
MetaStore offers a key-value store interface~\footnote{\url{https://cwiki.apache.org/confluence/display/hive/design}}.
Farsite~\cite{Adya:2003:Farsite} organizes multiple machines into virtual file
servers, each of which acts as the root of a distributed file system. Comet
describes a cloud oriented federated metadata service~\cite{federatedMetaData}.

None of these address the scaling problem that arises when data is analyzed and indexed
away from where it is stored.  Similarly, none of these address the integration
of sensitive locally stored information about personal usage of these objects.
Thus, one key benefit of better \emph{finding} is that
it should improve the efficiency of retrieving data stored across non-local
storage silos.

Some prior work explored using extrinsic usage information for \emph{finding}.
Placeless~\cite{placeless-tois} focused on using process level information extracted from their document processing
system to associate files together. Similarly,
Burrito~\cite{guo2012burrito} proposed \emph{activity context}, which they define as ``the
user's actions at a particular time.'' Both look at narrow instances of the
larger \emph{finding} problem.

Provenance uses observable information about construction of a file to augment file
search, which in turn improves findability~\cite{provsearch}. Provenance search
takes a narrow view of the activities of interest and are all largely
\emph{causality} focused.  However, humans tend to think
associatively~\cite{10.1145/1559845.1559992}, focusing on what else was
happening --- the cleaning crew came by their desk while they were writing that
document, the discussion at a meeting with others, their location when they
wrote a given document, or some other event that was happening around the time
they interacted with a given document.

While environmental information is not as obviously related as causal
relationships, prior work related to using statistical inference to establish
relationships within the storage domain has demonstrated such mechanisms can be
more efficient at identifying patterns that lead to higher
efficiency~\cite{10.1145/3035918.3064029}.

\section{Contributions}
\label{ch:intro:sec:contributions}

The research to support my thesis will contribute the following:

\begin{enumerate}
    \item Production of the \emph{Finding as a Service} (FaaS) dataset, a
          collection of meta-data and activity context from a local system, that
          will enable me to explore potentially useful information for informing
          activity context data collection. In addition, I will publicly
          share this data set to enable other researchers to develop new
          techniques for users to find digital items.

    \item \system~\footnote{\system is Xhosa for pragmatics.  In Linguistics,
              pragmatics is the study of meaning within a given context}, my
          architecture for a system that captures, stores, and
          disseminates \emph{activity context} without imposing excessive resource
          demand.

    \item \systemone~\footnote{\systemone is the Uzbek word for finding; the most
              recent graduate student from our research group is from Uzbekistan and
              has always been supportive of my research}, a single node
          implementation consistent with \system that provides
          \emph{FaaS} across multiple storage silos on a single
          system.

    \item \systemtwo~\footnote{\systemtwo is the Russian word for finding in
              recognition of the support for my research that I have received from both Ada Gavrilovska and
              Alexandra Fedorova.}, a distributed implementation of \system that provides
          \emph{Finding as a Service} (FaaS) across multiple systems using a
          combination of device private and cross-device shared storage
          silos.

    \item An evaluation demonstrating that it is possible to capture activity
          context without imposing excessive overhead, in either space or time.

\end{enumerate}

These projects focus on improving \emph{finding}.

The remainder of this document provides more specific insight into these
contributions and how I propose creating and disseminating them.
In \autoref{ch:background} I review the prior work that
underlies my thesis: what types of storage silos exist, what information we
already have available and why these are not sufficient to meet these use cases.
In \autoref{ch:research-questions} I set out the research questions that I
seek to answer to fully explore my thesis.
In  \autoref{ch:architecture} I describe the structure of the system I
propose building in order to support my thesis and how it addresses these use
cases.  In \autoref{ch:evaluation} I discuss how I propose evaluating
my system.  Specifically I attempt to address key questions, such as: ``how well does it address these use cases?'', ``what are the performance
and resource implications of using my system?'', and ``how well does it enable other
communities to construct more effective finding tools?''

\endinput

\section{Meeting Notes}
\tm{This section is to be removed.  It is my recording of notes from the meeting of October 19, 2021 with Margo and Sasha.}

The vast majority of prior work has focused on using properties that are
intrinsic to the file to facilitate search and naming and in large part this
project is about looking outside the file and including environmental
information which we call ``activity context.''

That leads to the story line that says if we look at how we have both

Differentiate \emph{naming} from \emph{finding}. We've always been talking about
this as \emph{naming} and I'm no longer convinced that we are solving a naming
problem, I feel that we are solving a \emph{finding} problem.

There's this nice linear history.  We start with the hierarchical name space.
It was a way to organize digital information in a manner that reflected physical
data organization.  Gifford points out that the digital world offers us a richer
space and therefore when we consider these intrinsic attributes we can
facilitate even better finding.  There have been \emph{hints} that expanding to
looking at factors outside the file might provide an improved experience.  That
is Guo [Burrito] and Soules [Provenance Search].  We are taking that hint of an
observation and broadening it into something we all ``activity context'' which
is about the environment in which an object exists and here are examples of the
kinds of queries that we want such a system to be able to satisfy.

\item The prior work suggests context helps (cite Soules).  We want to
enable HCI researchers to ask the broader question of whether broadening the
definition of environment [activity context] also helps.  We do not want to
\emph{answer} that question --- we want to enable a system that permits the
HCI research community to explore it.  To do that, we need to identify how
you get this information.

That then begs several research questions:

\begin{itemize}

    \item What events are useful in establishing relationships between objects?
          I might suggest the question you want to ask is ``What is the universe of
          relationships that I can extract and how can I do that?''  Then we can ask
          to the HCI researchers to ask them which would be of interest to them.
          ``What is the universe of relationships that we could extract?'' and ``what
          are the techniques that we need to develop to extract them.''  At some point
          we have to introduce this multi-silo world, but let's hold off.

    \item We would like to build an extensible framework that allows you to add
          all these things.  What we have discussed in the past is another mode for
          getting this internal information, which is ``I pipe it through the ML
          service, which indicates if there is a cat in it, or not.'' [Note that in
          the extrinsic view, this isn't interesting because the ML classifier is just
          another form of transducer].

          Your description of what one needs to do is largely spot on (from the
          spreadsheet).  I would hesitate to call this ``the thesis question''; as a
          systems researcher identify all the things that I can collect already and
          then develop a framework that makes it easy to add other things and then
          another piece where we bring them together in the multiple silo thing.

          Notice you can in fact do a bunch of this in a single silo; you can do this
          for a single system by broadening Guo and Soules work and saying ``ok we
          really want to capture''.  I'd go back and look at Guo carefully that even
          in the single systems world I think his work captures all the things we're
          talking about but it is worth going over it carefully and making sure of
          that.

              [Sasha typed] ``The current silos are conventional ones can we think about
          something that one might not think of as a silo?''  Discord, slack, e-mail
          are all silos.  \emph{Those are frustrating silos}. ``What I was browsing at
          a particular time.''  Two different dimensions here: the different kinds of
          silos and

          In the background chapter it is worth enumerating all these different silos
          (the ones we can think of) and going from obvious to crazy.

          The other dimension, the kinds of environmental factors we might want to
          think about.  What was I browsing, what was I listening to, what other
          documents were open, did I move the browser tab to a different window [how
                  long did I spend looking at the given tab - e.g., ``how long did this tab
                  have focus'' for example].

          Do we try to capture everything?  Winnowing process.

          Sasha: I have a thought on how this could easily turn into a systems
          problem.  We could try to categorize alll kinds of crazy silos,
          applications, activity contexts with respect to how we can collect their
          information.  For example, conventional things like chat and discord we just
          go into their meta-data.  Things like file systems we go into their files
          and file names.  More crazy things like ``what I was browsing'' we might
          need to capture screen or capture charactters.  Have a categorization of
          techniques for collecting activity context such as browser search and meta
          data versus capturing the screen versus something else. The categorization
          of silos and see how they match.  Which silos work with which methods.
          Where do we need to invent new methods.  This then leads to the question
          like the amount of data and parsing it and storing it and that quickly turns
          into a systems problem.

          ``What is the information?''
          ``How can we capture it?''
          ``What is the service that unifies this information and makes it
          available/usable?''



\end{itemize}

This is the diagram from the white board:

\begin{itemize}
\item Introduction
\begin{itemize}
    \item Linear history
          \begin{itemize}
              \item HNS
              \item SFS
              \item TFS [Tony FS].
          \end{itemize}

    \item Background
          \begin{itemize}
              \item Silos (Taxonomy).  Classification of silos that let you collect data.
                    \begin{itemize}
                        \item What can you collect?
                        \item How can you collect?
                    \end{itemize}

                    We need HCI feedback on this information.  Helps us focus our initial
                    data collection effort to focus on selecting data that is useful to
                    them.

                        [Margo] Asking them what they want means we are limited to the things
                    they can imagine.  Thus, we need to have a universe of things that we
                    can offer them. [TM: there are integration components already for quite
                            a range of applications. E.g., discord integrates with games, music
                            players integrate with other stuff as well]
          \end{itemize}

    \item Architecture

          How do you integrate info into a service?  ``Context as a Service'' (CaaS).
          This is Kwishut so start by lifting that work.

    \item Privacy/Security

          We agreed that \emph{security} is not really the core issue for this system:
          we aren't subverting the existing system, we aren't proposing some bypass
          for the existing system.  Thus, the security consideration is largely about
          protecting our own meta-data.  There are standard techniques for doing
          authentication that can be used for connecting from a client to a trusted
          server, there is no need for us to re-invent the wheel.  By deliberately
          designing the system so that activity context information is maintained on a
          trusted machine, whether local or remote, we can assert that it is \emph{by
              design} secure.  Similarly, by restricting access to only a trusted machine
          the system ensures privacy as well since only authorized agents can access
          the activity context.  This does not exclude the possibility of having
          different architectures for data sharing.

          Worrying about securing the system at this point is premature: a generalized
          multi-tenant activity context system could be useful for correlating
          activities across users and systems, but is premature until we have
          established that activity context is itself useful.

              [Shasha] I wonder if we are inventing a new type of privacy paradigm on a
          local device.  What we have right now is when I'm setting up my new
          device each app asks me what is allowed or not allowed.  Here, I want
          something else: I want to allow it as long as the information stays on
          the device and is for my personal use, or allowed by certain
          applications, but I don't allow it to leave the device, or to allow it
          to be exfiltrated but only if it is encrypted.

          Kosta in ECE usable privacy/security.  He will probably be able to
          point to related work and decide what to do with it.  Use it to
          prepare for responding to the issue.

              [Margo] extract from the draft, get it to him before the presentation
          could have more information to share at the presentation.

    \item Evaluation

          We show this data is useful because the HCI researchers have confirmed it is
          useful.  Now the evaluation will fall out: lots of data, difficult to gather
          it, how you store it, model it, query it.  How to build this thing and make
          it useful.  Each one of these items will have an evaluation piece.

          How do I enable query?  Here's the interface I'm going to do and here
          is how that addresses these N use cases that we presented that you
          can't do in other systems.  These are use cases that we hypothesize
          are useful and you can't do them today and here's how you do them with
          this proposed system.  The overhead: massive amounts of data, how do
          we manage it.  What is the size, what is the performance impact.  How
          easy is it to add another silo?  These metrics can be used to evaluate
          the system we're proposing to build.

          The efficacy comes down to ``assuming that we have got some HCI
          researchers off looking at the data we collected initially, by the
          time we have built a system we ought to be able to just re-run that
          evaluation across multiple silos with this distributed data.''

          The storyline for the thesis has to be this is about building a system
          to facilitate this other world.  We are hoping that we will have
          collaborators to do that and are actively pursuing them.  My question
          is ``assuming it is useful, can we even build such a system that will
          enable that?''

\end{itemize}

[Sasha]  Write the intro and background from scratch, watch the recording,
transcribe what you said and you'll have most of the background.


%% The following is a directive for TeXShop to indicate the main file
%%!TEX root = ../diss.tex

\chapter{Introduction}
\label{ch:introduction}

\begin{epigraph}
    \emph{
        What should I possibly have to tell you, oh venerable one? Perhaps that you're searching far too much? That in all that searching, you don't find the time for finding?
    } --- Siddhartha (1922), Hermann Hesse.
\end{epigraph}

\section{Activity Context}
\label{ch:introduction:sec:activitycontext}

In 1945 Vannevar Bush described the challenges to humans of
finding things in a codified system of records~\cite{bush1945we}:

\begin{quotation}
    \emph{Our ineptitude in getting at the record is largely
        caused by the artificiality of systems of indexing. When data of any
        sort are placed in storage, they are filed alphabetically or numerically, and
        information is found (when it is) by tracing it down from subclass to
        subclass. It can be in only one place, unless duplicates are used; one
        has to have rules as to which path will locate it, and the rules are
        cumbersome. Having found one item, moreover, one has to emerge from the
        system and re-enter on a new path.}

    \emph{
        The human mind does not work that way. It operates by association. With one
        item in its grasp, it snaps instantly to the next that is suggested by the
        association of thoughts, in accordance with some intricate web of trails
        carried by the cells of the brain. It has other characteristics, of course;
        trails that are not frequently followed are prone to fade, items are not
        fully permanent, memory is transitory. Yet the speed of action, the
        intricacy of trails, the detail of mental pictures, is awe-inspiring beyond
        all else in nature.}
\end{quotation}

This is as true in 2021 as it was in 1945.  Thus, the question that motivates my
research is: ``Can we build systems that get us closer to that ideal?''  My
argument that we can by capturing additional information that is not necessarily
useful to computers, but is useful to humans.

I call this additional captured information \emph{activity context}. While similar
to the ideas proposed in \emph{Burrito}~\cite{guo2012burrito}, I have broadened
that idea beyond just ``what a user is doing'' to incorporate information about
what the user is \emph{experiencing} that corresponds to more human-like context
information because it is useful for constructing \emph{association}.

Thus an ``activity context'' is an answer to the question:
\emph{what is going on in relation to the current event on a digital object?}
The job of the system becomes answering this question in a way that is useful.

More concretely, activity context is concerned with the environment in which a
given digital object is accessed.  Without restricting the abstract concept,
concrete examples of what I consider to be elements of an ``activity context''
might include:

\begin{itemize}
    \item Current weather.
    \item Notable news events.
    \item Focus website opened in a visible browser tab.
    \item User's mood.
    \item User's heart rate.
\end{itemize}

This is distinguished from \emph{what the digital object is}. While
understanding what something \emph{is} has merit, the context in which a digital
object is \emph{used} yields additional understanding about that digital object.

Further examples, in the form of ``use cases,'' can be found in
\autoref{ch:intro:sec:use-cases}.

\section{Thesis}
\label{ch:introduction:sec:thesis-statement}

\textbf{
    \input{thesis.tex}
}

\vspace{0.25cm}

``Intelligent use of files depends on having sufficient knowledge about them: their purposes, structures, and
contexts. Humans have traditionally made do by using their own methods for capturing and manipulating
such knowledge, but this is not available to programs, nor is it necessarily
convenient for humans~\cite{mogul1986representing}.''

Determining \emph{context} is a challenge with modern computer storage systems.
It is unrealistic to expect human users to provide that context.
Context is dynamic, imprecise, and not necessarily obvious, yet humans rely
upon context to create associations. By making environmental context information
available to programs, those same programs are able to present better options,
which \emph{is} convenient for humans.

%To improve the usefulness of naming systems within computer storage systems
%software must evolve to provide flexible, scalable, and multi-silo management
%of data object naming.


\section{Finding}
\label{ch:intro:sec:finding}

The volume of digital data is growing exponentially. Thus, it is not surprising
that solutions that worked when users were grappling with kilobytes (KB) or megabytes
(MB) of data do not work in the face of this growing deluge.

IBM's first magnetic disk drive could store up to 2.5
megabytes(MB)~\footnote{\url{https://www.7dayshop.com/blog/terabyte-evolution/}}
of digital data.  In 2020, we add 1.7MB of data \emph{per second per
    person}~\footnote{\url{https://techjury.net/blog/big-data-statistics/}}.  Today
we have become digital hoarders, collecting and keeping so much data that we often
cannot find specific objects when we need them.  How did we reach this point?

Early persistent storage systems used a simple flat directory structure
that gave a unique name to each object (``file'').  Such a simple structure was sufficient
to  name and identify distinct data objects (``files'').
\emph{Finding} the correct file was just a matter of scanning and picking from
the list.  This simple
structure did not scale well and was replaced by
a model based upon how paper documents were organized.
The hierarchical name
space~\cite{barnard1958,daley1965general,ritchie1973unix,Saltzer1978} is one in
which files (``digital objects'') are grouped into directories (sometimes called
folders).  Directories can also be grouped into other directories.  This model
mirrors how a filing cabinet works: multiple sheets of paper are gathered into a
folder, folders are organized into drawers, drawers into filing cabinets, filing
cabinets into rooms, etc. The directory and file metaphor was in use by
1958~\cite{barnard1958} and persists today as the common model despite the
volume of data being stored by a single computer storage device (``disk drive'') increasing by at least $10^6$~\footnote{Disk
    drives were measured in MB in 1965 and are measured in TB today.}.

In addition to the challenges of scaling, the file cabinet metaphor imposes
physical file limitations that are not valid for digital data.
Physical file cabinets do not allow a document to be in two folders at the same
time but there is no such restriction on electronic documents. The Multics
researchers addressed this by creating the \emph{link}, an idea that is
still used in many modern file systems~\cite{daley1965general}.

Computer networking enabled data sharing between users and computers but
complicated naming. Remote data access was typically represented to users either
as a hierarchical file system~\cite{nfs,howard1988scale} or an application program that programmatically
connected users to remote data~\cite{levin1979transport,10.1145/800216.806594,birrell1982grapevine}.

By 1990 the volume of data with which users interacted was so
large that researchers questioned the utility of the hierarchical name space~\cite{vicente1987assaying}.
The Semantic File System (SFS)~\cite{gifford1991semantic} suggested that
organization of digital objects be more fluid so that
users could group items together in ways that were semantically meaningful. The
meta-data generated from semantic information generated from file contents
permitted powerful query-based dynamic file organization.

While semantic file systems have not been widely adopted, the concept of
extracting semantic information from file content is present in modern file
indexing systems.  These indexing services employ ``transducers'' to extract
semantic information from files. Desktop search utilities (Windows
Search, Apple Spotlight, Station for Linux) rely upon indexing services to
provide their functionality.

\begin{figure}
    \centering
    \caption{First Wayback capture of google.com (google.stanford.edu) in 1998}
    \label{fig:google}
    \includegraphics[width=0.95\textwidth]{figures/1998-11-11-Google-Screenshot-showing-early-index-size.png}
\end{figure}

There are parallels between the challenges of indexing files and the challenges
of indexing Internet web pages. Early search engines used a curation model in which
humans decided what websites were of interest based upon the information within
the web page itself --- similar to the way that semantic information was
extracted from files in the Semantic File
System~\footnote{\url{https://www.hpe.com/us/en/insights/articles/how-search-worked-before-google-1703.html}}.

Internet web page indexing changed profoundly when two Stanford graduate
students proposed a novel way to exploit the structure of Internet web pages to
extract \emph{usage} information from web pages that did not depend upon
semantic content~\cite{page1999pagerank}.

Google's website used to state the number of pages that they indexed.  In 1998
the first capture of Google's website by archive.org shows they claimed to index
more than 25 million pages (\autoref{fig:google}). Google no longer publishes
that number but industry estimates indicate the number is at least $10^3$ more
now than it was then, and this only covers a few percent of the entire content
stored on the
Internet~\cite{bosch2016estimating}~\footnote{\url{https://www.worldwidewebsize.com/}}.

Could we utilize a similar technique for finding information within our own
trove of files? While there are similarities between the Internet and our file
collections, there are also significant differences. Files lack the level of common
structure present in web pages, preventing simple extraction of references
between files. Files (or digital objects) are stored in myriad
locations with different access mechanisms: local storage, cloud storage,
database, collaboration applications, e-mail programs, etc.  Sometimes these
overlap: your e-mail program stores some or all of your data on your local
computer, within your local storage. However, you do not expect to use the tools
for searching your local storage to find things within your e-mail software.
Thus, we should consider them to be distinct storage locations. I refer to these
distinct storage locations as \emph{storage silos} (or just \emph{silos}) to
emphasize their inherently separated nature.

Network storage is presented in many different formats: an inexpensive disk
drive attached to the local network represents ``Network Attached Storage''
(NAS) or a specialized parallel data cluster such as HDFS~\footnote{\url{https://hadoop.apache.org/docs/r1.2.1/hdfs_design.html}},
DAOS~\footnote{\url{https://www.intel.ca/content/www/ca/en/high-performance-computing/daos-high-performance-storage-brief.html}},
Lustre~\footnote{\url{https://www.lustre.org/}}, or
Ceph~\footnote{\url{https://docs.ceph.com/en/pacific/cephfs/index.html}}.
They typically support one or
more common data sharing protocols, such as NFS~\cite{sandberg1986sun} or
CIFS~\footnote{\url{https://docs.microsoft.com/en-us/openspecs/windows_protocols/ms-cifs/d416ff7c-c536-406e-a951-4f04b2fd1d2b}}.
They vary dramatically in how they are managed, accessed, and searched.  In most
cases there is no common interface --- each represents a unique ``storage
silo.''

Cloud storage is one specific type of network storage that is popular because it
allows you to access your data from any of your devices, provides a reliable
backup mechanism, and permits selective download to any given device. However,
these benefits are paired with challenges when it comes to finding specific
digital objects.  If files are not present on your local device, the indexing
services on those devices cannot assist you.  You could download all of the
content from the cloud storage providers to enable indexing, but that consumes
considerably more bandwidth and storage and is impractical for devices that have
resource constraints.  While we can use the cloud providers' search services, that
requires iteration over each of those services, using different interfaces with
variable results.

Our files come from multiple sources including websites, e-mails, databases,
and collaboration tools.  Those documents are stored both locally and remotely.
We create, access, and modify documents and then send them onwards using any of
the variety of silos and collaboration tools at our disposal.  Just a few days
after we last accessed them we struggle to find those
documents.

Given the diffusion of files across storage silos that occurs because of our
sharing and use, we often find versions and related files scattered across
multiple silos. We struggle to find these versions and determine when we have
found the ``right'' one.

Returning to the question of using contextual information for improving
\emph{finding}, the research community has observed that adding
contextual information, such as the current weather, to existing file
collections materially improves human ability to find the relevant digital
object~\cite{vianna2019thesis,10.1145/1559845.1559992,dumais2016stuff}. Thus, it
seems the answer is ``Yes, using contextual information improves
\emph{finding}.''

Google solves a simpler problem: an Internet web search need merely find
\emph{an} answer to the search query.  A personal file search needs provide
\emph{the} answer. Thus, it may not be possible to provide a definitive answer,
but narrowing the potential list of plausible answers leads users to the
relevant file, which is the goal of \emph{finding}.

The data that we need for creating activity context is already being collected
by our computers. Modern computers collect vast amounts of information about us:
what we do, where we are, with whom we communicate, the applications we use, the
files we access, the music we play, the web pages we visit, even how we
feel~\cite{chakriswaran2019emotion,8933554}.  We \emph{know} this
data exists because our own devices provide this information to third parties.
Given this data is already being collected, we know the additional cost will be
to store and make it accessible to applications.

\emph{Finding as a Service} (FaaS) will use this existing information to solve our data
finding problem. FaaS decouples \emph{finding} objects from \emph{storing}
objects.  FaaS facilitates \emph{finding} by exploiting contextual information
beyond the basic object characteristics widely available for searching today:
names, types, and dates. By relating information we already have with how our
digital objects are used, we provide the \emph{activity context} to enable
\emph{Finding as a Service} (FaaS).

\emph{Activity context} is important because it captures useful
information about the environment in which files are created,
consumed, and updated.  \emph{Activity context} need not be something the system
ordinarily relates to the digital object.  \emph{Activity context} captures key
information about the the user's wholistic environment.

\section{Use Cases}
\label{ch:intro:sec:use-cases}

The following use cases provide specific scenarios that cannot be
achieved using current systems.  I maintain that \emph{Finding as a Service}
addresses these use cases, which supports my thesis.

\begin{itemize}
    \item \label{use-case:e-mail}\textbf{The lost original.} Imagine that you received a spreadsheet
          from someone.  You begin to edit it, add information, sift through it.  At
          some point you realize that you sorted a subset of the columns, hopelessly
          scrambling the original information and your edits. You try to find the
          original source of the spreadsheet, only to realize that you cannot do so,
          even when you search in your e-mail program using the \emph{name} of the
          file that was saved on your local drive as part of your editing process. The
          system should permit you to find the original source of the information,
          even though the related digital objects are in different \emph{silos}.

    \item \label{use-case:misplaced-presentation}\textbf{The misplaced presentation.} You arrive at a meeting with your
          client after a long trip, only to realize the laptop computer you were
          using will not boot.  You have your smart phone and you \emph{think}
          you saved it to a cloud service.  How do you find it so you can share
          it with a colleague at the meeting? The system should permit you to
          find your own digital data in your cloud storage regardless of which
          device you used to create it.

    \item \label{use-case:multi-silo-problem}\textbf{The multi-silo relationship problem.} A colleague shares their
          experimental data with you, which was stored in NREL's
          High-Performance Computing Data
          Center~\footnote{\url{https://www.nrel.gov/computational-science/hpc-data-center.html}}.
          You then use that data as part of your own work, which you wish to
          share with a broader audience using Compute
          Canada~\footnote{\url{https://www.computecanada.ca/techrenewal/rdm/}}.
          You also shared your computational notebooks using your organization's
          account with Microsoft~\footnote{\url{https://visualstudio.microsoft.com/vs/features/notebooks-at-microsoft/}}.
          You created your slides in
          Prezi~\footnote{prezi.com} and presented them to a different research
          group on their Discord server~\footnote{\url{discord.com}}. One of the
          people that attended downloaded two of your notebooks and created a
          new notebook from them.  They then shared that notebook publicly via
          Google Colab~\footnote{\url{https://colab.research.google.com/}}.
          Your colleague knows that she shared her data with you and wants to be
          able to quickly find the documents that you and others have shared
          based upon that original data. Collaborative work like this is a
          modern reality and it is unlikely that all work product will be co-located.
          The system should permit your colleagues to find the work you and
          others have shared with her \emph{without} your intervention.

    \item \label{use-case:multi-silo-finding}\textbf{Multi-silo finding.} \persd is a visiting student from the
          country of Lemuria doing an internship with you in Camelot.  While arranging
          for this internship, \persd required \emph{numerous} different data objects:
          email messages with the host, offer letters, academic forms, a visa,
          boarding passes, project proposals, and more. The system should be able to
          provide you with a set of related files, regardless of their storage
          silo.

    \item \label{use-case:info-source}\textbf{The where did I get this information conundrum.}  In our modern
          world we often have one (or more) web pages open when we are authoring one
          (or more) documents.  A reasonable question to ask would then be ``what web
          pages did I look at while writing this document?'' Often it is not just that
          you looked at a given web page, but also if it was the last web page you
          looked at and how long you looked at it.  The system should be able to
          provide you with a list of that web activity.

    \item \label{use-case:privacy}\textbf{How do I share information while
              preserving privacy?} \persg, an investigative journalist who routinely receives
          sensitive information
          from third parties, is investigating the company from the prior use cases.
          \persg needs to be able store and access sensitive information, including
          information about the activity context of various e-mails, documents, pictures,
          and audio and video files. While \persg ensures that these data are
          encrypted, they need to also ensure
          that they can both find information and ensure that meta-data associated with
          those files is both usable and properly protected across silos.
          While \persg must protect their sources, they must also be able to associate
          evidence with those sources to make judgement calls about their validity.
          The system should support security and privacy policies for attributes that
          accomplish both.
\end{itemize}

This list of use cases is not exhaustive and is intended to provide cases that
resonate with readers.  They are use cases that are not addressed by existing
systems.  I review these existing systems and how they fail to address these use
cases in \autoref{ch:intro:sec:existing-solutions}.

\section{Existing Solutions Fall Short}
\label{ch:intro:sec:existing-solutions}

Prior work has addressed some of the challenges that I identified in the use
cases (\autoref{ch:intro:sec:use-cases}).  I briefly introduce key
aspects of how they fall short here and provide greater detail in
\autoref{ch:background}.

A simple solution to the multi-silo namespace challenge is to graft those
namespaces together. UNIX mount points~\cite{unix} are perhaps the first
instance of such federating namespaces. Distributed federation, as provided by
distributed file systems such as NFS~\cite{nfs} and AFS~\cite{howard1988scale}
emerged in the 1980s soon after adoption of high speed networks such as
Ethernet~\cite{digital1980ethernet}.

There is some work in cloud storage federated
namespaces~\cite{scfs,federatedMetaData}.
Nextcloud~\footnote{\url{https://nextcloud.com}} allows users to connect
multiple Nextcloud instances and integrate with FTP, CIFS, NFS and object
stores. This yields a classic hierarchical namespace structure with its known
limitations~\cite{vicente1987assaying,vicente1988accommodating}. It does nothing
to facilitate \emph{finding}. Peer-to-peer sharing networks (e.g., IPFS \cite{benet2014ipfs}) implement a
distributed file system where nodes advertise their files to users.
MetaStorage~\cite{metastorage} implements a highly available, distributed hash
table, similar to Amazon's DynamoDB~\cite{10.1145/2213836.2213945},
but with its data replicated and distributed across different cloud providers.
MetaStore offers a key-value store interface~\footnote{\url{https://cwiki.apache.org/confluence/display/hive/design}}.
Farsite~\cite{Adya:2003:Farsite} organizes multiple machines into virtual file
servers, each of which acts as the root of a distributed file system. Comet
describes a cloud oriented federated metadata service~\cite{federatedMetaData}.

None of these address the scaling problem that arises when data is analyzed and indexed
away from where it is stored.  Similarly, none of these address the integration
of sensitive locally stored information about personal usage of these objects.
Thus, one key benefit of better \emph{finding} is that
it should improve the efficiency of retrieving data stored across non-local
storage silos.

Some prior work explored using extrinsic usage information for \emph{finding}.
Placeless~\cite{placeless-tois} focused on using process level information extracted from their document processing
system to associate files together. Similarly,
Burrito~\cite{guo2012burrito} proposed \emph{activity context}, which they define as ``the
user's actions at a particular time.'' Both look at narrow instances of the
larger \emph{finding} problem.

Provenance uses observable information about construction of a file to augment file
search, which in turn improves findability~\cite{provsearch}. Provenance search
takes a narrow view of the activities of interest and are all largely
\emph{causality} focused.  However, humans tend to think
associatively~\cite{10.1145/1559845.1559992}, focusing on what else was
happening --- the cleaning crew came by their desk while they were writing that
document, the discussion at a meeting with others, their location when they
wrote a given document, or some other event that was happening around the time
they interacted with a given document.

While environmental information is not as obviously related as causal
relationships, prior work related to using statistical inference to establish
relationships within the storage domain has demonstrated such mechanisms can be
more efficient at identifying patterns that lead to higher
efficiency~\cite{10.1145/3035918.3064029}.

\section{Contributions}
\label{ch:intro:sec:contributions}

The research to support my thesis will contribute the following:

\begin{enumerate}
    \item Production of the \emph{Finding as a Service} (FaaS) dataset, a
          collection of meta-data and activity context from a local system, that
          will enable me to explore potentially useful information for informing
          activity context data collection. In addition, I will publicly
          share this data set to enable other researchers to develop new
          techniques for users to find digital items.

    \item \system~\footnote{\system is Xhosa for pragmatics.  In Linguistics,
              pragmatics is the study of meaning within a given context}, my
          architecture for a system that captures, stores, and
          disseminates \emph{activity context} without imposing excessive resource
          demand.

    \item \systemone~\footnote{\systemone is the Uzbek word for finding; the most
              recent graduate student from our research group is from Uzbekistan and
              has always been supportive of my research}, a single node
          implementation consistent with \system that provides
          \emph{FaaS} across multiple storage silos on a single
          system.

    \item \systemtwo~\footnote{\systemtwo is the Russian word for finding in
              recognition of the support for my research that I have received from both Ada Gavrilovska and
              Alexandra Fedorova.}, a distributed implementation of \system that provides
          \emph{Finding as a Service} (FaaS) across multiple systems using a
          combination of device private and cross-device shared storage
          silos.

    \item An evaluation demonstrating that it is possible to capture activity
          context without imposing excessive overhead, in either space or time.

\end{enumerate}

These projects focus on improving \emph{finding}.

The remainder of this document provides more specific insight into these
contributions and how I propose creating and disseminating them.
In \autoref{ch:background} I review the prior work that
underlies my thesis: what types of storage silos exist, what information we
already have available and why these are not sufficient to meet these use cases.
In \autoref{ch:research-questions} I set out the research questions that I
seek to answer to fully explore my thesis.
In  \autoref{ch:architecture} I describe the structure of the system I
propose building in order to support my thesis and how it addresses these use
cases.  In \autoref{ch:evaluation} I discuss how I propose evaluating
my system.  Specifically I attempt to address key questions, such as: ``how well does it address these use cases?'', ``what are the performance
and resource implications of using my system?'', and ``how well does it enable other
communities to construct more effective finding tools?''

\endinput

\section{Meeting Notes}
\tm{This section is to be removed.  It is my recording of notes from the meeting of October 19, 2021 with Margo and Sasha.}

The vast majority of prior work has focused on using properties that are
intrinsic to the file to facilitate search and naming and in large part this
project is about looking outside the file and including environmental
information which we call ``activity context.''

That leads to the story line that says if we look at how we have both

Differentiate \emph{naming} from \emph{finding}. We've always been talking about
this as \emph{naming} and I'm no longer convinced that we are solving a naming
problem, I feel that we are solving a \emph{finding} problem.

There's this nice linear history.  We start with the hierarchical name space.
It was a way to organize digital information in a manner that reflected physical
data organization.  Gifford points out that the digital world offers us a richer
space and therefore when we consider these intrinsic attributes we can
facilitate even better finding.  There have been \emph{hints} that expanding to
looking at factors outside the file might provide an improved experience.  That
is Guo [Burrito] and Soules [Provenance Search].  We are taking that hint of an
observation and broadening it into something we all ``activity context'' which
is about the environment in which an object exists and here are examples of the
kinds of queries that we want such a system to be able to satisfy.

\item The prior work suggests context helps (cite Soules).  We want to
enable HCI researchers to ask the broader question of whether broadening the
definition of environment [activity context] also helps.  We do not want to
\emph{answer} that question --- we want to enable a system that permits the
HCI research community to explore it.  To do that, we need to identify how
you get this information.

That then begs several research questions:

\begin{itemize}

    \item What events are useful in establishing relationships between objects?
          I might suggest the question you want to ask is ``What is the universe of
          relationships that I can extract and how can I do that?''  Then we can ask
          to the HCI researchers to ask them which would be of interest to them.
          ``What is the universe of relationships that we could extract?'' and ``what
          are the techniques that we need to develop to extract them.''  At some point
          we have to introduce this multi-silo world, but let's hold off.

    \item We would like to build an extensible framework that allows you to add
          all these things.  What we have discussed in the past is another mode for
          getting this internal information, which is ``I pipe it through the ML
          service, which indicates if there is a cat in it, or not.'' [Note that in
          the extrinsic view, this isn't interesting because the ML classifier is just
          another form of transducer].

          Your description of what one needs to do is largely spot on (from the
          spreadsheet).  I would hesitate to call this ``the thesis question''; as a
          systems researcher identify all the things that I can collect already and
          then develop a framework that makes it easy to add other things and then
          another piece where we bring them together in the multiple silo thing.

          Notice you can in fact do a bunch of this in a single silo; you can do this
          for a single system by broadening Guo and Soules work and saying ``ok we
          really want to capture''.  I'd go back and look at Guo carefully that even
          in the single systems world I think his work captures all the things we're
          talking about but it is worth going over it carefully and making sure of
          that.

              [Sasha typed] ``The current silos are conventional ones can we think about
          something that one might not think of as a silo?''  Discord, slack, e-mail
          are all silos.  \emph{Those are frustrating silos}. ``What I was browsing at
          a particular time.''  Two different dimensions here: the different kinds of
          silos and

          In the background chapter it is worth enumerating all these different silos
          (the ones we can think of) and going from obvious to crazy.

          The other dimension, the kinds of environmental factors we might want to
          think about.  What was I browsing, what was I listening to, what other
          documents were open, did I move the browser tab to a different window [how
                  long did I spend looking at the given tab - e.g., ``how long did this tab
                  have focus'' for example].

          Do we try to capture everything?  Winnowing process.

          Sasha: I have a thought on how this could easily turn into a systems
          problem.  We could try to categorize alll kinds of crazy silos,
          applications, activity contexts with respect to how we can collect their
          information.  For example, conventional things like chat and discord we just
          go into their meta-data.  Things like file systems we go into their files
          and file names.  More crazy things like ``what I was browsing'' we might
          need to capture screen or capture charactters.  Have a categorization of
          techniques for collecting activity context such as browser search and meta
          data versus capturing the screen versus something else. The categorization
          of silos and see how they match.  Which silos work with which methods.
          Where do we need to invent new methods.  This then leads to the question
          like the amount of data and parsing it and storing it and that quickly turns
          into a systems problem.

          ``What is the information?''
          ``How can we capture it?''
          ``What is the service that unifies this information and makes it
          available/usable?''



\end{itemize}

This is the diagram from the white board:

\begin{itemize}
\item Introduction
\begin{itemize}
    \item Linear history
          \begin{itemize}
              \item HNS
              \item SFS
              \item TFS [Tony FS].
          \end{itemize}

    \item Background
          \begin{itemize}
              \item Silos (Taxonomy).  Classification of silos that let you collect data.
                    \begin{itemize}
                        \item What can you collect?
                        \item How can you collect?
                    \end{itemize}

                    We need HCI feedback on this information.  Helps us focus our initial
                    data collection effort to focus on selecting data that is useful to
                    them.

                        [Margo] Asking them what they want means we are limited to the things
                    they can imagine.  Thus, we need to have a universe of things that we
                    can offer them. [TM: there are integration components already for quite
                            a range of applications. E.g., discord integrates with games, music
                            players integrate with other stuff as well]
          \end{itemize}

    \item Architecture

          How do you integrate info into a service?  ``Context as a Service'' (CaaS).
          This is Kwishut so start by lifting that work.

    \item Privacy/Security

          We agreed that \emph{security} is not really the core issue for this system:
          we aren't subverting the existing system, we aren't proposing some bypass
          for the existing system.  Thus, the security consideration is largely about
          protecting our own meta-data.  There are standard techniques for doing
          authentication that can be used for connecting from a client to a trusted
          server, there is no need for us to re-invent the wheel.  By deliberately
          designing the system so that activity context information is maintained on a
          trusted machine, whether local or remote, we can assert that it is \emph{by
              design} secure.  Similarly, by restricting access to only a trusted machine
          the system ensures privacy as well since only authorized agents can access
          the activity context.  This does not exclude the possibility of having
          different architectures for data sharing.

          Worrying about securing the system at this point is premature: a generalized
          multi-tenant activity context system could be useful for correlating
          activities across users and systems, but is premature until we have
          established that activity context is itself useful.

              [Shasha] I wonder if we are inventing a new type of privacy paradigm on a
          local device.  What we have right now is when I'm setting up my new
          device each app asks me what is allowed or not allowed.  Here, I want
          something else: I want to allow it as long as the information stays on
          the device and is for my personal use, or allowed by certain
          applications, but I don't allow it to leave the device, or to allow it
          to be exfiltrated but only if it is encrypted.

          Kosta in ECE usable privacy/security.  He will probably be able to
          point to related work and decide what to do with it.  Use it to
          prepare for responding to the issue.

              [Margo] extract from the draft, get it to him before the presentation
          could have more information to share at the presentation.

    \item Evaluation

          We show this data is useful because the HCI researchers have confirmed it is
          useful.  Now the evaluation will fall out: lots of data, difficult to gather
          it, how you store it, model it, query it.  How to build this thing and make
          it useful.  Each one of these items will have an evaluation piece.

          How do I enable query?  Here's the interface I'm going to do and here
          is how that addresses these N use cases that we presented that you
          can't do in other systems.  These are use cases that we hypothesize
          are useful and you can't do them today and here's how you do them with
          this proposed system.  The overhead: massive amounts of data, how do
          we manage it.  What is the size, what is the performance impact.  How
          easy is it to add another silo?  These metrics can be used to evaluate
          the system we're proposing to build.

          The efficacy comes down to ``assuming that we have got some HCI
          researchers off looking at the data we collected initially, by the
          time we have built a system we ought to be able to just re-run that
          evaluation across multiple silos with this distributed data.''

          The storyline for the thesis has to be this is about building a system
          to facilitate this other world.  We are hoping that we will have
          collaborators to do that and are actively pursuing them.  My question
          is ``assuming it is useful, can we even build such a system that will
          enable that?''

\end{itemize}

[Sasha]  Write the intro and background from scratch, watch the recording,
transcribe what you said and you'll have most of the background.



%    2. Main body
% Generally recommended to put each chapter into a separate file
\chapter{Background}
\label{ch:background}

\tm{Please note that this is a \textbf{work in progress} and I am providing this
    draft with this section partially written because I seek feedback on the overall
    structure and model that I present.  While this section will change in
    subsequent drafts, I do not expect its content to materially change the model
    that I present in \autoref{ch:model}.}

The prior work of Saltzer and Watson establish that the purpose of file systems
is to serve the naming needs of
\emph{users}~\cite{Saltzer1978,watson1981identifiers}.   However, the background
literature on file systems can be broadly broken up into two categories:

\begin{enumerate}
    \item \textbf{Storage Management} --- much of the prior work related to file
          systems focuses on the management of file data.  For physical media file
          systems --- those file systems that manage some sort of media, whether it is
          magnetic, optical, or solid state --- the concerns are about efficient use
          of the media itself.  For distributed (``network'') file systems, much of
          the prior work focuses on efficient protocols for providing access to file
          system data over a network.

    \item \textbf{Naming} --- a small amount of the prior work related to file
          systems focuses on the organization of naming.  There is some overlap
          between early storage management and naming literature.  Similarly,
          distributed file systems had to consider naming as well.

\end{enumerate}

I review this prior work in the subsequent sections
(\autoref{ch:background:sec:storage} and \autoref{ch:background:sec:naming}) and
then analyze these in the context of \system.  I then provide a brief review of
relevant linguistics literature pertinent to my thesis by reviewing the
linguistic field of \emph{semantics} and \emph{pragmatics}
(\autoref{ch:background:sec:linguistics}). Finally, I look at
the challenges that are not addressed in prior work that are necessary for me to
evaluate my thesis (\autoref{ch:background:sec:challenges}).

\section{Storage Management}
\label{ch:background:sec:storage}

Most of the research literature about media file systems involves management of how
data should be organized on the given media. While not central to my own thesis,
it is useful to understand that file systems research continues to be an active
and ongoing research area, but it is \emph{not} related to naming, and that work
really has little direct bearing on the naming aspects of file systems.

\tm{Multics}
\tm{TENEX}
\tm{BSD FFS}
\tm{Cedar}
\tm{LFS}
\tm{}


One of the earliest papers on file systems, ERMA,
The Multics file system was described by Daley in 1965~\cite{daley1965general},
and while it clearly lays out the hierarchical name space, it also provides
insight into very early file systems development, with important considerations
like parity checking, overflow sending, and the "printer bit and column
counter".

By 1965 media file systems development


\section{Naming}
\label{ch:background:sec:naming}

% \tm{1945 - Memex}

In 1945 Vannevar Bush described the challenges to humans of finding things in a
codified system of records, including those used by early computers:

\begin{quotation}
    Our ineptitude in getting at the record is largely
    caused by the artificiality of systems of indexing. When data of any
    sort are placed in storage, they are filed alphabetically or numerically, and
    information is found (when it is) by tracing it down from subclass to
    subclass. It can be in only one place, unless duplicates are used; one
    has to have rules as to which path will locate it, and the rules are
    cumbersome. Having found one item, moreover, one has to emerge from the
    system and re-enter on a new path.

    The human mind does not work that way. It operates by association. With one
    item in its grasp, it snaps instantly to the next that is suggested by the
    association of thoughts, in accordance with some intricate web of trails
    carried by the cells of the brain. It has other characteristics, of course;
    trails that are not frequently followed are prone to fade, items are not
    fully permanent, memory is transitory. Yet the speed of action, the
    intricacy of trails, the detail of mental pictures, is awe-inspiring beyond
    all else in nature.
\end{quotation}

This is as true in 2021 as it was in 1945.  While preparing this proposal I
spent time looking at the guides many libraries provided about the naming of
files. I found a body of recommendations about file naming
standards\footnote{Data Management for
    Researchers~\cite{briney2015data}}
\footnote{Smithsonian: \url{https://library.si.edu/sites/default/files/tutorial/pdf/filenamingorganizing20180227.pdf}}
\footnote{Stanford: \url{https://library.stanford.edu/research/data-management-services/data-best-practices/best-practices-file-naming}}
\footnote{NIST: \url{https://www.nist.gov/system/files/documents/pml/wmd/labmetrology/ElectronicFileOrganizationTips-2016-03.pdf}}.
Harvard Data Management suggests~\footnote{\url{https://datamanagement.hms.harvard.edu/collect/file-naming-conventions}}:

\begin{itemize}
    \item Think about your files
    \item Identify metadata (e.g., date, sample, experiment)
    \item Abbreviate or encode metadata
    \item Use versioning
    \item Think about how you will search for your files
    \item Deliberately separate metadata elements
    \item Write down your naming conventions
\end{itemize}

\tm{Note that the footnotes aren't rendering cleanly.  I will need to clean this up.}

Throughout these examples there are common themes: a name provides context for
\emph{what} the given object represents. Uniformity of information is also
important --- the ``naming convention'' permits not only identifying similarity
but key elements of \emph{difference} between any two named things.

\tm{It seems that this Harvard list is a serious indictment of the existing
    system: it pushes the cognitive load onto the users, talks about meta-data ,
    versioning, encoding, \emph{and} capturing the naming convention.}




Bush's observation here was about
the importance of developing storage systems that enabled human cognition by
forming an extended memory.  What is remarkable about this early work is how
well Bush captured the basic human need for such a storage system and how it
related to human cognition and associative thinking.  I note that we still do
not have any system that is as functional as Bush's Memex, though there are
aspects of the modern world wide web that are reminiscent of some of the
concepts he first described.

While I do not claim Bush's 1945 Atlantic article was file systems
research, it is useful to understand the perspective of work that
followed~\cite{bush1945we}.

% \tm{1956 - ERMA}

One of the earliest file system papers of which I am aware was
ERMA~\cite{barnard1958}.  This paper was more focused on documenting their
design process, but as part of this work they described the hierarchical
structure of files.  The diagrams clearly show the "folder/file" metaphor that
survives to this day.

\tm{1964 - A File Structure for the Complex, the Changing and the Indeterminate}

Nelson builds upon Bush's Memex description as he explores ``[t]he kinds of file
structures required if we are to use the computer for personal files and ad an
adjunct to creativity\ldots''  He points out the challenges inherent in the
task: ``They need to provide the capacity for intricate and idiosyncratic
arrangements, totally modifiable, undecided alternatives, and thorough internal
documentations.''

Nelson's work is usually cited as inspiration for the world wide web and its use
of \emph{hyperlinks} to associate related content together, yet this work was
primarily focused on how personal file information should be organized within a
computer system.  Key ideas here include the \emph{dynamic} nature of such
organization (``[i]t was also intended that the system would allow index
manipulations which we may call \underline{dynamci outlining} (or
\underline{dynamic indexing}).'')  Another important aspect of this work was his
repudiation of hierarchical organization: ``[n]o hierarchical file relations
were to be built in; the system would hold any shape imposed on
it.''~\cite{nelson19654}

\tm{1965 - Multics}

In 1965 the ``Fall Joint Computer Conference'' was dominated by papers about
General Electric's new Multics operating system.  One of those papers was
specific to the file system~\cite{daley1965general}.  From a naming perspective
the key observation was the use of the hierarchical file system structure,
clearly described and highly reminiscent of the hierarchical file system that
has been adopted first by UNIX and then enshrined in the POSIX interface.  One
observation I have with respect to this work is that while the initially propose
the hierarchical file system, the paper admits that this isn't sufficiently
descriptive and introduces the additional concept of a \emph{link}. This serves
as a good example of how the issues of storage efficiency (e.g., not wanting to
have duplicate copies of the file stored on disk) impacted naming (where the
fact that an object has multiple names need not be tied to the actual manner in
which this is implemented.)

\tm{This raises the question about mutability.  If you change the \emph{file} do
    you preserve the name or do you break it?  That's a really interesting question
    and I don't think anyone will be happy with pretty much any answer.  It's also
    the source of one of the real frustrations in modern computing, namely the use
    of shared libraries with the same \emph{name} but subtly different
    \emph{implementation}.  Hence we end up with \emph{libfoo.so} and
    \emph{libfoo.6.so} and \emph{libfoo.6.4.so}.  In Windows this became ``DLL
    hell'' and has lead to solutions like "side-by-side" installations of commonly
    used shared libraries.  There's a really difficult question lurking beneath this
    for naming: does the name specify something specific or something general?  This
    is the \emph{binding} problem that Saltzer refers to as well.
}

\tm{TENEX}

At the Fall Joint Computer Conference in 1972 Murphy presented a paper about the
storage organization of TENEX, including its naming scheme.  ``A powerful and versatile
directory and file naming facility is provided in which
a particular file is identified by a fixed-depth path which
includes device, directory name, file name, extension,
and version.''~\cite{murphy1972storage}

\tm{UNIX}

The 1974 Symposium on Operating Systems Principles included papers that
ultimately led to both of the primary ``families'' of operating systems that we
use today.  One of those introduced the UNIX operating system and that work
presented the model of ``everything is a file'' and the hierarchical name space
organization of those files~\cite{unix}.  Intriguingly, the need to add
additional contextual information was recognized even then: ``Besides the system
proper, the major programs available under UNIX are: ... and permuted index program.''

\tm{CAP}

The Cambridge Capability Protection operating system included a novel model for
using capabilities to protect files.  The namespace that CAP presented is
interesting in that it did consider non-hierarchical name spaces~\cite{needham1977cap}: ``By handling
directory capabilities in the same way as store capabilities we allow not only
hierarchical directories but also shared directories --- the structure can form
an arbitrary directed graph, which may even be cyclic.''  They also described
the concept of \emph{dynamic} directories: ``By allowing dynamic creation of
directories, we allow the manufacture of directories or of complex structures
of directories separate from the main filing system, accessible only to certain
programs (this is used for the password file, for example).''

\tm{NFS}

As we build communications networks to share information between computers the
idea of sharing files between computers quickly emerged.  Sun Microsystems'
Network File System (NFS) was not the first network file system but it is
certainly one of the most widely used and it continues to be used
today.~\cite{nfs}  It extended the hierarchical file system so that it
incorporated the name spaces of a remote computer system into the local file
system. While NFS provided subtly different behavior than a local file system,
it was ``close enough'' that in the absence of failures it ``just worked.''

\tm{AFS}

The Information Technology Center at Carnegie-Mellon University (CMU) was responsible
for the realization of a system of network attached computer workstations for
use in the CMU environment. Mahadev Satyanarayanan and his resarch group were
responsible for developing the Andrew File System (AFS), which was subsequently
commercialized by Transarc in the late 1990s.  AFS remains in use today.  AFS
was contemporaneous with NFS, but offered significantly different functionality,
including a \emph{global} name space, federated security (via MIT's Kerberos),
and a separate service that was responsible for resolving the global namespace
across various clients~\cite{howard1988scale}.

\tm{Universal Directory Service}

The concept of distributed name or directory services is one that is not new.
In 1985 Lantz provided a description of a ``Universal Directory Service'' that
is similar to some of the work that I anticipate while evaluating my own thesis.
This paper describes their services as offering the following
capabilities~\cite{10.1145/12481.12483}:
\begin{itemize}
    \item ``can span a heterogenous internetwork of existing naming domains;
    \item ``allows us to name, locate, and discover how to manipulate objects
          (including files, processes, mailboxes, people, and services);
    \item ``provides dynamic binding and context mechanisms; and
    \item ``can be integrated into most existing systems as a ``value-added''
          feature.''
\end{itemize}

While similar conceptually to the domain name service (DNS) that is used for
naming in the internet, it is considerably more general.  Intriguingly, while
there is subsequent work that cites to Lantz's work, I could
not find any later work that extended it.

\tm{Properties}
The idea of extended attribute information being associated with a file is one
that emerged in the 1980s.  Mogul's 1986 treatise ``Representing Information
About Files'' does an excellent job of summarizing the research up to that
point, including an explanation of the history regarding ``Leaf'' a protocol
that while developed at XEROX PARC for the Alto was only memorialized by him and
Brian Reid in 1981.  That appears to have been motivation for the creation of
what we now call ``extended attributes'', though Mogul refers to them as ``file
properties''~\cite{mogul1986representing}. Mogul's work provides considerable
insight into how file properties could be implemented, both by applications (as
part of the file itself) as well as within file system meta-data.  We actually
supported properties in Episode~\cite{chutani1992episode}.  An important argument Mogul
makes is that \emph{naming} should be separated from \emph{storage}:

\begin{quotation}

    The old monolithic system model usually implied a tight coupling between file system and directory
    system; in many cases the two are indistinguishable. While this approach might improve performance
    slightly, one of the recognized virtues of the client-server model is that it encourages separation of function.
    Separating file system and directory system into two distinct services (which might nevertheless interact as
    clients of each other) provides several benefits:
    \begin{itemize}
        \item \textbf{Modularity}: with attendant benefits of a cleaner service model, cleaner implementations, and
              the flexibility to substitute new implementations of one service without affecting the other.

              Modularity has additional value: it is easier to distribute specialized services than to build a
              generalized distributed system. For example, LOCUS uses its knowledge of the special
              semantics of directories to recover them after a partition; this cannot be done for files in
              general [Popek 81].

        \item \textbf{Crossing file system boundaries}: an integrated directory system can only store references to
              files in its associated file system. In a heterogeneous distributed system, we would like to
              construct a unified name space covering all file servers in the environment. Why should we
              prohibit users from storing in the same directory references to files stored by two distinct file
              servers? The Universal Directory System [Lantz 85] is an example of an approach to this
              question.

        \item \textbf{Non-file referents}: Directory systems can and should be used as general name-binding agents;
              they need not only refer to files. In fact, directories have been used to name non-file objects
              even in Unix (where devices appear in the file system name space). By separating directory
              from file service and giving non-file referents full citizenship in directory bindings, we can
              gain useful generality.
    \end{itemize}

    Not everyone agrees that directory service and file service should be separate; for example, the V-system
    takes the opposite point of view [Cheriton 84]. We prefer separation because it
    leads to a simpler model.

\end{quotation}

His definition of a file is useful as well: ``\emph{a \emph{file} is a named
    object that stores an arbitrary amount of data for an arbitrarily long time.}''

\begin{quotation}
    The Alto and WFS were research projects. Other groups within Xerox were working on commercial
    products, the Star professional workstation in particular. The original design for the Star file system [Dalal
            86] was tightly integrated with application-level functions. It was then realized that clearer separation of
    file service and application would yield a more open, flexible system, but the resulting file system retained
    some features of Star that were found to be generally useful. One of these features was support for
    extensible attributes, including attribute-based search.
\end{quotation}

My general sense remains that Mogul's work is the most closely related model of
the work I propose doing as part of this thesis.

\tm{Semantic}

Gifford introduced the concept of a \emph{semantic} file system at SOSP in
1991~\cite{gifford1991semantic}.  I reproduce the abstract from his paper to
provide the basic idea of his work.  While I \emph{started} from considering
semantic file systems as part of my PhD, my own work has moved beyond the model
of Gifford and is complementary to Gifford's own work.

\begin{quotation}
    A semantic file system is an information storage system that
    provides flexible associative access to the system's contents
    by automatically extracting attributes from files with file
    type specific transducers. Associative access is provided by a
    conservative extension to existing tree-structured file system
    protocols, and by protocols that are designed specifically for
    content based access. Compatibility with existing file system
    protocols is provided by introducing the concept of a
    virtual directory. Virtual directory names are interpreted as
    queries, and thus provide flexible associative access to files
    and directories in a manner compatible with existing software.
    Rapid attribute-based access to file system contents
    is implemented by automatic extraction and indexing of key
    properties of file system objects. The automatic indexing of
    files and directories is called "semantic" because user programmable
    transducers use information about the semantics
    of updated file system objects to extract the properties for
    indexing. Experimental results from a semantic file system
    implementation support the thesis that semantic file systems
    present a more effective storage abstraction than do traditional
    tree structured file systems for information sharing
    and command level programming.
\end{quotation}

A key point here is that semantics are defined based upon the \emph{content} of
the file itself. This is certainly useful but different than the model I
propose, which would introduce \emph{pragmatics}.  Pragmatics are defined based
upon the \emph{context} of how the given file is used.

Another insightul source of information about semantic file systems is from
Martin's PhD thesis~\cite{martin2008}.  He utilizes \emph{Formal Concept
    Analysis} as a mechanism for improving semantic file systems; his work continued
until a few years ago~\footnote{http://www.libferris.com}.  Some of this serves
as a caution about how challenging it is to build useful systems that augment
the file system namespace.

\tm{Tags}

The concept of using \emph{tags} on files is related to the earlier work on
semantics.  The distinction appears to be that tags were initially designed to
be added by users, though the possibility of combining transducer generated
tags, as in the semantic file systems work, with user created tags is certainly
not excluded~\cite{tagfs}.  Indeed, the Human-Computer Interface (HCI) community
has observed that ``[U]sers are unlikely to use metadata.  Even when the tagging
is somethign as convenient as voice tagging, users are not likely to make the
time investment required to assign metadata to
files.''~\cite{10.1145/642611.642682}  Thus, while it might be useful to allow
human driven tagging, it is unrealistic to consider it to be generally useful.

\tm{Layered}

In my own past work, we implemented a layered log-structured technique within a
single file that permitted us to add additional meta-data to files.  For
example, we were able to add NTFS features to FAT32 files on Windows, with a
file system filter driver responsible for interpreting the additional meta-data
and presenting it to the native operating system.

\tm{Graph}

\tm{360}

\tm{DelveFS}

Key take-away here (for me) was the use of pub/sub systems to provide
notification.  This dovetails nicely into the strong dynamic nature of naming
systems, since they have to be responsive.

\section{Linguistics}
\label{ch:background:sec:linguistics}

\tm{This is where I explain how linguistics describes semantics and pragmatics
    and tie them together with the relevant portions of my thesis.
}

\section{Human-Computer Interface (HCI)}

There is a rich body of literature about various ways of presenting file
information to human users.  Various approaches to this problem have been
considered, including: temporal organization, virtual directories, attribute/tag
organization, and even a table-top model~\cite{collins2007tabletop}.  The
literature seeems to be rather clear on the limitations of the hierarchical file
system for human usage.  Intriguingly, HCI researchers have argued that the
evolution of file systems up to the point of creating hiearchical name spaces
with links with the evolution of databases and to expand their abilities the
\emph{file browser} should evolve to support dynamic interaction on par with
dynamic data interaction in a relational database
works~\cite{marsden2003improving}. This body of prior work has identified that
one key challenge is the sheer
magnitude of how much data is being presented.  Thus, it is not useful: ``If we
were to visualoise all files in this list, as in the email client, the list
would be too long to afford rapid scanning...''

My reading of this prior work is that we need to support more dynamic naming and
naming support, the ability to use transducers to add key meta-data based upon
content and the ability to allow humans to add tags.

At the same time, the importance of hierarchical organization is increasingly
less useful.  While search is useful, particularly when combined with filters
(e.g., faceted search) it also is insufficient to the task.  The associative
data model, such as used in the table-top models~\cite{collins2007tabletop}, is
surprisingly powerful for collaborative work.  No single mechanism satisfies all
these needs; a robust system must enable multiple usage modalities.

\section{Challenges}
\label{ch:background:sec:challenges}

\tm{My concern with this section is that it may be premature --- until I
    introduce the model, identify the research questions to be explored, and the
    artifact(s) I propose to build to explore those research questions this section
    may be ``too soon''.
}


\section{Related Work}
\tm{Note: this text was moved from ``related work'' and needs to be incorporated into this section.}

Ways to organize data within a storage silo have been extensively
studied from multiple perspectives.  That there are so many different approaches
to evaluating the optimal way to organize things is a testament to both the
importance and complexity involved in approaching this topic.  This section will
consider the following related work:
\begin{itemize}
    \item The \textit{storage} perspective, which is primarily rooted in the
          systems perspective.  Storage in this context includes file systems and
          databases, each of which then has multiple different manifestations.

    \item The \textit{provenance} perspective, which is related to systems but
          focuses on creating a context of explainability that is distinct from
          storage.

    \item The \textit{human-computer interface} (HCI) perspective, which is related to
          how humans organize and find information within storage systems. There are a
          number of distinct perspectives to this work including: hieararchical
          data organization, personal information management, enhanced search, and
          cognitive data organization.
\end{itemize}

Each of these perspectives is complementary and assist in better understanding
the problem: systems focuses on being able to efficiently and reliably store and
recover information though often it is agnostic to the specifics of the
information; provenance focuses on accountability and explainability; and HCI
focuses on the human facing problems that need to be solved and tends to ignore
how those solutions are implemented.

\cite{mashwani2019360,9229638,vef2020delvefs,dourish2003the,harrison1996re,barreau1995finding,dourish1999getting,placeless-tois,giffordSFS,plan9,inversion,smartstore,tagfs,gfs,provsearch,uprove2,pindex,page1999pagerank,nfs,metastorage,howard1988scale,afs,Adya:2003:Farsite,unix,scfs,federatedMetaData,federatedACL,benet2014ipfs,guo2012burrito,mazurek2014toward,li2013horus,adya2002farsite,provprimer,camflow}

\nocite{*}

\section{Storage}
\label{ch:related-work:sec:storage}

Much of storage work is dominated by the realities of how media and networks
function, with a goal towards increasing both capacity and performance.  The
basic model of data organization within storage systems tends to separate into
\emph{structured} data --- typically what is found in databases --- and
\emph{unstrutured} data --- typically what is found in file systems or object
stores.

\tm{Describe structured data: why do people use databases, what are the
    strengths and weaknesses.  How does this relate to data organiztion?
}

\tm{Describe unstructured data: why do people use file systems or object stores,
    what are the strengths and weaknesses?  How does this relate to data
    organization?
}

\tm{Describe the work that has been done in semantic file systems, metadata
    augmentation, and non-hierarchical organization structure (e.g., Ground and
    Placeless).
}

\tm{This is a fairly large quotation from the Placeless Documents Project
    Archive that should probably be edited down, but for the moment I want to leave
    it here because I think the insight provides is quite useful.
}

From the Placeless Documents Project
Archive~\footnote{https://web.archive.org/web/20020210170621/http://www.parc.xerox.com/csl/projects/placeless/}:

\begin{quotation}

    Placeless Documents are documents that are organized and managed according
    to their properties, rather than according to their location. Document
    properties can be things you already know about your documents, like that
    they're published, or notes, or about the budget,or drafts, or source code,
    or important, or shared with your colleagues, or from your manager, or big,
    or from the Web, or... whatever suits you. Document properties can also be
    things that you want to be true about your documents, like that they are
    backed up, or replicated on your laptop, or can be purchased for a small
    fee. These latter properties carry the code to implement or interface with
    the desired functionality.  Document properties are statements about your
    documents that make sense to you, and affect what you're going to do with
    the documents.

    What does it mean?

    We live and work in an information-filled world. The project focuses on helping people cope with the large and diverse information spaces that are part of life in the networked world. Our approach is to provide information consumers with a unifying model for organizing and manipulating their information.

    Much of the information that we commonly use is in the form of documents, both physical and electronic. In fact, the use of electronic documents on our computer desktops is pervasive, and even though they unify much of desktop computing, the document metaphor falls short of providing an accessible and readily understood way to interact with all forms of information, whether electronic or physical. Instead, we resort to specialized applications for much of our computer-based work. Electronic documents are managed through different systems like mail, WWW, and file systems. Similarly, the many devices we use in the course of our work (including fax machines, scanners, televisions, VCRs, telephones) manipulate and store information, yet they do not integrate seamlessly with the rest of our on-line information.

    Furthermore, the means available for individuals or groups to organize and customize their information spaces are extremely poor and driven mostly by storage and distribution models, not user needs. The most common model for information organization is hierarchical. The use of folders as a fundamental organizing principle, and the restriction that documents (mail messages, files, URLs, etc.) appear in only one folder at a time, force users to create strict categorizations, resulting in inflexible organizations that tend to persist over time even though their needs evolve. Similarly, customization of the organization and the behavior of information according to individual requirements is cumbersome, if not impossible. Access to a shared document that one individual deems as corresponding to the budget and another project cannot be easily tailored for both specialized functional requirements are equally hard to achieve. If a user can express that a document is read-only, and that it is a Word document, why cant heexpress that updated copies should be faxed to a colleague once a week?

    This project is about removing these hurdles by using a novel infrastructure and proving its benefits through applications that exploit its capabilities. We plan not only to change the way people interact with their currently segregated world of documents, but we plan to exploit a single concept the document and its properties to allow users to interact with arbitrary information.

    Our vision is one of customizable, context-aware management of integrated information spaces, which:

    \begin{itemize}

        \item integrate information components from many sources: repositories (WWW, mail, file systems), devices (scanners, video-cameras, television, phone), and dynamic processes (workflow, source code management systems, search engines, and dynamic document content),
        \item allow customizable organization of the information based on properties of that information, e.g., budget related, read at home, shared with John, and From: petersen@parc.xerox.com,
        \item allow information properties to be arbitrary objects specified through many different mechanisms: explicitly by the users themselves, captured by physical context sensors, inferred from usage, automatically generated by content analysis, etc.
        \item allow information properties to be active and carry behaviors to automate information work, enabling functionality like fax to John at 5pm each day, translate to English, notarized, backed-up in Utah for safety, etc.
        \item scale to sizes anywhere between an individual and the enterprise,
        \item are available at all locations required by the users, and
        \item protect the privacy and intellectual property of users.

              In this world the focus is on information, customization, and functionality that
              extends beyond the abilities of monolithic applications. Essentially,
              information carries the behaviors and semantics needed to operate on it.
              Information is independent of location and becomes responsive to the
              environments it is used in and the contexts of individual users, and it is
              managed independently by both its consumers and providers.
    \end{itemize}
\end{quotation}

\tm{The big hole that I can see is their comment that they want to prove ``its
    benefits through applications that exploit its capabilities.''  The
}

\section{Provenance}
\label{ch:related-work:sec:provenance}

\tm{Explain provenance.  It is a more recent area of study, but it also captures
    more of the kinds of information that will be useful to me.
}

\section{Human-Computer Interface}
\label{ch:related-work:sec:hci}

The traditional primary organizational model provided to users has been the
file/folder metaphor\tm{Need citation to 1965 Daley and the 1956 storage
    papers}, which was itself derived from physical filing cabinets.  However,
electronic storage is not bound by the same restrictions as a physical filing
cabinet, as can be seen even in early work \tm{Again, this is Daley} where the
model added \emph{links}; the closest equivalent to this in filing cabinets
would be to make duplicate copies of a document --- I have seen exactly this
routinely practiced by bookkeeping and accounting professional, whether using
physical or electronic filing systems.  This works because the objects are
generally \emph{immutable}, but for electronic storage systems without
deduplication it is not particularly space efficient.

The Human-Computer Interface community has been studying human-centered data
organization models for decades.  For example, the HCI community observed that
hierarchical file structure is challenging for users with low spatial
abilities~\cite{vicente1988accommodating}. This suggests why storage developers
would not even see there is a problem here: the study of computer science
correlates well with the development of spatial
abilities~\cite{parkinson2018spatial}. In my own discussions with even senior
computer scientists it is the introduction of \emph{silos} that seems to make
the hierarchical abstraction break. ``Did I store that in Dropbox, or Google
Drive, or was it on my laptop or my desktop computer?''

The wealth of research here is astonishing, yet does not appear to influence the
design of storage systems to exploit the results of that research.  Indeed, the
systems community seems to be focused on a \emph{search based} solution while
the HCI community research suggests that search is \emph{not} preferred by users
--- instead, users want to \emph{navigate}:

\begin{quotation}
    \emph{When retrieving a file the user needs to choose between folder
        based navigation and query based search. There are obvious
        intuitive advantages of search for both retrieval and organization.
        Search seems to be more flexible and efficient for
        retrieval. It is flexible because it does not depend on users
        remembering the correct storage location; instead, in their
        query users can specify any file attribute they happen to
        remember...}

    \emph{In fact, regardless of search engine quality,
        people consistently use search only as a ‘last resort’ for that
        minority of cases where they cannot remember file
        locations...}~\cite{bergman2019search}

\end{quotation}

Thus, from an HCI perspective it would seem that one potential research
direction would be to consider potential interfaces that mimic navigation over a
search interface.

For example, recent work around data curation explores the idea of a ``data
dashboard'' since the first step of curation is finding specific data to
curate~\cite{Vitale_2020}. This work builds upon prior research showing that
when presenting data to users it is important to ignore storage silo boundaries.
Indeed, my reading of this work is that it presents what seems to be navigation
even though it is implemented using a search mechanism.  Equally important, this
work also points to the benefits of not changing the \emph{location} of data,
but rather allowing construction of useful relationships via metadata. Thus,
this work supports my ideas of ignoring silo boundaries and providing metadata
for use by a similar tool.  What it does not explore is a way for data storage
to provide enhanced metadata, dynamic update, and notification of changes, which
are important elements to making such a dashboard more useful for data
visualization.




\endinput

\reto{I feel like a lot of Challenges goes here... }

\begin{epigraph}
    \textit{Paradigm paralysis refers to the refusal or inability to think or see
        outside or beyond the current framework or way of thinking or seeing or
        perceiving things.  Paradigm paralysis is often used to indicate a general
        lack of cognitive flexibility and adaptability of thinking.} --- The Oxford
    Review Encyclopedia of Terms (2021).
\end{epigraph}

Computer storage systems continue to evolve and change at an increasing rate.
The difference between file systems, which provide the basic abstraction of
unstructured data, and database systems, which provide the basic abstraction of
structured data, is now filled with a growing array of \emph{semi-structured}
mechanisms including: key-value stores, object stores, document stores, no-SQL
databases, data warehouses, and data lakes.

Each new mechanism invented for storing data creates a new ``silo'' of
that specific storage system.  Each new storage system comes with its own
semantics and meta-data, and seldom with any explicit way of tying related
information together across such silos. Storage silos are often designed with
specific usage patterns in mind.  For example:

\reto{I'd go with categories here instead of products. as there are often
    multiple different solutions within the same category.
}

\begin{description}
    \item[\ac{HDFS}] --- Hadoop was created to solve the problem of large,
        dynamically written write-once files that are typical of machine data
        captured from multiple sources for additional analysis.

    \item[Google Drive] --- Google uses a common storage mechanism for its various
        services including e-mail, documents, spreadsheets, photos, and videos. It
        allows collaborate work, both by sharing access to the object as well as
        providing tools for simultaneous editing between human collaborators.

    \item[Intel DAOS]
        --- Intel's solution for large parallel \ac{HPC} storage
        needs that splits meta-data and data, with meta-data stored in persistent
        memory and data stored on \ac{NVMe} devices~\footnote{\url{https://www.intel.com/content/www/us/en/high-performance-computing/daos-high-performance-storage-brief.html}}.

    \item[Qumulo] --- Seattle-based start-up company that provides a large data
        storage management product that is used in specialized big-data industries.
        For example, Qumulo combines both local and cloud storage to provide video
        post-production work in which file sizes are quite large due to the size of
        high resolution video files. They claim to support secure storage of
        petabytes of file data across multiple tiers of storage.
\end{description}

\tm{I'm not sure these examples are specific enough.}

Each of these storage products attempts to address specific storage needs; each
new product creates a new storage silo.  A human user that can limit themselves
to using a single storage silo can exploit the functionality of that specific
silo and simplifies the challenging task of finding disparate but related
content spread across storage silos.

The proliferation of storage silos exacerbates the easy ability of humans to
navigate their own storage to find a specific file.  One of our own \ac{HCI}
researchers told me that they found their ability to find things on their own
computer broke when they added multiple distinct hard disks~\footnote{Private
    conversation with Joanna McGrenere at UBC CS-50 reception.} The challenge of
finding a specific file or set of files is one that consistently resonates with
many of those with whom I have discussed my research, both inside and outside
the computer science field.

The most commonly presented model of naming files (or objects) within a storage
silo is the hierarchical namespace.  The hierarchical namespace confounds the
storage \emph{location} with the information \emph{relationship}.  Note that
location here is not relative to the logical namespace of the storage silo, not
the physical storage device.

\reto{the challenges section provides quite a broad overview of how things are right now.
    I'm not sure if those are the challenges you will be facing with your thesis, or whether these are general challenges.
    Given your thesis statement,
    shouldn't there be something like a problem statement (or challenges that makes the thesis statement hard)
    I'd recommend to identify say 3-5 challenges and try to express them in one sentence each.
    each sentence then is the \subsection{} heading. where there is maybe a paragraph or two worth of information backing up the challenge/problem.
    ideally, your wock packages will correspond to one challenge/probleme each.
    (maybe those 3-5 challenges are in fact the research questions you plan to answer)
}

Ordinary applications familiar with using hierarchical file systems expect
related objects, of whatever type, to be in or close to the same directory.  The
hierarchical file system model has been highly successful for more than a
half-century, though that success is from the perspective of the storage
community.  The \ac{HCI} community has been pointing out that this model is
deficient for many users.

Storage silos are not a new invention.  The problem of cross-silo management,
including naming is not a new problem.  UNIX addressed multiple silos using
``mount points''.  A \emph{mount point} is a location in an hierarchical name
space where another namespace is logically connected to the existing namespace.
For example, UNIX mounts a new file system instance on top of an existing
directory.  On the Linux system where I am writing this document there are 33
distinct namespaces mounted on top of the base file system namespace --- the
``root'' namespace that starts at "/".  This model has served well as evidenced
by the fact that all mainstream consumer operating systems at present (Linux,
MacOS X, and Windows) support mount points.

NFS on UNIX is implemented using explicit mount points, much like media file
system instances while AFS used a global namespace mechanism to connect its
silos (``volumes'') together so that users were given a single consistent
namespace. Converting the hierarchical name of internet services has been used
to create internet-based namespaces, such as using \ac{WebDAV}, which converts
the internet \texttt{GET/PUT} model into an
hierarchical file systems namespace model.  Amazon's AWS S3 service provides an
object store model that is accessed using HTTP GET/PUT operations as well. This
can be accessed using the hierarchical storage model using any one of the
numerous "S3" FUSE file system implementations on \url{github.com}.  These FUSE
file system implementations of S3 exist not because they are fast or efficient
but because supporting the hierarchical model understood by existing
applications makes them \emph{usable} by existing applications.

The hierarchical name space is definitely powerful: it is a specific
construction of individual name space silos that are composed into a single
uniform namespace. However, the benefits of this single namespace have always
been limited: they tend to follow the \emph{computer} and not the \emph{user}.
Plan 9 was one of the few systems that focused on personal name spaces,
introducing key ideas such as context defined names --- such as the platform
architecture of the computer on which a program is currently running and
\emph{unions} in which multiple namespaces are not joined via a mount point but
rather via a merge of multiple namespaces mounted at the same location.
Xerox's Placeless research project was a system where documents ``are organized
and managed according to their properties, rather than according to their
location.''~\footnote{https://web.archive.org/web/20020210170621/http://www.parc.xerox.com/csl/projects/placeless/}

In the past two decades the complex multi-silo storage environment has exploded:
\begin{description}
    \item[WebDAV] --- the ability to map internet web servers into the
        hierarchical file system;
    \item[Dropbox] --- maintaining a mapping between a public internet remote
        storage solution and a portion of the local hierarchical namespace on
        existing storage across multiple devices;

\end{description}


\tm{This should come from the existing text.  It should explain \textit{how} we
    got to this point.
}

\section{Linguistics}
\label{ch:background:sec:linguistics}

\tm{An interesting way to look at this is from the \emph{linguistic}
    perspective.  Much of the prior work relates to semantics but in fact what I'm
    suggesting is that we also consider \emph{pragmatics}. Wikipedia defines this
    as: ``In linguistics and related fields, pragmatics is the study of how context
    contributes to meaning. Pragmatics encompasses phenomena including implicature,
    speech acts, relevance and conversation. Theories of pragmatics go
    hand-in-hand with theories of semantics, which studies aspects of meaning which
    are grammatically or lexically encoded. The ability to understand another
    speaker's intended meaning is called pragmatic competence. Pragmatics
    emerged as its own subfield in the 1950s after the pioneering work of J.L.
    Austin and Paul Grice.''  This actually seems to capture some of what we are
    trying to do quite well: semantics have been studied in terms of computer
    storage, but not pragmatic linguistics.
}

\begin{quotation}
    Pragmatics is a field of linguistics concerned with what a speaker implies
    and a listener infers based on contributing factors like the situational
    context, the individuals’ mental states, the preceding dialogue, and other
    elements.
\end{quotation}

\url{https://www.masterclass.com/articles/pragmatics-in-linguistics-guide}

\tm{Important take-away here is that Pragmatics is distinct from Semantics
    because it relates to how context affects the meaning of language.  This is
    actually quite close to what we are trying to do: to capture \emph{context} so
    we understand the meaning of the language we are using to describe things
    relative to that context.  However, what we are doing isn't a good mesh with
    pragmatics or computational pragmatics (yes, it exists) because that seems
    more focused on the question of using computational mechanisms for
    understanding pragmatics.
    \url{https://www.oxfordbibliographies.com/view/document/obo-9780199772810/obo-9780199772810-0264.xml}
    is interesting in that it really hits home on the dynamic nature of meaning.
}

\begin{quotation}
    Pragmatics as a branch of linguistics can be characterized as the study of
    the relations between linguistic properties of utterances on the one hand,
    and aspects of the context in which a given utterance is used on the other.
    Computational pragmatics is pragmatics with computational means, which
    include models of dialogue management processes, collections of language use
    data, annotation schemes and standards, software tools for corpus creation,
    annotation and exploration, process models of language generation and
    interpretation, context representations, and inference methods for
    context-dependent utterance generation and interpretation processes. The
    linguistic side of the relations that are studied in pragmatics is formed
    primarily by utterances in a conversation or sentences in a written text. In
    the case of written text the context side consists of the surrounding text
    and the setting in which the text is meant to function. In spoken or
    multimodal dialogue, the context of an utterance is formed by what has been
    said before and the interactive setting, but additionally by other
    perceptual, social, and mutual epistemic information (see Context Modeling).
    Much of this information is dynamic, as it changes during a dialogue and,
    more importantly, as a result of the dialogue, since the participants in a
    conversation influence each other’s state of information when they
    understand each other. Dialogue contexts are thus updated continuously as an
    effect of communication. Central to computational pragmatics is the
    development and use of computational tools and models for studying the
    relations between utterances and their context of use. Essential for
    understanding these relations are the use of inference and the description
    of language in terms of actions that are inspired by the context and that
    are intended to change the context. This bibliography therefore focuses on
    publications concerned with the computational modeling of dialogue in terms
    of communicative actions including the use of inference for utterance
    interpretation. It also considers the more static analysis of discourse
    coherence and semantic relations in text, and concludes with references to
    recent activities concerning the construction and use of resources in
    computational pragmatics, in particular annotation schemes, annotated
    corpora, and tools for corpus construction and use. The popularity of
    probabilistic approaches to natural language processing can also be seen in
    studies of pragmatic aspects of language use, although these approaches are
    so far not as important as in some other areas of language processing. The
    so-called rational speech acts (RSA) model treats language use as a
    recursive process in which probabilistic speaker and listener agents reason
    about each other’s intentions to enrich the literal semantics of their
    language along broadly Gricean lines. The core references for this approach
    are also included in this biography under Inference in Language Processing.
\end{quotation}

\tm{Good stuff here, which resonates with what I've been talking about.  Context
    isn't static, though semantics are static.
}



Computer storage focuses on \emph{content}, which is one form of
\emph{identity} such as is found in ``content addressing'' where the specific
identity of a file is described using a hash value computed from the content of
an object.  Some storage systems provide the ability to encode a limited amount
of information about relationships and properties.  \emph{Relationships} in
file systems are typically expressed using directories (or folders) and
\emph{context} is captured via a name.\reto{does it make sense to talk about an ever growing pile of files?
    somehow I feel like this is where the systems aspect comes into place:
    - humans have  a limited working set.
    - crawling through tons of files across different silos is impractical.
    - search is fuzzy and may not provide  the right results, nor work cross silo.
}
\reto{context is another thing. also: I wouldn't say that context is the
    name. }\tm{I disagree with Reto on this point: when we have hierarchical
    names, we embed the context in the hierarchical name \emph{because} that's
    the only way we have of capturing it.  Indeed, as Reto pointed out earlier,
    names have meaning within a particular context.  So... where do we
    \emph{get} this context.}


\reto{does it make sense to talk about an ever growing pile of files?
    somehow I feel like this is where the systems aspect comes into place:
    - humans have  a limited working set.
    - crawling through tons of files across different silos is impractical.
    - search is fuzzy and may not provide  the right results, nor work cross silo.
}\tm{I had text in at some point about the growing size and the problems
    with it.  The proliferation of silos is yet another issue, such as what is
    happening with in-memory compute, each of which ends up looking like its own
    silo.
}

The goal of making computer storage more flexible is not a new one.  Memex was
first proposed in 1945 by describing how computers might help human users by
serving as ``augmented memory''~\cite{bush1945we}.\reto{what does it do?}  Computer storage has evolved
dramatically in the 76 years since yet our storage systems have failed to
progress towards, let alone realize Memex's associative relationship
model.\reto{maybe say what the assoc relation ship model actually provides....}

\reto{mention the "search" and "lookup" of the right files again here? it appears in the abstract, but maybe could be a little bit expanded here?
    distinction between mete-data / content}
\tm{In fact, my original goal was to \emph{lead} with the thesis statement and
    then build on that.  Then I tried to add a small body of text to introduce the
    topic.  If that body of text is going to grow, I should move all of that after
    the thesis statement.}

\section{Naming}
\label{ch:background:sec:naming}
%\section{Why naming is a Computer Systems problem}
%\label{ch:introduction:sec:systems-problem}


\MIS{This feels like a classic related work paragraph, but I don't know where it
    fits in your story line.
}
Media file systems have long been organized using a simple internal key-value
store.  The BSD UNIX Fast File System used an ``index node'' (inode) as a
description of a data object, such as a file or directory~\cite{mckusick1984a}.  These nodes were
indexed using an identifier, which acts as a simple key.  The NTFS file system
is structured similarly~\cite{custer1994inside}.  Recent work explored the idea
of separating the namespace from storage with an emphasis on high performance
solid-state storage (SSD) devices~\cite{koo2021modernizing}.  This echos earlier
work suggesting using object storage devices (OSD) for file systems
implementations~\cite{seltzer2009hierarchical}.

Beyond separating naming from storage, there is also a trend to separate
meta-data from storage as well.  One early example of this is
Lustre~\cite{braam2019lustre}, with more recent work including Ceph and
Gluster~\cite{noronha2008imca,weil2006ceph}.  Each of these distributed file
systems separates the service of data from meta-data, which allows data paths to
focus on performance, including the use of parallel data paths.  Meta-data does
not benefit from high-performance parallel I/O paths and instead benefits from
low latency random access. This separation does not improve naming support
because that is not the goal of any of these systems.

Intel's DAOS system uses Intel DC Persistent Memory for meta-data and \ac{NVMe}
for data. DAOS supports two naming models: one is a POSIX-compatible
hierarchical name space, and the other is a key-value store interface. Storage
pools are accessed using one or the other of these interfaces but not both.

In a hierarchical name space the fully qualified path corresponds to the
\emph{logical} location of the file, that is the location of this file in the
namespace itself. In most file systems that logical location corresponds to a
specific file system instance and thus defines the storage device(s) on which
that file is stored.  Meta-data associated with that file is interpreted by the
file system to determine how to retrieve the data.  For a media file system that
usually corresponds to the address of the location(s) of the file data on the
corresponding media.  For a network file system that usually corresponds to the
network address used to request the correct file data from the computer storage
system where the data is actually stored.

Separating the file system meta-data from the file data is known to be
beneficial.~\cite{kawai2011a}  This approach logically makes sense when
considered in the multi-silo context as well, since a human user looking for
something actually does not care \textit{where} that the
thing is located they care \emph{what} the thing is.

This raises an intriguing question, one that is at the heart of my own research:
\emph{what is the purpose of naming?}  Applications are quite happy to use names
that are meaningless to humans, such as content hashes, UUIDs, or randomly
generated string names, such as are used for temporary application files.
Indeed, if an application knows that a name has a fixed format, the knowledge of
that fixed format can be used to simplify the implementation or to separate
application named files from other files.

Humans use file names to provide \emph{meaning} as to what is in the named
object.  Librarians assisting people in organizing data routinely suggest
embedding contextual information in the file names; directories are then used to
create additional organizational structure (or ``context''). There are numerous
limitations to this approach, however, because humans do not organize things in
the same fashion and even the same human does not organize things the same over
time. This is such a common phenomenon even the popular press pokes fun at it
(\autoref{fig:xkcd:1459}.)

The inability to find specific things is \textit{not} unique to computer data storage.
Indeed the current situation for organizing digital data seems to be descended
from the organizational structure of a library or a file cabinet, despite the
physical limitations of organizing paper files that are not constraints for
computer storage.  For example, when I am organizing my long play record
collection I have to chose one primary ``key'' to use, such as the time when the
album was released.  However, digital storage does not have this same
restriction - I can easily ask that they be presented to me by genre, artist,
album title, or even instrument type. Further, I can change my organizational
scheme dynamically.  Both Bell Lab's Plan 9~\cite{plan9} and Xerox Placeless
Documents~\cite{dourish1999getting}
observed that there is a dynamic nature to how information is organized.  The
hierarchical name space is basically static, like my album collection is static:
I can choose a different organizational scheme but to do so I have to physically
move them around.

Indeed, the system we have evolved works \emph{in spite} of the obvious
artificial limitations imposed by decades old decisions~\cite{dourish2003the}.  Rather than being
a deep insight that the current system works because it is the best of all
possible models, it is a testament to human ingenuity and a tolerance for
primitive, sub-standard systems.

There are a body of recommendations about file naming
standards\footnote{Data Management for
    Researchers~\cite{briney2015data}}
\footnote{Smithsonian: \url{https://library.si.edu/sites/default/files/tutorial/pdf/filenamingorganizing20180227.pdf}}
\footnote{Stanford: \url{https://library.stanford.edu/research/data-management-services/data-best-practices/best-practices-file-naming}}
\footnote{NIST: \url{https://www.nist.gov/system/files/documents/pml/wmd/labmetrology/ElectronicFileOrganizationTips-2016-03.pdf}}.
Harvard Data Management suggests~\footnote{\url{https://datamanagement.hms.harvard.edu/collect/file-naming-conventions}}:

\begin{itemize}
    \item Think about your files
    \item Identify metadata (e.g., date, sample, experiment)
    \item Abbreviate or encode metadata
    \item Use versioning
    \item Think about how you will search for your files
    \item Deliberately separate metadata elements
    \item Write down your naming conventions
\end{itemize}

Thus, there are common themes: a name provides context for \emph{what} the given
thing represents. Uniformity of information is also important --- the ``naming
convention'' permits not only identifying similarity but key elements of
\emph{difference} between any two named things.  Thus, to return to the original
question: ``what is the purpose of naming?''

\tm{It seems that this Harvard list is a serious indictment of the existing
    system: it pushes the cognitive load onto the users, talks about meta-data ,
    versioning, encoding, \emph{and} capturing the naming convention.}

It seems clear that human naming is
not the same as data storage naming, though we often treat them as the
same.  This can be seen in the data storage tendency to either use names that
are entirely about locating a specific object, such as is the case with a
key-value store, for example, or it combines location with identity, such as is
typified by the Uniform Resource Identifier (URI)~\cite{berners-lee1998uniform}.

Thus, it would seem that naming is about:

\begin{description}
    \item[Identity] --- the name of a thing should be sufficient to verify that
        it is the specific thing we seek.  An example of this is biometric data for
        humans or a cryptographic hash for a storage object;
    \item[Location] --- the name of a thing could provide information that
        directly or indirectly specifies where it is located.  A human example is
        how an identity card might specify a current residential address.  A data
        storage example is the U.S. Library of Congress Classification System, which
        can be used to find a particular work within very large bodies of work even
        though books are typically not thought of as a storage media;
    \item[Relationships]  --- the concept that one thing can be derived from
        another, whether in a direct form, such as a version of the same logical
        thing, or in a more indirect form, such as a provenance relationship showing
        that one thing was derived from another thing by applying some set of
        transformations to it;
    \item[Characteristics] --- this relates to something that is a property of
        the thing.  For humans this might be the date of birth (``creation date''),
        height, weight, eye color, etc.  For storage objects this might include
        timestamps, size, data type;
    \item[Context] --- at some some level, we often rely upon names to provide
        us with \emph{context}.  For humans, we assume that people with the same
        family name are likely to be members of the same family, even though
        ``\persa Smith'' is distinguished from ``\persb Smith''.
\end{description}

When considered from this perspective, it seems clear why ``naming is hard'' ---
it plays multiple distinct roles, sometimes overlapping, sometimes interfering.
These challenges are not well-served by existing naming support in computer
storage systems.

\tm{I feel that I should capture the dynamic nature of naming, which really does
    differ from the traditional static model of it.
}

\reto{Overall, I feel like the challenges are quite an overview of related work. I would try to focus on what challenges/problems arise when you try to desgin and implement your system?
    Challenge a) Multiple silos with different characteristics; and Challenge b) capturing activities, ... (or is that gone)? }

%\chapter{Background}
\label{ch:background}

\tm{Please note that this is a \textbf{work in progress} and I am providing this
    draft with this section partially written because I seek feedback on the overall
    structure and model that I present.  While this section will change in
    subsequent drafts, I do not expect its content to materially change the model
    that I present in \autoref{ch:model}.}

The prior work of Saltzer and Watson establish that the purpose of file systems
is to serve the naming needs of
\emph{users}~\cite{Saltzer1978,watson1981identifiers}.   However, the background
literature on file systems can be broadly broken up into two categories:

\begin{enumerate}
    \item \textbf{Storage Management} --- much of the prior work related to file
          systems focuses on the management of file data.  For physical media file
          systems --- those file systems that manage some sort of media, whether it is
          magnetic, optical, or solid state --- the concerns are about efficient use
          of the media itself.  For distributed (``network'') file systems, much of
          the prior work focuses on efficient protocols for providing access to file
          system data over a network.

    \item \textbf{Naming} --- a small amount of the prior work related to file
          systems focuses on the organization of naming.  There is some overlap
          between early storage management and naming literature.  Similarly,
          distributed file systems had to consider naming as well.

\end{enumerate}

I review this prior work in the subsequent sections
(\autoref{ch:background:sec:storage} and \autoref{ch:background:sec:naming}) and
then analyze these in the context of \system.  I then provide a brief review of
relevant linguistics literature pertinent to my thesis by reviewing the
linguistic field of \emph{semantics} and \emph{pragmatics}
(\autoref{ch:background:sec:linguistics}). Finally, I look at
the challenges that are not addressed in prior work that are necessary for me to
evaluate my thesis (\autoref{ch:background:sec:challenges}).

\section{Storage Management}
\label{ch:background:sec:storage}

Most of the research literature about media file systems involves management of how
data should be organized on the given media. While not central to my own thesis,
it is useful to understand that file systems research continues to be an active
and ongoing research area, but it is \emph{not} related to naming, and that work
really has little direct bearing on the naming aspects of file systems.

\tm{Multics}
\tm{TENEX}
\tm{BSD FFS}
\tm{Cedar}
\tm{LFS}
\tm{}


One of the earliest papers on file systems, ERMA,
The Multics file system was described by Daley in 1965~\cite{daley1965general},
and while it clearly lays out the hierarchical name space, it also provides
insight into very early file systems development, with important considerations
like parity checking, overflow sending, and the "printer bit and column
counter".

By 1965 media file systems development


\section{Naming}
\label{ch:background:sec:naming}

% \tm{1945 - Memex}

In 1945 Vannevar Bush described the challenges to humans of finding things in a
codified system of records, including those used by early computers:

\begin{quotation}
    Our ineptitude in getting at the record is largely
    caused by the artificiality of systems of indexing. When data of any
    sort are placed in storage, they are filed alphabetically or numerically, and
    information is found (when it is) by tracing it down from subclass to
    subclass. It can be in only one place, unless duplicates are used; one
    has to have rules as to which path will locate it, and the rules are
    cumbersome. Having found one item, moreover, one has to emerge from the
    system and re-enter on a new path.

    The human mind does not work that way. It operates by association. With one
    item in its grasp, it snaps instantly to the next that is suggested by the
    association of thoughts, in accordance with some intricate web of trails
    carried by the cells of the brain. It has other characteristics, of course;
    trails that are not frequently followed are prone to fade, items are not
    fully permanent, memory is transitory. Yet the speed of action, the
    intricacy of trails, the detail of mental pictures, is awe-inspiring beyond
    all else in nature.
\end{quotation}

This is as true in 2021 as it was in 1945.  While preparing this proposal I
spent time looking at the guides many libraries provided about the naming of
files. I found a body of recommendations about file naming
standards\footnote{Data Management for
    Researchers~\cite{briney2015data}}
\footnote{Smithsonian: \url{https://library.si.edu/sites/default/files/tutorial/pdf/filenamingorganizing20180227.pdf}}
\footnote{Stanford: \url{https://library.stanford.edu/research/data-management-services/data-best-practices/best-practices-file-naming}}
\footnote{NIST: \url{https://www.nist.gov/system/files/documents/pml/wmd/labmetrology/ElectronicFileOrganizationTips-2016-03.pdf}}.
Harvard Data Management suggests~\footnote{\url{https://datamanagement.hms.harvard.edu/collect/file-naming-conventions}}:

\begin{itemize}
    \item Think about your files
    \item Identify metadata (e.g., date, sample, experiment)
    \item Abbreviate or encode metadata
    \item Use versioning
    \item Think about how you will search for your files
    \item Deliberately separate metadata elements
    \item Write down your naming conventions
\end{itemize}

\tm{Note that the footnotes aren't rendering cleanly.  I will need to clean this up.}

Throughout these examples there are common themes: a name provides context for
\emph{what} the given object represents. Uniformity of information is also
important --- the ``naming convention'' permits not only identifying similarity
but key elements of \emph{difference} between any two named things.

\tm{It seems that this Harvard list is a serious indictment of the existing
    system: it pushes the cognitive load onto the users, talks about meta-data ,
    versioning, encoding, \emph{and} capturing the naming convention.}




Bush's observation here was about
the importance of developing storage systems that enabled human cognition by
forming an extended memory.  What is remarkable about this early work is how
well Bush captured the basic human need for such a storage system and how it
related to human cognition and associative thinking.  I note that we still do
not have any system that is as functional as Bush's Memex, though there are
aspects of the modern world wide web that are reminiscent of some of the
concepts he first described.

While I do not claim Bush's 1945 Atlantic article was file systems
research, it is useful to understand the perspective of work that
followed~\cite{bush1945we}.

% \tm{1956 - ERMA}

One of the earliest file system papers of which I am aware was
ERMA~\cite{barnard1958}.  This paper was more focused on documenting their
design process, but as part of this work they described the hierarchical
structure of files.  The diagrams clearly show the "folder/file" metaphor that
survives to this day.

\tm{1964 - A File Structure for the Complex, the Changing and the Indeterminate}

Nelson builds upon Bush's Memex description as he explores ``[t]he kinds of file
structures required if we are to use the computer for personal files and ad an
adjunct to creativity\ldots''  He points out the challenges inherent in the
task: ``They need to provide the capacity for intricate and idiosyncratic
arrangements, totally modifiable, undecided alternatives, and thorough internal
documentations.''

Nelson's work is usually cited as inspiration for the world wide web and its use
of \emph{hyperlinks} to associate related content together, yet this work was
primarily focused on how personal file information should be organized within a
computer system.  Key ideas here include the \emph{dynamic} nature of such
organization (``[i]t was also intended that the system would allow index
manipulations which we may call \underline{dynamci outlining} (or
\underline{dynamic indexing}).'')  Another important aspect of this work was his
repudiation of hierarchical organization: ``[n]o hierarchical file relations
were to be built in; the system would hold any shape imposed on
it.''~\cite{nelson19654}

\tm{1965 - Multics}

In 1965 the ``Fall Joint Computer Conference'' was dominated by papers about
General Electric's new Multics operating system.  One of those papers was
specific to the file system~\cite{daley1965general}.  From a naming perspective
the key observation was the use of the hierarchical file system structure,
clearly described and highly reminiscent of the hierarchical file system that
has been adopted first by UNIX and then enshrined in the POSIX interface.  One
observation I have with respect to this work is that while the initially propose
the hierarchical file system, the paper admits that this isn't sufficiently
descriptive and introduces the additional concept of a \emph{link}. This serves
as a good example of how the issues of storage efficiency (e.g., not wanting to
have duplicate copies of the file stored on disk) impacted naming (where the
fact that an object has multiple names need not be tied to the actual manner in
which this is implemented.)

\tm{This raises the question about mutability.  If you change the \emph{file} do
    you preserve the name or do you break it?  That's a really interesting question
    and I don't think anyone will be happy with pretty much any answer.  It's also
    the source of one of the real frustrations in modern computing, namely the use
    of shared libraries with the same \emph{name} but subtly different
    \emph{implementation}.  Hence we end up with \emph{libfoo.so} and
    \emph{libfoo.6.so} and \emph{libfoo.6.4.so}.  In Windows this became ``DLL
    hell'' and has lead to solutions like "side-by-side" installations of commonly
    used shared libraries.  There's a really difficult question lurking beneath this
    for naming: does the name specify something specific or something general?  This
    is the \emph{binding} problem that Saltzer refers to as well.
}

\tm{TENEX}

At the Fall Joint Computer Conference in 1972 Murphy presented a paper about the
storage organization of TENEX, including its naming scheme.  ``A powerful and versatile
directory and file naming facility is provided in which
a particular file is identified by a fixed-depth path which
includes device, directory name, file name, extension,
and version.''~\cite{murphy1972storage}

\tm{UNIX}

The 1974 Symposium on Operating Systems Principles included papers that
ultimately led to both of the primary ``families'' of operating systems that we
use today.  One of those introduced the UNIX operating system and that work
presented the model of ``everything is a file'' and the hierarchical name space
organization of those files~\cite{unix}.  Intriguingly, the need to add
additional contextual information was recognized even then: ``Besides the system
proper, the major programs available under UNIX are: ... and permuted index program.''

\tm{CAP}

The Cambridge Capability Protection operating system included a novel model for
using capabilities to protect files.  The namespace that CAP presented is
interesting in that it did consider non-hierarchical name spaces~\cite{needham1977cap}: ``By handling
directory capabilities in the same way as store capabilities we allow not only
hierarchical directories but also shared directories --- the structure can form
an arbitrary directed graph, which may even be cyclic.''  They also described
the concept of \emph{dynamic} directories: ``By allowing dynamic creation of
directories, we allow the manufacture of directories or of complex structures
of directories separate from the main filing system, accessible only to certain
programs (this is used for the password file, for example).''

\tm{NFS}

As we build communications networks to share information between computers the
idea of sharing files between computers quickly emerged.  Sun Microsystems'
Network File System (NFS) was not the first network file system but it is
certainly one of the most widely used and it continues to be used
today.~\cite{nfs}  It extended the hierarchical file system so that it
incorporated the name spaces of a remote computer system into the local file
system. While NFS provided subtly different behavior than a local file system,
it was ``close enough'' that in the absence of failures it ``just worked.''

\tm{AFS}

The Information Technology Center at Carnegie-Mellon University (CMU) was responsible
for the realization of a system of network attached computer workstations for
use in the CMU environment. Mahadev Satyanarayanan and his resarch group were
responsible for developing the Andrew File System (AFS), which was subsequently
commercialized by Transarc in the late 1990s.  AFS remains in use today.  AFS
was contemporaneous with NFS, but offered significantly different functionality,
including a \emph{global} name space, federated security (via MIT's Kerberos),
and a separate service that was responsible for resolving the global namespace
across various clients~\cite{howard1988scale}.

\tm{Universal Directory Service}

The concept of distributed name or directory services is one that is not new.
In 1985 Lantz provided a description of a ``Universal Directory Service'' that
is similar to some of the work that I anticipate while evaluating my own thesis.
This paper describes their services as offering the following
capabilities~\cite{10.1145/12481.12483}:
\begin{itemize}
    \item ``can span a heterogenous internetwork of existing naming domains;
    \item ``allows us to name, locate, and discover how to manipulate objects
          (including files, processes, mailboxes, people, and services);
    \item ``provides dynamic binding and context mechanisms; and
    \item ``can be integrated into most existing systems as a ``value-added''
          feature.''
\end{itemize}

While similar conceptually to the domain name service (DNS) that is used for
naming in the internet, it is considerably more general.  Intriguingly, while
there is subsequent work that cites to Lantz's work, I could
not find any later work that extended it.

\tm{Properties}
The idea of extended attribute information being associated with a file is one
that emerged in the 1980s.  Mogul's 1986 treatise ``Representing Information
About Files'' does an excellent job of summarizing the research up to that
point, including an explanation of the history regarding ``Leaf'' a protocol
that while developed at XEROX PARC for the Alto was only memorialized by him and
Brian Reid in 1981.  That appears to have been motivation for the creation of
what we now call ``extended attributes'', though Mogul refers to them as ``file
properties''~\cite{mogul1986representing}. Mogul's work provides considerable
insight into how file properties could be implemented, both by applications (as
part of the file itself) as well as within file system meta-data.  We actually
supported properties in Episode~\cite{chutani1992episode}.  An important argument Mogul
makes is that \emph{naming} should be separated from \emph{storage}:

\begin{quotation}

    The old monolithic system model usually implied a tight coupling between file system and directory
    system; in many cases the two are indistinguishable. While this approach might improve performance
    slightly, one of the recognized virtues of the client-server model is that it encourages separation of function.
    Separating file system and directory system into two distinct services (which might nevertheless interact as
    clients of each other) provides several benefits:
    \begin{itemize}
        \item \textbf{Modularity}: with attendant benefits of a cleaner service model, cleaner implementations, and
              the flexibility to substitute new implementations of one service without affecting the other.

              Modularity has additional value: it is easier to distribute specialized services than to build a
              generalized distributed system. For example, LOCUS uses its knowledge of the special
              semantics of directories to recover them after a partition; this cannot be done for files in
              general [Popek 81].

        \item \textbf{Crossing file system boundaries}: an integrated directory system can only store references to
              files in its associated file system. In a heterogeneous distributed system, we would like to
              construct a unified name space covering all file servers in the environment. Why should we
              prohibit users from storing in the same directory references to files stored by two distinct file
              servers? The Universal Directory System [Lantz 85] is an example of an approach to this
              question.

        \item \textbf{Non-file referents}: Directory systems can and should be used as general name-binding agents;
              they need not only refer to files. In fact, directories have been used to name non-file objects
              even in Unix (where devices appear in the file system name space). By separating directory
              from file service and giving non-file referents full citizenship in directory bindings, we can
              gain useful generality.
    \end{itemize}

    Not everyone agrees that directory service and file service should be separate; for example, the V-system
    takes the opposite point of view [Cheriton 84]. We prefer separation because it
    leads to a simpler model.

\end{quotation}

His definition of a file is useful as well: ``\emph{a \emph{file} is a named
    object that stores an arbitrary amount of data for an arbitrarily long time.}''

\begin{quotation}
    The Alto and WFS were research projects. Other groups within Xerox were working on commercial
    products, the Star professional workstation in particular. The original design for the Star file system [Dalal
            86] was tightly integrated with application-level functions. It was then realized that clearer separation of
    file service and application would yield a more open, flexible system, but the resulting file system retained
    some features of Star that were found to be generally useful. One of these features was support for
    extensible attributes, including attribute-based search.
\end{quotation}

My general sense remains that Mogul's work is the most closely related model of
the work I propose doing as part of this thesis.

\tm{Semantic}

Gifford introduced the concept of a \emph{semantic} file system at SOSP in
1991~\cite{gifford1991semantic}.  I reproduce the abstract from his paper to
provide the basic idea of his work.  While I \emph{started} from considering
semantic file systems as part of my PhD, my own work has moved beyond the model
of Gifford and is complementary to Gifford's own work.

\begin{quotation}
    A semantic file system is an information storage system that
    provides flexible associative access to the system's contents
    by automatically extracting attributes from files with file
    type specific transducers. Associative access is provided by a
    conservative extension to existing tree-structured file system
    protocols, and by protocols that are designed specifically for
    content based access. Compatibility with existing file system
    protocols is provided by introducing the concept of a
    virtual directory. Virtual directory names are interpreted as
    queries, and thus provide flexible associative access to files
    and directories in a manner compatible with existing software.
    Rapid attribute-based access to file system contents
    is implemented by automatic extraction and indexing of key
    properties of file system objects. The automatic indexing of
    files and directories is called "semantic" because user programmable
    transducers use information about the semantics
    of updated file system objects to extract the properties for
    indexing. Experimental results from a semantic file system
    implementation support the thesis that semantic file systems
    present a more effective storage abstraction than do traditional
    tree structured file systems for information sharing
    and command level programming.
\end{quotation}

A key point here is that semantics are defined based upon the \emph{content} of
the file itself. This is certainly useful but different than the model I
propose, which would introduce \emph{pragmatics}.  Pragmatics are defined based
upon the \emph{context} of how the given file is used.

Another insightul source of information about semantic file systems is from
Martin's PhD thesis~\cite{martin2008}.  He utilizes \emph{Formal Concept
    Analysis} as a mechanism for improving semantic file systems; his work continued
until a few years ago~\footnote{http://www.libferris.com}.  Some of this serves
as a caution about how challenging it is to build useful systems that augment
the file system namespace.

\tm{Tags}

The concept of using \emph{tags} on files is related to the earlier work on
semantics.  The distinction appears to be that tags were initially designed to
be added by users, though the possibility of combining transducer generated
tags, as in the semantic file systems work, with user created tags is certainly
not excluded~\cite{tagfs}.  Indeed, the Human-Computer Interface (HCI) community
has observed that ``[U]sers are unlikely to use metadata.  Even when the tagging
is somethign as convenient as voice tagging, users are not likely to make the
time investment required to assign metadata to
files.''~\cite{10.1145/642611.642682}  Thus, while it might be useful to allow
human driven tagging, it is unrealistic to consider it to be generally useful.

\tm{Layered}

In my own past work, we implemented a layered log-structured technique within a
single file that permitted us to add additional meta-data to files.  For
example, we were able to add NTFS features to FAT32 files on Windows, with a
file system filter driver responsible for interpreting the additional meta-data
and presenting it to the native operating system.

\tm{Graph}

\tm{360}

\tm{DelveFS}

Key take-away here (for me) was the use of pub/sub systems to provide
notification.  This dovetails nicely into the strong dynamic nature of naming
systems, since they have to be responsive.

\section{Linguistics}
\label{ch:background:sec:linguistics}

\tm{This is where I explain how linguistics describes semantics and pragmatics
    and tie them together with the relevant portions of my thesis.
}

\section{Human-Computer Interface (HCI)}

There is a rich body of literature about various ways of presenting file
information to human users.  Various approaches to this problem have been
considered, including: temporal organization, virtual directories, attribute/tag
organization, and even a table-top model~\cite{collins2007tabletop}.  The
literature seeems to be rather clear on the limitations of the hierarchical file
system for human usage.  Intriguingly, HCI researchers have argued that the
evolution of file systems up to the point of creating hiearchical name spaces
with links with the evolution of databases and to expand their abilities the
\emph{file browser} should evolve to support dynamic interaction on par with
dynamic data interaction in a relational database
works~\cite{marsden2003improving}. This body of prior work has identified that
one key challenge is the sheer
magnitude of how much data is being presented.  Thus, it is not useful: ``If we
were to visualoise all files in this list, as in the email client, the list
would be too long to afford rapid scanning...''

My reading of this prior work is that we need to support more dynamic naming and
naming support, the ability to use transducers to add key meta-data based upon
content and the ability to allow humans to add tags.

At the same time, the importance of hierarchical organization is increasingly
less useful.  While search is useful, particularly when combined with filters
(e.g., faceted search) it also is insufficient to the task.  The associative
data model, such as used in the table-top models~\cite{collins2007tabletop}, is
surprisingly powerful for collaborative work.  No single mechanism satisfies all
these needs; a robust system must enable multiple usage modalities.

\section{Challenges}
\label{ch:background:sec:challenges}

\tm{My concern with this section is that it may be premature --- until I
    introduce the model, identify the research questions to be explored, and the
    artifact(s) I propose to build to explore those research questions this section
    may be ``too soon''.
}


\section{Related Work}
\tm{Note: this text was moved from ``related work'' and needs to be incorporated into this section.}

Ways to organize data within a storage silo have been extensively
studied from multiple perspectives.  That there are so many different approaches
to evaluating the optimal way to organize things is a testament to both the
importance and complexity involved in approaching this topic.  This section will
consider the following related work:
\begin{itemize}
    \item The \textit{storage} perspective, which is primarily rooted in the
          systems perspective.  Storage in this context includes file systems and
          databases, each of which then has multiple different manifestations.

    \item The \textit{provenance} perspective, which is related to systems but
          focuses on creating a context of explainability that is distinct from
          storage.

    \item The \textit{human-computer interface} (HCI) perspective, which is related to
          how humans organize and find information within storage systems. There are a
          number of distinct perspectives to this work including: hieararchical
          data organization, personal information management, enhanced search, and
          cognitive data organization.
\end{itemize}

Each of these perspectives is complementary and assist in better understanding
the problem: systems focuses on being able to efficiently and reliably store and
recover information though often it is agnostic to the specifics of the
information; provenance focuses on accountability and explainability; and HCI
focuses on the human facing problems that need to be solved and tends to ignore
how those solutions are implemented.

\cite{mashwani2019360,9229638,vef2020delvefs,dourish2003the,harrison1996re,barreau1995finding,dourish1999getting,placeless-tois,giffordSFS,plan9,inversion,smartstore,tagfs,gfs,provsearch,uprove2,pindex,page1999pagerank,nfs,metastorage,howard1988scale,afs,Adya:2003:Farsite,unix,scfs,federatedMetaData,federatedACL,benet2014ipfs,guo2012burrito,mazurek2014toward,li2013horus,adya2002farsite,provprimer,camflow}

\nocite{*}

\section{Storage}
\label{ch:related-work:sec:storage}

Much of storage work is dominated by the realities of how media and networks
function, with a goal towards increasing both capacity and performance.  The
basic model of data organization within storage systems tends to separate into
\emph{structured} data --- typically what is found in databases --- and
\emph{unstrutured} data --- typically what is found in file systems or object
stores.

\tm{Describe structured data: why do people use databases, what are the
    strengths and weaknesses.  How does this relate to data organiztion?
}

\tm{Describe unstructured data: why do people use file systems or object stores,
    what are the strengths and weaknesses?  How does this relate to data
    organization?
}

\tm{Describe the work that has been done in semantic file systems, metadata
    augmentation, and non-hierarchical organization structure (e.g., Ground and
    Placeless).
}

\tm{This is a fairly large quotation from the Placeless Documents Project
    Archive that should probably be edited down, but for the moment I want to leave
    it here because I think the insight provides is quite useful.
}

From the Placeless Documents Project
Archive~\footnote{https://web.archive.org/web/20020210170621/http://www.parc.xerox.com/csl/projects/placeless/}:

\begin{quotation}

    Placeless Documents are documents that are organized and managed according
    to their properties, rather than according to their location. Document
    properties can be things you already know about your documents, like that
    they're published, or notes, or about the budget,or drafts, or source code,
    or important, or shared with your colleagues, or from your manager, or big,
    or from the Web, or... whatever suits you. Document properties can also be
    things that you want to be true about your documents, like that they are
    backed up, or replicated on your laptop, or can be purchased for a small
    fee. These latter properties carry the code to implement or interface with
    the desired functionality.  Document properties are statements about your
    documents that make sense to you, and affect what you're going to do with
    the documents.

    What does it mean?

    We live and work in an information-filled world. The project focuses on helping people cope with the large and diverse information spaces that are part of life in the networked world. Our approach is to provide information consumers with a unifying model for organizing and manipulating their information.

    Much of the information that we commonly use is in the form of documents, both physical and electronic. In fact, the use of electronic documents on our computer desktops is pervasive, and even though they unify much of desktop computing, the document metaphor falls short of providing an accessible and readily understood way to interact with all forms of information, whether electronic or physical. Instead, we resort to specialized applications for much of our computer-based work. Electronic documents are managed through different systems like mail, WWW, and file systems. Similarly, the many devices we use in the course of our work (including fax machines, scanners, televisions, VCRs, telephones) manipulate and store information, yet they do not integrate seamlessly with the rest of our on-line information.

    Furthermore, the means available for individuals or groups to organize and customize their information spaces are extremely poor and driven mostly by storage and distribution models, not user needs. The most common model for information organization is hierarchical. The use of folders as a fundamental organizing principle, and the restriction that documents (mail messages, files, URLs, etc.) appear in only one folder at a time, force users to create strict categorizations, resulting in inflexible organizations that tend to persist over time even though their needs evolve. Similarly, customization of the organization and the behavior of information according to individual requirements is cumbersome, if not impossible. Access to a shared document that one individual deems as corresponding to the budget and another project cannot be easily tailored for both specialized functional requirements are equally hard to achieve. If a user can express that a document is read-only, and that it is a Word document, why cant heexpress that updated copies should be faxed to a colleague once a week?

    This project is about removing these hurdles by using a novel infrastructure and proving its benefits through applications that exploit its capabilities. We plan not only to change the way people interact with their currently segregated world of documents, but we plan to exploit a single concept the document and its properties to allow users to interact with arbitrary information.

    Our vision is one of customizable, context-aware management of integrated information spaces, which:

    \begin{itemize}

        \item integrate information components from many sources: repositories (WWW, mail, file systems), devices (scanners, video-cameras, television, phone), and dynamic processes (workflow, source code management systems, search engines, and dynamic document content),
        \item allow customizable organization of the information based on properties of that information, e.g., budget related, read at home, shared with John, and From: petersen@parc.xerox.com,
        \item allow information properties to be arbitrary objects specified through many different mechanisms: explicitly by the users themselves, captured by physical context sensors, inferred from usage, automatically generated by content analysis, etc.
        \item allow information properties to be active and carry behaviors to automate information work, enabling functionality like fax to John at 5pm each day, translate to English, notarized, backed-up in Utah for safety, etc.
        \item scale to sizes anywhere between an individual and the enterprise,
        \item are available at all locations required by the users, and
        \item protect the privacy and intellectual property of users.

              In this world the focus is on information, customization, and functionality that
              extends beyond the abilities of monolithic applications. Essentially,
              information carries the behaviors and semantics needed to operate on it.
              Information is independent of location and becomes responsive to the
              environments it is used in and the contexts of individual users, and it is
              managed independently by both its consumers and providers.
    \end{itemize}
\end{quotation}

\tm{The big hole that I can see is their comment that they want to prove ``its
    benefits through applications that exploit its capabilities.''  The
}

\section{Provenance}
\label{ch:related-work:sec:provenance}

\tm{Explain provenance.  It is a more recent area of study, but it also captures
    more of the kinds of information that will be useful to me.
}

\section{Human-Computer Interface}
\label{ch:related-work:sec:hci}

The traditional primary organizational model provided to users has been the
file/folder metaphor\tm{Need citation to 1965 Daley and the 1956 storage
    papers}, which was itself derived from physical filing cabinets.  However,
electronic storage is not bound by the same restrictions as a physical filing
cabinet, as can be seen even in early work \tm{Again, this is Daley} where the
model added \emph{links}; the closest equivalent to this in filing cabinets
would be to make duplicate copies of a document --- I have seen exactly this
routinely practiced by bookkeeping and accounting professional, whether using
physical or electronic filing systems.  This works because the objects are
generally \emph{immutable}, but for electronic storage systems without
deduplication it is not particularly space efficient.

The Human-Computer Interface community has been studying human-centered data
organization models for decades.  For example, the HCI community observed that
hierarchical file structure is challenging for users with low spatial
abilities~\cite{vicente1988accommodating}. This suggests why storage developers
would not even see there is a problem here: the study of computer science
correlates well with the development of spatial
abilities~\cite{parkinson2018spatial}. In my own discussions with even senior
computer scientists it is the introduction of \emph{silos} that seems to make
the hierarchical abstraction break. ``Did I store that in Dropbox, or Google
Drive, or was it on my laptop or my desktop computer?''

The wealth of research here is astonishing, yet does not appear to influence the
design of storage systems to exploit the results of that research.  Indeed, the
systems community seems to be focused on a \emph{search based} solution while
the HCI community research suggests that search is \emph{not} preferred by users
--- instead, users want to \emph{navigate}:

\begin{quotation}
    \emph{When retrieving a file the user needs to choose between folder
        based navigation and query based search. There are obvious
        intuitive advantages of search for both retrieval and organization.
        Search seems to be more flexible and efficient for
        retrieval. It is flexible because it does not depend on users
        remembering the correct storage location; instead, in their
        query users can specify any file attribute they happen to
        remember...}

    \emph{In fact, regardless of search engine quality,
        people consistently use search only as a ‘last resort’ for that
        minority of cases where they cannot remember file
        locations...}~\cite{bergman2019search}

\end{quotation}

Thus, from an HCI perspective it would seem that one potential research
direction would be to consider potential interfaces that mimic navigation over a
search interface.

For example, recent work around data curation explores the idea of a ``data
dashboard'' since the first step of curation is finding specific data to
curate~\cite{Vitale_2020}. This work builds upon prior research showing that
when presenting data to users it is important to ignore storage silo boundaries.
Indeed, my reading of this work is that it presents what seems to be navigation
even though it is implemented using a search mechanism.  Equally important, this
work also points to the benefits of not changing the \emph{location} of data,
but rather allowing construction of useful relationships via metadata. Thus,
this work supports my ideas of ignoring silo boundaries and providing metadata
for use by a similar tool.  What it does not explore is a way for data storage
to provide enhanced metadata, dynamic update, and notification of changes, which
are important elements to making such a dashboard more useful for data
visualization.




\endinput

\reto{I feel like a lot of Challenges goes here... }

\begin{epigraph}
    \textit{Paradigm paralysis refers to the refusal or inability to think or see
        outside or beyond the current framework or way of thinking or seeing or
        perceiving things.  Paradigm paralysis is often used to indicate a general
        lack of cognitive flexibility and adaptability of thinking.} --- The Oxford
    Review Encyclopedia of Terms (2021).
\end{epigraph}

Computer storage systems continue to evolve and change at an increasing rate.
The difference between file systems, which provide the basic abstraction of
unstructured data, and database systems, which provide the basic abstraction of
structured data, is now filled with a growing array of \emph{semi-structured}
mechanisms including: key-value stores, object stores, document stores, no-SQL
databases, data warehouses, and data lakes.

Each new mechanism invented for storing data creates a new ``silo'' of
that specific storage system.  Each new storage system comes with its own
semantics and meta-data, and seldom with any explicit way of tying related
information together across such silos. Storage silos are often designed with
specific usage patterns in mind.  For example:

\reto{I'd go with categories here instead of products. as there are often
    multiple different solutions within the same category.
}

\begin{description}
    \item[\ac{HDFS}] --- Hadoop was created to solve the problem of large,
        dynamically written write-once files that are typical of machine data
        captured from multiple sources for additional analysis.

    \item[Google Drive] --- Google uses a common storage mechanism for its various
        services including e-mail, documents, spreadsheets, photos, and videos. It
        allows collaborate work, both by sharing access to the object as well as
        providing tools for simultaneous editing between human collaborators.

    \item[Intel DAOS]
        --- Intel's solution for large parallel \ac{HPC} storage
        needs that splits meta-data and data, with meta-data stored in persistent
        memory and data stored on \ac{NVMe} devices~\footnote{\url{https://www.intel.com/content/www/us/en/high-performance-computing/daos-high-performance-storage-brief.html}}.

    \item[Qumulo] --- Seattle-based start-up company that provides a large data
        storage management product that is used in specialized big-data industries.
        For example, Qumulo combines both local and cloud storage to provide video
        post-production work in which file sizes are quite large due to the size of
        high resolution video files. They claim to support secure storage of
        petabytes of file data across multiple tiers of storage.
\end{description}

\tm{I'm not sure these examples are specific enough.}

Each of these storage products attempts to address specific storage needs; each
new product creates a new storage silo.  A human user that can limit themselves
to using a single storage silo can exploit the functionality of that specific
silo and simplifies the challenging task of finding disparate but related
content spread across storage silos.

The proliferation of storage silos exacerbates the easy ability of humans to
navigate their own storage to find a specific file.  One of our own \ac{HCI}
researchers told me that they found their ability to find things on their own
computer broke when they added multiple distinct hard disks~\footnote{Private
    conversation with Joanna McGrenere at UBC CS-50 reception.} The challenge of
finding a specific file or set of files is one that consistently resonates with
many of those with whom I have discussed my research, both inside and outside
the computer science field.

The most commonly presented model of naming files (or objects) within a storage
silo is the hierarchical namespace.  The hierarchical namespace confounds the
storage \emph{location} with the information \emph{relationship}.  Note that
location here is not relative to the logical namespace of the storage silo, not
the physical storage device.

\reto{the challenges section provides quite a broad overview of how things are right now.
    I'm not sure if those are the challenges you will be facing with your thesis, or whether these are general challenges.
    Given your thesis statement,
    shouldn't there be something like a problem statement (or challenges that makes the thesis statement hard)
    I'd recommend to identify say 3-5 challenges and try to express them in one sentence each.
    each sentence then is the \subsection{} heading. where there is maybe a paragraph or two worth of information backing up the challenge/problem.
    ideally, your wock packages will correspond to one challenge/probleme each.
    (maybe those 3-5 challenges are in fact the research questions you plan to answer)
}

Ordinary applications familiar with using hierarchical file systems expect
related objects, of whatever type, to be in or close to the same directory.  The
hierarchical file system model has been highly successful for more than a
half-century, though that success is from the perspective of the storage
community.  The \ac{HCI} community has been pointing out that this model is
deficient for many users.

Storage silos are not a new invention.  The problem of cross-silo management,
including naming is not a new problem.  UNIX addressed multiple silos using
``mount points''.  A \emph{mount point} is a location in an hierarchical name
space where another namespace is logically connected to the existing namespace.
For example, UNIX mounts a new file system instance on top of an existing
directory.  On the Linux system where I am writing this document there are 33
distinct namespaces mounted on top of the base file system namespace --- the
``root'' namespace that starts at "/".  This model has served well as evidenced
by the fact that all mainstream consumer operating systems at present (Linux,
MacOS X, and Windows) support mount points.

NFS on UNIX is implemented using explicit mount points, much like media file
system instances while AFS used a global namespace mechanism to connect its
silos (``volumes'') together so that users were given a single consistent
namespace. Converting the hierarchical name of internet services has been used
to create internet-based namespaces, such as using \ac{WebDAV}, which converts
the internet \texttt{GET/PUT} model into an
hierarchical file systems namespace model.  Amazon's AWS S3 service provides an
object store model that is accessed using HTTP GET/PUT operations as well. This
can be accessed using the hierarchical storage model using any one of the
numerous "S3" FUSE file system implementations on \url{github.com}.  These FUSE
file system implementations of S3 exist not because they are fast or efficient
but because supporting the hierarchical model understood by existing
applications makes them \emph{usable} by existing applications.

The hierarchical name space is definitely powerful: it is a specific
construction of individual name space silos that are composed into a single
uniform namespace. However, the benefits of this single namespace have always
been limited: they tend to follow the \emph{computer} and not the \emph{user}.
Plan 9 was one of the few systems that focused on personal name spaces,
introducing key ideas such as context defined names --- such as the platform
architecture of the computer on which a program is currently running and
\emph{unions} in which multiple namespaces are not joined via a mount point but
rather via a merge of multiple namespaces mounted at the same location.
Xerox's Placeless research project was a system where documents ``are organized
and managed according to their properties, rather than according to their
location.''~\footnote{https://web.archive.org/web/20020210170621/http://www.parc.xerox.com/csl/projects/placeless/}

In the past two decades the complex multi-silo storage environment has exploded:
\begin{description}
    \item[WebDAV] --- the ability to map internet web servers into the
        hierarchical file system;
    \item[Dropbox] --- maintaining a mapping between a public internet remote
        storage solution and a portion of the local hierarchical namespace on
        existing storage across multiple devices;

\end{description}


\tm{This should come from the existing text.  It should explain \textit{how} we
    got to this point.
}

\section{Linguistics}
\label{ch:background:sec:linguistics}

\tm{An interesting way to look at this is from the \emph{linguistic}
    perspective.  Much of the prior work relates to semantics but in fact what I'm
    suggesting is that we also consider \emph{pragmatics}. Wikipedia defines this
    as: ``In linguistics and related fields, pragmatics is the study of how context
    contributes to meaning. Pragmatics encompasses phenomena including implicature,
    speech acts, relevance and conversation. Theories of pragmatics go
    hand-in-hand with theories of semantics, which studies aspects of meaning which
    are grammatically or lexically encoded. The ability to understand another
    speaker's intended meaning is called pragmatic competence. Pragmatics
    emerged as its own subfield in the 1950s after the pioneering work of J.L.
    Austin and Paul Grice.''  This actually seems to capture some of what we are
    trying to do quite well: semantics have been studied in terms of computer
    storage, but not pragmatic linguistics.
}

\begin{quotation}
    Pragmatics is a field of linguistics concerned with what a speaker implies
    and a listener infers based on contributing factors like the situational
    context, the individuals’ mental states, the preceding dialogue, and other
    elements.
\end{quotation}

\url{https://www.masterclass.com/articles/pragmatics-in-linguistics-guide}

\tm{Important take-away here is that Pragmatics is distinct from Semantics
    because it relates to how context affects the meaning of language.  This is
    actually quite close to what we are trying to do: to capture \emph{context} so
    we understand the meaning of the language we are using to describe things
    relative to that context.  However, what we are doing isn't a good mesh with
    pragmatics or computational pragmatics (yes, it exists) because that seems
    more focused on the question of using computational mechanisms for
    understanding pragmatics.
    \url{https://www.oxfordbibliographies.com/view/document/obo-9780199772810/obo-9780199772810-0264.xml}
    is interesting in that it really hits home on the dynamic nature of meaning.
}

\begin{quotation}
    Pragmatics as a branch of linguistics can be characterized as the study of
    the relations between linguistic properties of utterances on the one hand,
    and aspects of the context in which a given utterance is used on the other.
    Computational pragmatics is pragmatics with computational means, which
    include models of dialogue management processes, collections of language use
    data, annotation schemes and standards, software tools for corpus creation,
    annotation and exploration, process models of language generation and
    interpretation, context representations, and inference methods for
    context-dependent utterance generation and interpretation processes. The
    linguistic side of the relations that are studied in pragmatics is formed
    primarily by utterances in a conversation or sentences in a written text. In
    the case of written text the context side consists of the surrounding text
    and the setting in which the text is meant to function. In spoken or
    multimodal dialogue, the context of an utterance is formed by what has been
    said before and the interactive setting, but additionally by other
    perceptual, social, and mutual epistemic information (see Context Modeling).
    Much of this information is dynamic, as it changes during a dialogue and,
    more importantly, as a result of the dialogue, since the participants in a
    conversation influence each other’s state of information when they
    understand each other. Dialogue contexts are thus updated continuously as an
    effect of communication. Central to computational pragmatics is the
    development and use of computational tools and models for studying the
    relations between utterances and their context of use. Essential for
    understanding these relations are the use of inference and the description
    of language in terms of actions that are inspired by the context and that
    are intended to change the context. This bibliography therefore focuses on
    publications concerned with the computational modeling of dialogue in terms
    of communicative actions including the use of inference for utterance
    interpretation. It also considers the more static analysis of discourse
    coherence and semantic relations in text, and concludes with references to
    recent activities concerning the construction and use of resources in
    computational pragmatics, in particular annotation schemes, annotated
    corpora, and tools for corpus construction and use. The popularity of
    probabilistic approaches to natural language processing can also be seen in
    studies of pragmatic aspects of language use, although these approaches are
    so far not as important as in some other areas of language processing. The
    so-called rational speech acts (RSA) model treats language use as a
    recursive process in which probabilistic speaker and listener agents reason
    about each other’s intentions to enrich the literal semantics of their
    language along broadly Gricean lines. The core references for this approach
    are also included in this biography under Inference in Language Processing.
\end{quotation}

\tm{Good stuff here, which resonates with what I've been talking about.  Context
    isn't static, though semantics are static.
}



Computer storage focuses on \emph{content}, which is one form of
\emph{identity} such as is found in ``content addressing'' where the specific
identity of a file is described using a hash value computed from the content of
an object.  Some storage systems provide the ability to encode a limited amount
of information about relationships and properties.  \emph{Relationships} in
file systems are typically expressed using directories (or folders) and
\emph{context} is captured via a name.\reto{does it make sense to talk about an ever growing pile of files?
    somehow I feel like this is where the systems aspect comes into place:
    - humans have  a limited working set.
    - crawling through tons of files across different silos is impractical.
    - search is fuzzy and may not provide  the right results, nor work cross silo.
}
\reto{context is another thing. also: I wouldn't say that context is the
    name. }\tm{I disagree with Reto on this point: when we have hierarchical
    names, we embed the context in the hierarchical name \emph{because} that's
    the only way we have of capturing it.  Indeed, as Reto pointed out earlier,
    names have meaning within a particular context.  So... where do we
    \emph{get} this context.}


\reto{does it make sense to talk about an ever growing pile of files?
    somehow I feel like this is where the systems aspect comes into place:
    - humans have  a limited working set.
    - crawling through tons of files across different silos is impractical.
    - search is fuzzy and may not provide  the right results, nor work cross silo.
}\tm{I had text in at some point about the growing size and the problems
    with it.  The proliferation of silos is yet another issue, such as what is
    happening with in-memory compute, each of which ends up looking like its own
    silo.
}

The goal of making computer storage more flexible is not a new one.  Memex was
first proposed in 1945 by describing how computers might help human users by
serving as ``augmented memory''~\cite{bush1945we}.\reto{what does it do?}  Computer storage has evolved
dramatically in the 76 years since yet our storage systems have failed to
progress towards, let alone realize Memex's associative relationship
model.\reto{maybe say what the assoc relation ship model actually provides....}

\reto{mention the "search" and "lookup" of the right files again here? it appears in the abstract, but maybe could be a little bit expanded here?
    distinction between mete-data / content}
\tm{In fact, my original goal was to \emph{lead} with the thesis statement and
    then build on that.  Then I tried to add a small body of text to introduce the
    topic.  If that body of text is going to grow, I should move all of that after
    the thesis statement.}

\section{Naming}
\label{ch:background:sec:naming}
%\section{Why naming is a Computer Systems problem}
%\label{ch:introduction:sec:systems-problem}


\MIS{This feels like a classic related work paragraph, but I don't know where it
    fits in your story line.
}
Media file systems have long been organized using a simple internal key-value
store.  The BSD UNIX Fast File System used an ``index node'' (inode) as a
description of a data object, such as a file or directory~\cite{mckusick1984a}.  These nodes were
indexed using an identifier, which acts as a simple key.  The NTFS file system
is structured similarly~\cite{custer1994inside}.  Recent work explored the idea
of separating the namespace from storage with an emphasis on high performance
solid-state storage (SSD) devices~\cite{koo2021modernizing}.  This echos earlier
work suggesting using object storage devices (OSD) for file systems
implementations~\cite{seltzer2009hierarchical}.

Beyond separating naming from storage, there is also a trend to separate
meta-data from storage as well.  One early example of this is
Lustre~\cite{braam2019lustre}, with more recent work including Ceph and
Gluster~\cite{noronha2008imca,weil2006ceph}.  Each of these distributed file
systems separates the service of data from meta-data, which allows data paths to
focus on performance, including the use of parallel data paths.  Meta-data does
not benefit from high-performance parallel I/O paths and instead benefits from
low latency random access. This separation does not improve naming support
because that is not the goal of any of these systems.

Intel's DAOS system uses Intel DC Persistent Memory for meta-data and \ac{NVMe}
for data. DAOS supports two naming models: one is a POSIX-compatible
hierarchical name space, and the other is a key-value store interface. Storage
pools are accessed using one or the other of these interfaces but not both.

In a hierarchical name space the fully qualified path corresponds to the
\emph{logical} location of the file, that is the location of this file in the
namespace itself. In most file systems that logical location corresponds to a
specific file system instance and thus defines the storage device(s) on which
that file is stored.  Meta-data associated with that file is interpreted by the
file system to determine how to retrieve the data.  For a media file system that
usually corresponds to the address of the location(s) of the file data on the
corresponding media.  For a network file system that usually corresponds to the
network address used to request the correct file data from the computer storage
system where the data is actually stored.

Separating the file system meta-data from the file data is known to be
beneficial.~\cite{kawai2011a}  This approach logically makes sense when
considered in the multi-silo context as well, since a human user looking for
something actually does not care \textit{where} that the
thing is located they care \emph{what} the thing is.

This raises an intriguing question, one that is at the heart of my own research:
\emph{what is the purpose of naming?}  Applications are quite happy to use names
that are meaningless to humans, such as content hashes, UUIDs, or randomly
generated string names, such as are used for temporary application files.
Indeed, if an application knows that a name has a fixed format, the knowledge of
that fixed format can be used to simplify the implementation or to separate
application named files from other files.

Humans use file names to provide \emph{meaning} as to what is in the named
object.  Librarians assisting people in organizing data routinely suggest
embedding contextual information in the file names; directories are then used to
create additional organizational structure (or ``context''). There are numerous
limitations to this approach, however, because humans do not organize things in
the same fashion and even the same human does not organize things the same over
time. This is such a common phenomenon even the popular press pokes fun at it
(\autoref{fig:xkcd:1459}.)

The inability to find specific things is \textit{not} unique to computer data storage.
Indeed the current situation for organizing digital data seems to be descended
from the organizational structure of a library or a file cabinet, despite the
physical limitations of organizing paper files that are not constraints for
computer storage.  For example, when I am organizing my long play record
collection I have to chose one primary ``key'' to use, such as the time when the
album was released.  However, digital storage does not have this same
restriction - I can easily ask that they be presented to me by genre, artist,
album title, or even instrument type. Further, I can change my organizational
scheme dynamically.  Both Bell Lab's Plan 9~\cite{plan9} and Xerox Placeless
Documents~\cite{dourish1999getting}
observed that there is a dynamic nature to how information is organized.  The
hierarchical name space is basically static, like my album collection is static:
I can choose a different organizational scheme but to do so I have to physically
move them around.

Indeed, the system we have evolved works \emph{in spite} of the obvious
artificial limitations imposed by decades old decisions~\cite{dourish2003the}.  Rather than being
a deep insight that the current system works because it is the best of all
possible models, it is a testament to human ingenuity and a tolerance for
primitive, sub-standard systems.

There are a body of recommendations about file naming
standards\footnote{Data Management for
    Researchers~\cite{briney2015data}}
\footnote{Smithsonian: \url{https://library.si.edu/sites/default/files/tutorial/pdf/filenamingorganizing20180227.pdf}}
\footnote{Stanford: \url{https://library.stanford.edu/research/data-management-services/data-best-practices/best-practices-file-naming}}
\footnote{NIST: \url{https://www.nist.gov/system/files/documents/pml/wmd/labmetrology/ElectronicFileOrganizationTips-2016-03.pdf}}.
Harvard Data Management suggests~\footnote{\url{https://datamanagement.hms.harvard.edu/collect/file-naming-conventions}}:

\begin{itemize}
    \item Think about your files
    \item Identify metadata (e.g., date, sample, experiment)
    \item Abbreviate or encode metadata
    \item Use versioning
    \item Think about how you will search for your files
    \item Deliberately separate metadata elements
    \item Write down your naming conventions
\end{itemize}

Thus, there are common themes: a name provides context for \emph{what} the given
thing represents. Uniformity of information is also important --- the ``naming
convention'' permits not only identifying similarity but key elements of
\emph{difference} between any two named things.  Thus, to return to the original
question: ``what is the purpose of naming?''

\tm{It seems that this Harvard list is a serious indictment of the existing
    system: it pushes the cognitive load onto the users, talks about meta-data ,
    versioning, encoding, \emph{and} capturing the naming convention.}

It seems clear that human naming is
not the same as data storage naming, though we often treat them as the
same.  This can be seen in the data storage tendency to either use names that
are entirely about locating a specific object, such as is the case with a
key-value store, for example, or it combines location with identity, such as is
typified by the Uniform Resource Identifier (URI)~\cite{berners-lee1998uniform}.

Thus, it would seem that naming is about:

\begin{description}
    \item[Identity] --- the name of a thing should be sufficient to verify that
        it is the specific thing we seek.  An example of this is biometric data for
        humans or a cryptographic hash for a storage object;
    \item[Location] --- the name of a thing could provide information that
        directly or indirectly specifies where it is located.  A human example is
        how an identity card might specify a current residential address.  A data
        storage example is the U.S. Library of Congress Classification System, which
        can be used to find a particular work within very large bodies of work even
        though books are typically not thought of as a storage media;
    \item[Relationships]  --- the concept that one thing can be derived from
        another, whether in a direct form, such as a version of the same logical
        thing, or in a more indirect form, such as a provenance relationship showing
        that one thing was derived from another thing by applying some set of
        transformations to it;
    \item[Characteristics] --- this relates to something that is a property of
        the thing.  For humans this might be the date of birth (``creation date''),
        height, weight, eye color, etc.  For storage objects this might include
        timestamps, size, data type;
    \item[Context] --- at some some level, we often rely upon names to provide
        us with \emph{context}.  For humans, we assume that people with the same
        family name are likely to be members of the same family, even though
        ``\persa Smith'' is distinguished from ``\persb Smith''.
\end{description}

When considered from this perspective, it seems clear why ``naming is hard'' ---
it plays multiple distinct roles, sometimes overlapping, sometimes interfering.
These challenges are not well-served by existing naming support in computer
storage systems.

\tm{I feel that I should capture the dynamic nature of naming, which really does
    differ from the traditional static model of it.
}

\reto{Overall, I feel like the challenges are quite an overview of related work. I would try to focus on what challenges/problems arise when you try to desgin and implement your system?
    Challenge a) Multiple silos with different characteristics; and Challenge b) capturing activities, ... (or is that gone)? }

\chapter{Research Questions}
\label{ch:research-questions}

\begin{epigraph}
    \textit{For every particular thing to have a name is impossible. --- First,
        it is beyond the power of human capacity to frame and retain distinct
        ideas of all the particular things we meet with: every bird and beast men
        saw; every tree and plant that affected the senses, could not find a place
        in the most capacious understanding.} --- \textbf{John Locke}, \textit{An Essay
        Concerning Human Understanding}~\cite{locke1844locke}
\end{epigraph}

The goal of my research is to develop a framework for gathering and
disseminating rich data tracking activity across a user's silos and devices to
facilitate end-user \emph{finding}.  While the evaluation of these finding
questions lies in the domain of HCI researchers, my research must enable
the collection, aggregation, storage, and querying the data that these
researchers can use to evaluate different finding approaches.  Thus, my research
will answer the following questions:

\begin{enumerate}
    \item \label{rq:define-ac} \textbf{What data comprises activity context?} As
          previously observed
          (\autoref{ch:background:sec:useful-information-for-finding}) there is
          a wealth of potential information that could be included in the activity
          context.  This question asks: ``what data should be included in the activity
          context?''

    \item \label{rq:capture-ac} \textbf{How do we capture activity context?}
          While there are numerous different potential sources for activity
          information, how do we capture and store it?

    \item \label{rq:rich-data} \textbf{How do we collect rich data (including
              activity context) from multiple storage silos and make that rich data
              accessible efficiently?}
          We know from prior work that one can
          capture \emph{all} state for a single computer system with
          surprisingly moderate cost~\cite{devecsery2014eidetic}, but that
          approach does not make the collected state easily available.
          Thus, collecting and storing the rich data is not sufficient, we
          also need to make it accessible to applications. How do we make it
          available to applications in a manner whereby the benefits of better
          findability outweigh any impact on application performance?

    \item \label{rq:meta-data-query} \textbf{How do we facilitate query of this
              rich meta-data?} It is important that both users and applications have a
          mechanism via which they can exploit this extensive collection of rich
          meta-data.  What does the query interface for this rich meta-data look like?

    \item \label{rq:leverage-ac} \textbf{How can applications leverage this rich
              meta-data?} Applications must programmatically find the documents with which
          they interact.  Most applications use temporal relationships, such as
          ``recently accessed documents,'' as a primary mechanism to present users
          with a set of files they might wish to access.  When that fails, they fall
          back to offering the user files in a directory or list that the application
          deems `likely'.  This question asks: ``if an application has access to
          activity context data, how can it use the information to give users a better
          collection of candidate files from which to select?''

    \item \label{rq:privacy} \textbf{How do we preserve the privacy of sensitive
              meta-data?} In a model that incorporates extensive amounts of personally
          identifiable information there is a very real risk that this information
          will prove to be economically valuable to someone; how do we ensure that the
          user retains ownership and control of that information so that it is only
          released when they approve doing so?

\end{enumerate}

I intend to provide data that will allow other researchers to answer related
questions such as:

\begin{itemize}
    \item What relationships are most valuable in helping users find data?
    \item Does activity context provide better information for user search than
          semantic information alone?
    \item What interfaces best allow users to leverage rich meta-data search
          capabilities?
\end{itemize}

The questions themselves that are core to my proposed thesis work revolve around
exploring and broadening the of an \emph{activity context}. However, activity
context by itself does not replace the other prior work that
has added semantic understanding around files, e.g., semantic file systems,
formal concept analysis, tag file systems, graph file systems, etc. Thus, my
proposed system (\autoref{ch:architecture}) includes support for this prior work
but extends it by augmenting previous mechanisms with this additional rich usage
information that I call ``activity context.''

\endinput

The goal of my research is to explore a number of key research questions.  One
of the complications involved in this work is that defending my thesis will best
involve work by others.  Thus, in this section I identify key research
questions; from these I then identify those that relate to those I propose
constructing as part of my thesis.  The goal of the other questions then is to
provide guidance as to what I envision as the usage model for this work.

% Table generated by Excel2LaTeX from sheet 'Research Questions'

\begin{enumerate}
    \item\label{rq:events} \textbf{What events are useful in establishing relationships between
        objects?}  Prior work has explored a few relationships.  For example, the
    provenance guided search work identified \emph{causality} as being a better
    predictor of a relationship than temporal locality.  However, modern
    computer systems capture a vast array of information.  This provides us with
    intriguing opportunities to consider what of just the currently captured
    information is useful.  Beyond that, I propose there may be information that
    we do not currently capture that would also be useful.  The function of this
    research question then is to identify what information would be useful.
    My expectation is that answering this research question is likely best done
    in collaboration with researchers from other communities, such as HCI, in
    order to identify specific events that turn out to be useful.

    \item\label{rq:improve} \textbf{Does adding activity context improve associative access to
        files?}  More specifically, prior work has identified that using extracted
    semantic and specific tag information yield useful insight into file
    associations.  While the purpose of my work is to provide the tools
    necessary to answer this research question, I consider it to be beyond the
    scope of the work I will do in support of my thesis.  Ideally, I would work
    collaboratively with researchers in other fields better suited to evaluating
    this, likely as part of the work involved in answering question
    \ref{rq:events}.

    \item \label{rq:existing-apps} \textbf{How can existing applications benefit from the availability of
              activity context?} There are existing ideas in this space that have been
          explored such as ``virtual directories.'' While interesting, I don't
          consider this to be a core question to answer as part of my thesis.  I
          suggest this would be a good project to work with another systems researcher
          to explore in collaboration.

    \item \label{rq:providing-ac} \textbf{How do we provide activity context to enable others to exploit
              it to build better tools?}  This question presupposes that the answer to
          question \ref{rq:improve} is yes.  Thus, assuming that we find activity
          context does improve associative access to files, how do we provide access
          to that information.  There is a body of existing work upon which to draw
          for inspiration regarding meta-data queries of both static and dynamic
          sources, but it seems likely there will be at least systems-specific issues around
          processing a potentially high volume event stream efficiently.

    \item \label{rq:privacy} \textbf{How can we provide distributed meta-data services while
              preserving privacy?} Prior work in utilizing ``personal digital traces''
          raised similar concerns and the approach they took --- to store the data on
          the user's computer --- seems like the correct solution for the work that I
          do for my thesis~\cite{vianna2019searching}.  However, I note that it seems
          likely that this question is a logical follow-on if this system proves to be
          useful because I can envision benefits of being able to share activity
          context between multiple devices for a single user as well as the ability to
          share activity context with other users with whom the owning user is
          collaborating.  Thus, I do not propose fully addressing this question within the
          context of the work for my thesis.

\end{enumerate}

I do not propose answering all of these research questions.  My goal in
providing them is to lay the groundwork for understanding why the research I
propose doing in support of my thesis could have substantial impact.

\chapter{Architecture}
\label{ch:architecture}
\label{ch:model:sec:architecture:subsec:services} %TODO: remove this once I've
% cleaned up the document

\begin{epigraph}
    \emph{We are like dwarfs on the shoulders of giants, so that we can see more than
        they, and things at a greater distance, not by virtue of any sharpness of sight
        on our part, or any physical distinction, but because we are carried high and
        raised up by their giant size.} --- Metalogicon (1159) John of Salisbury.
\end{epigraph}

My goal with this architecture is to capture a broad description of the services
I anticipate are required to achieve my goals.  To that end, my architecture
seeks to provide a broad framework on which my own work as well as potential
future work can be constructed.

My focus in supporting my thesis will include building specific, likely limited,
implementations of tools that fit within my architecture. These tools will then
allow me to answer the research questions described in
\autoref{ch:research-questions}.

In keeping with good software architectural principles, I strive to ensure the
architecture is sufficiently general and not necessarily tailored narrowly to my
particular task: that narrowing can be done as part of my own tool development.

\section{Features}
\label{ch:architecture:sec:features}

The tool I propose constructing incorporates support for existing functionality
as well as the new functionality that I propose adding. In
\autoref{table:usecases} I have set out the specific features that I anticipate
providing by the tool. In turn, I use these features when constructing my
proposed architecture.

%%%%%%%%%%%%%%%%%%%%%%%%%%%%%%%%%%%%%%%%%%%%%%%%%%%%%%%%%%%%%%%%%%%%%%%%%%%%%%%%%%%%%%%%%%%%%%%%%%%%%%%%%
\begin{table*}[!th]
    \begin{adjustbox}{max width=\textwidth}
        {\renewcommand{\arraystretch}{1.5} %<- modify value to suit your needs
            \begin{tabular}{p{0.20\textwidth}p{0.4\textwidth}p{0.4\textwidth}}
                \hline
                \textbf{Feature}                                                                                                                                        & \textbf{Existing Technologies} & \textbf{No Solution} \\
                \hline
                \usecaseactivitycontext                                                                                                                                 &
                timestamps and geo-location, image recognition, browsing history, ticketing systems, application-specific solutions like Burrito~\cite{guo2012burrito}. &
                Link related activity across apps, record  browsing history and chat conversations relevant to the creation of the data object, storing it in ways that are secure and compact.
                \\
                %
                \usecasecrosssilosearch                                                                                                                                 &
                Search by name, creator, content across silos,
                app-specific searches (e.g., Spotlight)                                                                                                                 &
                Unified search across all kinds of storage, including file systems, object stores, apps and devices
                \\
                %
                \usecasedatarelationship                                                                                                                                &
                De-duplication of documents, versioning of specific files, git ancestor relation                                                                        &
                Explicit notion of data identity, tracking different versions across different silos as data is transformed
                \\
                %
                \usecasenotifications                                                                                                                                   &
                File watchers (INotify), synchronization status, manually inspecting modified time                                                                      &
                Ability to subscribe to specific changes on attributes
                \\
                %
                \usecasepersnamespace                                                                                                                                   &
                Hierarchy plus hard/soft links. Use of tags.                                                                                                            &
                Creating per\-son\-al\-ized name\-spaces with flexible data organization and views
                \\
                \hline
            \end{tabular}
        }
    \end{adjustbox}
    \caption{Use-case driven functional requirements.}
    \label{table:usecases}
\end{table*}
%%%%%%%%%%%%%%%%%%%%%%%%%%%%%%%%%%%%%%%%%%%%%%%%%%%%%%%%%%%%%%%%%%%%%%%%%%%%%%%%%%%%%%%%%%%%%%%%%%%%%%%%%

\subsection{Activity Context}

As Burrito demonstrated~\cite{guo2012burrito}, the \emph{context} in which data
were accessed or created is often a useful attribute on which users wish to
search, e.g., ``\emph{I'm looking for the document I was editing while emailing
    \persa about their favorite wines}.''

To the best of our knowledge, there is no modern system that supports queries
using rich context across applications.

I might be able to use timestamps or application-specific tags or history
information in queries, but it is laborious, if not impossible, to intersect
data from multiple applications and/or multiple silos.

\subsection{Cross-silo Search}
Users share documents in myriad ways: via messaging applications, on cloud
storage services, and via online applications. Users should not need to remember
which mechanism was used to share a particular document and should have some
easy way of organizing and searching through a collection of such distributed
documents.

\subsection{Data Relationships}

Documents can be related in arbitrary ways. This relationship information can be
used to facilitate and enable better search results. So far, I have identified
three specific relationships that are particularly important:

\begin{enumerate}
    \item \emph{copy} is a bit-for-bit identical replica of some data, in other
          words two items with different names store the same data. Deduplication
          functionality in storage systems frequently takes advantage of the
          prevalence of copies to reduce storage consumption. However, knowing that
          two items with different names are, in fact, the same is also valuable
          information for \emph{users}.

    \item \emph{conversion} is a reversible, repeatable transformation that
          changes the representation of data, without changing its semantics, e.g.,
          converting a CSV file into JSON.


    \item \emph{derivation} refers to data that has been computationally derived
          from another object by altering its content, e.g., adding a row to a
          spreadsheet.

\end{enumerate}

Classifying the relationship in such cases may not be obvious: if I
export an Excel format spreadsheet as a CSV file, I may lose data such that the
relationship is a \emph{derivation} rather than a \emph{conversion}.  A
conservative implementation would thus define this as a \emph{derivation} in the
absence of knowledge that an inverse transformation exists.

While storage systems can recognize copies, they cannot distinguish
conversions from derivations. However, from a user's perspective, these
operations are quite different: a conversion can be repeated, which is not
necessarily true of a derivation.

\subsection{Notifications}

Users frequently want to be notified when documents change, and many storage
services offer this functionality.

However, users might also want notification when data on which they directly or
indirectly depend changes. This requires both a notification system and an
awareness of the data relationship between different objects.

\subsection{Personalized Namespaces}

Users have different preferences and mental models to organize their documents,
frequently a source of conflict in a multi-user setting. I need a way to provide each user
the ability to personalize their document structure.

\begin{comment}
\section{Use Cases}
\label{ch:architecture:sec:use-cases}

To help motivate my proposed architecture, I explain how the features described
in \autoref{ch:architecture:sec:features} address a number of important use
cases.

\subsection{Data Processing}

\persa and \persc are preparing a report summarizing their work on a data analysis project for a customer.
\persc sends an email to \persa containing a CSV file with original data.
\persa opens this document in Excel, formats and filters it, adds additional data from a corporate storage silo,
and then returns the Excel document to \persc on Slack.
\persc is away from their desk when it arrives, so they open it on their phone, uploading it to a cloud drive.
\persc then sends the link to \persa for editing with update notifications.
Finally, \persc sends a PDF of the report to the compliance officer who promptly asks, ``Where did this data come from?''

This use case highlights the need for \usecasedatarelationship, as it has
instances of copies, conversions and derivations, \usecasecrosssilosearch, as
these items are located in multiple silos and accessed by multiple devices, and
\usecasenotifications, as update notifications need to be distributed to
designated users.

\subsection{Delete Request}

Some time later, the compliance officer requests that all documents containing a
customer's data must be deleted.

To help with finding all relevant customer data, \persb joins the project and
examines the report and requests the original data from which it was produced.

\persa remembers that they gave the original data to \persc shortly after
collecting it, but does not remember the name, location, or even how the
relevant files were transmitted. Thus, \persa has to manually search possible
locations and applications, sendsing references to documents to \persb, who then
starts organizing these files to methodically identify the ones that might
contain the customer's data. In the process, many of the other team members'
references to the documents stop working.

This use case illustrates the need for \usecaseactivitycontext to capture
data that has been collected while interacting with the customer, \usecasedatarelationship to identify related documents,
\usecasecrosssilosearch to easily locate relevant documents across data silos,
and \usecasepersnamespace to create a individual data organization.

\subsection{Security and Privacy}

\persg, an investigative journalist who routinely receives sensitive information
from third parties, is investigating the company from the prior use cases.

\persg needs to be able store and access sensitive information, including
information about the activity context of various e-mails, documents, pictures,
and audio and video files.

While \persg ensures that these data are encrypted, they need to also ensure
that they can both find information and ensure that meta-data associated with
those files is both usable and properly protected across silos.

This case requires both \usecasecrosssilosearch and
\usecaseactivitycontext to allow \persg to gather information obtained from
specific meetings or at a given time/place.

While \persg must protect their sources, they must also be able to associate
evidence with those sources to make judgement calls about their validity, so I
must design security and privacy policies for attributes that accomplish both.
\end{comment}

\subsection{From Use Cases to Architecture}

Recall that in \autoref{ch:intro:sec:use-cases} I provided a series of use cases
for consideration. Each use case and feature presents a situation that cannot be
solved with our existing mechanisms within the context of the multi-silo world.

In \autoref{table:usecases}, I identify existing technology that can be brought to
bear on the problem, while teasing apart the precise details that are missing.

Repeatedly, I find that critical information necessary to provide a feature is
unavailable, that providing such information is non-trivial, and that obtaining
it creates a collection of privacy challenges.

\section{Proposed Architecture}
\label{ch:architecture:sec:proposed-architecture}

\system is a family of services that enable sophisticated search and naming capabilities.
The key features that differentiate \system from prior work are:

\begin{enumerate}
    \item incorporating object relationships as first class meta-data,
    \item federating meta-data services,
    \item recording activity context,
    \item integrating storage from multiple silos, and
    \item enabling customizable naming services.
\end{enumerate}

Data continues to reside in existing and to-be-developed storage silos.
\system interacts with these silos, collects and captures metadata, and
provides a federated network of metadata and naming services to
meet the needs of the use cases in \autoref{table:usecases}.

\section{\system Services}

\begin{figure}[!tb]
    \centering
    \includegraphics[width=0.95\textwidth]{reference/hotstorage21/figures/Naming5-legend.png}
    \caption{\system Architecture (\autoref{ch:architecture:sec:proposed-architecture}).
        %Grey boxes indicate new components. AM=``Activity Monitor'', MS=``Meta-data Server'', UNS=``Update Notification Server'', NS=``Namespace Server''.
    }
    \label{fig:arch}
\end{figure}

\autoref{fig:arch} illustrates the \system architecture. \system allows for
different deployment scenarios. The five services can be run independently, they
can be co-located and bundled together to run on a local device, integrated into
an OS, or available as web-based services.

In the balance of this section, parenthesized numbers and letters refer to the arrows
in Figure \ref{fig:arch}. There are five main components:

\begin{enumerate}

    \item \textbf{Metadata servers (MS)} are responsible for storing attributes
          and provide a superset of capabilities found in existing metadata
          services~\cite{federatedMetaData,smartstore}. Users can register a
          Metadata Server with
          activity monitors or attribute services, which allows the Metadata Server to receive
          updated attributes from storage objects and activities (B). Thus, there can
          be multiple sources of attributes including the user itself. Metadata
          servers may retain the full or partial history of attribute updates or
          maintain only the most recent value.

    \item \textbf{Namespace servers (NS)} connect to one or more Metadata Server and use the metadata to provide users with a
          personalized namespace that allows both manual organization (i.e., a
          hierarchical namespace) and rich search capabilities.  The benefit of
          supporting a hierarchical namespace is that it provides a path for
          backwards compatibility as well as a mechanism for enabling virtual
          directories~\cite{gifford1991semantic} to enable existing applications
          to benefit from the enhanced capabilities of \system.  The benefit of
          rich search capabilities is that it enables us to build those virtual
          directories for use by the hierarchical components as well as propose
          and evaluate alternative data exploration tools.
          Users can register with a Namespace Server (R) that uses one or more Metadata Servers to obtain relevant
          attributes from them (C). Additionally, users can be part of a corporate Namespace Server that
          allows sharing of their select metadata with other users via standard enterprise
          public-key cryptography.

    \item \textbf{Activity monitors (AM)} run on the user's devices. Their main function is to observe temporal relations,
          activity context, and relationships between objects on a user's device and
          transmit them to a Metadata Server (D).

    \item \textbf{Attribute services (AS)} extract attributes from storage objects and transmit them to an Metadata Server (B). An Attribute Service
          might be invoked on updates, run once or periodically. For example, a file
          system Attribute Service would update the object's metadata with basic attributes such as size
          or modification time. There can be many Attribute Services that extract more ``interesting''
          attributes, e.g., image recognition, similarity, or other classifiers.

    \item \textbf{Update notification server (UNS)} provides notification mechanisms. Users can register interest in changes of
          attributes or underlying storage and will receive a message on change events (A)
          to which they have access.

\end{enumerate}

In \autoref{fig:single-system} I show a simplified image depicting how this
might work on a single computer system, with data collection from a variety of sources on
the local computer, including resources that are both local to the system and
remote resources accessible and in use on the computer.  Information is
ingested by the local \system components and then presented to applications via
a query interface; one likely use of this query interface would be using it to
form ``virtual directories'' that allow legacy application interactions with the
namespace. Unlike \autoref{fig:arch} this diagram omits details of the internal
structure and instead explains one way in which it could fit into the local
device's environment.

\begin{figure}
    \centering
    \includegraphics[width=0.95\textwidth]{figures/indaleko-arch.png}
    \caption{Single computer components for \system.}
    \label{fig:single-system}
\end{figure}


\section{\system Working Example}
\label{ch:architecture:sec:working-example}

To make the \system architecture concrete, I revisit our use-cases from \autoref{table:usecases}
and walk through the cases to illustrate how \system supports the various actions and events.

\subsection{Storing the e-mail attachment}

\persa's act of saving the CSV file that \persc sent in email corresponds to the
creation of a new object on the cloud storage silo, i.e., the file system (4). The
file server is \system-aware, so the Attribute Service co-located with it extracts attributes
from the document and forwards them to the Metadata Server (B).

The Activity Monitor on \persa's laptop detects that the CSV file came via company email from
\persc. It then captures the activity context identifying the relationship
between the e-mail and the CSV file and transmits it as additional metadata
about the CSV file to the Metadata Server (that already contains metadata extracted by the
Attribute Service). Moreover, because there is a company-wide namespace service, \system
establishes that the e-mail attachment, the CSV in the file server, and the one
on \persc's laptop (from which the file was sent) are exact copies of each
other.

Many applications already record some form of activity context, e.g., chat
history, browsing history. Such histories provide a rich source of additional
metadata. Other activity context, specifically the relationship between objects,
such as the fact that a particular file was saved to a local storage device from
an email message, requires more pervasive monitoring as found in, e.g., whole
provenance capture systems~\cite{camflow}. \system is agnostic about the precise
data that comprises activity context, but allows for storing and accessing
activity context as metadata.

\subsection{Creating the Excel file}

\persa opens the comma separated value (CSV) file using Excel and stores it as a spread sheet.
This creates a new object. The Activity Monitor detects that the newly created spreadsheet is
a conversion from the CSV file, either via a notification from \system-aware
Excel or by monitoring the system calls executed on the local system. \persa
proceeds to modify the data by filtering it in Excel and saving the changes. The
Activity Monitor records this event and updates the meta-data of the spreadsheet to record the
derivation-relationship. Ideally a \system-aware version of Excel specifies to
the Activity Monitor the exact type of the relationship (in this case a derivation); otherwise
the Activity Monitor informs the Metadata Server about an unspecified data relationship by observing the
opening of a CSV file and a subsequent creation of the Excel file.

\persa proceeds to upload the new Excel file on Slack, which triggers the
creation of a new storage object as Slack creates a local copy, the addition of
new metadata to Metadata Server via the AS, and the addition of a \emph{copy} data relationship
between the original Excel file and the Slack’s copy. The Activity Monitor notices (by
monitoring Slack chat) that the file was shared with user \persc and promptly
notifies the Metadata Server, which adds this detail to its metadata.

Once \persa is done, its local Metadata Server has been updated with three new objects: the
CSV file, the corresponding Excel file, and Slack’s copy of the Excel file.
There is a data relationship linking all three and metadata informing us
that the original CSV came from \persc and that the final Excel file was also
shared with that same person. If \persa wanted to remember what happened to
the data from the original CSV from \persc, they could query their local
personal Namespace Server, which would track down this history by querying the Metadata Server metadata.

\subsection{Sharing the spreadsheet}

\persc receives the Excel file from \persa via Slack on their phone, a
sequence of metadata events similar to those described earlier takes place,
except the phone does not run a local Namespace Server or Metadata Server. \persc now uploads the file to
the company's cloud drive (4). The Metadata Server (by way of the Attribute Service) reflects the creation of a
new object and records its remote location. The use of a company-wide namespace
and metadata service enables \system to record that the file in the cloud drive
is, in fact, a copy of the one received via Slack.  Further, \persc informs
their personal Namespace Server that they wish to notify \persa about all updates to the file
on the cloud drive. Thus, whenever an Attribute Service sends updated attributes to the Metadata Server,
\persc receives a notification.

The sharing relationship between the personal Namespace Server of \persa and \persc, and the
exchange of the relevant cryptographic credentials, would have been set up
earlier.

\subsection{Data origin and delete requests}

When the compliance officer asks about the origin of the data, \persc can query
the corporate Namespace Server to obtain the complete history of the report. This includes the
spreadsheet from which the report was derived and the e-mail or Slack messages
that transmitted the files.

The corporate Namespace Server was configured to be aware of the locations of the
collaborating users' personal Namespace Server. Moreover, because of the activity contexts
captured by the Activity Monitor, \system is able to identify documents that were created
during any activity involving the customer whose data must be deleted. Starting
from these documents, and by using the relationship of documents, \persb was
able to find all relevant objects and delete them, including the e-mail and
Slack messages.

\persa would have configured their personal Namespace Server to allow sharing of the metadata
associated with \persc with their corporate Namespace Server, and \persc would configure their
personal Namespace Server similarly. As a result, when \persc issues to the corporate Namespace Server a
query asking to trace the origins of the data in the final report, the corporate
Namespace Server is able to return all the history tracing back to the original CSV file.

Note that unlike existing systems, \system is able to efficiently find related
objects across storage silos. Operating systems already provide users with
indexing services to accelerate search of local files. This search can be made
cross-silo by mounting and enabling indexing on network shares (e.g., Windows
Desktop Search), or by interfacing with specific applications such as e-mail
(e.g., MacOS Spotlight, or Android search). The problems with indexing a
large remote storage repository are resource limitations such as bandwidth. In contrast,
\system addresses these limitations by delegating indexing and storage to one or
more services.

Namespace Servers are responsible for providing efficient search functionality.
\system uses Attribute Services to keep attributes up to date with object
modifications. Lastly, \system
supports coordinated search among one or more local and remote Namespace Server, allowing, for
example, a user to search across both their local Namespace Server as well as their employer's
Namespace Server.

\endinput

\section{Research Questions}
\label{ch:arhitecture:sec:research-questions}

Given the broad research questions that I provided in
\autoref{ch:research-questions}, it is important to also explain how my proposed
architecture is motivated to answer those questions.

\begin{comment}
% Moved to the original RQs
The questions themselves that are core to my proposed thesis work revolve around
exploring and broadening the earlier idea of an \emph{activity context}.
However, activity context by itself does not replace the other prior work that
has added semantic understanding around files, e.g., semantic file systems,
formal concept analysis, tag file systems, graph file systems, etc. Thus, my
proposed system includes support for this prior work but extends it by
augmenting previous mechanisms with this additional rich usage information that
I call ``activity context.''
\end{comment}


\subsection{Events}

Research questions \ref{rq:define-ac} and \ref{rq:capture-ac} relate to
identifying and capturing the specific events on a user's device that yield the
insights necessary to answer research question \ref{rq:leverage-ac}.  While
prior work~\cite{provsearch,vianna2014a} has proposed partial answers to these
questions, I will do the additional work necessary to address my own thesis by
looking broadly at combining this prior work as well as considering adding
additional contemporaneous events that seem likely to be beneficial.

My architecture is sufficiently general that it envisions having a stream of
event information that can then be used to better understand the context in
which digital objects are created and used.  While some of this has been
suggested previously under the guise of ``personal digital
traces''~\cite{vianna2019searching}, it is important to provide a simple
mechanism for extending the data collection and providing it in a uniform
fashion that can be used to further explore development of new tools. While I
have suggested a couple of potentially novel ideas, such as identifying the
music that was playing, I expect there will be other information that can be
collected.  For example, eye tracking information could be used to determine
more about the relationship between various elements.  Other sensory information
could be incorporated: the smells of cooking, the sounds of others within the
user's environment, their conversations, etc.  These become an extension of the
personal digital traces into a \emph{personal environmental trace}.

A key challenge here is to find a common language for describing these events;
while there is some effort at doing so in the personal digital traces work, it
is likely that I will need to extend that model to accommodate the various data
sources I have described (see \autoref{table:useful-information}.)


I envision a key aspect of this work will be to consider the personal digital
traces work and design an application programming interface (API) that I will
use for collecting the event traces into my implementations against this
interface.

Providing support for collecting, storing, and making this event information
accessible is a key element of the work that I propose providing in support of
my thesis.

\begin{comment}
\subsection{Improved Access}

Research question \ref{rq:improve} focuses on verifying that \emph{activity
    context} yields better outcomes.  Mentally, I consider this question to be
determining if the \emph{finding} outcome for digital objects is improved: does
it take less time to find things, does the user find things they had forgotten
about but wanted to find, does it improve their satisfaction, and does it
decrease them reaching the point of giving up their search.

While I do not propose answering this research question as part of my work for
this thesis, I do note that my proposed architecture provides the systems level
framework to better explore this space by enabling a simple model for collecting
activity context data as well as accessing that information.  Thus, by making
the collection and dissemination easier, it will be simpler for other
researchers to explore the effectiveness of activity context.
\end{comment}


\subsection{Compatibility}

I need to evaluate how existing applications can take advantage of the richer
meta-data capabilities I will make available in order to fully answer research
question \ref{rq:leverage-ac}.
Prior work has suggested ``virtual
directories'' as one way to achieve this and that may be sufficient, however
there are other models that have also been suggested, including graph name
spaces~\cite{gfs}, alternative browsers~\cite{9502515,collins2010escaping},
as well as hybrid models in which automated classifiers are either explicitly or
implicitly determined~\cite{10.1145/3209900.3209911}.

\subsection{Meta-Data Query}
\label{ch:architecture:sec:rq:subsec:meta-data-query}

Research question \ref{rq:meta-data-query} relates to asking how we provide
activity context to enable others to exploit it in building new tools and
exploring related research questions for refining and developing further tools.
There is a strong body of prior work regarding meta-data queries of both static
and dynamic
sources~\cite{Strong,revol2011universal,smartstore,pindex,federatedMetaData,huo2016mbfs,Suguna2015,Parker-Wood2014,watson2017exploring,leung2009magellan,leung2009spyglass,niazi2017hopsfs,van2011efficient}.

My architecture proposes using a meta-data service for storing and accessing
activity context information.  I did this precisely to allow support for a
flexible and fast query mechanism and thus the architecture does provide the
tools necessary to answer this research question as part of the work that I
propose in supporting my thesis.

\subsection{Privacy}

Research question \ref{rq:privacy} raises the issue around protecting sensitive
personal information.  My architecture permits a secure implementation of collecting detailed personal
information by giving the user the ability to restrict the sharing of that
information.  The architectural model envisions a more generalized framework in
which service providers can use existing or new authentication and encryption
mechanisms for supporting both ``Context as a Service'' models for service
providers, as well as enabling selective sharing of sensitive information with
other users.  While I do not propose exploring this research question broadly,
the architecture provided is sufficient to answer it narrowly.  Such a narrow
definition does not preclude future research into a more permissive model that
remains consistent with the proposed architecture.



%\chapter{Model}
\label{ch:model}

\begin{epigraph}
    \emph{For there is nothing lost, that may be found, if sought.} --- Edmund
    Spenser, \emph{Finding the Faerie Queene}, 1590.
\end{epigraph}



\reto{The model is one part of your contributions.
    Is it custom to have one contribution ready for your proposal?
    Shouldn't that go as 4.1 ?
}

\begin{quotation}
    \input{thesis.tex}
\end{quotation}

Testing this thesis requires an understanding of \emph{naming} and how I
envision pragmatics fitting as part of the larger naming model. As I have
pointed out when reviewing the background (\autoref{ch:background}) it is clear
that hierarchical models of naming were never intended to be the \emph{end} of
namespace development, though one could not tell that this is the case from
the storage systems that we use today.

The cracks have been showing for decades, the edifice of

\section{Naming}
\label{ch:model:sec:naming}

An essential part of my model is that it separates naming from storage: the
name does not need to incorporate the storage location.  This is not a new
observation, but it is one that is central to my thesis because addressing it is
an essential part of constructing the rich naming networks, which are central to
my thesis.

Thus, it is important to understand the fundamental purpose behind naming.
Salter'z work, which I describe in both \autoref{ch:introduction} and
\autoref{ch:background:sec:storage} captured the state of naming in 1978.  In
general, naming in 2021 is mostly the same as what Saltzer describes.

This paradigm is breaking down, however.  Recent popular press has discussed
that university instructors are now finding that students no longer understand
the folder/file
metaphor~\footnote{https://www.theverge.com/22684730/students-file-folder-directory-structure-education-gen-z}.
Instead, they have the \emph{laundry basket} metaphor, where all their content
goes into a small number of large collections.

\includegraphics[width=0.95\textwidth]{figures/laundry-basket-with-cat-on-top.png}

Given the prior work, the fact the exiting folder/file paradigm has finally
broken down --- at least for the younger generation.  It \emph{also} suggests an
important way to consider my own thesis, namely can the introduction of
pragmatic naming prove useful to both those using the old paradigm of folder and
file, or the new paradigm of a laundry basket?

\begin{comment}
One challenge in developing this work is the resistance of the computer systems
community to considering \emph{naming} as a systems problem.  One common
response is to posit that this is an \ac{HCI} problem.  While there are
certainly aspects of data visualization that \emph{are} \ac{HCI} problems, a
review of the current state of affairs suggests this is clearly untrue.

\begin{itemize}
    \item The computer systems community has \emph{already} taken over at least
          some aspect of naming; it is dereliction of responsibility to now insist the
          current state of affairs is not tightly tied to earlier choices by the
          computer systems community.

    \item The \ac{HCI} community has been evaluating, reviewing, and proposing
          potential alternatives to the current computer storage naming paradigm
          for \emph{decades} but these solutions fall short of being viable because it
          is not sufficient to change a single application --- even something as core
          to the problem as the file browser --- to resolve this problem.

    \item The computer storage community does admit to the challenges here and
          have implemented system-specific solutions to improving naming, but human
          users do not live in a reality in which such narrow solutions resolve the
          naming challenges of the larger system --- it is not sufficient to argue
          that any storage system has solved this problem when users are called upon
          to use multiple storage systems on a daily basis.
\end{itemize}
\end{comment}

My survey of the background work (\autoref{ch:background}) makes it clear that
the hierarchical directory structure was not considered to be the best possible
naming system, but it was considered to be a reasonable first step.  Similarly,
we know from that prior work the embedding of location data within the human
naming scheme was not productive.

The problem has become untenable at point: it is now as easy to just hunt
through a large collection of files without hierarchical structure as it is to
try and organize them.  Search may be the correct answer, though Whittaker
certainly argues that it is not --- that humans prefer to \emph{navigate} than
search~\cite{bergman2019search,bergman2019factors}.  Perhaps this work is now
dated that younger computer users have themselves become uninterested in
hierarchical organization structure at all.

One problem with search currently is that there is no universal search.
The \ac{HCI} community has also observed that cross-silo storage has become the
\emph{reality} of most computer users~\cite{Thereska2013}: ``Through user
studies and measurements, we find that users and application developers
increasingly have to deal with a \emph{de facto} distributed system of
specialized storage containers/file systems, each exposing complex data
structures, and each having different naming and metadata conventions, caching
and prefetching strategies and transactional properties. First, there is tension
between the traditional local file system and cloud storage containers.
Local file systems have high performance, but they lack
support for rich data structures, like graphs, that other
storage containers provide. Second, distinct cloud storage
containers provide different operational semantics
and data structures. Transferring data between these containers
is often lossy leading to added data management
complexity for users and developers.''

The idea of separating the namespace from storage has previously been
suggested~\cite{mogul1986representing,placeless-tois}.  The concept of using
semantic meaning as part of naming is also not new~\cite{gifford1991semantic}.
However, much of this is \emph{inward focused} on the file itself.  The
attributes of the file (sizes and timestamps) or characteristics of the contents
of the file (semantic meaning).  Prior work has only lightly included the concept of
\emph{context} of how the files have been used, such as Placeless document's use
of the ``process''~\cite{dourish1999getting}.

Instead, my proposed concept of context is much broader: what \emph{else} was
happening within the system that might have been related to particular files:
``Show me code that I wrote while listening to \emph{Lie to Me} by Depeche
Mode.'' We might want to do that because the human mind works by association.
This was Bush's \emph{point} in his 1945 article: humans use associative context
to locate things.  This can then be used to augment other data sources,
including file meta-data, semantic information, tags, extended attributes,
properties, and any other existing forms of information that are available.
This rich data can then in turn be used to create various types of namespaces
including the ``virtual directories'' proposed by
Gifford~\cite{gifford1991semantic} but also to allow exploration of other
potential interfaces including faceted search and graph browser interfaces.

\section{Research Questions}
\label{ch:model:sec:research-questions}

\tm{Here is where I should put the research questions that I seek to answer,
    since those will drive the choice of models.
}

I consider the following research questions as part of evaluating my
thesis:

\begin{enumerate}

    \item \label{rq:pragmatics} Does adding \emph{activity context} improve
          associative access to files beyond the benefits of \emph{semantic
              context}~\cite{gifford1991semantic}.

    \item \label{rq:no-hierarchy} Given a cross-silo storage architecture, how
          do we allow existing applications built to rely upon an hierarchical name
          space to work properly? \tm{Maybe the right answer here are the virtual
              directories of Gifford, but I'm not sure that it is the right answer,
              either.  This question is likely too broad.}


    \item \label{rq:events} What system events are useful in establishing
          relationships between objects?

    \item \label{rq:hci} How can we provide the \ac{HCI} community with the
          additional contextual information they require to build better tools?

    \item \label{req:privacy} How can we provide distributed meta-data services
          while preserving privacy?

    \item \label{req:security} How do we allow securely sharing meta-data?

\end{enumerate}

\tm{Note that a dynamic list of the research questions is maintained at
    \url{https://wamason-my.sharepoint.com/:x:/p/tony/EYVA1mlCLd5HkftoizGx-ksBr6KodPWcg3bYq3e5mhwf4w?e=icUFtU}.
}


\section{\system Architecture}
\label{ch:model:sec:architecture}

\tm{This is where the model emerges.  I'll start with the Kwishut model and then
    work through it to ensure that it provides what I need to answer those research
    questions.
}

\subsection{Services}
\label{ch:model:sec:architecture:subsec:services}

\tm{TBD}

\endinput

\subsection{Computer Storage Naming}

\MIS{What is this? What am I getting and putting? It seems to me that there is a
    high level narrative missing: Different silos us different kinds of names: HNS,
    key-based put/get, keyword-based search (e.g., the web), I don't even know
    exactly what I would call github, hierarchical?  So the work here is figuring
    out what the categories of namespaces are, how they are similar, how they are
    different and why having many is a problem (e.g., if everything were simply
    explosed as mount points, multi-silos might not be a problem; we'd still have
    problems finding things, but not due to how names are constructed).
}

\tm{This is a placeholder for this information.  I need to figure out where the
    right place to put this is.
}

In addition, network namespaces have proliferated, often distinct, sometimes
providing an hierarchical interface (e.g., the \ac{FTP}) and sometimes providing
a key-value interface (e.g., the HTTP GET/PUT operations).  The proliferation of
disjoint naming silos makes it more challenging to find related information
across silos.  Even with an hierarchical name space there is no simple mechanism
for finding related information that is physically part of distinct silos
because the relevant portion of the namespace is tied to the corresponding
portion of the namespace.

Thus, the underlying location of a given data object is defined by its storage
silo. The hierarchical name space obscures this somewhat but the underlying
system is still a composite of the namespaces on existing storage.

Some of this is historical: the only reliable place where applications can store
context is within the fully qualified path name to a file.  While \emph{some} file systems
support extended attributes, while others do not.  Even file systems that do
support extended attributes have subtle challenges associated with them,
such as differing limitations on them across file systems, the fact they are
lost when moved between file systems that support extended attributes and file
systems that do not, and how, unlike other file system operations, there is not
even a somewhat uniform API for managing extended attributed.  UNIX-like systems
have the various ``xattr'' operations but Win32 applications on Windows have
only an indirect mechanism for managing them (via backup APIs) though Windows
has its own very different set of native system calls for reading, writing, and
enumerating extended attributes on file systems that do support them.

Thus, the most common solution is to use the file name to embed context
(``meta-data''). \tm{I thought this was one of the insight }


Using file names to
embed context (``meta-data'') is well-understood~\cite{guo2012burrito}.  Yet,
embedding context in file names does not solve the problem of storing dissimilar
types of data in the same location, which defeats the purpose of having
specialized storage.  Similarly, it does not solve the problem of placing data
objects in their optimal storage location while preserving their relationship.

\tm{I'm not sure I like this example, but I'll leave it here for the time being.}
The need for this is increasingly clear.  For example, Qumulo has created a
distributed file system by constructing a cross-silo hierarchical namespace
separated from actual storage location with predictive data migration to drive
better support for huge data collections. Qumulo's approach attempts to address
the cross-silo approach but continues to rely upon the hierarchical name space
to do so.

\begin{figure}[!tbh]
    \centering
    \includegraphics[height=5cm]{chapters/figures/xkcd1459.png}
    \caption{XKCD: Never Look in Someone Else's Documents Folder}
    \label{fig:xkcd:1459}
\end{figure}

Common types of file systems include:

\begin{description}
    \item[Media] --- such file systems manage persistent storage, such as solid
        state disk (``ssd''), rotating magnetic media (``hard disk''), magnetic
        tape, and rotating optical storage (such as a CD-ROM). Media file systems
        manage the data storage provided by the media device.  A media device can be
        constructed by combining portions of multiple media devices, such as is done
        with \ac{RAID} storage.  Media file systems also manage the correspondence
        between the file system's logical storage elements, which are typically files and
        directories, and

    \item[Network] --- such file systems utilize a network protocol for
        transferring data from one or more computers accessed across a network so
        that it can be consumed by local applications.  Examples of network file
        systems include \ac{NFS}, \ac{AFS}, and \ac{CIFS}. Network file systems
        manage moving data between computer systems over the network, presenting a
        compatible name space, and manage data caches and coherence between local
        caches and remote storage.  Sometimes these are also referred to as
        \emph{distributed} file systems.

    \item[Pseudo] --- such file systems provide structured information extracted
        from non-storage locations and present it via an hierarchical name
        space.  For example, the \emph{proc} file system (\tm{From Plan 9?})
        presents information about the running operating system.  A pseudo file
        system manages its namespace as well as retrieving or storing data that
        relates to the corresponding system state.

\end{description}


\tm{Note that this is the point at which I have stopped during the re-write of
    this section.  Content beyond this point is being reworked.
}




In \autoref{ch:background:sec:challenges} I attempted to capture key aspects
of how naming is used by human users and from that proposed a basic list of key
elements to consider in a comprehensive naming model: identity, location,
relationship, characteristic, and context.  While this is a good place to begin
my analysis, more is needed to construct a robust model.  For example, one of the
challenges of naming is that it is \emph{dynamic}: the storage
location of a given data object might change.  This often means that the
\emph{name} also changes, because the name encodes location.  This is no more
intuitive to a human than it would be to insist someone change their name each
time they moved to a new home.

In other words, the \emph{object} is not different even though its storage
location changes.  Neither users nor applications care about this
specific detail until they go to actually retrieve it.  In the current model,
rather than bein forwarded to the correct object an error is returned indicating
that the object is no longer accessible.

To facilitate developing my naming model, I rely upon several use-cases that
have arisen during the course of my research around this topic, both working
with collaborators as well as discussions for ordinary users that are unfamiliar
with my research area. \tm{Margo suggested that I be specific here by pointing
    to particular people in footnotes for the time being.  I defer that for the
    moment but it is a good thing to add as I continue editing.}

\section{Use Cases}
\label{ch:model:sec:use-cases}

%%%%%%%%%%%%%%%%%%%%%%%%%%%%%%%%%%%%%%%%%%%%%%%%%%%%%%%%%%%%%%%%%%%%%%%%%%%%%%%%%%%%%%%%%%%%%%%%%%%%%%%%%
\begin{table*}[!tbh]
    {\renewcommand{\arraystretch}{1.3} %<- modify value to suit your needs
        \begin{tabular}{p{0.2\textwidth}p{0.4\textwidth}p{0.4\textwidth}}
            \hline
            \textbf{Feature}                                                                                                                                        & \textbf{Existing Technologies} & \textbf{No Solution} \\
            \hline
            \usecaseactivitycontext                                                                                                                                 &
            timestamps and geo-location, image recognition, browsing history, ticketing systems, application-specific solutions like Burrito~\cite{guo2012burrito}. &
            Link related activity across apps, record  browsing history and chat conversations relevant to the creation of the data object, storing it in ways that are secure and compact.
            \\
            %
            \usecasecrosssilosearch                                                                                                                                 &
            Search by name, creator, content across silos,
            app-specific searches (e.g., Spotlight)                                                                                                                 &
            Unified search across all kinds of storage, including file systems, object stores, apps and devices
            \\
            %
            \usecasedatarelationship                                                                                                                                &
            De-duplication of documents, versioning of specific files, git ancestor relation                                                                        &
            Explicit notion of data identity, tracking different versions across different silos as data is transformed
            \\
            %
            \usecasenotifications                                                                                                                                   &
            File watchers (INotify), synchronization status, manually inspecting modified time                                                                      &
            Ability to subscribe to specific changes on attributes
            \\
            %
            \usecasepersnamespace                                                                                                                                   &
            Hierarchy plus hard/soft links. Use of tags.                                                                                                            &
            Creating personalized namespaces with with flexible data organization and views
            \\
            \hline
        \end{tabular}
    }
    \caption{Use-case driven functional requirements.}
    \label{tab:usecases}
\end{table*}
%%%%%%%%%%%%%%%%%%%%%%%%%%%%%%%%%%%%%%%%%%%%%%%%%%%%%%%%%%%%%%%%%%%%%%%%%%%%%%%%%%%%%%%%%%%%%%%%%%%%%%%%%

To motivate the model I propose for \system, I first start with a series of
potential use cases. \tm{Note that I've started with the two from the original
    \system paper, but I think it might be worthwhile to add one or two more to
    provide a more well-rounded model.}

\begin{description}
    \item[Data Processing]
    \item[Compliance]
    \item[Memex]
    \item[Asset Management]
\end{description}

Using these uses cases as motivation, I propose that \system support the
features as shown in \autoref{tab:usecases} in greater detail in \autoref{ch:model:sec:features}.

\subsection{Data Processing}
\label{ch:model:sec:use-cases:subsec:data-processing}

\tm{This is from the HotStorage paper submission}

\persa and \persc are preparing a report summarizing their work on a data analysis project for a customer.
\persc sends an email to \persa containing a CSV file with original data.
\persa opens this document in Excel, formats and filters it, adds additional data from a corporate storage silo,
and then returns the Excel document to \persc on Slack.
\persc is away from their desk when it arrives, so they open it on their phone, uploading it to a cloud drive.
\persc then sends the link to \persa for editing with update notifications.
Finally, \persc sends a PDF of the report to the compliance officer who promptly asks, ``Where did this data come from?''

\subsection{Compliance}
\label{ch:model:sec:use-cases:subsec:compliance}

\noindent\textbf{Delete Request:~}
Some time later, the compliance officer requests that all documents containing a customer's data must be deleted.
To help with finding all relevant customer data, \persb joins the project and examines the report and requests the original data from which it was produced.
\persa remembers that they gave the original data to \persc shortly after
collecting it, but does not remember the name, location, or even how the
relevant files were transmitted. Thus, \persa has to manually search possible
locations and applications, sensing references to documents to \persb, who then
starts organizing these files to methodically identify the ones that might
contain the customer's data. In the process, many of the other team members'
references to the documents stop working.

\tm{
    Discussion with Ada: how do people do this \textit{already}?  Why are those
    solutions insufficient?  Some sites are accessible and others are not.  How
    do they \textit{prove} they are GDPR compliant?  What about when people move
    from dynamic to static memory?  Could I use storing the hash value as a
    mechanism for motivating this because it facilitates finding things.  Much
    like the Apple content hash for child p0rn.  How about extracting text from
    pictures to avoid censorship?
}

\MIS{
I think the current answer is 'not very well' -- it's an area of current
research and right now, I'm 95\% certain that aggregated data that was
influenced by individual data gets ignored. [Michael is just now submitting
a paper with MSR and Mickens on how to do better, but it's not like there
are solutions; instead people use a very narrow definition of what a user's
data really is.
}

\subsection{Memex}
\label{ch:model:sec:use-cases:subsec:memex}

The ``memex'' is a device posited by Vannevar Bush in 1945~\cite{bush1945we}:

\begin{quotation}
    \emph{``Consider a future device for individual use, which is a sort of mechanized
        private file and library. It needs a name, and, to coin one at random,
        "memex" will do. A memex is a device in which an individual stores all his
        books, records, and communications, and which is mechanized so that it may
        be consulted with exceeding speed and flexibility. It is an enlarged
        intimate supplement to his memory.''}
\end{quotation}

While the world wide web is certainly one interpretation of his forward thinking
article, the system he describes is also highly personal and appears to focus
not only on finding things but also capturing the context in which they are
found.

\tm{Seems like the key here is to extract the salient factors that would impact
    this model.  Note that \emph{context} is a remarkably critical element of
    understanding language in general, and naming particularly.  In Linguistics,
    they study \emph{pragmatics}, which are distinct from \emph{semantics}, and
    an important aspect of this is the context in which something --- including
    a name --- is used.  I need to explore this aspect of linguistics further
    because I have this sense that understanding the gap between the two naming
    systems is helpful in understanding why computer storage falls short.
}

Thus, when considering my naming model, I can draw upon the needs of Memex to
assist in ensuring the model is sufficient to meet the issues raised by
Bush~\cite{bush1945we}:

\begin{description}
    \item[Selection] --- ``The prime action of use is selection, and here we are
        halting indeed.'' The key here is being able to \emph{filter} items of
        interest at a given time. This becomes important as the number of objects
        being considered grows.  While computers are somewhat faster than they were
        in 1945, he points out the general problem of scaling and the need to be
        able to limit the actual size of the search space.  This is a very real
        consideration for our naming system: scalability.  Brute force search is not
        sufficient, as that is what we have \emph{now} and even as fast as storage
        systems are today this is not tenable.

    \item[Association] --- ``Our ineptitude in getting at the record is largely
        caused by the artificiality of systems of indexing. When data of any
        sort are placed in storage, they are filed alphabetically or numerically, and
        information is found (when it is) by tracing it down from subclass to
        subclass. It can be in only one place, unless duplicates are used; one
        has to have rules as to which path will locate it, and the rules are
        cumbersome. Having found one item, moreover, one has to emerge from the
        system and re-enter on a new path.

        ``The human mind does not work that way. It operates by association. With one
        item in its grasp, it snaps instantly to the next that is suggested by the
        association of thoughts, in accordance with some intricate web of trails
        carried by the cells of the brain. It has other characteristics, of course;
        trails that are not frequently followed are prone to fade, items are not
        fully permanent, memory is transitory. Yet the speed of action, the
        intricacy of trails, the detail of mental pictures, is awe-inspiring beyond
        all else in nature.''

        Thus, the key take-away here is to find associations.  This is part of
        the motivation for \emph{activity context}.

    \item[Trails] --- ``And his trails do not fade. Several years later, his
        talk with a friend turns to the queer ways in which a people resist
        innovations, even of vital interest. He has an example, in the fact that the
        outraged Europeans still failed to adopt the Turkish bow. In fact he has a
        trail on it. A touch brings up the code book. Tapping a few keys projects
        the head of the trail.''

        Thus, the key take-away here is to be able to show at least one kind of
        relationship: a ``trail,'' which seems to be similar to provenance.

\end{description}



\subsection{Asset Management}
\label{ch:model:sec:use-cases:subsec:asset-management}

One recurring theme in my conversations with the visual arts community, has revolved
around the management of \emph{assets}. This term is broadly used: web pages are
constructed of assets such as text, graphics, style sheets, and XML schema, all
of which are combined together to form a unique view of the given web page that
is relevant in the specific context of the viewer who may be using a graphical
computer, smartphone, tablet, or text-based web browser.

Similarly, computer games combine their own version of assets. Unity is a highly
popular framework for constructing interactive computer games and their website
defines an asset: ``Shorthand for anything that goes into a video game ---
characters, objects, sound effects, maps, environments,
etc.''~\footnote{https://unity.com/how-to/beginner/game-development-terms}

There is no single hierarchical organizational structure for assets that
satisfies the needs of this type of creative endeavor: a single game asset could
be classified by a myriad of characteristics.  When a game developer is looking
for a particular asset, they are often focused on those characteristics.  When
an audio engineer is attempting to construct specific sound effects they could
be looking for: the duration, number of channels, channel mapping, sampling
frequency, bit depth, instrument, or dynamic range for example.  It is clear
that what doesn't work in such a situation is a single directory filled with all
of the assets: that isn't useful.

Indeed, this use case seems to focus on being able to identify and use the
\emph{properties} of a given data object.  Simpler examples of this might be how
one organizes documents for accounting purposes: bank statements could be sorted
by the bank from which they came or the month that they cover.  In my
experience, accounting users actually will store copies of such a file because
they need to be able to identify it in \emph{both} formats.  Another similar
example is ``how do you organize your music collection?''  A music collection
could be organized by artist, or album, or year of release, or publisher or
lyricist, or genre. When we have a physical copy of recorded music (e.g., an
eight-track tape~\footnote{Invented in 1964 and thus as old as Multics, which
    adopted the hierarchical file system structure.} we are limited to the ways in
which we can organize it.  Digital files have no such limitations; that
limitation is an artifact of the current naming system.

\tm{What requirements are imposed by this use case?  It's relatable, but does it
    really impact the design?
}

\section{Features}
\label{ch:model:sec:features}

\system must provide the following features to meet the needs of the use cases:

\begin{description}
    \item[Activity Context] --- this is a mechanism by which we capture
        information for understanding the context in which data is created,
        transformed, and accessed.  This is not application or storage silo specific
        information. Examples of this might include timestamps, application specific
        meta-data, history information, provenance information, etc.

    \item[Search] --- while search is not the only use we envision for \system,
        it is one of the mechanisms that we expect to enable and should support
        searching across storage silos and exploiting the richer meta-data
\end{description}

\section{Architecture}
\label{ch:model:sec:architecture}

\tm{I'm starting with the \emph{Kwishut} architecture from HotStorage 2021.}

\begin{figure}[!tb]
    \centering
    \includegraphics[width=0.45\textwidth]{reference/hotstorage21/figures/Naming5-legend.png}
    \caption{\emph{\system} Architecture (see \autoref{ch:model:sec:architecture:subsec:services}).\\%
        Grey boxes indicate new components. AM=``Activity Monitor'', MS=``Meta-data Server'', UNS=``Update Notification Server'', NS=``Namespace Server''.
    }
    \label{ch:model:fig:arch}
\end{figure}

%\reto{Using meta-data service that must inter-operate (federated meta data) and relationships are not first-class citizen, cannot glue the meta-data service together with naming services to enable the things we want to do. }
%\reto{the storage location is independent on the notion of related files: meta-data service treats relationships as first-class citizens. }
%\reto{get the attributes out of the silos --> currently: this is done manually}

\emph{\system} is a family of services that enable sophisticated search and naming capabilities.
The key features that differentiate \emph{\system} from prior work are:

\begin{enumerate}[1)]
    \item incorporating object relationships as first class meta-data because
          prior work has not done so which interfers with the ability to ensure
          meta-data and naming services work together; and
    \item federating meta-data services, which is necessary to ensure
          efficiency, scalability, and security; and
    \item recording activity context, which is necessary to provide insight into
          how the data is used dynamically (over its lifetime) rather than statically
          (at the point of creation or last update); and
    \item integrating storage from multiple silos, which is necessary to clearly
          distinguish the storage characteristics that are focused on storage service
          optimization from the usage characteristics that establish associative
          context; and
    \item enabling customizable naming services, which are needed because we
          need the flexibility to support a broad range of existing as well as
          innovative new storage services.
\end{enumerate}

Data continues to reside in existing and to-be-developed storage silos.
\emph{\system} interacts with these silos, collects and captures metadata, and
provides a federated network of metadata and naming services to
meet the needs of users with the use cases in \S \ref{tab:usecases}
being our initial evaluation of our architecture and design.

\subsection{\emph{\system} Services}
\label{ch:model:sec:architecture:subsec:services}

Figure \ref{ch:model:fig:arch} illustrates the \emph{\system} architecture. \emph{\system} allows for
different deployment scenarios. The services can be run independently, they
can be co-located and bundled together to run on a local device, integrated into
an OS, or available as network-based services.

In the discussion below, parenthesized numbers and letters refer to the arrows
in Figure \ref{ch:model:fig:arch}. There are five main components:

\begin{enumerate}[1)]

    \item \textbf{Metadata servers (MS)} are responsible for storing attributes
          and provide a superset of capabilities found in existing metadata
          services~\cite{federatedMetaData,smartstore}. Users can register an MS with
          activity monitors or attribute services, which allows the MS to receive
          updated attributes from storage objects and activities (B). Thus, there can
          be multiple sources of attributes including the user itself. Metadata
          servers may retain the full or partial history of attribute updates or
          maintain only the most recent value.

    \item \textbf{Namespace servers (NS)}
          connect to one or more MS and use the metadata to provide users with a
          personalized namespace that allows both manual organization (i.e., a
          hierarchical namespace) and rich search capabilities.
          Users can register with an NS (R) that uses one or more MS to obtain relevant
          attributes from them (C). Additionally, users can be part of a corporate NS that
          allows sharing of their select metadata with other users via standard enterprise
          public-key cryptography.

    \item \textbf{Activity monitors (AM)}
          run on the user's devices. Their main function is to observe temporal relations,
          activity context, and relationships between objects on a user's device and
          transmit them to an MS (D).


    \item \textbf{Attribute services (AS)}
          extract attributes from storage objects and transmit them to an MS (B). An AS
          might be invoked on updates, run once or periodically. For example, a file
          system AS would update the object's metadata with basic attributes such as size
          or modification time. There can be many AS that extract more ``interesting''
          attributes, e.g., image recognition, similarity, or other classifiers.

    \item \textbf{Update notification server (UNS)}
          provides notification mechanisms. Users can register interest in changes of
          attributes or underlying storage and will receive a message on change events (A)
          to which they have access.

\end{enumerate}

\subsection{\emph{\system} working example}


To make the \emph{\system} architecture concrete, we revisit our use-cases from
\S\ref{tab:usecases} and walk through parts of it to illustrate how \emph{\system}
supports the various actions and events.

\noindent\textbf{Storing the e-mail attachment.}
\persa's act of saving the CSV file that \persc sent in email corresponds to the
creation of a new object on the file server silo, i.e., the file system (4). The
file server is \emph{\system}-aware, so the AS co-located with it extracts
attributes from the document and forwards them to the MS (B).

The AM on \persa's laptop detects that the CSV file came via company email from
\persc. It then captures the activity context identifying the relationship
between the e-mail and the CSV file and transmits it as additional metadata
about the CSV file to the MS (that already contains metadata extracted by the
AS). Moreover, because there is a company-wide namespace service, \emph{\system}
establishes that the e-mail attachment, the CSV in the file server, and the one
on \persc's laptop (from which the file was sent) are exact copies of each
other.

Many applications already record some form of activity context, e.g., chat
history, browsing history. Such histories provide a rich source of additional
metadata. Other activity context, specifically the relationship between objects,
such as the fact that a particular file was saved to a local storage device from
an email message, requires more pervasive monitoring as found in, e.g., whole
provenance capture systems~\cite{camflow}. \emph{\system} is agnostic about the precise
data that comprises activity context, but allows for storing and accessing
activity context as metadata.

\noindent\textbf{Creating the Excel file.}
Next, \persa opens the CSV file using Excel and stores it as a spread sheet.
This creates a new object. The AM detects that the newly created spreadsheet is
a conversion from the CSV file, either via a notification from \emph{\system}-aware
Excel or by monitoring the system calls executed on the local system. \persa
proceeds to modify the data by filtering it in Excel and saving the changes. The
AM records this event and updates the meta-data of the spreadsheet to record the
derivation-relationship. Ideally a \emph{\system}-aware version of Excel specifies to
the AM the exact type of the relationship (in this case a derivation); otherwise
the AM informs the MS about an unspecified data relationship by observing the
opening of a CSV file and a subsequent creation of the Excel file.

% \persa proceeds to upload the new Excel file on Slack, which triggers the creation of a new storage object as Slack creates a local copy, the addition of new metadata to MS via AS, and the addition of a \emph{copy} data relationship between the original Excel file and the Slack’s copy. The AM notices (by monitoring Slack chat) that the file was shared with user \persc and promptly notifies the MS, which adds this detail to its metadata.
% Once \persa is done, its local MS has been updated with three new objects: the CSV file, the corresponding Excel file, and Slack’s copy of the Excel file. There is a data relationship linking all three, and the metadata informing us that the original CSV came from \persc and that the final Excel file was also shared with that same person. If \persa wanted to remember what happened to the the data from the original CSV from \persc, they could query their local personal NS, which would track down this history by querying the MS metadata.

\noindent\textbf{Sharing the spreadsheet.}
As \persc  receives the Excel file from \persa via Slack on their phone, a
sequence of metadata events similar to those described earlier takes place,
except the phone does not run a local NS or MS. \persc now uploads the file to
the company's cloud drive (4). The MS (by way of AS) reflects the creation of a
new object and records its remote location. The use of a company-wide namespace
and metadata service enables \emph{\system} to record that the file in the cloud drive
is, in fact, a copy of the one received via Slack.  Further, \persc informs
their personal NS that they wish to notify \persa about all updates to the file
on the cloud drive. Thus, whenever an AS sends updated attributes to the MS,
\persc receives a notification.

% The sharing relationship between the personal NS of \persa and \persc, and the exchange of the relevant cryptographic credentials, would have been set up earlier.

\noindent\textbf{Data origin and delete requests.}
When the compliance officer asks about the origin of the data, \persc can query
the corporate NS to obtain the complete history of the report. This includes the
spreadsheet from which the report was derived and the e-mail or Slack messages
that transmitted the files.
The corporate NS was configured to be aware of the locations of the
collaborating users' personal NS. Moreover, because of the activity contexts
captured by the AM, \emph{\system} is able to identify documents that were created
during any activity involving the customer whose data must be deleted. Starting
from these documents, and by using the relationship of documents, \persb was
able to find all relevant objects and delete them, including the e-mail and
Slack messages.


% \persa would have configured their personal NS to allow sharing of the metadata associated with \persc with their corporate NS, and \persc would configure their personal NS similarly. As a result, when \persc issues to the corporate NS a query asking to trace the origins of the data in the final report, the corporate NS is able to return all the history tracing back to the original CSV file.

Note that unlike existing systems, \emph{\system} is able to efficiently find related
objects across storage silos. Operating systems already provide users with
indexing services to accelerate search of local files. This search can be made
cross-silo by mounting and enabling indexing on network shares (e.g., Windows
Desktop Search), or by interfacing with specific applications such as e-mail
(e.g., MacOS Spotlight, or Android search). The problems with indexing on a
large network storage repository are resource limitations such as bandwidth and
local storage that may render the system unusable during indexing. In contrast,
\emph{\system} addresses these limitations by delegating indexing and storage to one or
more services.

NS are responsible for providing efficient search functionality. \emph{\system} uses AS
to keep attributes up to date with object modifications. Lastly, \emph{\system}
supports coordinated search among one or more local and remote NS, allowing, for
example, a user to search across both their local NS as well as their employer's
NS.

\tm{Would it be useful to resurrect some of the excluded text from this section?}

\chapter{Evaluation}
\label{ch:evaluation}

\begin{epigraph}
    \emph{For there is nothing lost, that may be found, if sought.} --- Edmund
    Spenser, \emph{Finding the Faerie Queene}, 1590.
\end{epigraph}

An important aspect of supporting my thesis is to evaluate the system that I
have proposed in \autoref{ch:architecture} and ensure the proposed system
provides the information necessary to answer my research questions
(\autoref{ch:research-questions}.)

\section{Useful Events}
\label{ch:evaluation:sec:events}

Research question \ref{rq:define-ac} asks what constitutes an activity
context. Research question \ref{rq:capture-ac} asks how to capture this
information.  Research question \ref{rq:leverage-ac} asks how applications can
leverage activity context.  These three questions all rely upon identifying
which events are both useful and practical to collect and aggregate. While prior
work has identified events that are of interest I expect to find additional
potentially useful events to collect.  Thus, evaluating the overhead of adding
new events to the activity context is useful.  Such an evaluation of the
overhead associated with adding activity context would include: how difficult is
it to add a new activity context provider to the model and how difficult is it
to add support for the new activity context
in an existing tool.

I suggest these metrics because they reflect upon the performance of the
architectural model that I set out in \autoref{ch:architecture}.

The prior work, notably the personal information trace
work~\cite{vianna2019thesis}, has publicly available tools that could ease the
collection of data. I can then use my implementation against my own architecture
to ensure that resource cost of collection demonstrates the low overhead that I
expect for collecting such data.

Once I have demonstrated that my own tools implemented against my architecture
do not have substantial overhead, I can then look at the complexity of adding
additional data collection, using the suggestions in
\autoref{table:useful-information} as well as additional information that I can
identify as potentially useful.  Identifying such potentially useful activity
context data is an area in which I would expect collaboration could be quite
beneficial but I am not relying upon such collaboration to conduct my own
research.

An important metric in evaluating my architecture will be to consider both the
performance cost of adding additional data collection (measured in performance
impact) as well as the development effort (measured in code size).  Thus, I
propose collecting that information while I develop the tools and build
extensions to collect additional data.

\section{Usefuless of Activity Context}
\label{ch:evaluation:sec:activity-context}

Research question \ref{rq:leverage-ac} asks a critical question underlying my
thesis: that \emph{activity context} is itself useful.  There is at least one
prior work outside the systems field that indicates it
is~\cite{vianna2019searching} and thus I reasonably expect that I will be able
to reproduce their results.  It seems logical to consider their evaluation
methodology as one way to measure the effectiveness of our activity context
driven model.  This, however, is not an ideal fit as the personal digital traces
work was evaluated against synthetic existing benchmarks.  Thus, other prior
work that suggests other potential metrics including the time it takes for a
user to perform their search and whether or not the search itself was successful
(using the \emph{abandonment rate}).

Assuming that activity context is useful, a more traditional systems evaluation
seems justified: what is the cost of collecting and disseminating the activity
context, what is the time to process queries~\cite{ames2013qmds}, how difficult
is it to add additional activity context providers, and what is the potential
added complexity for applications to utilize \emph{activity context}.

\textbf{Note:} While I expect there will be substantial benefit to collaborating
with others interested in the human-computer interface (HCI) and information
retrieval (IR) potential for using my work, based upon consultation with my
supervisors I do not assume that this will be the case.  Thus, the possibility of
collaboration has the potential to provide considerable impact if my research
supports my thesis, my thesis proposal is not dependent upon such collaborative
work. Thus, I intend on being sufficiently flexible to take advantage of
collaborative opportunities that do arise, but also realistic in completing my
own work in order to complete my thesis.

\section{Backwards Compatibility}
\label{ch:evaluation:sec:existing-apps}

Prior work, such as with semantic file systems~\cite{gifford1991semantic}, has
been realized by using indexing services.  Similarly, personal digital traces
have been used to augment indexing services~\cite{vianna2019searching,Xu2014}.
These works have evaluations for the effectiveness of their solutions. Thus,
virtual directory solutions and indexing solutions have prior evaluations that
can be leveraged to develop a more extensive evaluation model.

With respect to my own thesis related work, the availability of this type of
indexing and/or virtual directory mechanism would be helpful in understanding
the costs and performance of my own architecture to ensure that it can
adequately meet the needs of my target tool-building community.

\section{Access to Activity Context}
\label{ch:evaluation:sec:providing-ac}


Research question \ref{rq:meta-data-query} asks how to provide activity context to
enable users and application developers to exploit the enhanced rich meta-data
of \system.  Much of this work
relates to performing efficient meta-data queries against a potentially large
collection of such data. There is a strong body of prior work regarding
meta-data queries of both static
and dynamic
sources~\cite{Strong,revol2011universal,smartstore,pindex,federatedMetaData,huo2016mbfs,Suguna2015,Parker-Wood2014,watson2017exploring,leung2009magellan,leung2009spyglass,niazi2017hopsfs,van2011efficient}.
The prior work includes a model for providing evaluation.  In addition,
file system meta-data query specific research has also created a framework for
evaluation that relates to the performance speed of meta-data
queries~\cite{ames2013qmds}.

In addition to the performance of such queries, I expect it will also be useful
to determine the generality and ability to form specific queries.  My
expectation is that these queries will not ordinarily be initiated directly by
users but it may be useful, as part of evaluating the interface, to determine if
human-provided queries are viable and the level of ease with which they can be
constructed.  I expect further refinement on how to evaluate the flexibility of
the query mechanism may be avoided by adopting an existing query
language~\cite{francis2018cypher,van2016pgql}.

\section{Privacy}
\label{ch:evaluation:sec:privacy}

Research question \ref{rq:privacy} asks about how to ensure the privacy of users
is preserved when capturing detailed personal information in the \emph{activity
    context} I propose recording.  Because I have explicitly stated that for my own
thesis work I will be assuming the user maintains complete control of their
activity context, I do not propose any specific model for evaluating this
security because it is \emph{by design} as secure as the user's own data.

Ensuring privacy of meta-data is an active research area, in terms of
extraction, dissemination, and
sharing~\cite{10.1007/978-3-030-72465-8_14,eskandarian2021express,budzko2019architecture}.
Thus, future work that is not envisioned as part of my thesis should be done in
a context where emerging work is used to evaluate privacy concerns of a more
general meta-data service.




%\chapter{Plan}
\label{ch:plan}

\reto{
    I recall, we had this to be structured in work packages (of course with some more information):\\%
    WP1: Literature Survey\\%
    - Duration: 3m\\%
    - State: Completed\\%
    WP2: Model definition\\%
    - Duration: 6m\\%
    - State: On going\\%
    WP3: storage silo implementation\\%
    - Duration: 6m\\%
    - State: not started. \\%
    WPN: write thesis\\%
    - Duration: 3m\\%
    - State: not started. \\%
}

\section{Contributions}
\label{ch:introduction:sec:contributions}

\begin{epigraph}
    \textit{I don't let the cleaners in\ldots because \textbf{I} know where
        everything in this room is.  All the books, the papers --- and the moment
        they start cleaning, those things get hopelessly organized and tucked away
        and I can never find them again.} --- Crown of Midnight, Sarah J. Maas
\end{epigraph}

This thesis proposal describes what I intend to contribute to our understanding
of pragmatic file systems:

\begin{enumerate}
    \item \label{contribution:model} An explicit model for what computer storage
          naming is: how and why we name data objects, what a name \emph{should be} in a
          computer storage context to correspond how names are used and formed \reto{that's probably going to be super related to Saltzers work... };
          and
    \item \label{contribution:arch} An architecture and design of a potential
          naming system that supports the naming model (\autoref{contribution:model}
          that is sufficiently flexible to work across the full range of current and
          prospective computer storage needs: in-memory compute systems, small
          devices, computer workstations, enterprise scale storage systems, and
          distributed, geo-replicated cloud storage systems, which permits the
          construction of new purpose-built storage systems with the strong naming
          support as well
          as integration of existing storage systems; and
    \item \label{contribution:purpose-built} Implement and evaluate a novel
          storage system implementation based on this design (\autoref{contribution:arch}) that
          demonstrate strong naming support within a single storage silo model; and
    \item \label{contribution:existing} Implement and evaluate an implementation
          of the richer naming system (\autoref{contribution:arch}) on an existing
          computer storage system; and
    \item \label{contribution:combined-system}  Implement and evaluate the
          combination of both the novel storage system implementation
          (\autoref{contribution:purpose-built}) and the enhanced legacy system
          (\autoref{contribution:existing}) into a unified system consistent with the
          model (\autoref{contribution:arch}).

\end{enumerate}

To provide these contributions I must implement a number of novel new
approaches to naming in computer storage systems.  In this chapter I provide a
high level overview of my plan to accomplish this.

\section{Research Questions}
\label{ch:plan:sec:research-questions}

\reto{Somehow I feel like the research questions should go into the problem statement section.
    (unless that's required to behere)
}

This plan is focused on answering the following research questions:

\begin{itemize}
    \item \textbf{Can a storage silo implementation that separates naming, meta-data,
              and storage management into distinct components be constructed without
              sacrificing performance?}\label{ch:plan:rq:naming-separation-performance}  It is easy to talk about \emph{functionality} but
          there is a real risk with a radically new architecture that it will
          not be usable due to the performance costs. Part of the challenge in
          answering this research question is to pick appropriate benchmarks.  My goal
          is to demonstrate there is \emph{no} significant cost associated with I/O
          performance (less than 5\%), and that there is at most modest cost
          associated with meta-data performance (creating, opening, deleting,
          renaming) versus existing systems (less than 10\%).  While I think it would
          be useful to work with someone interested in data visualization to
          demonstrate improvements in data analysis and access, I do not propose doing
          this as part of my own work.  I discuss this in more detail in
          \autoref{ch:plan:sec:storage-silo-implementation}.

    \item \textbf{Can a federated meta-data service be constructed to provide meaningful
              security guarantees against information leakage?}
          \label{ch:plan:rq:security} By increasing the exposure
          of information via enhanced/augmented meta-data there are clear benefits for
          compliance measures, but these are achieved at the expense of making
          additional meta-data available.  Part of the original design was to address
          these concerns and this research question evaluates the efficacy of that
          design.  To do this will require examining the security model my system
          uses, the potential information exposure, and at least a first-order
          analysis of the impact of information leaks.  Depending upon my analysis of
          the answer to this question it may be necessary to augment the \system
          security model to better protect the information.

    \item \textbf{How can \system, implemented as a cross-silo naming and meta-data service,
              provide measurable benefits to those using this system?}
          \label{ch:plan:rq:cross-silo-naming-benefits} The research
          question itself is vague at this point because while I can suggest
          anticipated benefits in terms of cross-silo data location it is difficult to
          identify the ``benefits'' let alone quantify them.  Despite this, both will
          be necessary, which makes this a ``high risk'' research question. \tm{I
              don't think this is a good research question as written, so I need to
              re-work this\ldots again.  The difficulty here is that measuring this is not
              going to be particularly easy, unless we turn this into an \ac{HCI}
              activity, such as by looking at ``abandonment rates'' when people start
              searching for things.  That \emph{may} be what is required, though, to make
              this argument because a performance eval of the software doesn't make much
              sense --- to what would I compare this?}

\end{itemize}

My model for \system is quite broad.  Building the pieces
to satisfy this architecture from scratch is a monumental undertaking that
outstrips the time available while I complete my PhD.

The next step then is to determine what is possible given current technological
components and what extensions are necessary to expand those components to
answer the research questions. I propose achieving this by combining those
existing technological components, identifying key elements that are necessary,
and analyzing how to minimize the risk involved by focusing on answering the
research questions rather than re-implementing functionality that \emph{already
    exists} and is sufficient to the task.

One key area in which I expect to develop new techniques is meta-data extraction
and construction from existing data and storing that meta-data into pre-existing
technology components.  I describe this further in the following sections.

\section{Storage Silo Implementation}
\label{ch:plan:sec:storage-silo-implementation}

While my objective is to construct \system, which will support multiple silos
combined with federated naming, activity, and meta-data services, my proposed
initial step is to do this on a single system consisting of two storage silos.

This section constitutes an initial ``straw person'' proposal for what I
anticipate constructing.  \MIS{I would call this the MVP --- how do we most
    simply demonstrate the key ideas behind this work in a constrained setting.}
In doing so I expect to evaluate at least two of the following research questions:

\begin{itemize}
    \item \textbf{Can the separation of naming from the file system(s) be achieved
              without compromising performance?} This is related to the broader question articulated earlier in this
          chapter (\autoref{ch:plan:rq:naming-separation-performance}.)

    \item \textbf{Can this two-silo naming system be used to demonstrate more effective
              data location?}  The original motivation is that rich, dynamic naming will
          enable ``finding my stuff''.  One approach to this would be to build on top
          of prior work done in our department related to \emph{data curation} to see
          if a richer naming system can be used to simplify that
          task~\cite{Vitale_2020}.

    \item \textbf{Can a non-traditional storage mechanism be effective when used as a storage silo
              within this separated naming and meta-data service model?}  This
          specifically speaks to the non-hierarchical name system idea that was
          previously proposed in ``Hierarchical File Systems Are
          Dead''~\cite{seltzer2009hierarchical}.  While not a primary motivator for
          the main thesis, I am intrigued by the idea of taking a persistent memory
          key-value store along the lines of ``Modernizing File System through
          In-Storage Indexing''~\cite{koo2021modernizing} where the key-value store is
          in persistent memory, which allows me to leverage my own prior
          work. \MIS{you should take a look at the Inversion file system of
              Mike Olson (circa 1990-3). It's basically building a file system
              on top of an RDBMS -- think about exactly what you are trying to
              do on the KV store that is different.
          }\tm{I have looked at inversion previously.  I do not think that
              maps onto the ideas I've expressed elsewhere about this approach,
              but perhaps this isn't a good research question for \emph{this}
              proposal.
          }

    \item \textbf{Does the explicit separation of meta-data and naming from the file
              system enable the creation of dynamic per-user name\-spaces?} Traditional
          name\-spaces are static in that the names of objects within them
          are not changed by the system, yet relationship based name\-spaces could be
          dynamic.  Further, the namespace of interest to me now may not be the same
          as the namespace of interest to someone using the same data but in the
          context of a different role --- by not tying the namespace to the data does
          it provide a more functional namespace?  \tm{This question is not fleshed
              out enough at this point.}

    \item \textbf{Does capturing activity context related to the way that data is
              constructed and used allow construction of enhanced data management
              mechanisms?} One obvious way of demonstrating this might be to show where it
          does permit that type of enhanced data management mechanism.  For example,
          this might be related to the data curation aspect of prior work. It is
          likely there are other similar data management activities that might yield a
          useful basis for finding information of interest.

\end{itemize}

I anticipate that this work will require developing specific components, which
largely correspond to the services described in
\autoref{ch:model:sec:architecture:subsec:services}.

One key aspect that is not well-defined in the model section but is
important to do as part of this work is to define what an \emph{activity
    context} is, how it is created, and how it is used within the system.  This will
be an important aspect of this work, even if the actual implementation is quite
simple, possibly capturing only a tiny amount of information.\MIS{I continue to
    claim that an AC is the result of a query applied to a rich MD store.}\tm{This
    is an area in which I think the anthropological linguistic perspective is
    useful. Activity context in a human sense would involve who is present, what
    you've discussed before that, where you are located, etc.  So my model is quite
    a bit different than a query capture, but this is a good thing to explore
    further.
}
Initial
techniques for doing this can likely be based upon existing information
gathering systems, such as \emph{eBPF}\tm{add to glossary, define out}\MIS{I
    find glossaries unhelpful; you want to define terms in context. No reader is
    going to remember them from the glossary nor are they going to want to turn back
    to look them up.
} in Linux
or \emph{ETW}\tm{add to glossary, define out} on Windows, both of which are
pre-existing systems with well-defined mechanisms for extracting information
from production systems.

To review, the services that will be implemented in this phase of my work are:

\begin{description}
    \item[Metadata Servers] --- there are existing models for meta-data servers,
        such as Egeria~\footnote{\url{https://github.com/odpi/egeria}} or
        Amundsen~\footnote{\url{https://github.com/amundsen-io/amundsen}} (a
        demonstrative, but not definitive list) that might be sufficient. If
        not, using a key-value store such as
        WiredTiger~\footnote{\url{https://github.com/wiredtiger/wiredtiger}} to
        construct a metadata server is also viable. \tm{I'd rather not build
            anything I don't have to but my concern is that this is going to be a
            core bit of the work; the right thing to do is to do a more thorough
            search for these technologies and figure out which one looks like a good
            fit. \emph{Building} one is possible but seems like a big project on its
            own.}

    \item[Namespace Servers] --- my expectation is that this will need to be
        constructed; it must interoperate with the metadata servers.  Work here will
        include defining the interface to constructing a new namespace or retrieving
        a previously constructed namespace. While the ultimate goal is to have a
        federated namespace service, this initial effort need only provide a single,
        non-federated namespace even though that is more limited than the actual
        design.

    \item[Activity Monitors] --- these create the activity context information I
        described previously.  I would expect that only a single activity monitor
        will be built for this initial system prototype and the details of this will
        be driven by the chosen definition of the activity context and relevant
        activity providers.

    \item[Attribute Services] --- because \system is a naming system, we need a
        mechanism that retrieves storage attributes. Initially, this will consist of
        a source for scanning current content and then a monitoring mechanism
        (likely similar to the activity provider) that notifies the attribute
        service to update the attributes of files that have changed.

    \item[Update Notification Server] --- existing systems provide change
        notifications.  In a cross-silo system this sort of dynamic state monitoring
        will also be useful.  An initial implementation for a local system could
        consist simply of using the existing change notification mechanism(s) for
        watching changes.  A more robust implementation could use information from
        the meta-data service(s) to determine if a given change is of interest.
        This dynamic change tracking could then be federated as part of the later
        planned work.
\end{description}


This proposal is quite broad and consists of no working software at the present
time.  To achieve this I propose taking the initial architecture and
constructing\MIS{I was expecting this system to do something like, "I need to
    widget that does X. I can do that by starting with existing widget A and
    extending it in the following ways."
}
a design based upon this architecture.  To evaluate the design, I
propose choosing at least two distinct storage silos and at least one target
operating system.\MIS{You must have thought about what you'd pick -- why not say
    that?
}  To the extent possible, initial implementation would focus on
combining existing components as much as possible.  For example, Using MinIo and
Sparkle Share as storage silos, with Windows Cloud Sync would provide documented
and generally well-understood technologies on which to construct these
services.\MIS{OK, so you have -- then you need to talk a bit more about these
    technologies, exactly what they bring to the table and what they don't -- i.e.,
    what you will have to do (not at the code level, but the conceptual level) to
    leverage them.}
The Meta-data server can be constructed using one of the available key-value
stores (e.g., WiredTiger).  The Notification service could start with Emitter.
Initial activity context work can be built using eBPF or other inbuilt tracing
mechanisms.  This leaves how to build the attribute services as an open area to
be further refined.

\tm{TBD}

\section{Federated Naming/Meta-data Security}
\label{ch:plan:sec:federated-naming-security}

\tm{TBD}

\section{Cross-Silo Naming/Meta-data Benefits}
\label{ch:plan:sec:cross-silo-naming-benefits}

\tm{TBD}

\section{Timeline}
\label{ch:plan:sec:timeline}

% Table generated by Excel2LaTeX from sheet 'Sheet1'
\begin{table}[htbp]
  \centering
  \caption{Proposed Schedule for competion of Doctoral Thesis}
  \begin{tabular}{p{0.15\columnwidth}p{0.85\columnwidth}}
    \textbf{Month} & \textbf{Activity}                                          \\
    \rowcolor[HTML]{C0C0C0} \multicolumn{2}{c}{2021}                            \\
    September      & Draft thesis proposal to supervisors                       \\
    October        & Updated draft thesis proposal to supervisors               \\
    November       & Final thesis proposal to committee                         \\
    December       & Defend thesis proposal - reach candidacy                   \\
    \rowcolor[HTML]{C0C0C0} \multicolumn{2}{c}{2022}                            \\
    January        & Begin proposed research Part 1 (Storage Silo)              \\
    April          & Milestone 1 Storage Silo (Alpha)                           \\
    June           & Milestone 2: Storage Silo (Beta)                           \\
    August         & Milestone 3: Storage Silo (Complete)                       \\
    September      & Storage Silo paper submission ready                        \\
    September      & Begin proposed research Part 2 (Federated Naming/Security) \\
    December       & Milestone 1: Federated Naming/Security (Alpha)             \\
    \rowcolor[HTML]{C0C0C0} \multicolumn{2}{c}{2023}                            \\
    March          & Milestone 2: Federated Naming/Security (Beta)              \\
    June           & Milestone 3: Federated Naming/Security (Complete)          \\
    July           & Federated Naming/Security Paper submission ready.          \\
    September      & Milestone 1: Cross-Silo Naming (Alpha)                     \\
    December       & Milestone 2: Cross-Silo Naming (Beta)                      \\
    \rowcolor[HTML]{C0C0C0} \multicolumn{2}{c}{2024}                            \\
    March          & Milestone 3: Cross-Silo Naming (Complete)                  \\
    April          & Cross-silo Naming Paper submission ready                   \\
    June           & Draft thesis.                                              \\
    August         & Final thesis.                                              \\
    October        & Defend thesis.                                             \\
  \end{tabular}%
  \label{ch:plan:tab:schedule}%
\end{table}%


The purpose of the schedule shown in \autoref{ch:plan:tab:schedule} is subject to adjustment as the research
progresses.


%
% This describes what I'm proposing to implement
\chapter{Local Naming}
\label{ch:local}

The first contribution that I propose for my thesis is to identify,
implement, and evaluate a system in which naming and storage are distinctly
separated.  This system will be consistent with the underlying goals and basic
architecture of \autoref{ch:model} but focus on a small-scale system: that of a
pair of devices belonging to a single individual.  I call this the ``local''
model.

By focusing on the modest scale of a single individual, I will focus on
answering key issues that are fundamental to \system.  I propose answering the
following questions:

\begin{enumerate}
    \item What is necessary to implement the rich namespace model of \system
          while efficiently supporting legacy applications?
    \item How can existing applications utilize the cross-silo naming model of
          \system without being modified?
    \item How effective is \system when used within this local context?
\end{enumerate}

Despite the modest scale, this work will require implementing at least basic
versions of most of the components described in
\autoref{ch:model:sec:architecture}, which will lay the groundwork for other
elements of this thesis.

%\chapter{Organization Level Naming}
\label{ch:organization}

The second contribution that I propose for my thesis is to identify,
implement, and evaluate naming at the organizational level.  This system will
expand upon the work done for \autoref{ch:local} with an emphasis on considering
the issues in naming for organizations.  The key issues that I propose
addressing are: organizational roles and adaptability, working collaboration,
and security.

\begin{description}
    \item[Organizational roles and adaptability] --- individuals play some role within an
        organization, which means that the responsibilities of a given role may be
        given to a new member of the organization.  Part of the assignment of
        existing responsibilities to someone new includes ``ownership'' of data and
        information that are part of the role.  In this scenario the ability to find
        things becomes important in a way different than considered in
        \autoref{ch:local} because there is no cognitive model for the
        organizational structure of the existing information.

    \item[Working collaboration] --- teams of people may be called upon to
        collaborate on various projects over time.  Consideration of this model is
        core to the design principles I set forth previously (\autoref{ch:model})
        but were not a core consideration for the earlier proposed work
        (\autoref{ch:local}).

    \item[Security] --- the initial model of security for local naming
        (\autoref{ch:local}) was deliberately simple, since the owner of the data
        had access to it, and others did not.  With multiple members in an
        organization, security becomes a more relevant consideration. The focus here
        is not on \emph{data} security, as that has been well-addressed previously,
        but rather \emph{meta-data} security.  As we collect more meta-data within
        the system, it becomes possible to infer information about the data. Because
        an organization might have information that is considered sensitive,
        security is a consideration for \system.
\end{description}

\tm{This is still way too vague.}


%\chapter{Global Naming}
\label{ch:global}

The third contribution that I propose for my thesis is to identify,
implement, and evaluate naming at global level. Unlike the work proposed in
\autoref{ch:local} and \autoref{ch:organization}, this work will explore on
considering naming as a top-level service that might provide ``Naming as a
Service''. Thus, this proposed work is an amplification of the prior work to
consider how \system can be effectively implemented at scale.


%\include{chapters/discussion}
%\chapter{Timeline}
\label{ch:timeline}




%\include{chapters/relatedwork}
\section{Conclusion}
\label{sec:conclusion}

We have presented our position that we need \system, a storage architecture that decouples naming from the storage location of documents and data objects, provides customizable and personalized namespaces, and that makes relationships between documents a first class citizen. With \system, users will be able to organize, share and find their data conveniently across multiple storage silos using a rich set of attributes breaking away from the rigid, hierarchical organization.

We expect \system will enable a broad area of research in HCI exploring new ways to visualize and interact with data using the mechanism's provided by \system. Moreover, we expect \system to provide interesting scenarios for security and privacy research in storage systems.



%    3. Notes
%    4. Footnotes

%    5. Bibliography
\begin{singlespace}
    \raggedright
    \printbibliography
    %\bibliographystyle{abbrvnat}
    %\bibliography{bib/indaleko}
\end{singlespace}

%\appendix
%    6. Appendices (including copies of all required UBC Research
%       Ethics Board's Certificates of Approval)
%\include{reb-coa}	% pdfpages is useful here
%\chapter{Supporting Materials}

This would be any supporting material not central to the dissertation.
For example:
\begin{itemize}
    \item additional details of methodology and/or data;
    \item diagrams of specialized equipment developed.;
    \item copies of questionnaires and survey instruments.
\end{itemize}

\autoref{ch:appendix:section:evolution} and
\autoref{ch:appendix:section:kwishut} are write-ups that I did originally as
part of preparing this thesis proposal.  They are included as background and I
expect to remove them prior to submission.

\section{One-Page Proposal: Evolution}
\label{ch:appendix:section:evolution}

File systems have evolved slowly: development approaches, operating systems
interaction, and interfaces reflect decisions made decades ago.  However, scale,
technology, and systems have evolved.  Past decisions reflect appropriate
solutions that no longer fit the modern world.  File systems must evolve to meet
modern demands, utilize modern technologies, and encourage further innovation.

\MIS{I think the messaging is right here, but the order is wrong: That is you
    start with FS evolution, but if the world were static, we wouldn’t care about FS
    evolution. So I think the motivation is that data has changed dramatically in
    the past 40 years: we generate $10^x$ times more data each year; we have rich data
    (audio, video, VR, etc).  However, file system s are largely unchanged (slight
    exaggeration).
}

\MIS{I think I would next say something to the effect of, “My dissertation will
    examine N ways in which file systems must evolve, demonstrating both why and how
    they must do so. Then list the N things and go into the next paragraph.
}
In the early 1990s file system caching underwent rapid transformation, with
tight integration between the virtual memory manager and the file system: cached
files were now mapped.  I/O operations from applications were thus satisfied
using the memory mapping (\MIS{Expand to what this means; using standard memory
    copying library calls (which, in turn, means accessing file data via load and store instructions).
    There are some/many who would argue that mmap cannot possibly make it faster
    to access data – keith is one such person – he fundamentally doesn’t believe it,
    even after reading sasha’s post carefully.}).   This greatly improved performance and utilization of
memory.  File systems could thus handle fixed (page size) operations, with
needed data faulted into the cache in multi-page unities that were highly
efficient for storage devices.  The introduction of persistent memory \MIS{Over
    the past N years, the industry has seen the emergence of persistent memory
    accessible from the memory bus --
} in this
environment required a new change, the “direct access” model, which continued to
use the page based management model with direct mapping.  However, this model no
longer makes sense and has interesting implicit costs associated with it because small
objects (files) in persistent memory still consume blocks of storage, which is
inefficient, and the I/O patterns of applications, which provide insight into
the structure of the data within the files, is ignored.  The DaxFS project
builds upon my prior work with persistent memory and combines it with work I
have done to find useful insights in recognizing and using implicit object
structures to provide efficient storage and sub-block sized object efficiencies,
including deduplication and “patching support”. \MIS{At this point, the key
    issue is not so much how it builds upon and/or relates to your own work but
    precisely what DaxFS is designed to do/address.
}

Modern production file systems are typically specialized kernel-level
software.\MIS{I would just say, “File systems are most frequently implemented as
    part of the operating system, running at supervisor privilege.” Or something
    like that.  It’s not a development model – it’s a deployment model.
}
This development model hobbles innovation and exploration because it relegates
user-mode file systems to specialized use, such as network file systems, or toy
examples.  There is no fundamental reason for this: software does not run faster
in kernel mode than it does in user mode. Indeed, most operating systems make
choices that make the fundamental operation of copying data between memory
locations slower in kernel mode than user mode.  Prior work has explored
techniques for improving this performance using in-kernel techniques such as
optimized communications and specialized kernel extensions as well as a single
development model for user and kernel file systems.
I propose finding mechanisms that avoid kernel
interaction at all, as a form of “kernel bypass” reminiscent of how file systems
are implemented in micro-kernel systems, with the goal being to leverage the
simpler user mode development environment to foster further innovation building
file systems.\MIS{I think this requires more precision to differentiate from the
    gazillion papers on kernel bypass.
}

File systems have been using a simple naming scheme with nominal
meta-data\MIS{More precise “hierarchical namespace” – the meta data issue that
    the current meta data is designed for use by the FILE SYSTEM not people.},
hearkening back to a time when users did not store related files in unrelated
locations.  This simple model has served us well but has serious limitations in
modern storage systems.  I have identified specific issues\MIS{Vague – be more
    precise about what you have identified.}
related to capturing data relationships in a fashion that permits association of related data,
including files, stored in unrelated locations, such as storage systems with
different characteristics and object stores.  This work is to construct a system
that captures activity context\MIS{Which is not defined…} and then permits me to create namespace views
driven by those activity contexts to show data relationships.  This will allow
me to answer the question of whether or not such a system can simplify data
location, reduce data duplication, and ensure efficient use of storage by
allowing related data to be stored in unrelated storage locations.\MIS{What is
    the ultimate problem you are trying to solve?  People can’t find their data!
    Just say that – then say that you have concrete ideas about how to make it
    easier – and it’s all about activity context, which is ….. and enables …
}

Thus, the goal of my thesis\MIS{Think about this in terms of a thesis statement
    you will ultimately be able to defend. E.g., Adding features  X, Y, and Z to
    file systems enables them to better support modern file system usage, such as ….
} is to revisit old decisions in light of new
information, find old or new paradigms for storing and organizing information,
with a goal towards simplifying development of new file systems, as well as
providing greater utility and performance.


\section{One-Page Proposal: Kwishut}
\label{ch:appendix:section:kwishut}

Existing storage systems often operate in isolation.\MIS{Compare this opening to
    the following, ‘users today store and access data from myriad different
    locations’}
This does not match the
usage or expectations of users: large static files may be stored in an optimized
storage silo, while smaller dynamic files constructed from those large files may
be stored in a collaborative storage silo, such as Google Docs or Excel Online.
This situation is exacerbated by a naming convention that either uses
proprietary meta-data management mechanisms or forces users to embed critical
meta-data into the file and path name structure.  Prior work, such as Placeless
and Ground, have attempted to address aspects of this\MIS{But what is “this” ?} but have fallen short:
Placeless fails to capture the “activity context”\MIS{Not defined} information so critical to
tying related data objects together and Ground identifies interesting research
questions without addressing them.\MIS{Here is what I think this paragraph is
    saying, “Users today store and access data in myriad different places. This
    distributed storage makes finding objects difficult. I claim that users possess
    so much data with many complex and varying relationships that the inability to
    find things is  due to a fundamental mismatch between the way users want to
    interact with their data and the way that systems enable them to interact with
    them. Kwishmut is the name we give to an architecture designed to address this
    fundamental mismatch.
}

Kwishut sets out to extend this prior work and address key research questions
that we have identified: the cost and benefits of: (1) explicitly separating
meta-data management, including naming, from storage silo location and
management; (2) securing, protecting the privacy and ensuring the integrity of
meta-data; and (3) constructing usable namespace views, including traditional
hierarchical, search driven, and novel data visualization strategies.
\MIS{It’s not entirely clear that we are at a point to cast this in
terms of cost/benefit.  That might be a way to evaluate it, but I’m not sure
that’s the question we are trying to answer – I think you are claiming (i.e., a
thesis statement) might be of the form that “Explicit separation of meta data
and naming, enabling the construction of per-user namespaces, and capturing
information about the context in which data is constructed and accessed are
critical to supporting modern data management needs.”  [that’s not quite right,
but it’s close.]
}

This system involves the design, development, and evaluation of key components:
(1) a meta-data service that extends the capabilities of existing meta-data
services and permits capturing the richer meta-data Kwishut enables; (2) a
namespace service that utilizes the meta-data services to realize usable
cross-storage silo namespace servers; (3) activity monitors that record specific
activity context on local devices and store that in a meta-data service so that
this rich context is available to the namespace services; (4) attribute services
that extract attribute information from existing storage silos and store that in
a meta-data service; and (5) an update notification service that permits
registration for specific events and notification when those events are
observed.
\MIS{Is there an MVP subset? (E.g., could you design an architecture that
    includes notification, necessary for certain applications, but focus on those
    apps that do not require it?)
}

This proposal is quite broad and consists of no working software at the present
time.  To achieve this I propose taking the initial architecture and
constructing a design based upon this architecture.  To evaluate the design, I
propose choosing at least two distinct storage silos and at least one target
operating system.  To the extent possible, initial implementation would focus on
combining existing components as much as possible.  For example, Using MinIo and
Sparkle Share as storage silos, with Windows Cloud Sync would provide documented
and generally well-understood technologies on which to construct these services.
The Meta-data server can be constructed using one of the available key-value
stores (e.g., WiredTiger).  The Notification service could start with Emitter.
Initial activity context work can be built using eBPF or other inbuilt tracing
mechanisms.  This leaves how to build the attribute services as an open area to
be further refined. \MIS{I think the message here is, “At present we have only a
    high level design of the system. We can develop a prototype implementation by
    leverage existing technology for most of the components, implementing only those
    parts necessary to address the key research questions. For example, we do not
    need to build a new file system, but can leverage existing ones such as … Nor do
    we need to implement a meat-data store. Instead we need to develop techniques to
    extract meta-data from data and then store that meta-data in off the shelf
    technology, such as …}


The key research questions are: (1) what benefits can be realized in terms of
using cross-silo meta-data to create namespaces based upon logical associations
rather than traditional silo/path/name style organization; (2) what are the
costs related to these; (3) what are the security issues we can identify and how
do we address them in Kwishut; (4) how can this system be used to improve data
governance; and (5) can we enable construction of innovative new data
visualizations of cross-silo data that are not presently possible.

\section{HotOS 2019}
\label{ch:appendix:section:hotos19}

% Not dead yet

\subsection{Abstract}

Rumors of the demise of the hierarchical file system namespace have been
greatly exaggerated.
While there seems to be wide spread agreement that most users make no use
of the hierarchical name space, we continue to use it both as the underlying
storage structure and as the default user interface.
To compensate for this mismatch between the native organization and the
way users interact with their data,
we have produced myriad search tools atop the file system.
This approach, however, has some limitations.
In particular, we focus on the absence of generalized relationships
as first class entities.

The hierarchical namespace elevates the \emph{contains} relationship
above all else.
Applications elevate the \emph{was created by me} relationship.
These are only two particular relationships among many;
items accessed around the same time share a temporal relationship;
attachments to email messages share a relationship;
a collection of documents, email message, notes, etc. form another
relationship.
We claim that elevating arbitrary and generalized relationships as first class
file system elements provides a better user experience and leads to entirely
different implementation approach.

\subsection{A Modest Proposal}
\label{hotos19:graphfs}

% Because our focus is on the \textit{naming} system and not the storage system,
% for the present time we will not consider the issues that will arise in the
% storage management layer to support the new model, though we admit that there
% are likely to be concerns that will need to be addressed in future work.

Our proposed file system focuses on \textit{relationships} between our files.
We use an analogy between social graphs and file systems to explore this
approach.
Facebook's graph is a collection of typed objects
(e.g., users, actions, places) and associations (e.g., friend, authored, tagged).
File system objects map to users and files; \textit{contains} is, perhaps, the only
association captured in a file system.
For the rest of this discussion, we treat directories as the embodiment of the
\emph{contains} relationship, not objects.

We consider two strawman implementations for elevating relationships to
first class file system objects.

\subsubsection{File System as a Graph Database}

In considering a graph based file system, first we consider
implementing a file system in a graph database, of which there are
many (\S \ref{hotos19:background}). Their primary focus is the storage of
graph-structured \textit{data}.  Our focus is in the use of graph-structured
data as critical meta-data inside of a storage system.
As such, there is a mismatch in design targets between a file system and
existing graph databases: nodes in graph databases are small; nodes in
file systems are large. Graph databases tend to favor a navigation-based API;
file systems need a point query and search API. Graph databases assume that
attributes and relationships are provided; file systems will frequently derive
attributes and relationships.
These differences suggest to us that existing graph databases are not suitable
as the basis for file systems.
Nonetheless, we encourage others to consider such an arrangement, should they
have compelling reason to do so.

\subsubsection{File System as Social Network}

Next, we consider implementing a file system in the same way Facebook
implements their social network graph.
Facebook's original implementation stored the social graph in MySQL, queried
it from PHP, and cached the results in memcache.
More recently, Facebook introduced Tao, which is a service that more directly
implements the fundamental objects and relationships that comprise the
social graph~\cite{bronson2013tao}.
While Tao is specifically designed to support the widely distributed,
replicated, and rapidly changing social network scenario, it provides the
starting point for conceptualizing a data model premised on the primality of
relationships.
Tao stores both objects and associations in a MySQL database and presents
the graph abstraction via Association and Query APIs in the caching layer.
Is this a viable structure for a file system?

% in fact, facebook has a separate storage system for videos and pictures
% because they are large.
Unlike objects in Facebook, files are large.
Although prior work has considered using relational
database~\cite{olson1993design} and other index-based
structures~\cite{spillane2013vttree}  to store files,
the community seems to have
concluded that such storage is not ideal. We agree, suggesting that
an RDBMS is not the desired storage system.

What about relationships? Is it appropriate
to use one persistent representation (e.g., a relational one) and a second
memory representation (e.g., a graph-structured on) or
should we use a single graph-structured representation both in persistent store
and in-memory.
We propose the latter for two reasons.
First, the rumored era of non-volatile main memory seems to be around the
corner, so a modern file system design should embrace a single
representation.
Second, while it is reasonable for Facebook to construct the entire graph in
a distributed pool of main memory, file systems must work on a more limited
scale and therefore cannot ensure that the realized graph structure will fit
in main memory.

As neither strawman design seems suitable for our relationship-centric file
system, we present a new model and file system design.

\subsection{Graph FS Model}
\label{hotos19:graphfs:model}

We set out a basic description of our core objects in Table \ref{hotos19:table:graphfs:terminology}
%\footnote{We took inspiration for this model from https://github.com/opencypher.}
and a demonstrative set of example relationships in Table \ref{hotos19:table:relationship-examples}.
We do not consider either of these to be exhaustive, but rather propose them as an initial
basis for discussion.
The presented model can encompass
the functionality of the existing hierarchical file system model.

% https://github.com/opencypher/openCypher/blob/master/docs/property-graph-model.adoc

\begin{table}[h]
    \captionsetup{justification=centering}
    \begin{tabular}{p{2cm}p{5cm}}
        Term                          & Definition\tabularnewline\hline
        \multirow{1}{*}{file}         &
        \multirow{1}{*}{\parbox{4.8cm}{Uniquely identified storage unit}}
        \tabularnewline
        \multirow{1}{*}{relationship} &
        \multirow{1}{*}{\parbox{4.8cm}{Directional file association}}
        \tabularnewline
        \multirow{1}{*}{labels}       &
        \multirow{1}{*}{\parbox{4.8cm}{A binary attribute, e.g., executable}}
        \tabularnewline
        \multirow{1}{*}{property}     &
        \multirow{1}{*}{\parbox{4.8cm}{Key/Value attribute}}
        \tabularnewline
    \end{tabular}
    \caption{Graph File Systems Terminology}\label{hotos19:table:graphfs:terminology}
    %    \Description{Graph File Systems Terminology}
\end{table}



Every file has a unique identifier, such as a \textbf{UUID}, similar to
an inode number or object ID.
We do not rely upon \textit{names}
as they are simply mutable properties.

\begin{table}[h]
    \begin{tabular}{p{1.9cm}p{5.5cm}}
        Relationship                           & Description\tabularnewline
        \hline
        %        \multirow{1}{*}{\textit{is}} &
        %        \multirow{1}{*}{\parbox{5.4cm}{Attribute of a file, e.g. size or timestamp}}
        %        \tabularnewline
        \multirow{1}{*}{\textit{similar}}      &
        \multirow{1}{*}{\parbox{5.4cm}{Similarity measure, e.g., \cite{masci2014multimodal}}}
        \tabularnewline
        \multirow{1}{*}{\textit{precedes}}     &
        \multirow{1}{*}{\parbox{5.4cm}{temporal relationship (e.g., versioning)}}
        \tabularnewline
        \multirow{1}{*}{\textit{succeeds}}     &
        \multirow{1}{*}{\parbox{5.4cm}{temporal relationship (e.g., versioning)}}
        %        \tabularnewline
        %        \multirow{1}{*}{\textit{located}} &
        %        \multirow{1}{*}{\parbox{5.4cm}{link or url}}
        \tabularnewline
        \multirow{1}{*}{\textit{contains}}     &
        \multirow{1}{*}{\parbox{5.4cm}{directory/file relationship}}
        \tabularnewline
        \multirow{1}{*}{\textit{contained by}} &
        \multirow{1}{*}{\parbox{5.4cm}{directory/file relationship}}
        \tabularnewline
        \multirow{1}{*}{\textit{derived from}} &
        \multirow{1}{*}{\parbox{5.4cm}{provenance (e.g., .o to .c)}}
        \tabularnewline
    \end{tabular}
    \caption{Graph File System Relationship Examples}\label{hotos19:table:relationship-examples}
    %    \Description{Graph File System Relationship Examples}
\end{table}



A \textit{relationship} is a directional association between two files.  We expect there
to be far fewer relationships than files, though many more \textit{instances} of
relationships (i.e., the number of edges in our graph exceeds the number of vertices).
Relationships may be either uni-directional (e.g., derived from) or
bi-directional (e.g., similar).
Table \ref{hotos19:table:relationship-examples}
provides a set of sample relationships; the universe of relationships
is extensible.
As in RDF, relationships are triples: two files and the relationship.

As files have attributes in a conventional file system, both files and
relationships have attributes in a graph file system.
A \textit{label} is a simple binary attribute (e.g., executable),
while a \textit{property} is an arbitrary name/value pair, much like
an extended attribute, but they are native to the model, not
an afterthought.

% We will extend our terminology as needed, using the existing terminology of the
% relationship graph as inspiration for usable models.

\subsubsection{Interface}\label{hotos19:graphfs:interface}

\begin{table}[b]
    \small
    \captionsetup{justification=centering}
    \begin{tabular}{p{2cm}p{5cm}}
        Operation               & Description\tabularnewline\hline
        \multirow{1}{*}{create} &
        \multirow{1}{*}{\parbox{4.8cm}{Insert new file into graph}}
        \tabularnewline
        \multirow{1}{*}{relate} &
        \multirow{1}{*}{\parbox{4.8cm}{Insert new edge into graph}}
        \tabularnewline
        \multirow{1}{*}{label}  &
        \multirow{1}{*}{\parbox{4.8cm}{Insert new labels}}
        \tabularnewline
        \multirow{1}{*}{set}    &
        \multirow{1}{*}{\parbox{4.8cm}{Insert new properties}}
        \tabularnewline
        \multirow{1}{*}{remove} &
        \multirow{1}{*}{\parbox{4.8cm}{Remove something from graph}}
        \tabularnewline
    \end{tabular}
    \caption{Graph File Systems Operations}\label{hotos19:table:graphfs:operations}
    %    \Description{Graph File Systems Operation Examples}
\end{table}


One of the lessons from Plan 9 is that everything can be represented as a file~\cite{pike1992use};
we expect to
continue with this paradigm as it has served us well over the years.  While we generally
think of files as a blob of \textit{persistent} data, in fact it is useful to
think of them as abstract \textit{generators} of byte stream data.  This fits well
with our model of separating namespace from storage; how the storage
providers return data to us is orthogonal to the namespace we use to retrieve it.
For example, the \textit{procfs} file system creates a synthetic namespace and supports
I/O operations for reading and modifying data contents of the pseudo-files.

From the namespace perspective, our file system must support operations that manipulate that
namespace. This includes the ability to create files, relationships,
relationship instances \textit{between} files, labels, and properties.
Similarly, we need the ability to remove each of these.

Our model is simple, yet powerful.  It captures interesting concepts such as versioning, using relationships such as
\textit{precedes} and \textit{succeeds},
and provenance, using relationships around derivation and use,
and application specific relationships, such as \textit{indexes} so a database
system can expose the relationship between its primary data and the
corresponding index files.
Although relationships are binary,
we can create clusters of related files by asking for all the vertices connected
by a specific relationship.

Where do relationships, labels, and properties come from?
We identify at least the following five sources:
1) the system itself will generate traditional
attributes (e.g., \textit{size}, \textit{read time}) and some
relationships (e.g., contains);
2) tools that extract meta-data from different file
types~\cite{soules2004toward,bloehdorn2006tagfs} will produce more attributes;
3) applications will generate both attributes and relationships;
4) users may generate attributes and relationships, although history
suggests that asking users to annotate data is a losing
proposition~\cite{soules2003can}; and
5) kernel extensions, e.g., provenance tracking systems~\cite{pasquier17camflow}
will generate attributes and relationships.

Several interesting possibilities emerge from this design.
Hard links are multiple \textit{name} properties attached to the
same file, potentially in different namespaces.
Soft links are a relationship between two names.
The system can capture relationships that extend beyond the file system.
For example, the \textit{derived from} relationship from
Table \ref{hotos19:table:relationship-examples} might describe a file that came
from a particular email or web site.

\begin{figure}[bt]
    \captionsetup{justification=centering}
    \includegraphics[width=0.9\linewidth]{reference/hotos19/figures/model-graph.eps}
    \caption{Graph File System}\label{hotos19:fig:graphfs-example}
    %\Description{Simplistic Graph File System Picture}
\end{figure}

Figure \ref{hotos19:fig:graphfs-example} provides a simplified visualization of our graph file
system model.
Our inclusion of disjoint graphs captures the notion that the system
naturally supports multiple namespaces, implemented as disconnected graph
components.

\subsection{Aspects of Implementation}

In the absence of space to provide a full implementation, we offer a few
strategies that make a graph file system both feasible and novel.
The underlying storage structure for files is essentially an object store~\cite{factor2005object} and
attribute storage is largely a solved problem
(although the last time
one of the authors said that, her colleague disagreed~\cite{mao2012cache}), so we
focus on fast and efficient graph storage and query.

Today's systems either provide graph storage~\cite{rudolf2013graph,webber2018programmatic,microsoft:cosmosdb}
or graph processing~\cite{shun2013ligra,gonzalez2014graphx,malewicz2010pregel,salihoglu2013gps,nguyen2013lightweight,low2014graphlab,kyrola2012graphchi},
but a graph file system needs a high performance, space-efficient, mutable and queryable
native graph representation.
We have found that mutable compressed sparse row representations~\cite{macko2015llama}
meet all these requirements (we used them as the query and storage mechanism in the
SHEEP graph partitioner~\cite{margo2015scalable}).
Just as high performance key/value stores are considered reasonable implementation
strategies for attribute storage and management in file systems, similarly efficient
structures supporting graph storage and management should be adopted in file systems
as well.

\subsubsection{Search}\label{hotos19:search}

The driving force behind our graph file system design is to provide the
infrastructure to make it easier for users to find data.
Users do not navigate to data, they \textit{search} for it, so we
consider more effective search models to further
motivate the graph file system.

We observe that there are two different models of ``search'': application
search and user search.
Applications need to be able to open files quickly
using a \textit{key}. For example, both NFS~\cite{sandberg1986sun}
and AFS~\cite{sidebotham1986volumes} use the file system \textit{inode number} as their
mechanism for identifying the specific file or directory being accessed,
because it is fast, avoiding a costly namespace traversal.
Similarly, NTFS supports the ability of applications to open a file by
identifier~\cite{sreenivas2011bypass}.
They did this to support their implementation
of the Apple File Protocol (``service for Macintosh'') but has subsequently been used
for other uses. Indeed, it has been further extended to permit files to be opened by an
application-defined identifier (a UUID); Microsoft continues to support file IDs in
ReFS~\cite{microsoft:refs:features}. The Google File System~\cite{Ghemawat2003}
observation was similar: applications can use keys to find their files.

Modern applications tend to either create files that they use internally, often going to great lengths
to hide their location from the user; or maintain a list of recently used items with a full path name,
which breaks when the path changes, even if the file did not change.  A key interface for applications
better fits this usage model. Thus, a ``search by key'' interface is sufficient.

The more challenging problem is user-focused search.
\begin{comment}
Many of the
characteristics in a good human usable search system do not benefit the programs
directly.
\end{comment}
For human users, we want to enable a model like the
\textit{memex}~\cite{bush1945we}: ``A memex is a device in which an individual stores
all his books, records, and communications, and which is mechanized so that it may be
consulted with exceeding speed and flexibility. It is an enlarged intimate supplement
to his memory.''

The HCI community has a long history researching
more effective search, including such efforts as
SIS~\cite{dumais2016stuff} and faceted search~\cite{arenas2016faceted,tunkelang2009faceted,hearst2006design,klungre2018evaluating,walton2017looking,cleverley2015retrieving}.
Critical to this work is the idea that search is most effective
when \textit{not} bound to a specific taxonomic order --- very much the opposite of today's
hierarchical search model, which enforces a rigid
order on the structure of information.

How does a graph file system then enable modern search?
First, support for a broad and extensible set of attributes and
relationships brings search engine technology to bear in the
service of file systems.
There is some irony that the success of web search, and in particular the
primacy of relationships in those algorithms~\cite{page1999pagerank}, has had
virtually no impact on how we find our own local data.
Second, the generalized graph structure, which no longer elevates any single
organzation gets rid of the \textit{specific taxnomic order} that HCI
researchers determine to be counterproductive.
Third, although some degree of temporal query is possible using \texttt{find},
its interface is not especially accessible to the typical user, and it requires
a series of manual operations to express natural queries such as
``Show me the documents I wrote last summer after I got back from my Amazon
rafting trip.''

Our goal is not to specify the entire range of searches that can be realized, but rather to
explore file system structures that enable the creation and mining of relationships to help users to find relevant data.

\begin{comment}
To help motivate our work, we consider the \textit{Graph Query
    Language}~\footnote{https://gql.today/wp-content/uploads/2018/05/a-proposal-to-the-database-industry-not-three-but-one-gql.pdf}
(GQL) as a starting point.  GQL is an emerging language
attuned to the needs of \textit{property graphs}, which happen to be similar to the model that we envision for our new
file system.  It attempts to merge the strengths of three existing graph database query languages into a single, standard,
query language for property graphs~\cite{van2016pgql,francis2018cypher,angles2018g}.

% This choice is motivated by our realization that the new model we propose is a property graph~\cite{rudolf2013graph}.
\end{comment}


\subsubsection{Related Work}\label{hotos19:background}
The need for better name spaces in file systems is hardly a new
topic, with many solutions being proposed and implemented over
the years.

\textit{Search utilities} are successors to the permuted index program.  They permit us to find
files based on \textit{content} and \textit{attributes}.  MacOS X has \textit{spotlight}, which
provides an extensible, index-driven search service~\cite{apple:spotlight-extensions}.  Similarly,
Windows offer a similar extensible service~\cite{microsoft:data-add-in}.  These enable searching
based upon attributes, e.g., file suffix, date, size, etc., and context-sensitive content, e.g.,
music files by artist, composer, song title, or even \textit{rights},
but limited, if any, ability to search by relationship.

%.  A number of other examples
%of similar search mechanisms
% exist~\cite{Suguna2015,huo2016mbfs,leung2009magellan}.

%\textit{Databases} enable developers willing to pre-define their data's structure to enable searching
%on the specifics of the data.  Databases come in a rich array of models: relational, column, document,
%and graph, for example.  File systems constructed from databases have been extensively
%explored~\cite{olson1993design,balabine1999file,balabine2002database,kashyap2004file,murphy2002design}.
%As we noted previously (\S \ref{sec:graphfs:model}) such approaches have failed to yield clear
%results.

\textit{Tag Systems} were an early approach to improving hierarchical file systems searchability%
~\cite{Parker-Wood2014,chou2015findfs,ma2009file,laursen2014,nayuki2017,Andrews2012,Up2016,Jones2016,aws:s3:object:tagging,ames2006lifs,leung2009magellan,frieder2012hierarchical}.
Automatic tagging systems have become a more common approach here as manual tagging by users
has proven to be impractical~\cite{soules2003can,soules2004toward}.
The addition of \textit{semantic} information
~\cite{di2017gfs,hua2016real,martin2004formal,Martin2005,martin2008,martin2014,gifford1991semantic,Faubel2008,harlan2011joinfs,Suguna2015,Andrews2012,ngo2007integrating,Omvlee2009,wang2003managing,gopal1999integrating,Codocedo2015,Jones2016,Mahalingam2003,Parker-Wood2014}
is useful but falls short of addressing the fundamental need to understand
data relationships, because like more conventional systems,
these semantically-aware approaches still focus on the file,
not on the relationships between the files.
As such, they are simply an add-on to
the hierarchical model, not a replacement for it.
Such approaches can provide useful functionality
in our graph file system model as well.
Indeed, we even pointed this out (perhaps subconsciously) in prior
work when we said ``How many of them [files] are \textit{related} to
each other?'' [emphasis added]~\cite{Seltzer2009} .

Files are rich with relationships.  However, these relationships are not limited to what the
file system can ``see''.
Narrowing our vision to the closed pool of file system relationships
hobbles our ability to capture them.
For example, the obvious relationship between
a file and the e-mail from whence it originated is not exploitable in
any system of which we are aware.
The academic papers we generate refer to other papers.  An enlightened document application would provide an identifier that can
be used to find the corresponding paper - a \textit{refers to} relationship,
whether on our local system or elsewhere.
Of course, in the current model, we likely can't
recall where we stored it when we downloaded it.  As we create new works, we refer to older works --- our own
documents, web pages, Jupyter notebooks, spreadsheets, pictures, etc. Capturing these relationships permits
us to reconstruct the process taken to produce an output.
This is the fundamental problem that the provenance community has been
addressing, but few systems~\cite{pasquier17camflow,reddy06pass} demonstrate
an understanding of the role our file systems play in making this
possible.

Versioning is a feature that continues to reappear in various guises.
This is simply one example of temporal locality; a
relationship that we have not yet deeply mined.
While it is now common practice for
individual applications to ``remember'' the last few files you have accessed,
there are few cross-application examples.
In lieu of the right tools, users invent creative solutions.
For example, one of the authors \textit{attaches} documents to e-mail
immediately after reviewing them
specifically to establish temporal relationship.
The ability to establish temporal relationships across applications should
provide powerful capabilities.

We do not know which relationships are useful. One of the hallmarks of good file systems design over the decades
has been \textit{not} to impose a specific restricted model on what files can be --- we leave that to databases.
We do not intend to establish a definitive set of relationships any more than we focus on defining
the structure of file contents.
However, we encourage others to explore this area, encourage best practices,
and build tools that produce and use such relationships, leaving the
storage and retrieval of relationships to the file system.

% Relationships between files is not new --- this is the quintessential relationship of the modern Internet, with its
% vast web of content that references across the domains --- but those models are certainly more recent than that
% of the hierarchical file system.  Unlike the internet, where information is shared, we seek to enable
% similar relationships useful to our specific usage and with data we do not necessarily wish to share.

Much of the raw data that applications generate would be better \textit{not} injected into
the hierarchical name space: the location of our personal email database, financial
software files, binaries, temporary files, etc.  Their presence in the name space clutter
it and make our existing brute force search slower yet no more useful.

Application programs benefit being
unfettered from the hierarchical name space~\cite{Ghemawat2003}, both
in terms of their efficiency as well as the benefits of \textit{not} commingling the private files of individual
applications --- but the hierarchical file system requires
they be stored somewhere within its domain.
Applications routinely hide most of their files in out of the way
locations, as they are only useful to the application itself.
Thus we end up with ``System Volume Information'' and ``.ssh'' and a myriad of obscure locations where applications
hide data from us.
This is a side effect of the name space model we have used for the past
50 years.

Prior attempts to address this have done so within the confines of a narrow perspective of what is needed to fully
enable the ability of us to find our data.  The HCI community have been poking at the
edges of this problem for decades as well --- observing the frailties of the hierarchical model and suggesting
alternatives~\cite{harper2013file,lindley2018exploring,khan2018forgotten,vitale2018hoarding,boardman2003too,nayuki2017,martin2014,Jan2011,Andrews2012,Mander1992,Omvlee2009}.

Our graph file system permits them to escape the existing paradigm \textit{without} giving up support
of existing applications.


\section{HotOS 2021}
\label{ch:appendix:section:hotos21}

\subsection{Absract}
File systems use names to store and retrieve file data. A system can use a
machine-generated name to locate an object, while a user needs a semantically
meaningful name to locate it. Most systems either lump these two names into one,
or tie them together within a particular storage silo, e.g., a file system,
Google docs, or a cloud store. As a result, it is difficult to search and
navigate objects across silos and view them within a meaningful context that is
not necessarily tied to file names.

We propose Nirvana, an architecture decoupling system-level \textit{location
    names} from human-friendly \textit{context/semantic names}. Nirvana generalizes
the definition of a semantic name to be a set of metadata attributes, which may
or may not include a traditional user-friendly name. We explain how Nirvana can
enhance user experience and describe system support needed to realize it.

\subsection{Introduction}
\label{hotos21:introduction}

Today's file systems provide two primary functions: a way to store chunks of data and names that provide users with the familiar
metaphor of a file cabinet (i.e., folders and documents).
Much has changed since we adopted this design.
Now most data does not reside on local file systems because instead data is distributed across myriad services such as Dropbox, Google docs, Amazon S3, Microsoft OneDrive, and Github as well as attached to communication mechanisms and applications like email, chat, and Slack.
Today, just like local file systems, each of these storage solutions provides its own storage and naming mechanisms.
And therein lies the problem: names were and are designed for users, but multiple disparate namespaces hurt the user experience.
The user bears the burden of remembering personally irrelevant information: \textit{where} they stored their documents.
In the worst case, they must search through the individual storage silos. There is no practical/convenient way to quickly locate the item.

Existing solutions tie namespaces to storage silos, but this model does not fit with the way we use storage.
We routinely access data from multiple silos, locating items by navigating or searching silo-by-silo.

We observe that names serve two purposes: names specify \textit{location} and/or \textit{context or semantic meaning} of the object.
Context/meeting is most important to users; location is most important to the storage silo.
Typically, storage systems fuse these naming purposes in such a way that it is not possible to relegate
fixing the context/semantic naming problem to the Human Computer Interface (HCI) community.

The following (real world) scenario highlights some of the challenges.

A student from the country of Lemuria is destined for a summer internship in Camelot. Arranging for this internship
requires many (many, many) different data objects: email messages with the host, offer letters, academic forms, a visa,
boarding passes, project proposals, and more. How can our overwhelmed intern organize and/or find all the documents
associated with their internship? We consider three approaches.

\textit{Meticulous:} Place/copy every document into a single directory on their personal machine.
Pros: It is simple to find everything.
Cons: It requires conscious effort, pre-planning, and prescience.  Everything is saved: emails,
copies of physical documents, text messages, and project proposals. Our intern hopes they have correctly predicted what they will
need in future.  With more files and people involved, record keeping takes more time and people have different opinions on how to organize the files.

%When the student begins the process by sending tens of email messages, must they all be exported from mail and saved away?
%Each time the shared project proposal changes, the student must update their local copy.
%Further, this approach does not scale because as the number of files and people involved increases, the complexity increases
%exponentially, which becomes untenable.

\textit{Haphazard:} Leave everything where it is and search for it when you need it.
Desktop search utilities such as Spotlight~\cite{apple-search} and application-specific search tools make this possible,
but this approach frequently requires searching across multiple silos, returns many more documents than intended, and misses those our intern~\cite{bergman2019factors}.

\textit{Nirvana:} Our tool on the intern's local machine creates a \textit{personal namespace}.
It combines conventional file system metadata with new (optional) user-provided metadata, relationships between items (e.g., an email
message and the document attached to it, email messages with largely identical contents, and documents shared among the same sets of people)
and provides a navigation interface that uses this metadata to create collections of semantically related objects~\cite{gifford1991semantic}.

We tried to implement Nirvana outside the storage system~\cite{ashish}.
We developed a visualizer that constructed personal namespaces by extracting existing and new properties (primarily relationships similar to those listed above) from multiple distinct data storage silos.
We were surprised at the effectiveness of this approach to find related files, even across storage silos, particularly when combined with additional metadata that we generated.
%Figure \ref{fig:ashish-demo} shows a screen capture from a demonstration of that visualizer.
%\mis{I find this particular visualization confusing; is it possible to create the one where all the documents just magically showed up? The relationships here dominate and don't necessarily make sense in this context.}

%Based on this experience, we concluded that dynamically generated, per-user namespaces are a promising solution to today's multi-silo reality, but that the promise of such systems cannot be realized without storage system support and integration.
%\mis{do better than this: The rest of this paper explores exactly what kind of system support is required.}

Building Nirvana within the confines of existing storage silos has limitations:
(1) using static file system attributes hindered us from providing useful ways to search and navigate data,
(2) extracting file relationships across storage silos without explicit support for relationships as \textit{attributes} was tedious,
(3) our visualizer had no mechanism of maintaining privacy and security when storing these attributes outside the storage silos.
Our observation is that dynamically generated, per-user namespaces show promise when used with today's multi-silo reality.
However, this promise cannot be realized without storage system support and integration.
We propose the Nirvana architecture, which separates \textit{location names} from \textit{context/semantic names}, and provides
the latter \textit{outside} of the storage system via a separate namespace service.

The idea of having distinct namespace providers delivering a global namespace is not new~\cite{howard1988scale,kazar1990decorum},
nor is the idea of using file metadata to generate namespaces ~\cite{gifford1991semantic}.
Nirvana is novel in that it combines these pre-existing ideas, introduces new approaches, and is specifically designed to support today's multi-silo world.
Nirvana combines 1) support for multiple namespace providers,
2) an expansive view of storage meta-data,
3) embracing search as an essential component of naming, and
4) a separation of user-visible names from silo-local names.

The rest of this paper introduces Nirvana and explores what system support is required to realize this vision.

\subsection{Background}
\label{hotos21:background}

We are certainly not the first to propose either that naming is important~\cite{pike-naming},
the need for better file system search~\cite{gifford1991semantic,mogul1986representing,Seltzer2009},
or personal namespaces~\cite{10.1145/155848.155861}.
The HCI community has similarly been observing both issues and potential alternatives for
decades and it remains an open area of active research~\cite{malone1983how,bergman2019factors,CHS:Medium:2019,vitale2020personal}.
We discuss existing technologies to determine those that we can use in constructing our solution.

\paragraph{Separation of location and semantic naming}

Cloud storage systems, such as Google Cloud and AWS S3 already decouple location names from human-readable names.
In both systems, a user assigns to an object a \textit{key name} and places the object in a \textit{bucket}~\cite{google-cloud-storage,aws-s3}.
The bucket hierarchy is flat, and a richer namespace is available for user convenience outside the object store itself.
For example, S3 allows the user to simulate logical hierarchy using key name prefixes and delimiters, but the support
for inferring this hierarchy is part of the AWS S3 console, not the S3 storage itself.
Likewise, the S3 console supports a concept of folders.
Google cloud provides similar features~\cite{google-cloud-naming}. Neither objects nor buckets can be explicitly renamed within the storage system,
but an entity \textit{external} to it, Google Cloud Console, will simulate the renaming for users' convenience.

In each of these storage silos, something builds a namespace on top of them for us to access, but each of these constructed namespaces is \textit{tied to the particular storage silo}.
We do not know of a convenient method to navigate data using a common namespace across silos.

\paragraph{Application-defined namespaces}
As with naming, there are different kinds of namespaces in use today.

First, there are virtual machines or containers that provide applications, and even entire operating systems, with their own namespace for resource management.
Within their isolated environment they are at liberty to organize objects as they please while the namespace provided by the hypervisor or the OS maps the VM/container-local name onto the backing store on disk.

Secondly, there are many application defined and managed namespaces.
Examples include groupwares like Google Workspace or Microsoft Outlook managing contacts, emails, calendar events, notes, etc., all of which may contain attachments.
Similarly, multimedia services store, organize and index videos and pictures providing a rich media library interface to the user~\cite{orr2020sample}.
Customer Relationship Management (CRM) systems store information about customers, orders, employees including documents for regulatory aspects.
Our final example, wikis, hold any kind of information organized in pages and links between them.
Each of these applications defines relevant meta-data and decides how to store it.
Even when it is possible to access the underlying assets, often useful information is lost in the process.
%While it may be possible to access assets that underlie these applications, e.g., a photograph, doing so typically loses useful information (i.e., meta-data), such as the enclosing album, geo-location, captions, etc. In other cases, accessing the underlying assets may simply be impossible.

Third, users often resort to embedding metadata in the only place they have/know: the file and path name~\cite{9229638, guo2012burrito}.
There are limitations to this approach: file and path names are constrained in length, character sets are restricted, and as names grow longer they are less useful to humans.
For example, photographs from a camera or smart
phone are devoid of anything but metadata, which means humans typically rely on thumbnail images to find things.  In addition, these names with embedded metadata are inherently prone to conflicts: changing an attribute of the file can mean changing its name, thereby breaking connections between the object and other objects or programs that may have referenced it.

\paragraph{Search} While conventional search (on a desktop or a mobile device) allows combining results of
a keyword search across different applications and the web, it does not allow for discovery of objects linked
by relationships more complex than sharing a keyword: e.g., lineage, time context (e.g., viewing one document
while simultaneously writing a second document), etc.
Furthermore, each search engine operates on whatever object metadata \textit{it} can extract (and deems useful),
but it could be faster and more effective if it could operate on explicit collections of metadata attributes securely and
persistently maintained across storage silos. %In addition, search engines are dynamic in their ability to index based upon what is being requested by users.  Thus, any static scheme becomes brittle, because it relies upon us to pre-determine the best answer. \mis{I do not understand that last phrase.}  In the example we provided earlier (Figure \ref{fig:ashish-demo}) we used \textit{similarity} metrics to find relationships between files and simple clustering mechanisms to find related content.

\paragraph{Combining many namespaces}

Federated namespaces are an approach to bridging the gap between storage silos by binding two or more naming contexts~\cite{namespace-federation-ibm, huawei, kubernetes, cloudera, netapp-patent}.
The simplest example is file system mount points, another is CORBA (Common Object Request Broker Architecture), and its CosNaming service~\cite{cos-naming-oracle, omg-naming}.
A crucial limitation of federated namespaces is the assumption that bound namespaces must use similar ways to name and navigate objects.
For example, on Unix you can mount another hierarchical file system, but not a key-value store. With CORBA, you can bind different CosNaming contexts, but not, say, an LDAP name server implementation.
Furthermore, the federated model assumes that a storage system provides a human-readable namespace.
We propose to separate the object storage from namespaces (\autoref{sec:arch}). This model makes adoption and integration of existing storage silos easier.

\paragraph{Extended Attributes}


%\mis{Need to document the history/evolution of EA before citing the problems with them. What ancient file systems had them? Relate EAs to property lists and Appl forks/data streams as per Tony's comment that this is replacing. I think the story line that needs to be filled in is something like what follows.}

% . The version of FAT32 for OS/2 contains support for extended attributes in the form of a special file.
% nwfs   (1986)  - Novell
% HPFS   (1988)
% ext2   (1992)
% NTFS   (1993)
% HFS+   (1998)  - Apple
% ods-5  (1998)  - DEC
% UFS2   (1999)  - UNIX
% newer: BeeGFS, ReFS, ext3/4, lustre, f2fs, gpfs, gfs, reiserfs, xfs, jfs, qfs, bfs, advfs, nss, apfs, vxfs, udf, zfs, btrfs, squashfs

% UFS2 implementation: http://fxr.watson.org/fxr/source/ufs/ufs/ufs_extattr.c

Extended attributes have been supported by file system for more than 30 years, first appearing in the Novell Network File System in 1986.
Since then many file systems have supported them, including HFS+ (Apple), UFS2 (UNIX), NTFS (Windows), and ext2 (Linux).
Today, extended attributes are a common file system feature.
Some systems support multiple extended attribute namespaces that separate user-defined attributes from system or security attributes,
which require different permission levels to access~\cite{man-xattr}.
POSIX v1.e initially had support for extended attributes as a part of access control lists (ACL) but dropped it due to lack of interest\cite{posix-acl-linux}.
OpenBSD dropped their support for the same reason\cite{extattr-openbsd}.
The result is that we cannot rely on the file system to provide a standard
and reliable way of saving user-defined metadata with a file and thus are not a viable solution for
our problem. A list of issues that are caused by the current state of extended attribute support:

\emph{Space Limitations:\ }Extended attributes are limited in size, e.g., ext4 allows up to 4KB, Amazon S3 objects support up to 2KB. The extended attribute space must be shared among multiple applications, so neither an application nor a user can know whether it is even possible to add a new attribute or what will happen if a security application does not have enough room left for its attributes.

\emph{Unique names:\ } Since extended attribute space is shared by different applications, there is a potential for a clash in the names of the attributes from different applications. An extended-attribute key that works on one particular silo might not work on a different silo.

\emph{Loss of information:\ } Not all underlying file systems support extended attributes. Therefore, we cannot save metadata on all file systems and cannot preserve metadata while moving a file from one file system to another.

%\emph{Change of information:\ } Moving a file from one file system to another may alter the metadata associated with the file. For example, moving a file from ext4 to FAT32
% For example, copying a file between ext4 and NTFS can result its permissions being changed from 600 to 777.
% This issue is not related to EAs, it is related to having disparate security models and the lack of urgency in constructing a robust mapping between them.

\emph{Uncertified attribute modification:\ } There is no method to prevent an application from accidentally or otherwise modifying an attribute that it does not own. Furthermore, since modifications are uncertified, it is impossible to determine the origin of the modifications and trust the values of the attributes.

\emph{Non-compatibility:\ }Support and conventions are not consistent across implementations. Some file system attributes limit the length of attribute names; others limit the usable character set.

%Copying files between two different systems can result in loss of metadata, if its not specifically stored using a different mechanism. For example, Microsoft SharePoint allows adding tags or properties to files, and create versions. This information is not available when accessing files through the WebDav interface or copy the file onto a mobile medium such as an USB drive, worse it may even change the ownership information if the SharePoint user and the user downloading the file differ.

%\tm{This is actually an important problem/issue - we need to deal with this at systems level because \textbf{any} solution that tries to externalize this effectively means that this external metadata is fragile.  In some of my prior work, we used a layered log structured file system approach to extending file capabilities and one of its strengths was that it keeps the data and metadata together.  Of course, systems like Ceph and Lustre certainly explicitly separate the data/metadata, but I would argue that this is both a problem we have to solve \textbf{and} a reason why this remains a systems problem: only the system can define how you connect the dots.  One interesting observation I had when thinking about what happens in a system with relationships is that it is suddenly very easy to keep back pointers from the object to all the things that reference said object (e.g., it's just a bidirectional edge).  That's quite useful, since it helps us get back to potential meta-data.  My original example here was the hard link problem - the only solution today is to do a brute force search of the namespace.  With edges, this becomes trivial: "find all the edges where the target is ID x".  While the edge $(v_1,v_2)$ can be required to be unique, we aren't requiring that $v_2$ only appear in a single edge.}


%JKN - not useful here: \joel{Joel's reading list\cite{10.1145/765891.765977,id3fs,tagfs,10.1145/2485732.2485741,10.1145/2611354.2611367,10.1145/2843043.2843868,10.1145/2901318.2901350,9079563}}


%\tm{So, what this says is "existing mechanism are not enough."  I agree with that.  At the same time, I'd argue that the 4KB limitation of ext4 (which surprises me, we picked 64KB in 1990, and that seemed somewhat small to me back then) is driven by the lack of perceived need for these.  For example, we didn't store ACLs in EAs (we called them property lists), and certainly NTFS doesn't do that either, precisely because you can't deal well with failure semantics.  So, why choose such a small number?  Nobody uses them, probably because they have limited utility as they are presently realized.}
%\jkn{Many file systems don't have the 4KB limitation - that is imposed by the VFS implementation on Linux. If we actually started to use EA's we could patch VFS to not have that limit. Interoperability would still be a problem though.}
%
%\tm{With respect to the "some attributes are special", I also agree. It likely makes more sense to simply have an enumerative interface reminiscent of how we handle files now: is there an "X" attribute associated with this object, give me all the attributes associated with this object, give me a subset of the attributes associated with this object, etc.  Then a file system can create and manage some of those attributes.
%As for what we do about file systems that have limitations, I think that's a really good question.  One thing we did with Episode (which had some funky semantics to support DCE/DFS) was defined an expanded set of vnode operations (the VNOPX operations, as I recall) that simulated some of those, while returning explicit errors for others.  For example, a COW-snapshot \textit{could} fail, and applications using that special feature had to be written to know it might fail.  Not a great solution, but a pragmatic one that worked OK at the time for allowing some interoperability.}
%
%\tm{So, how about building a namespace that relies upon an underlying KV store.  As long as the KV store provides get/put semantics, it's simple enough that we can integrate existing file systems (using the path as the key) as well as flat namespace systems (S3, Google Drive, etc.)  Then we'd need an interface that, given the URL for the underlying key, can return the set of attributes and relationships that WE define as working.  It's not an ideal solution, but it is a pragmatic one, and it doesn't preclude building a fully functional file system that provides the same interface (or even augmenting an existing system to do so).}
%\jkn{Sound good to me. We should stop calling it a namespace though - that word is overused and becoming ambiguous. The extended attribute names also belong to a "namespace", which during our Zoom meeting, Margo defined as a kind of owner (who created the attribute).}

%\tm{This helps explain why it's a systems problem: this needs a common interface, we've already claimed responsibility for arbitrary attribute storage, we've just done a poor job of it, and that's easily seen by the fact that nobody uses them and thus implementations are severely restricted.}

%JKN I summarized the features we want the FS to support here. I don't know if this is necessary or the right place, so feel free to move/modify/delete.
Ideally, all file systems would have a function to allow storing user-defined metadata.
It should provide a mechanism to define multiple unique attribute namespaces to avoid collisions.
It should also secure the namespaces so that application access can be restricted, e.g., only authenticated applications can write attributes in that namespace.
%It should use key-value store semantics for the common operations (insert, delete, update, search).
Finally, it should allow users to define their trust level with distinct creators of attributes within that namespace.
When we construct our global namespace, we can utilize such existing metadata, combining it with our own metadata, and storing it, possibly
outside the storage silo containing the specific object.

%\subsubsection{Example of a FS that uses metadata for improved performance}
%\textit{f2fs} \ref{lee2015f2fs} is a log structured filesystem (LFS) \ref{rosenblum1992design} designed for flash that uses its file based metadata to improve its performance. LFS writes all modifications sequentially to one active segment in a log-like structure, thus converting random writes into sequential writes. As a file gets modified, the original data blocks become obsolete and create holes in the segments. These holes need to be compacted into contiguous free space that can be used by LFS for future sequential writes; the process to do this compaction is called garbage collection (GC). GC involves collecting all the valid blocks in a segment and writing it to the end of an active segment.

%Writing data blocks based on temperature reduces the number of times a block is written by GC. Files that are never modified are cold and files that are frequently modified are hot. Files with access patterns in between are termed as warm. Writing files with the same access pattern based temperature reduces the number of times that GC has to write blocks. For eg: writing only non changing cold blocks in a segment results in a segment that does not require GC. As against this, mixing blocks that are hot and cold in a segment results in a lot of invalid blocks caused by overwrites to the hot blocks. During GC, the unchanging cold blocks from such a segment have to be written out to a new segment; these cold blocks continue to undergo GC as long as they are mixed with other hot blocks. To avoid this unnecessary writing, \textit{f2fs} writes data and metadata separately based on their access pattern; data/metadata is categorized into hot, warm and cold. To achieve this segregation, \textit{f2fs} uses filenames to decide how frequently a file will be accessed. At mkfs time, \textit{f2fs} is configured with a list of file name extensions for the three access pattern based temperature. As an example: media files ending with .mp3, .jpeg, etc are not expected to be modified and hence are always written to cold segments. A filename ending in .db is expected to be modified frequently and is thus written to a hot segment. \textit{f2fs} uses file based metadata to separate the writes to segments with different temperature, thus improving it's GC performance.


%\textbf{Why hierarchical namespaces are a pain}
%\tm{I'm not sure that it does.  It explores what makes \textit{implementing} them difficult on file systems (\textbf{rename} is the argument that I used in 1989 to include transactions and journaling in Episode versus ``careful update.'')  However, that probably doesn't have much to do with the direction we've taken.}

%Rename (and perhaps create) are very challenging and introduce complex ordering requirements for metadata persistence.

%Directories create problems: updating across directories creates inherent race conditions (e.g., "rename a/b c/d" and "rename "c/d a/b"), bottleneck/hot-spots (they are variable sized data structures with locality ramifications), multi-directory dependencies ("path walk"), scalability challenges, path length challenges (what happens when nobody expects to get a 32K character long path name?), iteration (how do you reliably and performantly enumerate across an arbitrary sized list of arbitrary sized objects without creating bottlenecks).

%Why only have one namespace?  Multiple views to the same data is used in visualization to improve cognitive understanding. Why is a single view (hierarchical name space) sufficient for file data?


%\reto{are we tackling application scenarios only, or does this apply to kernel level as well (e.g. /boot/initrd)}

%\tm{Is there any reason we should restrict it?  If anything, I'd argue that the kernel has zero benefit from the hierarchical namespace; it could just use a well-known identifier and skip the entire path walk.  Files that are used by \textit{programs} exclusively don't really seem to benefit from an hierarchical name space, but they need some location mechanism (the ``inode number''). At some level all storage is a KV store, we just do offset based indexing in some cases to keep it simple. Ceph explicitly separates data and meta-data in a way that basically distinguishes namespace from storage as well; they did so for performance. \cite{weil2006ceph}}

\begin{figure}[!tb]
    \centering
    \begin{tabular}{c}
        \includegraphics[width=0.95\columnwidth]{reference/hotos21/figures/nirvana-arch-8.png}
    \end{tabular}
    \caption{Nirvana Systems Architecture.}  %Separate Namespace service(s) from storage service(s).  Kernel namespace service permits use throughout OS lifetime, including boot.  User namespace service permits integration with remote namespace service and local namespace service, providing integration over the network.  Namespace as a Service (Naas) provides the ability to maintain shared namespaces across computational device boundaries; can provide primary or secondary name services.}
    \label{fig:systemsarchitecture}
\end{figure}

\subsection{Architecture}
\label{hotos21:architecture}

\newcommand{\REF}{reference}
\newcommand{\PROJECTION}{view}

\section{The Nirvana Architecture}\label{sec:arch}

%\begin{figure}
%  \begin{center}
%  \begin{footnotesize}
%    \begin{tabular}{|cp{0.75\columnwidth}|}\hline
%        silo & stores objects\\
%        object & the file's content\\
%        meta-data & the files attributes/properties\\
%        key & the unique identifier of the object\\
%        namespace & defines name context, high-dimensional attribute space \\
%        projection & construction a lower-dimensional namespace from a higher-dimensional namespace\\
%        name & a friendly/meaningful object name within the namespace. \\
%        locate & the process of obtaining an object's key within a context/namespace. \\
%        search & the process of locating an object based on a search query. \\
%        navigate & the process of locating an object based on following meaningful names \\
%        Overall & I think this is powerful: the user can define the namespace and its projections.
%    \\\hline\end{tabular}
%\end{footnotesize}
%    \end{center}
%  \vspace{-4mm}
%    \caption{Terminology (will go away for final submission)}
%  \label{fig:terminology}
%\end{figure}


% Surbhi asks that I include multiple cloud storage silos in the picture.  I wonder if we could add network file systems as well; I just didn't want them to be below
% the namespace.
% Puneet indicates that the text is too small.  We both agreed this needs to be centered and a bit larger (it feels like I'm wasting space).

%\mis{I did a commit before changing this section, but I wanted to tweak some terminology and provide a gentler introduction to this section. I worry that asking the reader to adopt a whole new terminology before telling them what we are really proposing is too great a burden.}
%\tm{I actually agree; defining the terms up front is, however, a useful tool for us when writing the text.}

We propose separating context/semantic naming from location and exposing this to users.
This separation will enable users to navigate/search data across multiple silos independent of where data is stored, and their organization preferences.
To realize this, we present the Nirvana architecture.
In Nirvana, storage silos need not change how they store objects, assign them silo-local \textit{location} names,
support their choice of attributes, and provide ways for users and other services to access them.
More interestingly, we absolve storage silos from providing \textit{context/semantic} names.
Instead, we introduce \emph{Nirvana namespaces} and \emph{namespace services}, which integrate one or more silos to provide \textit{context/semantic} names and access capabilities. In the following discussion, we use the term \emph{object} to refer to a discrete unit of storage, such as a file stored in an existing storage silo. Objects can reside on device-local file systems or in the cloud. Users access objects via requests to namespace services, which can be either local or cloud-based (see Fig.~\ref{fig:systemsarchitecture}).

\paragraph{Namespaces} In a Nirvana namespace, the name of a data object simply becomes a \textit{collection of metadata attributes}. Attributes are tuples consisting of, e.g., type, name, value, and author.
To date, we have identified three attribute types: \emph{authoritative}, \emph{annotated}, and \emph{autogenerated}.
Authoritative attributes are statements of fact provided by the silo, e.g., file size, creation time, content hash.
The author of an authoritative attribute is the silo containing the object; if the object is replicated in multiple silos, the author is the originator of the object.
Annotated attributes are provided by users or applications, e.g., Word document attributes, user-supplied tags, image metadata.
The author of such attributes is the user or program providing the annotations.
Finally, autogenerated attributes are produced by services, e.g., the cat video detector, a keyword extraction service.
Authors are responsible for defining the semantic meaning of their attributes.
Nirvana allows an unbounded number of attributes per object and requires only one: the authoritative location attribute,
which is a set of identifiers that refer to the silo and silo-local name(s) by which an object can be accessed.


%In the following discussion, we use the term \emph{object} to refer to a data element stored on and managed by Nirvana.
%Objects can reside on device-local file
%systems, network attached file systems (e.g., CIFS/SMB, NFS),
%distributed storage that projects onto a local storage silo (e.g., %OneDrive, Dropbox), and
%cloud storage systems of any type
%(e.g., Amazon S3, Azure Cosmos, Oracle Server, MongoDB).
%Users access objects via requests to namespace services, which can be either local or cloud-based services.\footnote{If a cloud-based services is used, we assume a caching layer on the local device to ensure access to data during disconnection. In the simplest case, the caching layer degenerates to the local file system namespace.}

\paragraph{Namespace Services}

The namespace service provides support for three basic operations: (1) the ability to create a Nirvana \REF, which is a set of metadata attributes;
(2) the ability to map a {\REF} to the underlying storage location and vice versa; and
(3) the ability to query the namespace to find {\REF}s that match a  pattern.  These {\REF}s are immutable: deletion is achieved using a tombstone and update by creating a new {\REF}, with a new creating timestamp.  This preserves history and allows versioning, assuming storage silo support.

% Moved from Intro to here.  See if we can capture it.
% Location is usually a one-to-one, or one-to-many relationship.
% Context/meaning are many-to-many relationships.

%A Nirvana namespace service runs on a local system or in the cloud, may work with an OS-integrated
A \textit{namespace provider} may also provide an API used by applications to interact with namespaces, allowing construction of new visualization and search tools.
For example, if you are searching from your smartphone, you can find a specific document, regardless of where
it is stored, whether that is your cloud storage system, a desktop computer, or some other storage location and even if it is not presently accessible.
Conversely, this model can be constructed to permit maintaining multiple distinct namespaces, so that each user is able to see \emph{{\PROJECTION}s} of their own data.

\paragraph{Locating specific objects} From our description so far, it seems that Nirvana is all about providing fertile ground for effective search, but it may be unclear how the user could \textit{locate} the specific object they need. We provide an example to illustrate how Nirvana can support this.

% Traditionally. Name and hierarchy of directories.
% In Nirvana, these can become attributes, and a namespace provider can simulate a legacy hierarchical namespace. Support users who find it useful and legacy tools, such as compilers.
% At the same time, allows users to gently depart from this model. For example, HPC community embed metadata elements of their objects into file names, users can now store these  elements explicitly with the name service, while also retaining the desired part of an old-fashioned name as a separate metadata element.

In traditional systems, a user gives a file a memorable name, e.g., \texttt{eddie.txt}.
In addition, hierarchical directory structure provides additional context for the object, helping the user locate the precise instance of \texttt{eddie.txt} that they need.
For example, there can be a \texttt{faculty/recommendations/eddie.txt} and \texttt{papers/fantasstic/eddie.txt}.
Nirvana supports locating objects via memorable names by assigning the object a user-provided attribute, e.g., \texttt{eddie.txt}.
Similarly, to simulate navigation by directory hierarchy, an object can be assigned attributes corresponding to the path components.
In this way, Nirvana can also support legacy tools relying on hierarchical directory namespaces.

At the same time, Nirvana can help users go beyond using memorable names.
For example, in the experimental science community, users traditionally embed the experiment context into the name of their data files~\cite{guo2012burrito}.
In Nirvana, users can explicitly store those metadata elements in a namespace.
Then, grepping for, say, \texttt{experiment-data.txt} can return multiple instances, but their metadata attributes can be explained via \texttt{stat}, and a \texttt{``diffstat''} command could also show the differences in the attributes between the two {\REF}s.

While hierarchical file system provide us with path and file names for search, our approach can make use of other augmented metadata elements, allowing us to narrow the results.

% (1) We don't lock ourselves into a set of attributes, so we don't need prescience. We don't need a big centralized indexer that knows all about all types of files.
% (2) We can customize these "tags" based upon the user's own behavior, and can be added to an indexing list.
% We can use the attributes to provide the "most relevant" results in the visualization tool(s).


\paragraph{Sharing, Access Control and Privacy}
% - access control  -> who can access the file, what part of the file, what attributes (all, just a subset, ?)
Enabling safe sharing requires a clear model of access control and a clear understanding of possible privacy implications.
While traditional file systems do have a vague notion of namespace access control using directory permissions, we are now faced with two explicit levels requiring access control: the object itself and the metadata.

What rights exist on each level and how do they relate to each other?
In Nirvana, possession of the {\REF} does not imply any rights to access the object, since access is managed by the storage silo itself.
The namespace provider has the ability to encode encrypted values within metadata elements, which also allows storing of sensitive information within the name without compromising security --- anyone with the Nirvana {REF} must still also have the relevant key to understand the value of specific attributes.
Similarly, this permits storing metadata in a third party service without compromising the contents of the metadata attribute.
%Utilizing digital signatures to protect the metadata element can be used to verify that the decryption key used is correct, or it can be used to verify that the value itself did originate with the specified authority to prevent spoofing.  This ability to secure individual metadata elements against tampering protects from malicious actors tampering with metadata elements as well as permits users to define their own unique rings of trust based  upon their own (or their organization's) criteria.

%Intuitively, a user may only perform an operation above if they posses the corresponding rights. There are two access control methods: access control lists and capabilities. Which form will prevail for namespaces and for the objects themselves?

%In addition, there might be certain rights a user needs to have in order to further share the object (or namespace) with other users or add the object to another name space. How can we revoke the rights from users and who is allowed to revoke -- is this an explicit right or implicitly given to the owner?

%Namespaces may contain many properties of their objects that in turn contain many bytes of data. Can we assign rights to particular properties or bytes?
%\reto{github access token}
%\tm{Remember in Twizzler, where they constructed objects via composition?  I'd argue that you \textit{could} do exactly the same thing and use it to control logical regions.  That's definitely not something we do with the existing APIs, but it could be something a file system permitted via a more flexible interface.}

%Given that a name in Nirvana includes the metadata necessary to locate the object in the global storage space using a URI, it is easy for names to be shared between namespaces.  There is no inherent requirement that the recipient of names need trust the metadata elements of that namespace in the same way the original owner did --- each user can define their own unique "rings of trust" with respect to individual metadata elements.

An important element of the Nirvana namespace model is the ability to identify not only the \textit{type} of a metadata attribute,
but also the \textit{authority} that provided such data.
To accomplish this, we need to have a mechanism for allowing verifiable identification for the metadata attributes that make up the name.
However, rather than propose a new identity and access management (IAM) service,
we simply note that we expect Nirvana to work with a least one such service (Fig.~\ref{fig:systemsarchitecture}).
In addition, this model also allows for a specific treatment of trust relationships that need not rely upon the namespace itself.
In essence, this allows the user to define their own ``security rings of trust.''

Nirvana Namespaces detect storage changes via a publish/subscribe model, similar to prior work~\cite{birman1987exploiting,9229638}.

\subsection{Research Opportunities}\label{hotos21:research}

%The use of namespaces has changed, though the design and implementation has not evolved substantially.  Recent initiatives by commercial
%entities such as Google, Amazon, NetApp, and Qumulo to improve naming within storage silos indicates we have reached the breaking point.
%While allowing per-silo namespace solutions is one possible future, we view it as important from the \textit{user} perspective to avoid
%this type of single-vendor lock-in.  However, to achieve our Nirvana model will require research to satisfy a number of outstanding
%questions that our design has raised.

%We identify five open areas: what is left of a file system when the namespace is removed, what operations should we support on namespaces
%versus on silos, what does access control look like, what are the implications of privacy and security, how can storage silos utilize meta-data
%to smartly optimize storage.

\paragraph{File System Evolution}
% - what's left of the file system if we have the two kinds of naming

With a distinct namespace implementation, we need to consider questions on how to optimize their interaction.
For example, should we permit the namespace implementation to store its metadata within the storage silo?
How can the storage silo access metadata from the namespace?  How can we optimize storage silos when
they are not burdened with the different needs of namespace and storage management?  Can we provide
mechanisms to notify the namespace when changes occur within the storage silo?
These questions arise because we separate the traditional, hierarchical file system structure we have known for more than a half century.
%Is there a better implementation for extended attributes such that other ecosystem can use them? The current implementation problems are not unique to our ecosystem but general.
%How will the layout of the FS change when we associate more user specific, application specific contextual metadata with files?
%Can the algorithms to access these be improved? For eg: can we provide a hook for user space functions to run to create the contextual metadata when the file changes?
% By separating the two kinds of naming in namespaces breaks with the traditional, hierarchical file system structure as we know it for half a decade. The question arises of what is left from the traditional file system?
% This seems to translate into concept of a ``thin-waist'' for storage, an object store like construct where the objects are retrieved solely based on their key.

%\subsection{Supported Operations}
% - operations (interoperability between name spaces)
Traditionally, users see storage as a convolution of location and its context within the storage, i.e.~the path of the file defining its location and context.
Performing a file system operation was reflected on both the file system context and the backing store at the same time.
A clean separation of this two concepts into the storage (location name) and namespace (semantic/context name) raises fundamental questions on what kind of operations are supported by each of those two concepts, how  they relate to each other, and how closely coupled they are.

%\paragraph{Store Operations}
%Store operations are applied on the object itself. Users should be able to create/delete objects, as well as read/write them.
%While those operations seem obvious, it is less clear for other aspects like rename and versioning: how does creating a new version or renaming it play together with the namespaces this object may appear in, and how do we refer to a particular version?
%Given users will use the namespace to search for their objects, does renaming even make sense?
%Based on the four operations mentioned previously, one can build more complex operations like copy or move to, for example, move an object to another silo.
%By doing so, should the object identifier be preserved? What happens if we forget the identifier?
%If not, how are we keeping the namespaces in sync with the new location of the object? Is there an equivalent for the HTTP 3xx status codes indicating temporary or permanent redirects?
%Lastly, users want to share access to objects, e.g.~for collaborative editing.
%We will talk about the rights and permissions involved in the next section with respect to sharing objects.

%\paragraph{Namespace Operations}
%Namespace operations are applied on the namespace. Similar to the objects, a user may want to create its own namespace, or delete it again if its no longer used. The central question here is where do these namespaces reside and how do we refer to a particular namespace?
%Is there some kind of namespace service, or are they running on a users local machine and thus may not always be accessible? Is there some kind of root namespace to find other namespaces? Is there some kind of register operation?

%Note, that an object may be added to to multiple namespaces, and it may be the case that the alst instace of it within an address space is deleted. What happens then? Can we even know that the last instance is deleted?

%Within a namespace, a user will likely want to add or remove objects, modify their meta-data information, and search the namespace for possible relevant objects based on a query or other objects with are alike a located one. What does it mean that two objects are relate?

%Besides searching, enumerating all objects that have been added to a namespace might also be desirable -- a potentially expensive operation. To what extent can we use the file contents to pre-populate the meta-data of the file? For example, CSCOPE builds an index of referenced symbols of source code files.

%Besides creating or deleting namespaces, a user may want to merge or unifiy two namespaces, or create a new subnamespace containing only the objects related to a project for instance -- an operation which may be expressed as create plus merge of a search result.

%Similar to the objects, a user may want to share an address space with other users to collaboratively work on a project, or because a business process requires approval of multiple people.

%Lastly, what is the interface to an namespace and how do they interoperate wich each other?


\paragraph{Differential Privacy and Security}
% - differential privacy & security (GDPR compliant)
The notion of access control mentioned above raises the question of how the security model used in this two layer system, as well as potential privacy issues: the meta-data may reveal much information about the object itself, even its presence, replication factor or temporary unavailability within a namespace is leaking information. How do we control this kind of information flow on the namespaces and the backing store?

Can we provide some form of differential privacy to the user of the name space? Effectively by prohibiting the user from enumerating the content of the namespace, and restricting queries to those who only produce one result -- in the sense you need to know precisely what you are looking for. As a more general aspect: how do we effectively prevent data leakage?

%In the context of privacy regulations like GDPR a user has the right to be forgotten. This implies we need to be able to not only find the relevant objects, but also the corresponding meta data. Is this feasible?
%Besides the namespace and object ownership, is there a concept of ``content'' ownership?
For example, in current namespaces there is no simple mechanism for finding all of the references to a specific object (e.g., the embarrassing photo that needs to be forgotten).
Nirvana provides the ability to find such references.  Can we create a GDPR compliance system that guarantees to expunge such references?


\paragraph{Using Metadata to Optimize Storage}
% - use meta data to organize the underlying storage
Currently, we have to explicitly decide on where we want to store our data by manually selecting from various data silos providing different properties, guarantees and methods of sharing.
Having a rich namespace containing meta-data for the objects, could we leverage this information to deduce the best method of storage?
For instance, the origin of the information may influence the selection of the data silo, whether data is to be encrypted, an expiration date assigned, creating multiple instances of the object in different silos, or whether we may want to keep track of different versions of the object.
Moreover, one may want to keep multiple, synchronized copies of the same object to support different access pattern or ensure locality.
How does regulation come into play (e.g., a object must not leave a certain jurisdiction)?
One of the central aspects of storage optimization is the question of which entity performs this?
This will likely require some form of synchronization among multiple namespaces that know about the object to ensure stability and avoid frequent changes to the storage.

\subsection{Conclusion}
\label{hotos21:conclusion}

We demonstrated providing useful ways to navigate/search across storage silos is not simply an HCI problem: it requires new system support.
We proposed Nirvana that, at its heart, separates location names from context/semantic names, and maintains those names in a cross-silo
name service separate but complementary from the storage. We presented research questions to address when building Nirvana.

% Nirvana

\section{HotStorage 2021}
\label{ch:appendix:section:hotstorage21}

\subsection{abstract}

Users store data in multiple storage silos, such as Google drive, Slack, email,
Dropbox, and local file systems that mostly rely on traditional user-assigned
names. A user who wants to locate a document that she saved while having a
conversation with her colleague on a specific subject last month will have a
hard time finding that document if she doesn’t remember in which silo it was
stored or what name it was given.

Prior work that introduced a \emph{Placeless} storage architecture enabled
cross-silo search using semantically meaningful attributes, while other prior
work used data provenance to construct a user's \emph{activity context} (e.g.,
what they were doing at the time they created or accessed data) to aid in
document location. We take a position that despite these prior systems that
demonstrated rich semantic search capabilities, we still cannot provide these
capabilities using existing system APIs and abstractions.

We explain that this is not simply an HCI problem and identify the systems
problems that must be solved to realize this vision. We present \emph{Kwishut}, an
architectural blueprint for enabling semantically rich, multi-silo data
management.

\subsection{Introduction}

Today's file systems provide two primary functions: a way to store chunks of data and names that provide users with the familiar
metaphor of a file cabinet (i.e., folders and documents).
Much has changed since we adopted this design.

Now most data resides not on local file systems but on myriad services such as
Dropbox, Google docs, Amazon S3, Microsoft OneDrive, and Github, as well as
attached to communication mechanisms and applications such as email, chat, and
collaborative communication platforms (e.g., Slack, Discord).

Today, just like local file systems, each of these storage solutions provides
its own storage and naming mechanisms. Pity the user Alice who wants to find the
file that was sent to her by Bob while they were having a slack conversation
about cool papers in HotStorage 2020, if she remembers neither the name of the
file nor whether she stored it in Dropbox, Google drive or her local file
system.

The \emph{Placeless} architecture~\cite{placeless-tois} provided an elegant
solution to this problem by enabling search across different storage silos using
semantically meaningful names. Each file was annotated with a rich set of
attributes, determined by the user or generated by software, and a naming
service, spanning silos, searched the entire collection of user files using
these semantically meaningful attributes. \emph{Placeless} was a huge
improvement over isolated storage silos and semantically-poor names, but it did
not solve Alice’s problem. Finding Alice's document requires that we track files
across storage silos and \emph{activity contexts}. An activity context describes
the other activities that were happening at the same time a document was
accessed. This context might include applications, such as Slack, email or a web
browser, which are not file systems in any traditional sense, but can be sources
and destinations for data and for its semantically meaningful context.

The only solution of which we are aware that captures activity context is
Burrito~\cite{guo2012burrito}, which used data provenance to keep track of a
user’s activity context, i.e., the applications they were running and actions
they were taking while examining a particular file. Unfortunately, Burrito is a
desktop application that was intended neither to work across multiple devices
nor to span multiple, remote storage silos. While it introduced the idea of
activity context sensitive search, it did not address any of the semantic
searching issues of Placeless and it did not consider the privacy consequences
of storing user context in a distributed environment.

We posit that \textbf{1) user naming should be entirely decoupled from local naming and 2) users need customizable and personal namespaces.}
We present \emph{Kwishut}\footnote{\emph{Kwishut} means ``naming'' in a native North American language.}, an architecture embodying this position, leveraging existing infrastructure where possible and extending it where necessary.
\emph{Kwishut} uses separate metadata and naming services coupled with user activity monitors.
\emph{Kwishut} is designed to allow incorporation of existing storage, metadata, and naming services without modification, while providing enhanced functionality when services support \emph{Kwishut} features.
We limit discussion to the systems infrastructure required to realize this vision; current storage management interfaces (e.g., file browser) can use \emph{Kwishut} namespaces directly, while the availability of rich metadata in \emph{Kwishut} enables HCI research on better ways for users to identify and find their data.

We begin with use cases motivating the need for \emph{Kwishut} (\S\ref{hotstorage:use-cases}), highlighting specific features missing from today's storage and naming services. Next, we present the \emph{Kwishut} architecture (\S\ref{hotstorage21:arch}) and future research directions it enables (\S\ref{hotstorage:future}).
We then discuss how \emph{Kwishut} builds upon prior work
(\S\ref{hotstorage:background}) and conclude summarizing our position (\S\ref{hotstorage:conclusion}).

\subsection{Why we need \emph{Kwishut}}
\label{hotstorage:use-cases}

%%%%%%%%%%%%%%%%%%%%%%%%%%%%%%%%%%%%%%%%%%%%%%%%%%%%%%%%%%%%%%%%%%%%%%%%%%%%%%%%%%%%%%%%%%%%%%%%%%%%%%%%%
\begin{table*}[!th]
    {\renewcommand{\arraystretch}{1.3} %<- modify value to suit your needs
        \begin{tabular}{p{0.11\textwidth}p{0.4\textwidth}p{0.425\textwidth}}
            \hline
            \textbf{Feature}                                                                                                                                        & \textbf{Existing Technologies} & \textbf{No Solution} \\
            \hline
            \usecaseactivitycontext                                                                                                                                 &
            timestamps and geo-location, image recognition, browsing history, ticketing systems, application-specific solutions like Burrito~\cite{guo2012burrito}. &
            Link related activity across apps, record  browsing history and chat conversations relevant to the creation of the data object, storing it in ways that are secure and compact.
            \\
            %
            \usecasecrosssilosearch                                                                                                                                 &
            Search by name, creator, content across silos,
            app-specific searches (e.g., Spotlight)                                                                                                                 &
            Unified search across all kinds of storage, including file systems, object stores, apps and devices
            \\
            %
            \usecasedatarelationship                                                                                                                                &
            De-duplication of documents, versioning of specific files, git ancestor relation                                                                        &
            Explicit notion of data identity, tracking different versions across different silos as data is transformed
            \\
            %
            \usecasenotifications                                                                                                                                   &
            File watchers (INotify), synchronization status, manually inspecting modified time                                                                      &
            Ability to subscribe to specific changes on attributes
            \\
            %
            \usecasepersnamespace                                                                                                                                   &
            Hierarchy plus hard/soft links. Use of tags.                                                                                                            &
            Creating personalized namespaces with with flexible data organization and views
            \\
            \hline
        \end{tabular}
    }
    \caption{Use-case driven functional requirements.}
    \label{hotstorage21:usecases}
\end{table*}
%%%%%%%%%%%%%%%%%%%%%%%%%%%%%%%%%%%%%%%%%%%%%%%%%%%%%%%%%%%%%%%%%%%%%%%%%%%%%%%%%%%%%%%%%%%%%%%%%%%%%%%%%

%Each storage silo provides a specific set of features facilitating certain use-case scenarios, for example, the local disk provides offline access to data, while cloud-based solutions allow users to collaboratively work on a shared document.

We identified five categories of information that are necessary to facilitate the integration of semantically meaningful naming with user activity context.
Unfortunately, to varying degrees, these features cannot be provided by today's storage system architecture.
We introduce these categories, summarize them in table~\autoref{hotstorage21:usecases},
and then present use cases demonstrating how they facilitate data management.

\subsubsection{Feature Wish List}
\label{hotstorage:features}
\noindent\textbf{Activity Context: }
As Burrito demonstrated~\cite{guo2012burrito}, the \emph{context} in which data were accessed or created is often a useful attribute on which users wish to search, e.g., ``\emph{I'm looking for the document I was editing while emailing \persa about their favorite wines}.''
To the best of our knowledge, there is no modern system that supports queries using rich context across applications.
We might be able to use timestamps or application-specific tags or history information in queries, but it is laborious, if not impossible, to intersect data from multiple applications and/or multiple silos.

\noindent\textbf{Cross-Silo Search: }
Users share documents in myriad ways: via messaging applications, on cloud storage services, and via online applications. Users should not need to remember which mechanism was used to share a particular document and should have some easy way of organizing and searching through a collection of such distributed documents.

\noindent\textbf{Data Relationships: }
Documents can be related in arbitrary ways. This relationship information can be used to facilitate and enable better search results. So far, we have identified three specific relationships that are particularly important:

\noindent\emph{1) copy} is a bit-for-bit identical replica of some data, in other words two items with different names store the same data.
% Deduplication functionality in storage systems frequently takes advantage of the prevalence of copies to reduce storage consumption. However, knowing that two items with different names are, in fact, the same is also valuable information for users.

\noindent \emph{2) conversion} is a reversible, repeatable transformation that changes the representation of data, without changing its semantics, e.g., converting a CSV file into JSON.

\noindent \emph{3) derivation} refers to data that has been computationally derived from another object by altering its content, e.g., adding a row to a spreadsheet.

While storage systems can recognize copies, they cannot distinguish conversions from derivations. However, from a user's perspective, these operations are quite different: a conversion can be repeated, which is not necessarily true of a derivation.

\noindent\textbf{Notifications:}
Users frequently want to be notified when documents change, and many storage services offer this functionality.
However, users might also want notification when data on which they directly or indirectly depend changes. This requires both a notification system and an awareness of the data relationship between different objects.

\noindent\textbf{Personalized Namespaces:}
Users have different preferences and mental models to organize their documents, a source of conflict in a multi-user setting. We need a way to provide each user the ability to personalize their document structure.

\subsection{Use Cases}
The following use cases illustrate how the features described above arise in common place activities.

\noindent\textbf{Data Processing:~}
\persa and \persc are preparing a report summarizing their work on a data analysis project for a customer.
\persc sends an email to \persa containing a CSV file with original data.
\persa opens this document in Excel, formats and filters it, adds additional data from a corporate storage silo,
and then returns the Excel document to \persc on Slack.
\persc is away from their desk when it arrives, so they open it on their phone, uploading it to a cloud drive.
\persc then sends the link to \persa for editing with update notifications.
Finally, \persc sends a PDF of the report to the compliance officer who promptly asks, ``Where did this data come from?''

\noindent\textit{Feature Analysis:~}
This use case highlights the need for 1) \usecasedatarelationship, as it has instances of copies, conversions and derivations, 2) \usecasecrosssilosearch, as these items are located in multiple silos and accessed by multiple devices, and 3) \usecasenotifications, as update notifications need to be distributed to designated users.

\noindent\textbf{Delete Request:~}
Some time later, the compliance officer requests that all documents containing a customer's data must be deleted.
To help with finding all relevant customer data, \persb joins the project and examines the report and requests the original data from which it was produced.
\persa remembers that they gave the original data to \persc shortly after collecting it, but does not remember the name, location, or even how the relevant files were transmitted. Thus, \persa has to manually search possible locations and applications, sendsing references to documents to \persb, who then starts organizing these files to methodically identify the ones that might contain the customer's data. In the process, many of the other team members' references to the documents stop working.

\noindent\textit{Analysis:~}
This use case illustrates the need for 1) \usecaseactivitycontext to capture
data that has been collected while interacting with the customer, 2)
\usecasedatarelationship to identify related documents, 3)
\usecasecrosssilosearch to easily locate relevant documents across data silos,
and 4) \usecasepersnamespace to create a individual data organization.

% \noindent\textbf{Security and Privacy:~}
% \persg, an investigative journalist who routinely receives sensitive information from third parties, is investigating the company from the prior use cases.
% \persg needs to be able store and access sensitive information, including information about the activity context of various e-mails, documents, pictures, and audio and video files.
% While \persg ensures that these data are encrypted, they need to also ensure that they can both find information and ensure that meta-data associated with those files is both usable and properly protected across silos.

% \noindent\textit{Feature Analysis:~}
% This case requires both \usecasecrosssilosearch and
% \usecaseactivitycontext to allow \persg to gather information obtained from specific meetings or at a given time/place.
% While \persg must protect their sources, they must also be able to associate evidence with those sources to make judgement calls about their validity, so we must design security and privacy policies for attributes that accomplish both.

\subsubsection{From Use Cases to Architecture}

Each use case and feature class suggests capabilities that are unavailable
today.

In~\autoref{hotstorage21:usecases} we identify existing technology that can be brought to
bear on the problem while teasing apart the precise details that are missing.

Repeatedly, we find that critical information necessary to provide a feature is
unavailable, that providing such information is non-trivial, and that obtaining
it creates a collection of privacy challenges.

\subsection{Architecture}
\label{hotstorage21:arch}

\begin{figure}[!tb]
    \centering
    \includegraphics[width=0.45\textwidth]{reference/hotstorage21/figures/Naming5-legend.png}
    \caption{\emph{Kwishut} Architecture (\S \ref{hotstorage21:arch}).
        %Grey boxes indicate new components. AM=``Activity Monitor'', MS=``Meta-data Server'', UNS=``Update Notification Server'', NS=``Namespace Server''.
    }
    \label{fig:arch}
\end{figure}

%\reto{Using meta-data service that must inter-operate (federated meta data) and relationships are not first-class citizen, cannot glue the meta-data service together with naming services to enable the things we want to do. }
%\reto{the storage location is independent on the notion of related files: meta-data service treats relationships as first-class citizens. }
%\reto{get the attributes out of the silos --> currently: this is done manually}

\emph{Kwishut} is a family of services that enable sophisticated search and naming capabilities.
The key features that differentiate \emph{Kwishut} from prior work are:
\textit{1)} incorporating object relationships as first class meta-data,
\textit{2)} federating meta-data services,
\textit{3)} recording activity context,
\textit{4)} integrating storage from multiple silos, and
\textit{5)} enabling customizable naming services.
Data continues to reside in existing and to-be-developed storage silos.
\emph{Kwishut} interacts with these silos, collects and captures metadata, and
provides a federated network of metadata and naming services to
meet the needs of the use cases in \S \ref{hotstorage:use-cases}.

\subsection{\emph{Kwishut} Services}

Figure \ref{fig:arch} illustrates the \emph{Kwishut} architecture. \emph{Kwishut} allows for
different deployment scenarios. The five services can be run independently, they
can be co-located and bundled together to run on a local device, integrated into
an OS, or available as web-based services.

In the discussion below, parenthesized numbers and letters refer to the arrows
in Figure \ref{fig:arch}. There are five main components:

\noindent\textbf{1) Metadata servers (MS)}
are responsible for storing attributes and provide a superset of capabilities found in existing metadata services~\cite{federatedMetaData,smartstore}. Users can register an MS with activity monitors or attribute services, which allows the MS to receive updated attributes from storage objects and activities (B). Thus, there can be multiple sources of attributes including the user itself. Metadata servers may retain the full or partial history of attribute updates or maintain only the most recent value.

\noindent\textbf{2) Namespace servers (NS)}
connect to one or more MS and use the metadata to provide users with a
personalized namespace that allows both manual organization (i.e., a
hierarchical namespace) and rich search capabilities.
Users can register with an NS (R) that uses one or more MS to obtain relevant
attributes from them (C). Additionally, users can be part of a corporate NS that
allows sharing of their select metadata with other users via standard enterprise
public-key cryptography.


\noindent\textbf{3) Activity monitors (AM)}
run on the user's devices. Their main function is to observe temporal relations,
activity context, and relationships between objects on a user's device and
transmit them to an MS (D).


\noindent\textbf{4) Attribute services (AS)}
extract attributes from storage objects and transmit them to an MS (B). An AS
might be invoked on updates, run once or periodically. For example, a file
system AS would update the object's metadata with basic attributes such as size
or modification time. There can be many AS that extract more ``interesting''
attributes, e.g., image recognition, similarity, or other classifiers.

\noindent\textbf{5) Update notification server (UNS)}
provides notification mechanisms. Users can register interest in changes of
attributes or underlying storage and will receive a message on change events (A)
to which they have access.

\subsection{\emph{Kwishut} working example}

To make the \emph{Kwishut} architecture concrete, we revisit our use-cases from
\S\ref{hotstorage:use-cases} and walk through parts of it to illustrate how \emph{Kwishut}
supports the various actions and events.

\noindent\textbf{Storing the e-mail attachment.}
\persa's act of saving the CSV file that \persc sent in email corresponds to the creation of a new object on the file server silo, i.e., the file system (4). The file server is \emph{Kwishut}-aware, so the AS co-located with it extracts attributes from the document and forwards them to the MS (B).

The AM on \persa's laptop detects that the CSV file came via company email from
\persc. It then captures the activity context identifying the relationship
between the e-mail and the CSV file and transmits it as additional metadata
about the CSV file to the MS (that already contains metadata extracted by the
AS). Moreover, because there is a company-wide namespace service, \emph{Kwishut}
establishes that the e-mail attachment, the CSV in the file server, and the one
on \persc's laptop (from which the file was sent) are exact copies of each
other.

Many applications already record some form of activity context, e.g., chat
history, browsing history. Such histories provide a rich source of additional
metadata. Other activity context, specifically the relationship between objects,
such as the fact that a particular file was saved to a local storage device from
an email message, requires more pervasive monitoring as found in, e.g., whole
provenance capture systems~\cite{camflow}. \emph{Kwishut} is agnostic about the precise
data that comprises activity context, but allows for storing and accessing
activity context as metadata.

\noindent\textbf{Creating the Excel file.}
Next, \persa opens the CSV file using Excel and stores it as a spread sheet.
This creates a new object. The AM detects that the newly created spreadsheet is
a conversion from the CSV file, either via a notification from \emph{Kwishut}-aware
Excel or by monitoring the system calls executed on the local system. \persa
proceeds to modify the data by filtering it in Excel and saving the changes. The
AM records this event and updates the meta-data of the spreadsheet to record the
derivation-relationship. Ideally a \emph{Kwishut}-aware version of Excel specifies to
the AM the exact type of the relationship (in this case a derivation); otherwise
the AM informs the MS about an unspecified data relationship by observing the
opening of a CSV file and a subsequent creation of the Excel file.

% \persa proceeds to upload the new Excel file on Slack, which triggers the creation of a new storage object as Slack creates a local copy, the addition of new metadata to MS via AS, and the addition of a \emph{copy} data relationship between the original Excel file and the Slack’s copy. The AM notices (by monitoring Slack chat) that the file was shared with user \persc and promptly notifies the MS, which adds this detail to its metadata.
% Once \persa is done, its local MS has been updated with three new objects: the CSV file, the corresponding Excel file, and Slack’s copy of the Excel file. There is a data relationship linking all three, and the metadata informing us that the original CSV came from \persc and that the final Excel file was also shared with that same person. If \persa wanted to remember what happened to the the data from the original CSV from \persc, they could query their local personal NS, which would track down this history by querying the MS metadata.

\noindent\textbf{Sharing the spreadsheet.}
As \persc  receives the Excel file from \persa via Slack on their phone, a
sequence of metadata events similar to those described earlier takes place,
except the phone does not run a local NS or MS. \persc now uploads the file to
the company's cloud drive (4). The MS (by way of AS) reflects the creation of a
new object and records its remote location. The use of a company-wide namespace
and metadata service enables \emph{Kwishut} to record that the file in the cloud drive
is, in fact, a copy of the one received via Slack.  Further, \persc informs
their personal NS that they wish to notify \persa about all updates to the file
on the cloud drive. Thus, whenever an AS sends updated attributes to the MS,
\persc receives a notification.

% The sharing relationship between the personal NS of \persa and \persc, and the exchange of the relevant cryptographic credentials, would have been set up earlier.

\noindent\textbf{Data origin and delete requests.}
When the compliance officer asks about the origin of the data, \persc can query
the corporate NS to obtain the complete history of the report. This includes the
spreadsheet from which the report was derived and the e-mail or Slack messages
that transmitted the files.
The corporate NS was configured to be aware of the locations of the
collaborating users' personal NS. Moreover, because of the activity contexts
captured by the AM, \emph{Kwishut} is able to identify documents that were created
during any activity involving the customer whose data must be deleted. Starting
from these documents, and by using the relationship of documents, \persb was
able to find all relevant objects and delete them, including the e-mail and
Slack messages.


% \persa would have configured their personal NS to allow sharing of the metadata associated with \persc with their corporate NS, and \persc would configure their personal NS similarly. As a result, when \persc issues to the corporate NS a query asking to trace the origins of the data in the final report, the corporate NS is able to return all the history tracing back to the original CSV file.

Note that unlike existing systems, \emph{Kwishut} is able to efficiently find related
objects across storage silos. Operating systems already provide users with
indexing services to accelerate search of local files. This search can be made
cross-silo by mounting and enabling indexing on network shares (e.g., Windows
Desktop Search), or by interfacing with specific applications such as e-mail
(e.g., MacOS Spotlight, or Android search). The problems with indexing on a
large network storage repository are resource limitations such as bandwidth and
local storage that may render the system unusable during indexing. In contrast,
\emph{Kwishut} addresses these limitations by delegating indexing and storage to one or
more services.

NS are responsible for providing efficient search functionality. \emph{Kwishut} uses AS
to keep attributes up to date with object modifications. Lastly, \emph{Kwishut}
supports coordinated search among one or more local and remote NS, allowing, for
example, a user to search across both their local NS as well as their employer's
NS.


% \sasha{There are a few remaining pieces that we did not mention. Please look at the commented text at the end of architecture-new.latex to see if you want to restore some of that text.}
%\subsection{The remaining pieces}

%To complete the description of \emph{Kwishut} here we fill in some of the missing details.

%\textbf{Metadata deletion and updates.} Metadata associated with a storage object can be deleted underlying storage object is deleted or when the user is required to comply with legal requirements, such as the “right to be forgotten”. The metadata of the object is updated (via push or pull by the AS) if the object changes in the underlying storage silo or if the user (or \emph{Kwishut}-aware applications) choose to create additional metadata, e.g., run image recognition on photo files and record the names of identified places and persons.

%\textbf{Security} \sasha{I am just copying what we had in the original document, but I don't know if this is useful. Cut this? } We base our security requirements on a simple threat model that considers (1) protection of the meta-data itself and (2) protecting information that might be gleaned from activity within \emph{Kwishut}.  We assume the primary threat here is inadvertent disclosure, particularly through the use of third-party service providers. Secondary threats include the ability to verify attribute information provided by external parties, including anti-repudiation as well as tampering. \MIS{I don't understand that last sentence; do you meant that the threat is attribute spam?}

%Using public key mechanisms for signing attributes provides a standard mechanism for verifying the authenticity of the attribute itself; forged signatures can be detected using other information from the \emph{Kwishut} object.  For example, an object stored in a given silo would require use of a known set of digital signatures from the relevant authoritative name service. \MIS{That last sentence seems backwards to me; I don't understand our claims.}

%\textbf{Data relationships} While \emph{Kwishut} treats the designated data relationships automatically, users and applications can create metadata for other relationships. For example, the corporate compliance officer might find it helpful if implementations of data security policies were explicitly linked to the appropriate regulation or mandate.

\subsection{Future Directions}
\label{hotstorage:future}
We now explore a few research directions that \emph{Kwishut} suggests.

\noindent\textbf{Attribute Security and Integrity.}
\emph{Kwishut} decouples naming and attributes from the storage object. This opens up a
research direction on the security model of attributes themselves. Are the
permissions on the attributes similar to the ones on the storage objects
themselves? Can a user change an attribute in its local namespace, but not in
the company wide one? This segues into the question of attribute
integrity/quality: not all attribute sources have the same trust-level. For
instance, a user might label an image ``dog,'' while the image recognition AS
might label it ``cat''.

\noindent\textbf{Privacy.}
\emph{Kwishut} collects a lot of metadata across multiple communicating channels and
storage silos, including activity data. This raises the question of how to
manage these metadata in a privacy-preserving manner.

\noindent\textbf{Interface Design.}
We presented a system architecture that provides a rich context to search and
organize storage objects. We envision that this will provide the foundation for
new directions in HCI research: By using individual namespaces, we can
dynamically organize and visualize documents and other storage objects, and
seamlessly navigate and locate related documents providing a new user
experience.

\noindent\textbf{Relationship-based Queries.}
\emph{Kwishut} tracks relationships between storage objects. These relationships
provide minimal data provenance~\cite{provprimer} allowin users to locate the
chain of related documents originating in a specific activity context.

These relationships are most naturally expressed as graphs, where nodes are
objects, and edges are the relationships between objects. Edges could have
weighted-labels, indicating the type and importance of their relationship.

This enables more sophisticated data-analysis beyond pure content-based indexing
by using graph queries.

For instance, lineage queries (i.e., tracing the history of an object) are path
traversals, which are challenging to implement efficiently in conventional
storage systems. This suggests that the NS and/or MS require sophisticated
storage and query mechanisms.

\subsection{Related Work}
\label{hotstorage:background}

\emph{Kwishut} draws on prior work in semantic file systems, using search to locate
documents, and federated naming systems.

\noindent\textbf{Semantic File Systems.}

Although we introduced our desire for semantically meaningful names with
reference to the Placeless architecture, the idea originated in the systems
community with the semantic file system~\cite{giffordSFS}, SFS.

SFS used automatically extracted attributes to construct virtual directories
that contained collections of semantically related documents.

There exist many extensions or variants on this theme such as per-process
namespaces~\cite{plan9}, inverting the database/file system layering to build
file systems on top of queriable databases~\cite{inversion},
manually tagged file systems~\cite{tagfs},
constructing semantic metadata stores from distributed
storage~\cite{smartstore}, and systems that manage conventional and semantic
structures in parallel~\cite{gfs}.
\emph{Kwishut} represents another step in this evolution.
It extends prior work by combining semantic naming with user activity context
and is designed for today's multi-silo'd storage encompassing everything from
mobile devices to desktops to object stores to cloud storage.


\noindent\textbf{Search.}
An alternative to creating a semantically meaningful name space is to enable extensive metadata-based search.
Desktop tools such as Apple’s Spotlight, Linux KDE Baloo, and Windows
Desktop Search adopt this approach.
However, breakthroughs in web search (i.e., incorporation of pagerank~\cite{page1999pagerank}) demonstrated that the relationship among objects is at least as important as the metadata itself.
The efficacy of provenance-assisted search~\cite{provsearch,uprove2,pindex} demonstrates that history, in addition to relationships enhance users' ability to locate documents.
However, searching for documents is fundamentally different from naming.
Search-based approaches rely either on a user to select the correct item from many presented or on the sufficiency of providing \emph{any} relevant document.
However, naming requires the ability to identify a specific document. \emph{Kwishut} is designed to support both searching and uniquely identifying a specific document.

\noindent\textbf{Multi-silo Data Aggregation.}
%\MIS{There is an entire market for consultants who help people manage data in multiple silos; who knew?}
%\MIS{It seems that the biologists are also really interested in multi-silo search.}
UNIX mount points~\cite{unix} are perhaps the first instance of federating namespaces.
Distributed federation, as provided by distributed file systems such as NFS~\cite{nfs} and AFS~\cite{howard1988scale} followed soon after adoption of local area networks.
With the advent of cloud storage, there has been work in federated namespaces that span
cloud stores~\cite{scfs,federatedMetaData}.
Nextcloud (\url{https://nextcloud.com}) allows users to connect multiple Nextcloud instances and integrate with
FTP, CIFS, NFS and Object stores. Yet, documents are still organized in a classic
hierarchical structure. Peer-to-peer sharing networks (e.g., IPFS \cite{benet2014ipfs}) implement a distributed
file system where nodes advertise their files to users.
MetaStorage~\cite{metastorage} implements a highly available, distributed hash table,
% similar to Amazon's DynamoDB, %% trying to cut a line or two to get us under the limit and if we leave this here, it needs a reference
but with
its data replicated and distributed across different cloud providers.
% MetaStore offers a key-value store interface.
Farsite~\cite{Adya:2003:Farsite} organizes multiple machines into virtual file servers, each of which acts as the root of a distributed file system. Comet describes a cloud oriented federated metadata service~\cite{federatedMetaData}.

\subsection{Conclusion}
\label{hotstorage:conclusion}

We have presented our position that we need \emph{Kwishut}, a storage
architecture that decouples naming from the storage location of documents and
data objects, provides customizable and personalized namespaces, and that makes
relationships between documents a first class citizen. With \emph{Kwishut},
users will be able to organize, share and find their data conveniently across
multiple storage silos using a rich set of attributes breaking away from the
rigid, hierarchical organization.

We expect \emph{Kwishut} will enable a broad area of research in HCI exploring
new ways to visualize and interact with data using the mechanism's provided by
\emph{Kwishut}. Moreover, we expect \emph{Kwishut} to provide interesting
scenarios for security and privacy research in storage systems.


\backmatter
%    7. Index
% See the makeindex package: the following page provides a quick overview
% <http://www.image.ufl.edu/help/latex/latex_indexes.shtml>


\end{document}
